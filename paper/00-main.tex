% This is samplepaper.tex, a sample chapter demonstrating the
% LLNCS macro package for Springer Computer Science proceedings;
% Version 2.21 of 2022/01/12
%
\documentclass[runningheads]{llncs}
%
\usepackage[T1]{fontenc}
% T1 fonts will be used to generate the final print and online PDFs,
% so please use T1 fonts in your manuscript whenever possible.
% Other font encondings may result in incorrect characters.
%
\usepackage{graphicx}
% Used for displaying a sample figure. If possible, figure files should
% be included in EPS format.
%
% If you use the hyperref package, please uncomment the following two lines
% to display URLs in blue roman font according to Springer's eBook style:
\usepackage{color}
\usepackage{xcolor}
%\usepackage{url}
\usepackage{amsmath}
\usepackage{caption}
\usepackage{hyperref}
\renewcommand\UrlFont{\color{blue}\rmfamily}
\urlstyle{rm}

\usepackage{soul}
%

\newcommand{\squeezeup}{\vspace{-3mm}}

\begin{document}
%
\title{Towards 
%Inter-Document 
Disambiguation of Mathematical Terms based on Semantic Representations}
%
%\titlerunning{Abbreviated paper title}
% If the paper title is too long for the running head, you can set
% an abbreviated paper title here
%
%\author{First Author\inst{1}\orcidID{0000-1111-2222-3333} \and Second Author\inst{2,3}\orcidID{1111-2222-3333-4444} \and Third Author\inst{3}\orcidID{2222--3333-4444-5555} }
%
\authorrunning{F. Author et al.}
% First names are abbreviated in the running head.
% If there are more than two authors, 'et al.' is used.
%
%\institute{Princeton University, Princeton NJ 08544, USA \and Springer Heidelberg, Tiergartenstr. 17, 69121 Heidelberg, Germany \email{lncs@springer.com }\\
 
%
\maketitle              % typeset the header of the contribution
%
\begin{abstract}
In mathematical literature, terms can have multiple meanings based on context. Manual disambiguation across scholarly articles demands massive efforts from mathematicians. 
%This paper addresses the challenge of \st{automatically determining whether two definitions of a mathematical term are semantically different.}\ann{definiendum disambiguation, or automatically determining the equivalence of mathematical terms according to their definitions.}
This paper addresses the challenge of automatically determining whether two definitions of a mathematical term are semantically different.
%Unlike existing approaches in mathematical term disambiguation, which are based on document-level information or syntactic features, 
Specifically, the difficulties and how contextualized textual representation can help resolve the problem, are investigated.
%We construct a 
A new dataset MathD2 for %fine-grained 
mathematical term disambiguation is constructed with ProofWiki's disambiguation pages. %Given a definition containing an ambiguous term, the goal is to link it to its corresponding identifier, indicated by its page title in ProofWiki. %We then study two 
Then two approaches based on the contextualized textual representation %of the mathematical contents
are studied: (1) supervised classification based on the embedding of concatenated definition and title and (2) zero-shot prediction based on semantic textual similarity(STS) %the cosine similarity 
between definition and title. 
Both approaches achieve accuracy and macro F1 scores greater than 0.9 on the ground truth dataset, demonstrating the effectiveness of our methods for the automatic disambiguation of mathematical definitions. Our dataset, code and experimental results are available here: \url{https://anonymous.4open.science/r/ecir2025-4137-5B97} %Our results show challenges and promising applications of sentence embeddings for mathematical language processing and information retrieval.

\keywords{%Word Sense Disambiguation\and 
Entity Linking \and Text Similarity \and Transformers \and Mathematical Definition.}
\end{abstract}
%

\section{Introduction}
%Context and Motivation: 

Mathematical scholarly articles contain highly structured statements, such as axioms, theorems, and proofs, which cannot be easily navigated or explored through traditional keyword searches. 
Several initiatives have emerged to improve knowledge discovery from academic papers such as the
Open Research Knowledge Graph (ORKG)~\cite{auer2020orkg}, a project that aims to describe research papers in a structured manner to facilitate the comparison among papers,
or TheoremKB~\cite{mishra2024first,mishraPS21}, a project that tries to build a knowledge base of interlinked theorems and proofs across scientific literature. 
This work aims to automatically construct a knowledge base of mathematical definitions. %Such a knowledge base allows researchers to look up terms efficiently, and helps to index relevant mathematical statements and articles with the terms therein.  
%It can also help writers compare their definitions with others’ formulations to detect errors such as missing conditions.

% decribe the mathematical definitiosns ang the challeges
Existing works in this research area involve extraction of mathematical definitions~\cite{berlioz2023hierarchical,asperti_extraction_2004,sun2023discoveringregex,vanetik2020automatedlstmcnn} and the terms defined therein, called \textit{definienda} or \textit{definiendum} for its singular form~\cite{berliozargot2021,jiang2023extracting}. These tasks are extended by disambiguating or linking the newly extracted definition-term pair to existing concepts in a reference glossary, or otherwise, extend the glossary.
%\st{Indeed, there could be different but equivalent definitions of the same concept and polysemous notions as definienda (see in Table).} 
%\ann{
Definiendum disambiguation is challenging particularly in cases where identical terms for the same concept are defined in various ways (e.g., ``path''), or while polysemous terms (e.g., ``block'') refer to distinct concepts (see in Table~\ref{tab:paper_def_example}). %Table ~\ref{tab:paper_def_example} provides examples illustrating these scenarios.
%Examples illustrating these scenarios are provided in Table~\ref{tab:paper_def_example}.
%}
A possible heuristic is that if the definienda of two definitions are linked to two different concepts in a reference knowledge base, then these two definitions are different. This scenario is simplified by assuming that each definition defines one definiendum. 
\squeezeup
\begin{table}
    \centering
    \caption{Definitions extracted from different scholarly articles~\cite{jiang2023extracting}. %The definition of ``path'' has different formulations. The notion of ``block'' has different meanings.
    }
    \begin{tabular}{|l|p{0.8\linewidth}|}
          \hline
          Definiendum&Definition and Source Article\\\hline
          path&If the vertices $v_0,v_1,\ldots,v_k$ of a walk $W$ are distinct then $W$ is called a \emph{Path}. A path with $n$ vertices will be denoted by $P_n$. $P_n$ has length $n-1$.~\cite{kalayathankal2015operationspath1}\\\hline
          path&Let $G=(V,E)$ be a graph. A \emph{path} in a graph is a sequence of vertices such that from each of its vertices there is an edge to the next vertex in the sequence. This is denoted by $P=(u=v_0,v_1\ldots,v_k=v)$, where $(v_i,v_{i+1}) \in E$ for $0 \le i \le k-1$.~\cite{perera2012bipartitionpath2} \\\hline
          block&A \emph{block} in  $H$ is a maximal set of tightly-connected hyperedges.~\cite{ergemlidze20193block2}\\\hline
          block&A \emph{block} of indices is a set of numbers $S$ where every term $SG_{a,b}(s)$ depends on the same value via division, for all $s\in S$.~\cite{kupin2011subtractionblock1}\\\hline
    \end{tabular}
    \label{tab:paper_def_example}
\end{table}

%ProofWiki

%\st{ProofWiki~\footnote{~\url{https://ProofWiki.org/wiki/Main_Page}} is a crowdsourced online collection of mathematical results. It provides 500 definition disambiguation pages~\footnote{~\url{https://ProofWiki.org/wiki/Category:Definition_Disambiguation_Pages}} by the time of writing.} 
%\ann{
For the purpose of this study, ProofWiki~\footnote{\url{https://ProofWiki.org/wiki/Main_Page}} is used as the reference list. It is a crowd-sourced online collection of mathematical proofs, which includes 500 disambiguation pages. 
Similar to Wikipedia, these disambiguation pages contain lists of identical terms, each linking to its corresponding definition page. Each definition page contains a unique page title, the definition %of the term,
and a topic or category where this term can be found (e.g., algebra or geometry)%, and a list of references
. Specifically, the page title contains the definiendum along with its category and serves as the identifier of the definition page within ProofWiki.
%}
%\st{Each definition page is identified with a unique title that contains the definiendum and information about the category, like algebra or geometry.}

This work 
%proposes to leverage these disambiguation pages as ground truth~\footnote{Our dataset, code and experimental results are available here: \url{https://anonymous.4open.science/r/ecir2025-4137-5B97}} to: address 
addresses the following research questions \textit{\textbf{RQ1:} How well can contextualized text representation help the disambiguation of mathematical terms? \textbf{RQ2:} Which architectures and pretraining strategies is best suited for the task?} The following are the main contributions of this work: %The contributions of this work are summarized as follows:
%\vspace{-3mm}
\begin{itemize}
    \item \textbf{MathD2} - a new dataset for \textbf{Math}ematical \textbf{D}efiniendum \textbf{D}isambiguation
    \item Exploration of \textbf{two different approaches} that show how the disambiguation task can benefit from contextualized semantic representations 
    \item \textbf{Experiment-supported evidence} %is provided
    highlighting the efficiency of sentence embeddings for the addressed disambiguation task.
\end{itemize}
%After discussing related work in Section~\ref{sec:related_work}, the two proposed approaches are presented in Section ~\ref{sec:methology}, while conducted experiments and result analysis are discussed in Section~\ref{sec:experiments}, followed by findings and conclusions in Section~\ref{sec:conclusion}.

\section{Related Work}
\label{sec:related_work}
The challenges posed by this task are (a) the lack of labeled datasets for equivalent mathematical definitions, (b) the limited number of disambiguation pages, and (c) 
%\st{the interweave of discourse, notations, and formulas that differentiate mathematical content from text in general domain}
%\ann{
the unstructured nature of definitions that combine mathematical notations, formulas, and general discourse%}
~\cite{jiang2023extracting,vanetik2020automatedlstmcnn}.  To address (a), entity linking and sentence similarity approaches for mathematical terms are reviewed. To tackle (b) and (c), transformer models~\cite{vaswani2023attentionneed} are employed for their capabilities to produce rich, contextualized representations.% enhanced by domain-adapted pretraining, potentially beneficial for mathematical language processing.


Contextualized representations produced by BERT (Bidirectional Encoder Representations from Transformers)~\cite{devlin2019bert} encode the meaning of a word according to its context. This means that polysemous words have several, more accurate representations depending on the sentence where they appear. BERT is pretrained on two key tasks: Masked Language Modeling (MLM), where random tokens in a sentence are masked and predicted based on context, and Next Sentence Prediction (NSP), which trains BERT to determine whether a sentence logically follows another.
%understand sequential relationships between sentences by distinguishing if a sentence logically follows another.  
Pretraining with MLM is widely applied for domain adaptation 
%and is proven to improve the performance of downstream tasks
, especially when there is a dearth of data for finetuneing~\cite{mishraPS21,jiang2022choubert}. In addition, finetuning BERT for specific downstream tasks and domains is straightforward. For instance, by combining BERT's output with a classification layer, it has been adapted for mathematical notation prediction~\cite{jo2021notation}, definiendum extraction~\cite{jiang2023extracting} and mathematical statement extraction~\cite{mishra2024first}. 
%Pretraining with MLM is widely applied for domain adaptation and is proven to improve the performance of downstream tasks, especially when there is not much data for finetuneing~\cite{mishraPS21,jiang2022choubert}.
The Natural Language Inferernce (NLI) datasets~\cite{bowman-etal-2015-large,williams-etal-2018-broad} used by BERT's NSP pretraining are related to the task at hand. A piece of supporting evidence is AcroBERT~\cite{chen2023gladis}, an entity linker that reuses BERT for NSP's pretrained weights and is finetuned to link acronyms to their long forms. AcroBERT outperforms BERT and other domain-adapted BERT-based models.% by its pretraining and triplet framework. 

%Another related task in automatic scientific document analysis is the disambiguation of mathematical identifiers~\cite{asakura2024intra_identifier}.

%However, due to the nature of the BERT's pretraining tasks, it is not suitable for measuring semantic similarity. 
However, the nature of the BERT's pretraining tasks makes it unsuitable for measuring semantic similarity. Sentence BERT (SBERT)~\footnote{\url{https://huggingface.co/sentence-transformers/all-mpnet-base-v2}} ~\cite{reimers2019sentence} modifies BERT's architecture to produce semantically meaningful sentence embeddings that can be compared using cosine-similarity.
%MiniLM~\cite{wang2020minilm} is a deep self-attention distillation approach to simply and effectively compress large transformer-based pretrained models. It works by having the student model mimic the teacher model's self-attention modules, particularly focusing on the distributions and value relations in the teacher's final layer.
%\ann{
Out-of-the-box SBERT achieves superior performance across varied classification tasks involving mathematical texts~\cite{steinfeldt2024evaluation}. In one such task, the proponents measure the similarity of SBERT embeddings between an input text and the combination of titles and abstracts of mathematical publications in arXiv~\footnote{\url{https://arxiv.org/}} and zbMATH~\footnote{\url{https://zbmath.org/}} to predict the classification code of the respective repositories.
%}
%\st{Regarding the application of SBERT to mathematical text, proposes to use titles, abstracts, and the classification codes from mathematical publication databases as a benchmark to evaluate similarity models for short mathematical texts. This study shows that the out-of-box SBERT achieves good results in recommending classification codes from textual data.} 
In the same vein, this study aims to evaluate the effectiveness of semantic textual similarity in linking definitions to titles. Since BERT for NSP and SBERT require different domain adaptation strategies~\cite{reimers2019sentence,steinfeldt2024evaluation}, this work first identifies the architecture that performs better for the task.
%Two small SBERT-like models derived from MiniLM~\cite{wang2020minilm}, namely SBERT-all-MiniLM-L6-v2~\footnote{\url{https://huggingface.co/sentence-transformers/all-MiniLM-L6-v2}} and SBERT-all-MiniLM-L12-v2~\footnote{\url{https://huggingface.co/sentence-transformers/all-MiniLM-L12-v2}}, are also studied in~\cite{steinfeldt2024evaluation}, both perform slightly inferior to SBERT.


\section{Methodology}
\label{sec:methology}
Term disambiguation is formalized as an entity linking task, where the entities refer to the definition page titles in ProofWiki. That is, given (1) %\st{a definition that defines an ambiguous term (we assume that the definiendum has already been extracted) } 
a definition and an ambiguous definiendum
and (2) a dictionary that maps the ambiguous definiendum to entities% (the page titles of the definition in ProofWiki)
, the goal is to find the title that best matches the definition. The proposed method is described in two steps. First, the ground truth dataset is constructed. Second, two applicable approaches
%with related models 
%to achieve the goal 
are considered.
\subsection{Construction of the MathD2 Dataset}
A dump of the whole ProofWiki was extracted on the 9th July, 2024 using WikiTeam~\cite{wikiteamwikiteam_2024}.
%We use WikiTeam~\cite{wikiteamwikiteam_2024} to get a dump of the whole WikiProof on 9th July, 2024.  
This dump is then parsed to get all the definition statements and titles from all definition disambiguation pages. The extracted definitions is converted to plain text.  By mapping ambiguous terms and the corresponding definition titles is finally constructed.
%We convert the extracted definitions to plain text.  We build a dictionary by mapping ambiguous terms and the corresponding definition titles. 
%We remark that s
Some definitions might contain other definitions(e.g., the definition of ``Loop''~\footnote{\url{https://proofwiki.org/wiki/Definition:Loop_(Topology)}}), which also happens to definitions in scholarly papers. If both definition titles are mapped to a common ambiguous term, only the nested definition and its title are kept, because otherwise the outer definition should be mapped to two titles: its title and the one of the nested definition. %Finally, we remove terms that are mapped to less than two titles. 
Finally, terms mapped to less than two titles are removed. 
Table~\ref{tab:proofwikidata} shows (definition, title) pairs extracted from the disambiguation page of ``Bilinear Form'~\footnote{\url{https://proofwiki.org/wiki/Definition:Bilinear_Form}}.
\squeezeup
\begin{table}
    \centering
    \caption{Data extracted from a ProofWiki disambiguation page.}
    \begin{tabular}{|p{0.65\linewidth}|p{0.3\linewidth}|}
          \hline
           Definition&Title\\\hline
            Let $R$ be a ring. Let $R_R$ denote the $R$-module $R$. Let $M_R$ be an $R$-module. A bilinear form on $M_R$ is a bilinear mapping $B : M_R \times M_R \to R_R$.&Definition:Bilinear Form (Linear Algebra) \\\hline
            A bilinear form is a linear form of order $2$.&Definition:Bilinear Form (Polynomial Theory)\\\hline
    \end{tabular}
    \label{tab:proofwikidata}
\end{table}
For the finetuning in Section~\ref{sec:nsp}, %we split the sets
the dataset is split based on the ambiguous terms at the ratio of 8:2, making a training dataset of 275 ambiguous terms with 1436 (definition, title) pairs and a test dataset of 68 ambiguous terms with 433 (definition, title) pairs.  In the finetuning %phase 
of Section~\ref{sec:nsp}, for each ambiguous term, %we randomly select 
two definitions and their titles are randomly selected to make positive pairs, and the titles of two other definitions to make negative pairs. %Figure~\ref{fig:train_sample} shows an example. 
%We evaluate both
Both approaches are evaluated on the training and test datasets, except for the finetuned model. 


\subsection{Classification Based on One Concatenated Embedding}
~\label{sec:nsp}

Following the finetuning setup of AcroBERT~\cite{chen2023gladis}, %we adapt 
BERT for NSP is adapted to build a supervised sentence pair classifier to link definitions to their page titles in ProofWiki. Every pair of (definition, candidate title with the matching ambiguous term in ProofWiki) is concatenated as an input sequence. The sequence begins with a [CLS] token, followed by a candidate title, a [SEP] token, and then the definition, ending with [SEP]. The input sequence passes through BERT's transformer layers. These layers produce contextual embedding for each token in the sequence. Then, the embedding of [CLS] is fed into a softmax classification layer, which outputs a score to judge how coherent the concatenated sequence is. %We select the
The pair with the highest score is selected as the final predicted output. %We first use the
First the out-of-box BERT for NSP serves as the baseline to see how well the pre-retained natural language inference model can describe the entailment between the titles and definitions. %We then finetune the 
Then the pretrained BERT for NSP is finetuned with the training set using a triplet loss function $\mathcal{L} = \max \left \{ 0, \lambda - d_{\text{neg} } + d_{\text{pos} } \right \}$ that aims to assign higher scores to the correct titles that match the input definition while reducing the scores of irrelevant candidates, where $\lambda = 0.2$ is the margin value, and $d_{\text{pos}}$ and $d_{\text{neg}}$ are the distances for positive and negative pairs, respectively. %We implement this 
This approach is implemented with PyTorch~\cite{paszke2019pytorch} and transfomers~\cite{wolf-etal-2020-transformers}. %We use a 
A batch size of 16 and Adam optimizer with learning rate 1e-5 are used. The learning rate is exponentially decayed at a rate of 0.95 every 1000 steps. %We train the 
The model is trained with the training dataset for 100 epochs and evaluated with the test dataset after each epoch. %All our experiments are run with an NVIDIA Tesla V100S-PCIE 32GB GPU.





\subsection{Zero-shot Prediction Based on Semantic Textual Similarity}
\label{sec:sim}
A shortcoming of the previous solution is that the NSP inference has to be run for every (definition, title) pair mapped to an ambiguous term. Motivated to make a computationally more efficient solution, %e explore 
the sentence embeddings of the definitions and titles are explored. 
%We suggest that a definition and its title should be more related than paired with other titles. 
In this setup, %we calculate 
the sentence embedding of the titles and the definitions only need to be calculated once. For the definition and each candidate title with the matching ambiguous term, %we select 
the title with the highest cosine similarity to the embedding of the definition is selected as the final predicted output.  To explore the potential benefits of different pretraining corpus and related tasks, %we use 
the best-performing sentence transformers for Semantic Textual Similarity(STS) tasks as reported in~\cite{steinfeldt2024evaluation} and out-of-box SBERT are used% and also other BERT-based models pretrained for mathematical language processing
. Following SBERT's default setting~\cite{reimers2019sentence}, %we use 
the mean pooling strategy is used to calculate the sentence embeddings. 

\section{Results and Discussion}
\label{sec:experiments}
%We use a
Accuracy and the average of the $F_1$ score for each ambiguous term (macro $F_1$ score) are used to measure how well both approaches can link a definition to the correct title. Table~\ref{tab:all_scores} shows the experimental results of both methods. Overall, our finetuned NSP model performs best, validating AcroBERT's set-up and the helpfulness of BERT for NSP's pretrained weights. Notably, the out-of-the-box SBERT demonstrated excellent performance with much less inference time.
%The performance of prediction based on STS with SBERT models are aligned with the results of ~\cite{reimers2019sentence} and ~\cite{steinfeldt2024evaluation}. 
Given that both BERT for NSP and SBERT are pretrained on NLI tasks~\cite{devlin2019bert,reimers2019sentence}, it may be deduced that i) compared to using the [CLS] representation of concatenated sequence, using separated sentence embeddings captures more information for our task, and/or ii) SBERT's pretraining on (title, abstract) pairs from S2ORC dataset~\cite{lo-wang-2020-s2orc} helps to better understand the entailment between titles and body texts. However, 
%Bert-MLM\_arXiv-MP-class\_zbMath, a
the domain-adapted SBERT model~\footnote{\url{https://huggingface.co/math-similarity/Bert-MLM_arXiv-MP-class_arXiv}}  that the authors of ~\cite{steinfeldt2024evaluation} finetuned with multiple tasks using titles and abstracts of mathematical papers does not yield better results. This might be due to MLM being solely trained on titles and abstracts, diminishing the model's representational capacity for general text. 

\textbf{Limitations:} An interesting finding %of this work
is that SBERT for STS and the finetuned BERT for NSP make some common mistakes, indicating the limits of using only semantic representations. The most common error is when the definition statement includes nested definitions.  Another typical error is that the predicted result is in the correct category but not the definiendum, mainly when the definition contains morphemes in the predicted title or when the definition does not contain some morphemes in the expected title. For example, the definition of ``Consequence Function'' starts with ``Let $\mathbf{G}$ be a game...''~\footnote{\url{https://proofwiki.org/wiki/Definition:Consequence_Function}}
, and the predicted title is ``Definition:Consequence(Game Theory)'~\footnote{\url{https://proofwiki.org/wiki/Definition:Consequence_(Game_Theory)}}
. Thus, enhancing sentence embedding's comprehension of semantic and syntactic knowledge of mathematical definitions is still worth investigating. Other common mistakes reveal the noises in the dataset due to automatic scrapping and \LaTeX conversion of irregular ProofWiki pages.

\squeezeup
\begin{table}
    \centering
    \caption{Accuracy and macro $F_1$ scores. Values are reported as $\rho \cdot 100$.}
    \begin{tabular}{|p{0.4\linewidth}|p{0.15\linewidth}|p{0.1\linewidth}|p{0.1\linewidth}|p{0.1\linewidth}|p{0.1\linewidth}|}
    \hline
        Model & Approach& \multicolumn{2}{l|}{Test} & \multicolumn{2}{l|}{Train}\\ \cline{3-6}
         & & $F_1$ & Acc. & $F_1$ & Acc. \\ \hline
        BERT~\cite{devlin2019bert} & NSP& 80,9 & 84,8 & 79,8 & 83,9 \\ \hline
        Finetuned BERT & NSP& \textbf{92,1} & \textbf{93,8} & - & - \\ \hline
   %     BERT & STS& 26,1 & 35,3 & 30,2 & 40,3 \\ \hline
   %     CC-BERT~\cite{mishraPS21} &STS& 28,2 & 37,9 & 35,6 & 45,4 \\ \hline
        %SBERT-all-MiniLM-L6-v2&STS& 90,1 & 92,4 & 88,1 & 91,0 \\ \hline
        %SBERT-all-MiniLM-L12-v2&STS& 91,2 & 93,3 & \textbf{89,4} & \textbf{92,1} \\ \hline
        SBERT-all-mpnet-base-v2~\cite{reimers2019sentence}
        &STS& \textbf{91,4} & \textbf{93,5} & 89,0 & 91,6 \\ \hline
        %Bert-MLM\_arXiv~\cite{steinfeldt2024evaluation} &STS& 28,3 & 38,3 & 32,8 & 42,7 \\ \hline
        Adapted SBERT~\cite{steinfeldt2024evaluation}%~\footnote{\url{https://huggingface.co/math-similarity/Bert-MLM_arXiv-MP-class_arXiv}} 
        &STS& 43,8 & 52,2 & 54,0 & 61,5 \\ \hline
    \end{tabular}
    \label{tab:all_scores}
\end{table}






\squeezeup
\section{Conclusion and Future Works}
~\label{sec:conclusion}
This work introduces a new dataset for %fine-grained
mathematical term disambiguation with ProofWiki. Two entity linking approaches have been implemented and shown to yield advantages in the usage of contextualized embeddings to differentiate mathematical definitions. The experimental results proved the efficiency and effectiveness of using out-of-the-box SBERT. %This work indicates the need for further study on finetuning general purpose pretrained models for new domains and tasks. %not conventionally considered ideal for transfer learning.
%This work opens up future research on building sentence embeddings while benefiting from domain-specific MLM and task-related pretraining.} 
Further work is planned on applying the proposed approaches on scholarly papers. In addition, the current approach is to be extended to include document-level representation and citation information to differentiate definitions in scholarly papers. This work also indicates the need for further study on building sentence transformers that benefit from domain-specific MLM and task-related pretraining.

%For future work, we will apply the study to scholarly papers and not just ProofWiki
%, especially text extracted from PDF articles because the LaTeX source is not always available
%We also plan to combine sentence representation with document-level representation and citation information to differentiate definitions in scholarly papers. 


%\begin{credits}
%\subsubsection{\ackname} A bold run-in heading in small font size at the end of the paper is used for general acknowledgments, for example: This study was fundedby X (grant number Y).

%\subsubsection{\discintname}
%It is now necessary to declare any competing interests or to specifically state that the authors have no competing interests. Please place the statement with a bold run-in heading in small font size beneath the(optional) acknowledgments\footnote{If EquinOCS, our proceedings submissionsystem, is used, then the disclaimer can be provided directly in the system.},for example: The authors have no competing interests to declare that are relevant to the content of this article. Or: Author A has received research grants from Company W. Author B has received a speaker honorarium from Company X and owns stock in Company Y. Author C is a member of committee Z.
%\end{credits}
%
% ---- Bibliography ----
%
% BibTeX users should specify bibliography style 'splncs04'.
% References will then be sorted and formatted in the correct style.
%
\bibliographystyle{splncs04}
\bibliography{99-Bibliography}

\end{document}
