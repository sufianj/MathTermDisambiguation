\section{Introduction}
%Context and Motivation: 

Mathematical scholarly articles contain highly structured statements, such as axioms, theorems, and proofs, which cannot be easily navigated or explored through traditional keyword searches. 
Several initiatives have emerged to improve knowledge discovery from academic papers such as the
Open Research Knowledge Graph (ORKG)~\cite{auer2020orkg}, a project that aims to describe research papers in a structured manner to facilitate the comparison among papers,
or TheoremKB~\cite{mishra2024first,mishraPS21}, a project that tries to build a knowledge base of interlinked theorems and proofs across scientific literature. 
This work aims to automatically construct a knowledge base of mathematical definitions. %Such a knowledge base allows researchers to look up terms efficiently, and helps to index relevant mathematical statements and articles with the terms therein.  
%It can also help writers compare their definitions with others’ formulations to detect errors such as missing conditions.

% decribe the mathematical definitiosns ang the challeges
Existing works in this research area involve extraction of mathematical definitions~\cite{berlioz2023hierarchical,asperti_extraction_2004,sun2023discoveringregex,vanetik2020automatedlstmcnn} and the terms defined therein, called \textit{definienda} or \textit{definiendum} for its singular form~\cite{berliozargot2021,jiang2023extracting}. These tasks are extended by disambiguating or linking the newly extracted definition-term pair to existing concepts in a reference glossary, or otherwise, extend the glossary.
%\st{Indeed, there could be different but equivalent definitions of the same concept and polysemous notions as definienda (see in Table).} 
%\ann{
Definiendum disambiguation is challenging particularly in cases where identical terms for the same concept are defined in various ways (e.g., ``path''), or while polysemous terms (e.g., ``block'') refer to distinct concepts (see in Table~\ref{tab:paper_def_example}). %Table ~\ref{tab:paper_def_example} provides examples illustrating these scenarios.
%Examples illustrating these scenarios are provided in Table~\ref{tab:paper_def_example}.
%}
A possible heuristic is that if the definienda of two definitions are linked to two different concepts in a reference knowledge base, then these two definitions are different. This scenario is simplified by assuming that each definition defines one definiendum. 
\squeezeup
\begin{table}
    \centering
    \caption{Definitions extracted from different scholarly articles~\cite{jiang2023extracting}. %The definition of ``path'' has different formulations. The notion of ``block'' has different meanings.
    }
    \begin{tabular}{|l|p{0.8\linewidth}|}
          \hline
          Definiendum&Definition and Source Article\\\hline
          path&If the vertices $v_0,v_1,\ldots,v_k$ of a walk $W$ are distinct then $W$ is called a \emph{Path}. A path with $n$ vertices will be denoted by $P_n$. $P_n$ has length $n-1$.~\cite{kalayathankal2015operationspath1}\\\hline
          path&Let $G=(V,E)$ be a graph. A \emph{path} in a graph is a sequence of vertices such that from each of its vertices there is an edge to the next vertex in the sequence. This is denoted by $P=(u=v_0,v_1\ldots,v_k=v)$, where $(v_i,v_{i+1}) \in E$ for $0 \le i \le k-1$.~\cite{perera2012bipartitionpath2} \\\hline
          block&A \emph{block} in  $H$ is a maximal set of tightly-connected hyperedges.~\cite{ergemlidze20193block2}\\\hline
          block&A \emph{block} of indices is a set of numbers $S$ where every term $SG_{a,b}(s)$ depends on the same value via division, for all $s\in S$.~\cite{kupin2011subtractionblock1}\\\hline
    \end{tabular}
    \label{tab:paper_def_example}
\end{table}

%ProofWiki

%\st{ProofWiki~\footnote{~\url{https://ProofWiki.org/wiki/Main_Page}} is a crowdsourced online collection of mathematical results. It provides 500 definition disambiguation pages~\footnote{~\url{https://ProofWiki.org/wiki/Category:Definition_Disambiguation_Pages}} by the time of writing.} 
%\ann{
For the purpose of this study, ProofWiki~\footnote{\url{https://ProofWiki.org/wiki/Main_Page}} is used as the reference list. It is a crowd-sourced online collection of mathematical proofs, which includes 500 disambiguation pages. 
Similar to Wikipedia, these disambiguation pages contain lists of identical terms, each linking to its corresponding definition page. Each definition page contains a unique page title, the definition %of the term,
and a topic or category where this term can be found (e.g., algebra or geometry)%, and a list of references
. Specifically, the page title contains the definiendum along with its category and serves as the identifier of the definition page within ProofWiki.
%}
%\st{Each definition page is identified with a unique title that contains the definiendum and information about the category, like algebra or geometry.}

This work 
%proposes to leverage these disambiguation pages as ground truth~\footnote{Our dataset, code and experimental results are available here: \url{https://anonymous.4open.science/r/ecir2025-4137-5B97}} to: address 
addresses the following research questions \textit{\textbf{RQ1:} How well can contextualized text representation help the disambiguation of mathematical terms? \textbf{RQ2:} Which architectures and pretraining strategies is best suited for the task?} The following are the main contributions of this work: %The contributions of this work are summarized as follows:
%\vspace{-3mm}
\begin{itemize}
    \item \textbf{MathD2} - a new dataset for \textbf{Math}ematical \textbf{D}efiniendum \textbf{D}isambiguation
    \item Exploration of \textbf{two different approaches} that show how the disambiguation task can benefit from contextualized semantic representations 
    \item \textbf{Experiment-supported evidence} %is provided
    highlighting the efficiency of sentence embeddings for the addressed disambiguation task.
\end{itemize}
%After discussing related work in Section~\ref{sec:related_work}, the two proposed approaches are presented in Section ~\ref{sec:methology}, while conducted experiments and result analysis are discussed in Section~\ref{sec:experiments}, followed by findings and conclusions in Section~\ref{sec:conclusion}.