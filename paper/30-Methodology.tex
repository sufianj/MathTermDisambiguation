\section{Methodology}
\label{sec:methology}
Term disambiguation is formalized as an entity linking task, where the entities refer to the definition page titles in ProofWiki. That is, given (1) %\st{a definition that defines an ambiguous term (we assume that the definiendum has already been extracted) } 
a definition and an ambiguous definiendum
and (2) a dictionary that maps the ambiguous definiendum to entities% (the page titles of the definition in ProofWiki)
, the goal is to find the title that best matches the definition. The proposed method is described in two steps. First, the ground truth dataset is constructed. Second, two applicable approaches
%with related models 
%to achieve the goal 
are considered.
\subsection{Construction of the MathD2 Dataset}
A dump of the whole ProofWiki was extracted on the 9th July, 2024 using WikiTeam~\cite{wikiteamwikiteam_2024}.
%We use WikiTeam~\cite{wikiteamwikiteam_2024} to get a dump of the whole WikiProof on 9th July, 2024.  
This dump is then parsed to get all the definition statements and titles from all definition disambiguation pages. The extracted definitions is converted to plain text.  By mapping ambiguous terms and the corresponding definition titles is finally constructed.
%We convert the extracted definitions to plain text.  We build a dictionary by mapping ambiguous terms and the corresponding definition titles. 
%We remark that s
Some definitions might contain other definitions(e.g., the definition of ``Loop''~\footnote{\url{https://proofwiki.org/wiki/Definition:Loop_(Topology)}}), which also happens to definitions in scholarly papers. If both definition titles are mapped to a common ambiguous term, only the nested definition and its title are kept, because otherwise the outer definition should be mapped to two titles: its title and the one of the nested definition. %Finally, we remove terms that are mapped to less than two titles. 
Finally, terms mapped to less than two titles are removed. 
Table~\ref{tab:proofwikidata} shows (definition, title) pairs extracted from the disambiguation page of ``Bilinear Form'~\footnote{\url{https://proofwiki.org/wiki/Definition:Bilinear_Form}}.
\squeezeup
\begin{table}
    \centering
    \caption{Data extracted from a ProofWiki disambiguation page.}
    \begin{tabular}{|p{0.65\linewidth}|p{0.3\linewidth}|}
          \hline
           Definition&Title\\\hline
            Let $R$ be a ring. Let $R_R$ denote the $R$-module $R$. Let $M_R$ be an $R$-module. A bilinear form on $M_R$ is a bilinear mapping $B : M_R \times M_R \to R_R$.&Definition:Bilinear Form (Linear Algebra) \\\hline
            A bilinear form is a linear form of order $2$.&Definition:Bilinear Form (Polynomial Theory)\\\hline
    \end{tabular}
    \label{tab:proofwikidata}
\end{table}
For the finetuning in Section~\ref{sec:nsp}, %we split the sets
the dataset is split based on the ambiguous terms at the ratio of 8:2, making a training dataset of 275 ambiguous terms with 1436 (definition, title) pairs and a test dataset of 68 ambiguous terms with 433 (definition, title) pairs.  In the finetuning %phase 
of Section~\ref{sec:nsp}, for each ambiguous term, %we randomly select 
two definitions and their titles are randomly selected to make positive pairs, and the titles of two other definitions to make negative pairs. %Figure~\ref{fig:train_sample} shows an example. 
%We evaluate both
Both approaches are evaluated on the training and test datasets, except for the finetuned model. 
