\usepackage{XZVT-private}
dd_page_title,term,definition,def_page_title,categories,plain_text_def
Definition:Above,Above,"In the context of numbers, above means greater than.

Note that this applies to:
:the natural numbers $\N$
:the integers $\Z$
:the rational numbers $\Q$
:the real numbers $\R$

but specifically not the complex numbers $\C$ because the complex numbers do not have a usual ordering.
",Definition:Above (Number),['Definitions/Language Definitions'],"In the context of numbers, above means greater than.

Note that this applies to:
:the natural numbers 
:the integers 
:the rational numbers 
:the real numbers 

but specifically not the complex numbers  because the complex numbers do not have a usual ordering.
"
Definition:Above,Above,"Let $a$ and $b$ be points in $3$-dimensional Euclidean space $\R^3$.

Let $P$ be an arbitrary plane embedded in $S$ be distinguished and defined as horizontal.

Then:
:$a$ is above $b$
:
:the height of $a$  $P$ is greater than the height of $b$  $P$.
",Definition:Above (Solid Geometry),"['Definitions/Language Definitions', 'Definitions/Solid Geometry']","Let a and b be points in 3-dimensional Euclidean space ^3.

Let P be an arbitrary plane embedded in S be distinguished and defined as horizontal.

Then:
:a is above b
:
:the height of a  P is greater than the height of b  P.
"
Definition:Above,Above,"Let $a$ and $b$ be points in the cartesian plane $\R^2$.

Then:
:$a$ is above $b$
:
:the $y$ coordinate of $a$ is greater than the $y$ coordinate of $b$.
",Definition:Above (Plane Geometry),"['Definitions/Language Definitions', 'Definitions/Plane Geometry']","Let a and b be points in the cartesian plane ^2.

Then:
:a is above b
:
:the y coordinate of a is greater than the y coordinate of b.
"
Definition:Absolute,Absolute,"An absolute number is a number in an expression which has a single value.

It is either expressed using actual figures, in an agreed number system, or by a symbol which is understood to represent that specific number.",Definition:Absolute Number,"['Definitions/Algebra', 'Definitions/Numbers']","An absolute number is a number in an expression which has a single value.

It is either expressed using actual figures, in an agreed number system, or by a symbol which is understood to represent that specific number."
Definition:Absolute,Absolute,"A constant is a name for an object (usually a number, but the concept has wider applications) which does not change during the context of a logical or mathematical argument.


A constant can be considered as an operator which takes no operands.

A constant can also be considered as a variable whose domain is a singleton.",Definition:Constant,"['Definitions/Constants', 'Definitions/Algebra']","A constant is a name for an object (usually a number, but the concept has wider applications) which does not change during the context of a logical or mathematical argument.


A constant can be considered as an operator which takes no operands.

A constant can also be considered as a variable whose domain is a singleton."
Definition:Absolute,Absolute,,Definition:Absolute Convergence,"['Definitions/Absolute Convergence', 'Definitions/Convergence']",
Definition:Absolute,Absolute,"Let $f: \R^n \to \R$ be a real-valued function.

Let $f$ be such that, for all $\mathbf x := \tuple {x_1, x_2, \ldots, x_n} \in \R^n$:
:$\map f {\mathbf x} = \map f {\mathbf y}$
where $\mathbf y$ is a permutation of $\tuple {x_1, x_2, \ldots, x_n}$.

Then $f$ is an absolutely symmetric function.",Definition:Symmetric Function/Absolute,['Definitions/Symmetric Functions'],"Let f: ^n → be a real-valued function.

Let f be such that, for all 𝐱 := x_1, x_2, …, x_n∈^n:
:f 𝐱 =  f 𝐲
where 𝐲 is a permutation of x_1, x_2, …, x_n.

Then f is an absolutely symmetric function."
Definition:Absolute,Absolute,"Let $V$ be a Banach space.

Let $\family {v_i}_{i \mathop \in I}$ be an indexed family of elements of $V$.


Then $\ds \sum \set {v_i: i \in I}$ converges absolutely  $\ds \sum \set {\norm {v_i}: i \mathop \in I}$ converges.

This nomenclature is appropriate as we have Absolutely Convergent Generalized Sum Converges.",Definition:Generalized Sum/Absolute Net Convergence,"['Definitions/Group Theory', 'Definitions/Generalized Sums']","Let V be a Banach space.

Let v_i_i ∈ I be an indexed family of elements of V.


Then ∑v_i: i ∈ I converges absolutely  ∑v_i: i ∈ I converges.

This nomenclature is appropriate as we have Absolutely Convergent Generalized Sum Converges."
Definition:Absolute,Absolute,"Let $T_1 = \struct {S_1, \tau_1}$ be a topological space.

Let $T_2 = \struct {S_2, \tau_2}$ be a topological subspace of $T_1$.


Let $T_2$ be a retract of $T_1$.


$T_2$ is an absolute retract of $T_1$ :
:for every closed subspace $B$ of a $T_4$ space $T$ such that $B$ is homeomorphic to $A$, then $B$ is a retract of $T$.",Definition:Retract (Topology)/Absolute,['Definitions/Topology'],"Let T_1 = S_1, τ_1 be a topological space.

Let T_2 = S_2, τ_2 be a topological subspace of T_1.


Let T_2 be a retract of T_1.


T_2 is an absolute retract of T_1 :
:for every closed subspace B of a T_4 space T such that B is homeomorphic to A, then B is a retract of T."
Definition:Absolute,Absolute,"Let $x_0$ be an approximation to a (true) value $x$.


The absolute error of $x_0$ in $x$ is defined as:

:$\Delta x := x_0 - x$


=== Correction ===
",Definition:Error/Absolute,['Definitions/Errors'],"Let x_0 be an approximation to a (true) value x.


The absolute error of x_0 in x is defined as:

:Δ x := x_0 - x


=== Correction ===
"
Definition:Absolute,Absolute,"Let $S$ be a sample or a population.

Let $\omega$ be a qualitative variable, or a class interval of a quantitative variable.


The frequency of $\omega$ is the number of individuals in $S$ satisfying $\omega$.",Definition:Frequency (Descriptive Statistics),"['Definitions/Frequency (Descriptive Statistics)', 'Definitions/Class Intervals', 'Definitions/Qualitative Variables', 'Definitions/Descriptive Statistics']","Let S be a sample or a population.

Let ω be a qualitative variable, or a class interval of a quantitative variable.


The frequency of ω is the number of individuals in S satisfying ω."
Definition:Absolute,Absolute,An absolute measure of dispersion is a measure of dispersion that indicates how spread out or scattered a set of observations with respect to the actual values of those observations.,Definition:Absolute Measure of Dispersion,['Definitions/Dispersion (Statistics)'],An absolute measure of dispersion is a measure of dispersion that indicates how spread out or scattered a set of observations with respect to the actual values of those observations.
Definition:Absolute,Absolute,Absolute geometry is the study of Euclidean geometry without the parallel postulate.,Definition:Absolute Geometry,"['Definitions/Absolute Geometry', 'Definitions/Geometry', 'Definitions/Branches of Mathematics']",Absolute geometry is the study of Euclidean geometry without the parallel postulate.
Definition:Absolute,Absolute,"A real number $r$ is absolutely normal if it is normal  every number base $b$.

That is,  its basis expansion in every number base $b$ is such that:
:no finite sequence of digits of $r$ of length $n$ occurs more frequently than any other such finite sequence of length $n$.


In particular, for every number base $b$, all digits of $r$ have the same natural density in the basis expansion of $r$.",Definition:Absolutely Normal Number,"['Definitions/Absolutely Normal Numbers', 'Definitions/Normal Numbers', 'Definitions/Numbers']","A real number r is absolutely normal if it is normal  every number base b.

That is,  its basis expansion in every number base b is such that:
:no finite sequence of digits of r of length n occurs more frequently than any other such finite sequence of length n.


In particular, for every number base b, all digits of r have the same natural density in the basis expansion of r."
Definition:Absolute,Absolute,"Absolute temperature is a measure of the amount of heat energy in a body.

It is defined as:
:$T = \dfrac 1 k \paren {\dfrac {\partial U} {\partial \ln g} }$
where:
:$k$ is a constant that relates the mean kinetic energy and absolute temperature of the body $B$
:$U$ is the total energy of $B$
:$g$ is the number of possible states in which $B$ can be.


=== Dimension ===

",Definition:Temperature/Absolute,['Definitions/Temperature'],"Absolute temperature is a measure of the amount of heat energy in a body.

It is defined as:
:T =  1 k ∂ U∂ln g
where:
:k is a constant that relates the mean kinetic energy and absolute temperature of the body B
:U is the total energy of B
:g is the number of possible states in which B can be.


=== Dimension ===

"
Definition:Absolute,Absolute,"Absolute zero is the lowest temperature which can theoretically be achieved.

It is the temperature where all motion due to thermal effects stops.

Before that temperature can be reached, quantum effects come into play.",Definition:Absolute Zero,['Definitions/Temperature'],"Absolute zero is the lowest temperature which can theoretically be achieved.

It is the temperature where all motion due to thermal effects stops.

Before that temperature can be reached, quantum effects come into play."
Definition:Absolute,Absolute,"Let $\mathbf A$ be an array.


When referring to a specific element of $\mathbf A$ directly by its indices $\tuple {i, j}$, those indices can be referred to as the absolute address of that element.

This terminology is most often seen in the context of computer spreadsheet programs.",Definition:Array/Element/Absolute Address,['Definitions/Arrays'],"Let 𝐀 be an array.


When referring to a specific element of 𝐀 directly by its indices i, j, those indices can be referred to as the absolute address of that element.

This terminology is most often seen in the context of computer spreadsheet programs."
Definition:Action,Action,"Let $B_1$ and $B_2$ be bodies in space.

Let $B_1$ apply a force $\mathbf F$ on $B_2$.

The force $\mathbf F$ is known as the action that $B_1$ applies to $B_2$.
",Definition:Action (Physics),['Definitions/Force'],"Let B_1 and B_2 be bodies in space.

Let B_1 apply a force 𝐅 on B_2.

The force 𝐅 is known as the action that B_1 applies to B_2.
"
Definition:Action,Action,"The action for a given segment from point $A$ to point $B$ on the trajectory of a dynamical system is defined by the line integral:

:$\ds \sum_i \int_A^B p_i \rd q_i$

where:
:$q_i$ are the generalized coordinates
:$p_i$ are the generalized momenta.
",Definition:Action (Dynamics),['Definitions/Dynamics'],"The action for a given segment from point A to point B on the trajectory of a dynamical system is defined by the line integral:

:∑_i ∫_A^B p_i  q_i

where:
:q_i are the generalized coordinates
:p_i are the generalized momenta.
"
Definition:Action,Action,"The action applied by a system from state $1$ to state $2$ is defined as the definite integral of the Lagrangian over time from state $1$ to state $2$:

:$\ds S_{12} = \int_{t_1}^{t_2} \LL \rd t$
where:
:$S_{12}$ is the action from $1$ to $2$
:$t$ is time
:$\LL$ is the Lagrangian.",Definition:Action Applied by System,"['Definitions/Hamiltonian Mechanics', 'Definitions/Dimensions of Measurement']","The action applied by a system from state 1 to state 2 is defined as the definite integral of the Lagrangian over time from state 1 to state 2:

:S_12 = ∫_t_1^t_2 t
where:
:S_12 is the action from 1 to 2
:t is time
: is the Lagrangian."
Definition:Action,Action,"Let $X$ be a set.

Let $\struct {G, \circ}$ be a group whose identity is $e$.


A (left) group action is an operation $\phi: G \times X \to X$ such that:

:$\forall \tuple {g, x} \in G \times X: g * x := \map \phi {g, x} \in X$

in such a way that the group action axioms are satisfied:

Let $X$ be a set.

Let $\struct {G, \circ}$ be a group whose identity is $e$.


A right group action is a mapping $\phi: X \times G \to X$ such that:

:$\forall \tuple {x, g} \in X \times G : x * g := \map \phi {x, g} \in X$

in such a way that the right group action axioms are satisfied:

",Definition:Group Action,['Definitions/Group Actions'],"Let X be a set.

Let G, ∘ be a group whose identity is e.


A (left) group action is an operation ϕ: G × X → X such that:

:∀g, x∈ G × X: g * x := ϕg, x∈ X

in such a way that the group action axioms are satisfied:

Let X be a set.

Let G, ∘ be a group whose identity is e.


A right group action is a mapping ϕ: X × G → X such that:

:∀x, g∈ X × G : x * g := ϕx, g∈ X

in such a way that the right group action axioms are satisfied:

"
Definition:Action,Action,"Let $R$ be a ring.

Let $M$ be an abelian group.

Let $\circ : R \times M \to M$ be a mapping from the cartesian product $R \times M$.


$\circ$ is a left linear ring action of $R$ on $M$  $\circ$ satisfies the left ring action axioms:

Let $R$ be a ring.

Let $M$ be an abelian group.

Let $\circ : M \times R \to M$ be a mapping from the cartesian product $M \times R$.


$\circ$ is a right linear ring action of $R$ on $M$  $\circ$ satisfies the right ring action axioms:

",Definition:Linear Ring Action,"['Definitions/Module Theory', 'Definitions/Linear Ring Actions']","Let R be a ring.

Let M be an abelian group.

Let ∘ : R × M → M be a mapping from the cartesian product R × M.


∘ is a left linear ring action of R on M  ∘ satisfies the left ring action axioms:

Let R be a ring.

Let M be an abelian group.

Let ∘ : M × R → M be a mapping from the cartesian product M × R.


∘ is a right linear ring action of R on M  ∘ satisfies the right ring action axioms:

"
Definition:Acyclic,Acyclic,"Let $\mathbf A$ be an abelian category with enough injectives.

Let $\mathbf B$ be an abelian category.

Let $F : \mathbf A \to \mathbf B$ be a left exact functor.

Let $X$ be an object of $\mathbf A$.


Then $X$ is $F$-acyclic  $\mathrm R^i \map F X = 0$ for all positive integers $i \in \Z_{i \mathop \ge 1}$.

In the above $\mathrm R^i F$ denotes the $i$-th right derived functor of $F$.


Category:Definitions/Homological Algebra",Definition:Acyclic Object,['Definitions/Homological Algebra'],"Let 𝐀 be an abelian category with enough injectives.

Let 𝐁 be an abelian category.

Let F : 𝐀→𝐁 be a left exact functor.

Let X be an object of 𝐀.


Then X is F-acyclic  R^i  F X = 0 for all positive integers i ∈_i ≥ 1.

In the above R^i F denotes the i-th right derived functor of F.


Category:Definitions/Homological Algebra"
Definition:Acyclic,Acyclic,"Let $\mathbf A$ be an abelian category with enough injectives.

Let $\mathbf B$ be an abelian category.

Let $F : \mathbf A \to \mathbf B$ be a left exact functor.

Let $X$ be an object of $\mathbf A$.


Then $X$ is $F$-acyclic  $\mathrm R^i \map F X = 0$ for all positive integers $i \in \Z_{i \mathop \ge 1}$.

In the above $\mathrm R^i F$ denotes the $i$-th right derived functor of $F$.


Category:Definitions/Homological Algebra
",Definition:Acyclic Resolution,['Definitions/Homological Algebra'],"Let 𝐀 be an abelian category with enough injectives.

Let 𝐁 be an abelian category.

Let F : 𝐀→𝐁 be a left exact functor.

Let X be an object of 𝐀.


Then X is F-acyclic  R^i  F X = 0 for all positive integers i ∈_i ≥ 1.

In the above R^i F denotes the i-th right derived functor of F.


Category:Definitions/Homological Algebra
"
Definition:Acyclic,Acyclic,"An acyclic graph is a graph or digraph with no cycles.
",Definition:Acyclic Graph,['Definitions/Graph Theory'],"An acyclic graph is a graph or digraph with no cycles.
"
Definition:Acyclic,Acyclic,,Definition:Acyclic,['Definitions/Language Definitions'],
Definition:Additive Function,Additive Function,"Let $f: S \to S$ be a mapping on an algebraic structure $\struct {S, +}$.


Then $f$ is an additive function  it preserves the addition operation:
:$\forall x, y \in S: \map f {x + y} = \map f x + \map f y$
",Definition:Additive Function (Conventional),"['Definitions/Analysis', 'Definitions/Abstract Algebra', 'Definitions/Linear Algebra']","Let f: S → S be a mapping on an algebraic structure S, +.


Then f is an additive function  it preserves the addition operation:
:∀ x, y ∈ S:  f x + y =  f x +  f y
"
Definition:Additive Function,Additive Function,"Let $R$ be a unique factorization domain.

Let $f : R \to \C$ be a complex-valued function.


Then $f$ is additive :
:For all coprime $x, y \in R$: $\map f {x y} = \map f x + \map f y$


=== Arithmetic Function ===
",Definition:Additive Function on UFD,['Definitions/Ring Theory'],"Let R be a unique factorization domain.

Let f : R → be a complex-valued function.


Then f is additive :
:For all coprime x, y ∈ R: f x y =  f x +  f y


=== Arithmetic Function ===
"
Definition:Additive Function,Additive Function,"Let $\struct {R, +, \times}$ be a ring.

Let $f: R \to R$ be a mapping on $R$.


Then $f$ is described as completely additive :

:$\forall m, n \in R: \map f {m \times n} = \map f m + \map f n$


That is, a completely additive function is one where the value of a product of two numbers equals the sum of the value of each one individually.",Definition:Completely Additive Function,"['Definitions/Ring Theory', 'Definitions/Number Theory']","Let R, +, × be a ring.

Let f: R → R be a mapping on R.


Then f is described as completely additive :

:∀ m, n ∈ R:  f m × n =  f m +  f n


That is, a completely additive function is one where the value of a product of two numbers equals the sum of the value of each one individually."
Definition:Additive Function,Additive Function,"Let $\SS$ be an algebra of sets.

Let $f: \SS \to \overline \R$ be a function, where $\overline \R$ denotes the set of extended real numbers.


Then $f$ is defined to be additive :

:$\forall S, T \in \SS: S \cap T = \O \implies \map f {S \cup T} = \map f S + \map f T$

That is, for any two disjoint elements of $\SS$, $f$ of their union equals the sum of $f$ of the individual elements.


Note from Finite Union of Sets in Additive Function that:

:$\ds \map f {\bigcup_{i \mathop = 1}^n S_i} = \sum_{i \mathop = 1}^n \map f {S_i}$

where $S_1, S_2, \ldots, S_n$ is any finite collection of pairwise disjoint elements of $\SS$.",Definition:Additive Function (Measure Theory),['Definitions/Measure Theory'],"Let  be an algebra of sets.

Let f: → be a function, where  denotes the set of extended real numbers.


Then f is defined to be additive :

:∀ S, T ∈: S ∩ T = Ø f S ∪ T =  f S +  f T

That is, for any two disjoint elements of , f of their union equals the sum of f of the individual elements.


Note from Finite Union of Sets in Additive Function that:

:f ⋃_i  = 1^n S_i = ∑_i  = 1^n  f S_i

where S_1, S_2, …, S_n is any finite collection of pairwise disjoint elements of ."
Definition:Additive Function,Additive Function,"Let $\Sigma$ be a $\sigma$-algebra.

Let $f: \Sigma \to \overline \R$ be a function, where $\overline \R$ denotes the set of extended real numbers.


Then $f$ is defined as countably additive :
:$\ds \map f {\bigcup_{n \mathop \in \N} E_n} = \sum_{n \mathop \in \N} \map f {E_n}$

where $\sequence {E_n}$ is any sequence of pairwise disjoint elements of $\Sigma$.


That is, for any countably infinite set of pairwise disjoint elements of $\Sigma$, $f$ of their union equals the sum of $f$ of the individual elements.",Definition:Countably Additive Function,"['Definitions/Set Systems', 'Definitions/Measure Theory']","Let Σ be a σ-algebra.

Let f: Σ→ be a function, where  denotes the set of extended real numbers.


Then f is defined as countably additive :
:f ⋃_n ∈ E_n = ∑_n ∈ f E_n

where E_n is any sequence of pairwise disjoint elements of Σ.


That is, for any countably infinite set of pairwise disjoint elements of Σ, f of their union equals the sum of f of the individual elements."
Definition:Adjacent,Adjacent,"Let $G = \struct {V, E}$ be a graph.

Two vertices $u, v \in V$ of $G$ are non-adjacent  they are not adjacent.
",Definition:Adjacent (Graph Theory)/Vertices,"['Definitions/Adjacency (Graph Theory)', 'Definitions/Vertices of Graphs']","Let G = V, E be a graph.

Two vertices u, v ∈ V of G are non-adjacent  they are not adjacent.
"
Definition:Adjacent,Adjacent,"Let $G = \struct {V, E}$ be a graph.

Two edges $u, v \in V$ of $G$ are non-adjacent  they are not adjacent.
",Definition:Adjacent (Graph Theory)/Edges,"['Definitions/Edges of Graphs', 'Definitions/Adjacency (Graph Theory)']","Let G = V, E be a graph.

Two edges u, v ∈ V of G are non-adjacent  they are not adjacent.
"
Definition:Adjacent,Adjacent,":

Let $G = \struct {V, E}$ be a planar graph.

Two faces of $G$ are adjacent  they are both incident to the same edge (or edges).

In the above diagram, $BCEF$ and $ABF$ are adjacent, but $BCEF$ and $AFG$ are not adjacent.


Note that faces which are both incident to the same vertex are not considered adjacent unless they are also both incident to the same edge.
:

Let $G = \struct {V, E}$ be a planar graph.

Two faces of $G$ are adjacent  they are both incident to the same edge (or edges).

In the above diagram, $BCEF$ and $ABF$ are adjacent, but $BCEF$ and $AFG$ are not adjacent.


Note that faces which are both incident to the same vertex are not considered adjacent unless they are also both incident to the same edge.
:

Let $G = \struct {V, E}$ be a planar graph.

Two faces of $G$ are adjacent  they are both incident to the same edge (or edges).

In the above diagram, $BCEF$ and $ABF$ are adjacent, but $BCEF$ and $AFG$ are not adjacent.


Note that faces which are both incident to the same vertex are not considered adjacent unless they are also both incident to the same edge.
",Definition:Adjacent (Graph Theory)/Faces,"['Definitions/Adjacency (Graph Theory)', 'Definitions/Faces of Graphs']",":

Let G = V, E be a planar graph.

Two faces of G are adjacent  they are both incident to the same edge (or edges).

In the above diagram, BCEF and ABF are adjacent, but BCEF and AFG are not adjacent.


Note that faces which are both incident to the same vertex are not considered adjacent unless they are also both incident to the same edge.
:

Let G = V, E be a planar graph.

Two faces of G are adjacent  they are both incident to the same edge (or edges).

In the above diagram, BCEF and ABF are adjacent, but BCEF and AFG are not adjacent.


Note that faces which are both incident to the same vertex are not considered adjacent unless they are also both incident to the same edge.
:

Let G = V, E be a planar graph.

Two faces of G are adjacent  they are both incident to the same edge (or edges).

In the above diagram, BCEF and ABF are adjacent, but BCEF and AFG are not adjacent.


Note that faces which are both incident to the same vertex are not considered adjacent unless they are also both incident to the same edge.
"
Definition:Adjacent,Adjacent,,Definition:Adjacent (Geometry),"['Definitions/Adjacent (Geometry)', 'Definitions/Geometry']",
Definition:Adjacent,Adjacent,"Two angles are adjacent if they have an intersecting line in common:

:",Definition:Angle/Adjacent,['Definitions/Angles'],"Two angles are adjacent if they have an intersecting line in common:

:"
Definition:Adjacent,Adjacent,"Each vertex of a polygon is formed by the intersection of two sides.

The two sides that form a particular vertex are referred to as the adjacents of that vertex, or described as adjacent to that vertex.
Two sides of a polygon that meet at the same vertex are adjacent to each other.
Each side of a polygon intersects two other sides, and so is terminated at either endpoint by two vertices.

The two vertices that terminate a particular side are referred to as the adjacents of that side, or described as adjacent to that side.
Each side of a polygon intersects two other sides, and so is terminated at either endpoint by two vertices.


Those two vertices are described as adjacent to each other.
",Definition:Polygon/Adjacent,"['Definitions/Adjacent (Polygons)', 'Definitions/Polygons']","Each vertex of a polygon is formed by the intersection of two sides.

The two sides that form a particular vertex are referred to as the adjacents of that vertex, or described as adjacent to that vertex.
Two sides of a polygon that meet at the same vertex are adjacent to each other.
Each side of a polygon intersects two other sides, and so is terminated at either endpoint by two vertices.

The two vertices that terminate a particular side are referred to as the adjacents of that side, or described as adjacent to that side.
Each side of a polygon intersects two other sides, and so is terminated at either endpoint by two vertices.


Those two vertices are described as adjacent to each other.
"
Definition:Adjacent,Adjacent,"Each vertex of a polygon is formed by the intersection of two sides.

The two sides that form a particular vertex are referred to as the adjacents of that vertex, or described as adjacent to that vertex.",Definition:Polygon/Adjacent/Side to Vertex,['Definitions/Adjacent (Polygons)'],"Each vertex of a polygon is formed by the intersection of two sides.

The two sides that form a particular vertex are referred to as the adjacents of that vertex, or described as adjacent to that vertex."
Definition:Adjacent,Adjacent,"Each side of a polygon intersects two other sides, and so is terminated at either endpoint by two vertices.

The two vertices that terminate a particular side are referred to as the adjacents of that side, or described as adjacent to that side.",Definition:Polygon/Adjacent/Vertex to Side,['Definitions/Adjacent (Polygons)'],"Each side of a polygon intersects two other sides, and so is terminated at either endpoint by two vertices.

The two vertices that terminate a particular side are referred to as the adjacents of that side, or described as adjacent to that side."
Definition:Adjacent,Adjacent,Two sides of a polygon that meet at the same vertex are adjacent to each other.,Definition:Polygon/Adjacent/Sides,['Definitions/Adjacent (Polygons)'],Two sides of a polygon that meet at the same vertex are adjacent to each other.
Definition:Adjacent,Adjacent,"Each side of a polygon intersects two other sides, and so is terminated at either endpoint by two vertices.


Those two vertices are described as adjacent to each other.",Definition:Polygon/Adjacent/Vertices,['Definitions/Adjacent (Polygons)'],"Each side of a polygon intersects two other sides, and so is terminated at either endpoint by two vertices.


Those two vertices are described as adjacent to each other."
Definition:Adjacent,Adjacent,"Each vertex of a polyhedron is formed by the intersection of a number of faces.

The faces that form a particular vertex are referred to as the adjacents of that vertex, or described as adjacent to that vertex.
Each edge of a polyhedron is formed by the intersection of two faces.

The faces that form a particular edge are referred to as the adjacents of that edge, or described as adjacent to that edge.
Two faces of a polyhedron that meet at the same vertex are adjacent to each other.
The edges that intersect at a particular face are referred to as the adjacents of that face, or described as adjacent to that face.
Two edges of a polyhedron that intersect at a particular vertex are referred to as adjacent to each other.
The vertices that intersect a particular face are referred to as the adjacents of that face, or described as adjacent to that face.
",Definition:Polyhedron/Adjacent,"['Definitions/Adjacent (Polyhedra)', 'Definitions/Polyhedra']","Each vertex of a polyhedron is formed by the intersection of a number of faces.

The faces that form a particular vertex are referred to as the adjacents of that vertex, or described as adjacent to that vertex.
Each edge of a polyhedron is formed by the intersection of two faces.

The faces that form a particular edge are referred to as the adjacents of that edge, or described as adjacent to that edge.
Two faces of a polyhedron that meet at the same vertex are adjacent to each other.
The edges that intersect at a particular face are referred to as the adjacents of that face, or described as adjacent to that face.
Two edges of a polyhedron that intersect at a particular vertex are referred to as adjacent to each other.
The vertices that intersect a particular face are referred to as the adjacents of that face, or described as adjacent to that face.
"
Definition:Adjacent,Adjacent,"Each vertex of a polyhedron is formed by the intersection of a number of faces.

The faces that form a particular vertex are referred to as the adjacents of that vertex, or described as adjacent to that vertex.",Definition:Polyhedron/Adjacent/Face to Vertex,['Definitions/Adjacent (Polyhedra)'],"Each vertex of a polyhedron is formed by the intersection of a number of faces.

The faces that form a particular vertex are referred to as the adjacents of that vertex, or described as adjacent to that vertex."
Definition:Adjacent,Adjacent,"The vertices that intersect a particular face are referred to as the adjacents of that face, or described as adjacent to that face.",Definition:Polyhedron/Adjacent/Vertex to Face,['Definitions/Adjacent (Polyhedra)'],"The vertices that intersect a particular face are referred to as the adjacents of that face, or described as adjacent to that face."
Definition:Adjacent,Adjacent,"Each edge of a polyhedron is formed by the intersection of two faces.

The faces that form a particular edge are referred to as the adjacents of that edge, or described as adjacent to that edge.",Definition:Polyhedron/Adjacent/Face to Edge,['Definitions/Adjacent (Polyhedra)'],"Each edge of a polyhedron is formed by the intersection of two faces.

The faces that form a particular edge are referred to as the adjacents of that edge, or described as adjacent to that edge."
Definition:Adjacent,Adjacent,"The edges that intersect at a particular face are referred to as the adjacents of that face, or described as adjacent to that face.",Definition:Polyhedron/Adjacent/Edge to Face,['Definitions/Adjacent (Polyhedra)'],"The edges that intersect at a particular face are referred to as the adjacents of that face, or described as adjacent to that face."
Definition:Adjacent,Adjacent,Two faces of a polyhedron that meet at the same vertex are adjacent to each other.,Definition:Polyhedron/Adjacent/Faces,['Definitions/Adjacent (Polyhedra)'],Two faces of a polyhedron that meet at the same vertex are adjacent to each other.
Definition:Adjacent,Adjacent,"The two sides of a triangle which form a particular vertex are referred to as adjacent to that angle.

Similarly, the two vertices of a triangle to which a particular side contributes are referred to as adjacent to that side.


Category:Definitions/Triangles",Definition:Triangle (Geometry)/Adjacent,['Definitions/Triangles'],"The two sides of a triangle which form a particular vertex are referred to as adjacent to that angle.

Similarly, the two vertices of a triangle to which a particular side contributes are referred to as adjacent to that side.


Category:Definitions/Triangles"
Definition:Adjacent,Adjacent,"The two sides of a triangle which form a particular vertex are referred to as adjacent to that angle.

Similarly, the two vertices of a triangle to which a particular side contributes are referred to as adjacent to that side.


Category:Definitions/Triangles
",Definition:Triangle (Geometry)/Right-Angled/Adjacent,['Definitions/Right Triangles'],"The two sides of a triangle which form a particular vertex are referred to as adjacent to that angle.

Similarly, the two vertices of a triangle to which a particular side contributes are referred to as adjacent to that side.


Category:Definitions/Triangles
"
Definition:Adjacent Faces,Adjacent Faces,":

Let $G = \struct {V, E}$ be a planar graph.

Two faces of $G$ are adjacent  they are both incident to the same edge (or edges).

In the above diagram, $BCEF$ and $ABF$ are adjacent, but $BCEF$ and $AFG$ are not adjacent.


Note that faces which are both incident to the same vertex are not considered adjacent unless they are also both incident to the same edge.",Definition:Adjacent (Graph Theory)/Faces,"['Definitions/Adjacency (Graph Theory)', 'Definitions/Faces of Graphs']",":

Let G = V, E be a planar graph.

Two faces of G are adjacent  they are both incident to the same edge (or edges).

In the above diagram, BCEF and ABF are adjacent, but BCEF and AFG are not adjacent.


Note that faces which are both incident to the same vertex are not considered adjacent unless they are also both incident to the same edge."
Definition:Adjacent Faces,Adjacent Faces,Two faces of a polyhedron that meet at the same vertex are adjacent to each other.,Definition:Polyhedron/Adjacent/Faces,['Definitions/Adjacent (Polyhedra)'],Two faces of a polyhedron that meet at the same vertex are adjacent to each other.
Definition:Adjoint,Adjoint,"Let $\HH$ and $\KK$ be Hilbert spaces.

Let $\map \BB {\HH, \KK}$ be the set of bounded linear transformations from $\HH$ to $\KK$.

Let $A \in \map \BB {\HH, \KK}$ be a bounded linear transformation.


By Existence and Uniqueness of Adjoint, there exists a unique bounded linear transformation $A^* \in \map \BB {\KK, \HH}$ such that:

:$\forall h \in \HH, k \in \KK: {\innerprod {\map A h} k}_\KK = {\innerprod h {\map {A^*} k} }_\HH$

where $\innerprod \cdot \cdot_\HH$ and $\innerprod \cdot \cdot_\KK$ are inner products on $\HH$ and $\KK$ respectively.


$A^*$ is called the adjoint of $A$.


The operation of assigning $A^*$ to $A$ may be referred to as adjoining.",Definition:Adjoint Linear Transformation,"['Definitions/Adjoints', 'Definitions/Linear Transformations on Hilbert Spaces']","Let  and  be Hilbert spaces.

Let , be the set of bounded linear transformations from  to .

Let A ∈, be a bounded linear transformation.


By Existence and Uniqueness of Adjoint, there exists a unique bounded linear transformation A^* ∈, such that:

:∀ h ∈, k ∈:  A h k_ =  h A^* k_

where ··_ and ··_ are inner products on  and  respectively.


A^* is called the adjoint of A.


The operation of assigning A^* to A may be referred to as adjoining."
Definition:Adjoint,Adjoint,"Let $\struct {S, \preceq}$ and $\struct {T, \precsim}$ be ordered sets.

Let $g: S \to T$, $d: T \to S$ be mappings.

Let $\tuple {g, d}$ be a Galois connection.


Then:

:$g$ is called the upper adjoint of the Galois connection.
",Definition:Galois Connection/Upper Adjoint,['Definitions/Galois Connections'],"Let S, ≼ and T, ≾ be ordered sets.

Let g: S → T, d: T → S be mappings.

Let g, d be a Galois connection.


Then:

:g is called the upper adjoint of the Galois connection.
"
Definition:Adjoint,Adjoint,"Let $\struct {S, \preceq}$ and $\struct {T, \precsim}$ be ordered sets.

Let $g: S \to T$, $d: T \to S$ be mappings.

Let $\tuple {g, d}$ be a Galois connection.


Then:

:$d$ is called the lower adjoint of the Galois connection.
",Definition:Galois Connection/Lower Adjoint,['Definitions/Galois Connections'],"Let S, ≼ and T, ≾ be ordered sets.

Let g: S → T, d: T → S be mappings.

Let g, d be a Galois connection.


Then:

:d is called the lower adjoint of the Galois connection.
"
Definition:Adjoint,Adjoint,"
",Definition:Adjoint Functor,['Definitions/Category Theory'],"
"
Definition:Adjoint,Adjoint,"Let $\mathbf C$, $\mathbf D$ be locally small categories.

Let $F : \mathbf D \to \mathbf C$ and $G : \mathbf C \to \mathbf D$ be functors.

$F$ is a left adjoint functor of $G$  there exists an adjunction $\struct {F, G, \alpha}$.",Definition:Left Adjoint Functor,['Definitions/Category Theory'],"Let 𝐂, 𝐃 be locally small categories.

Let F : 𝐃→𝐂 and G : 𝐂→𝐃 be functors.

F is a left adjoint functor of G  there exists an adjunction F, G, α."
Definition:Adjoint,Adjoint,"Let $\mathbf C$, $\mathbf D$ be locally small categories.

Let $F : \mathbf D \to \mathbf C$ and $G : \mathbf C \to \mathbf D$ be functors.

$G$ is a right adjoint functor of $F$  there exists an adjunction $\struct {F, G, \alpha}$.
",Definition:Right Adjoint Functor,['Definitions/Category Theory'],"Let 𝐂, 𝐃 be locally small categories.

Let F : 𝐃→𝐂 and G : 𝐂→𝐃 be functors.

G is a right adjoint functor of F  there exists an adjunction F, G, α.
"
Definition:Adjoint,Adjoint,,Definition:Adjoint Matrix,"['Definitions/Matrix Algebra', 'Definitions/Linear Algebra']",
Definition:Affine,Affine,"Let $K$ be a field.

Let $A = K \sqbrk {X_1, \ldots, X_n}$ be the ring of polynomial functions in $n$ variables over $K$.


Then a subset $X \subseteq K^n$ is an affine algebraic set  it is the zero locus of some set $T \subseteq A$.",Definition:Affine Algebraic Set,"['Definitions/Algebraic Geometry', 'Definitions/Affine Geometry']","Let K be a field.

Let A = K X_1, …, X_n be the ring of polynomial functions in n variables over K.


Then a subset X ⊆ K^n is an affine algebraic set  it is the zero locus of some set T ⊆ A."
Definition:Affine,Affine,"Let $K$ be a field.

Let $A = K \sqbrk {X_1, \ldots, X_n}$ be the ring of polynomial functions in $n$ variables over $K$.


Then a subset $X \subseteq K^n$ is an affine algebraic set  it is the zero locus of some set $T \subseteq A$.
",Definition:Affine Algebraic Variety,['Definitions/Algebraic Geometry'],"Let K be a field.

Let A = K X_1, …, X_n be the ring of polynomial functions in n variables over K.


Then a subset X ⊆ K^n is an affine algebraic set  it is the zero locus of some set T ⊆ A.
"
Definition:Affine,Affine,An affine scheme is a ringed space which is isomorphic to the spectrum of a commutative ring with unity.,Definition:Affine Scheme,['Definitions/Schemes'],An affine scheme is a ringed space which is isomorphic to the spectrum of a commutative ring with unity.
Definition:Affine,Affine,,Definition:Affine Dimension,['Definitions/Affine Geometry'],
Definition:Affine,Affine,"An affine monoid is a monoid that is:
: finitely generated
and:
: isomorphic to a submonoid of a free abelian group $\Z^d$, for some $d \in \Z_{\ge 0}$.",Definition:Affine Monoid,['Definitions/Monoids'],"An affine monoid is a monoid that is:
: finitely generated
and:
: isomorphic to a submonoid of a free abelian group ^d, for some d ∈_≥ 0."
Definition:Age,Age,"The age of a physical object is defined as the period of time over which it has been in existence.


Category:Definitions/Applied Mathematics
",Definition:Age (Time),['Definitions/Applied Mathematics'],"The age of a physical object is defined as the period of time over which it has been in existence.


Category:Definitions/Applied Mathematics
"
Definition:Age,Age,"Let $\MM$ be an $\LL$-structure.


An age of $\MM$ is a class $K$ of $\LL$-structures such that:
* if $\AA$ is a finitely generated $\LL$-structure such that there is an $\LL$-embedding $\AA \to \MM$, then $\AA$ is isomorphic to some structure in $K$,
* no two structures in $K$ are isomorphic, and
* $K$ does not contain any structures which are not finitely generated or do not embed into $\MM$.

That is, $K$ is an age of $\MM$  it contains exactly one representative from each isomorphism type of the finitely-generated structures that embed into $\MM$.



Category:Definitions/Model Theory for Predicate Logic",Definition:Age (Model Theory),['Definitions/Model Theory for Predicate Logic'],"Let  be an -structure.


An age of  is a class K of -structures such that:
* if Å is a finitely generated -structure such that there is an -embedding Å→, then Å is isomorphic to some structure in K,
* no two structures in K are isomorphic, and
* K does not contain any structures which are not finitely generated or do not embed into .

That is, K is an age of   it contains exactly one representative from each isomorphism type of the finitely-generated structures that embed into .



Category:Definitions/Model Theory for Predicate Logic"
Definition:Aggregation,Aggregation,"Parenthesis is a syntactical technique to disambiguate the meaning of a logical formula.

It allows one to specify that a logical formula should (temporarily) be regarded as being a single entity, being on the same level as a statement variable.

Such a formula is referred to as being in parenthesis.

Typically, a formal language, in defining its formal grammar, ensures by means of parenthesis that all of its well-formed words are uniquely readable.


Generally, brackets are used to indicate that certain formulas are in parenthesis.

The brackets that are mostly used are round ones, the left (round) bracket $($ and the right (round) bracket $)$.",Definition:Parenthesis,"['Definitions/Parenthesis', 'Definitions/Symbolic Logic', 'Definitions/Algebra', 'Definitions/Arithmetic']","Parenthesis is a syntactical technique to disambiguate the meaning of a logical formula.

It allows one to specify that a logical formula should (temporarily) be regarded as being a single entity, being on the same level as a statement variable.

Such a formula is referred to as being in parenthesis.

Typically, a formal language, in defining its formal grammar, ensures by means of parenthesis that all of its well-formed words are uniquely readable.


Generally, brackets are used to indicate that certain formulas are in parenthesis.

The brackets that are mostly used are round ones, the left (round) bracket ( and the right (round) bracket )."
Definition:Aggregation,Aggregation,"A set is intuitively defined as any aggregation of objects, called elements, which can be precisely defined in some way or other.

We can think of each set as a single entity in itself, and we can denote it (and usually do) by means of a single symbol.


That is, anything you care to think of can be a set. This concept is known as the .


However, there are problems with the . If we allow it to be used without any restrictions at all, paradoxes arise, the most famous example probably being Russell's Paradox.


Hence some sources define a set as a  'well-defined' collection of objects, leaving the concept of what constitutes well-definition to later in the exposition.",Definition:Set,"['Definitions/Set Theory', 'Definitions/Sets']","A set is intuitively defined as any aggregation of objects, called elements, which can be precisely defined in some way or other.

We can think of each set as a single entity in itself, and we can denote it (and usually do) by means of a single symbol.


That is, anything you care to think of can be a set. This concept is known as the .


However, there are problems with the . If we allow it to be used without any restrictions at all, paradoxes arise, the most famous example probably being Russell's Paradox.


Hence some sources define a set as a  'well-defined' collection of objects, leaving the concept of what constitutes well-definition to later in the exposition."
Definition:Aggregation,Aggregation,"An aggregation, in the context of physics, is a set of bodies (but usually particles) all of which are under the same or similar conditions, and which are assumed to behave (in certain aspects) as one body.

The concept is deliberately left vague.",Definition:Aggregation (Physics),"['Definitions/Aggregations (Physics)', 'Definitions/Physics', 'Definitions/Applied Mathematics']","An aggregation, in the context of physics, is a set of bodies (but usually particles) all of which are under the same or similar conditions, and which are assumed to behave (in certain aspects) as one body.

The concept is deliberately left vague."
Definition:Algebra,Algebra,"Algebra is the branch of mathematics which studies the techniques of manipulation of objects and expressions.
",Definition:Algebra (Mathematical Branch),['Definitions/Branches of Mathematics'],"Algebra is the branch of mathematics which studies the techniques of manipulation of objects and expressions.
"
Definition:Algebra,Algebra,"Linear algebra is the branch of algebra which studies vector spaces and linear transformations between them.
Algebra is the branch of mathematics which studies the techniques of manipulation of objects and expressions.
",Definition:Linear Algebra (Mathematical Branch),"['Definitions/Linear Algebra', 'Definitions/Algebra', 'Definitions/Linearity', 'Definitions/Branches of Mathematics']","Linear algebra is the branch of algebra which studies vector spaces and linear transformations between them.
Algebra is the branch of mathematics which studies the techniques of manipulation of objects and expressions.
"
Definition:Algebra,Algebra,"An algebra loop $\struct {S, \circ}$ is a quasigroup with an identity element.
:$\exists e \in S: \forall x \in S: x \circ e = x = e \circ x$",Definition:Algebra Loop,['Definitions/Abstract Algebra'],"An algebra loop S, ∘ is a quasigroup with an identity element.
:∃ e ∈ S: ∀ x ∈ S: x ∘ e = x = e ∘ x"
Definition:Algebra,Algebra,"=== Definition 1 ===



=== Definition 2 ===
",Definition:Algebra of Sets,"['Definitions/Set Systems', 'Definitions/Algebras of Sets', 'Definitions/Rings of Sets']","=== Definition 1 ===



=== Definition 2 ===
"
Definition:Algebra,Algebra,"An algebraic structure with $1$ operation is an ordered pair:
:$\struct {S, \circ}$
where:
:$S$ is a set
:$\circ$ is a binary operation defined on all the elements of $S \times S$.
",Definition:B-Algebra,"['Definitions/Algebraic Structures', 'Definitions/B-Algebras']","An algebraic structure with 1 operation is an ordered pair:
:S, ∘
where:
:S is a set
:∘ is a binary operation defined on all the elements of S × S.
"
Definition:Algebra,Algebra,,Definition:Algebraic,[],
Definition:Algebraic,Algebraic,Algebraic topology is a branch of topology which uses tools from abstract algebra to study topological spaces.,Definition:Algebraic Topology,"['Definitions/Branches of Mathematics', 'Definitions/Abstract Algebra', 'Definitions/Algebraic Topology', 'Definitions/Topology']",Algebraic topology is a branch of topology which uses tools from abstract algebra to study topological spaces.
Definition:Algebraic,Algebraic,"Algebraic geometry is the branch of geometry which studies objects in multi-dimensional space using the techniques of abstract algebra.

In particular, techniques from commutative algebra are mainly used.

It also encompasses the study of algebraic varieties.
An algebraic variety is the solution set of a system of simultaneous polynomial equations:







",Definition:Algebraic Geometry,"['Definitions/Algebraic Geometry', 'Definitions/Geometry', 'Definitions/Branches of Mathematics']","Algebraic geometry is the branch of geometry which studies objects in multi-dimensional space using the techniques of abstract algebra.

In particular, techniques from commutative algebra are mainly used.

It also encompasses the study of algebraic varieties.
An algebraic variety is the solution set of a system of simultaneous polynomial equations:







"
Definition:Algebraic,Algebraic,"An algebraic variety is the solution set of a system of simultaneous polynomial equations:






",Definition:Algebraic Variety,"['Definitions/Algebraic Varieties', 'Definitions/Algebraic Geometry']","An algebraic variety is the solution set of a system of simultaneous polynomial equations:






"
Definition:Algebraic,Algebraic,"Let $L / K$ be a field extension.

Let $A \subseteq L$ be a subset of $L$.

Let $\map K {\set {X_\alpha: \alpha \in A} }$ be the field of rational functions in the indeterminates $\family {X_\alpha}_{\alpha \mathop \in A}$.


Then $A$ is algebraically independent over $K$  there exists a homomorphism:
:$\phi: \map K {\set {X_\alpha: \alpha \in A} } \to L$
such that, for all $\alpha \in A$:
:$\map \phi {X_\alpha} = \alpha$",Definition:Algebraically Independent,['Definitions/Field Extensions'],"Let L / K be a field extension.

Let A ⊆ L be a subset of L.

Let K X_α: α∈ A be the field of rational functions in the indeterminates X_α_α∈ A.


Then A is algebraically independent over K  there exists a homomorphism:
:ϕ:  K X_α: α∈ A→ L
such that, for all α∈ A:
:ϕX_α = α"
Definition:Algebraic,Algebraic,"Algebraic number theory is the branch of abstract algebra which studies structures in which the usual number fields are embedded.

As such it can also be considered to be a branch of number theory.",Definition:Algebraic Number Theory,"['Definitions/Branches of Mathematics', 'Definitions/Algebraic Number Theory', 'Definitions/Number Theory']","Algebraic number theory is the branch of abstract algebra which studies structures in which the usual number fields are embedded.

As such it can also be considered to be a branch of number theory."
Definition:Algebraic,Algebraic,"An algebraic number field is a finite extension of the field of rational numbers $\Q$.
",Definition:Algebraic Integer,['Definitions/Algebraic Number Theory'],"An algebraic number field is a finite extension of the field of rational numbers .
"
Definition:Algebraic,Algebraic,"Let $y$ be a solution to the polynomial equation:

:$\map {p_0} x + \map {p_1} x y + \dotsb + \map {p_{n - 1} } x y^{n - 1} + \map {p_n} x y^n = 0$

where $\map {p_0} x \ne 0, \map {p_1} x, \dotsc, \map {p_n} x$ are real polynomial functions in $x$.


Then $y = \map f x$ is a (real) algebraic function:
Let $w$ be a solution to the polynomial equation:

:$\map {p_0} z + \map {p_1} z w + \dotsb + \map {p_{n - 1} } z w^{n - 1} + \map {p_n} z w^n = 0$

where $\map {p_0} z \ne 0, \map {p_1} z, \dotsc, \map {p_n} z$ are complex polynomial functions in $z$.


Then $w = \map f z$ is a (complex) algebraic function:
",Definition:Algebraic Function,"['Definitions/Analysis', 'Definitions/Algebraic Functions']","Let y be a solution to the polynomial equation:

:p_0 x + p_1 x y + … + p_n - 1 x y^n - 1 + p_n x y^n = 0

where p_0 x  0, p_1 x, …, p_n x are real polynomial functions in x.


Then y =  f x is a (real) algebraic function:
Let w be a solution to the polynomial equation:

:p_0 z + p_1 z w + … + p_n - 1 z w^n - 1 + p_n z w^n = 0

where p_0 z  0, p_1 z, …, p_n z are complex polynomial functions in z.


Then w =  f z is a (complex) algebraic function:
"
Definition:Algebraic,Algebraic,An algebraic number field is a finite extension of the field of rational numbers $\Q$.,Definition:Algebraic Number Field,"['Definitions/Algebraic Number Theory', 'Definitions/Number Fields']",An algebraic number field is a finite extension of the field of rational numbers .
Definition:Algebraic,Algebraic,"An algebraic system is a mathematical system $\SS = \struct {E, O}$ where:

:$E$ is a non-empty set of elements

:$O$ is a set of finitary operations on $E$.",Definition:Algebraic System,['Definitions/Abstract Algebra'],"An algebraic system is a mathematical system = E, O where:

:E is a non-empty set of elements

:O is a set of finitary operations on E."
Definition:Algebraic,Algebraic,"Let $\struct {S, \preceq}$ be an ordered set.


Then $\struct {S, \preceq}$ is algebraic 
:(for all elements $x$ of $S$: $x^{\mathrm{compact} }$ is directed)
and:
:$\struct {S, \preceq}$ is up-complete and satisfies the axiom of $K$-approximation:
where $x^{\mathrm{compact} }$ denotes the compact closure of $x$.",Definition:Algebraic Ordered Set,['Definitions/Order Theory'],"Let S, ≼ be an ordered set.


Then S, ≼ is algebraic 
:(for all elements x of S: x^compact is directed)
and:
:S, ≼ is up-complete and satisfies the axiom of K-approximation:
where x^compact denotes the compact closure of x."
Definition:Alphabet,Alphabet,"Let $\LL$ be a formal language.


The alphabet $\AA$ of $\LL$ is a set of symbols from which collations in $\LL$ may be constructed.

An alphabet consists of the following parts:

:The letters
:The signs.

Depending on the specific nature of any particular formal language, these too may be subcategorized.







",Definition:Formal Language/Alphabet,"['Definitions/Alphabets (Formal Language)', 'Definitions/Formal Languages']","Let  be a formal language.


The alphabet Å of  is a set of symbols from which collations in  may be constructed.

An alphabet consists of the following parts:

:The letters
:The signs.

Depending on the specific nature of any particular formal language, these too may be subcategorized.







"
Definition:Alphabet,Alphabet,"The alphabet of a natural language $\LL$ is the set of symbols, called letters, which are used to represent the sounds of $\LL$.


=== English ===

",Definition:Alphabet of Natural Language,"['Definitions/Natural Language', 'Definitions/Language Definitions']","The alphabet of a natural language  is the set of symbols, called letters, which are used to represent the sounds of .


=== English ===

"
Definition:Alternant,Alternant,"Let $p \lor q$ be a compound statement whose main connective is the disjunction:
:$p \lor q$  $p$ is true or $q$ is true or both are true.


The substatements $p$ and $q$ are known as the disjuncts.",Definition:Disjunction/Disjunct,['Definitions/Disjunction'],"Let p  q be a compound statement whose main connective is the disjunction:
:p  q  p is true or q is true or both are true.


The substatements p and q are known as the disjuncts."
Definition:Alternant,Alternant,"An alternant is a determinant of order $n$ such that the element in the $i$th row and $j$th column is defined as:
:$\map {f_i} {r_j}$
where:
:the $f_i$ are $n$ mappings
:the $r_j$ are $n$ elements.",Definition:Alternant (Linear Algebra),"['Definitions/Alternants', 'Definitions/Linear Algebra']","An alternant is a determinant of order n such that the element in the ith row and jth column is defined as:
:f_ir_j
where:
:the f_i are n mappings
:the r_j are n elements."
Definition:Alternative,Alternative,"Let $p \lor q$ be a compound statement whose main connective is the disjunction:
:$p \lor q$  $p$ is true or $q$ is true or both are true.


The substatements $p$ and $q$ are known as the disjuncts.",Definition:Disjunction/Disjunct,['Definitions/Disjunction'],"Let p  q be a compound statement whose main connective is the disjunction:
:p  q  p is true or q is true or both are true.


The substatements p and q are known as the disjuncts."
Definition:Alternative,Alternative,"Let $\circ$ be a binary operation.


Then $\circ$ is defined as being alternative on $S$ :

:$\forall T := \set {x, y} \subseteq S: \forall x, y, z \in T: \paren {x \circ y} \circ z = x \circ \paren {y \circ z}$

That is, $\circ$ is associative over any two elements of $S$.


For example, for any $x, y \in S$:
:$\paren {x \circ y} \circ x = x \circ \paren {y \circ x}$
:$\paren {x \circ x} \circ y = x \circ \paren {x \circ y}$
and so on.",Definition:Alternative Operation,['Definitions/Abstract Algebra'],"Let ∘ be a binary operation.


Then ∘ is defined as being alternative on S :

:∀ T := x, y⊆ S: ∀ x, y, z ∈ T: x ∘ y∘ z = x ∘y ∘ z

That is, ∘ is associative over any two elements of S.


For example, for any x, y ∈ S:
:x ∘ y∘ x = x ∘y ∘ x
:x ∘ x∘ y = x ∘x ∘ y
and so on."
Definition:Altitude,Altitude,An altitude of a polygon is the longest perpendicular from the base to a vertex most distant from the base.,Definition:Altitude of Polygon,"['Definitions/Altitudes (Geometry)', 'Definitions/Polygons']",An altitude of a polygon is the longest perpendicular from the base to a vertex most distant from the base.
Definition:Altitude,Altitude,"Let $\triangle ABC$ be a triangle.

Let $h_a$ be the altitude of $A$:

:


The point at which $h_a$ meets $BC$ (or its production) is the foot of the altitude $h_a$.


Category:Definitions/Triangles
",Definition:Altitude of Triangle,"['Definitions/Altitudes (Geometry)', 'Definitions/Triangles']","Let ABC be a triangle.

Let h_a be the altitude of A:

:


The point at which h_a meets BC (or its production) is the foot of the altitude h_a.


Category:Definitions/Triangles
"
Definition:Altitude,Altitude,":

An altitude of a parallelogram is a line drawn perpendicular to its base, through one of its vertices to the side opposite to the base (which is extended if necessary).

In the parallelogram above, line $DE$ is an altitude of the parallelogram $ABCD$.


The term is also used for the length of such a line.",Definition:Quadrilateral/Parallelogram/Altitude,['Definitions/Parallelograms'],":

An altitude of a parallelogram is a line drawn perpendicular to its base, through one of its vertices to the side opposite to the base (which is extended if necessary).

In the parallelogram above, line DE is an altitude of the parallelogram ABCD.


The term is also used for the length of such a line."
Definition:Altitude,Altitude,"An altitude of a polygon is the longest perpendicular from the base to a vertex most distant from the base.
An altitude of a polyhedron is the longest perpendicular from the base to a vertex most distant from the base.
:

Let a perpendicular $AE$ be dropped from the apex of a cone to the plane containing its base.

The line segment $AE$ is an altitude of the cone.
:

An altitude of a cylinder is a line segment drawn perpendicular to the base and its opposite plane.


In the above diagram, $HJ$ is an altitude of the cylinder $ACBDFE$.
:

An altitude of a prism is a line which is perpendicular to the bases of the prism.

In the above diagram, a line of length $h$ is an altitude of the prism $AJ$.
:

An altitude of a pyramid is a straight line perpendicular to the plane of the base to its apex.

In the above diagram, an altitude is a straight line length is $h$.
",Definition:Altitude of Geometric Figure,"['Definitions/Altitudes (Geometry)', 'Definitions/Geometric Figures']","An altitude of a polygon is the longest perpendicular from the base to a vertex most distant from the base.
An altitude of a polyhedron is the longest perpendicular from the base to a vertex most distant from the base.
:

Let a perpendicular AE be dropped from the apex of a cone to the plane containing its base.

The line segment AE is an altitude of the cone.
:

An altitude of a cylinder is a line segment drawn perpendicular to the base and its opposite plane.


In the above diagram, HJ is an altitude of the cylinder ACBDFE.
:

An altitude of a prism is a line which is perpendicular to the bases of the prism.

In the above diagram, a line of length h is an altitude of the prism AJ.
:

An altitude of a pyramid is a straight line perpendicular to the plane of the base to its apex.

In the above diagram, an altitude is a straight line length is h.
"
Definition:Altitude,Altitude,An altitude of a polyhedron is the longest perpendicular from the base to a vertex most distant from the base.,Definition:Altitude of Polyhedron,"['Definitions/Altitudes (Geometry)', 'Definitions/Polygons']",An altitude of a polyhedron is the longest perpendicular from the base to a vertex most distant from the base.
Definition:Altitude,Altitude,":

Let a perpendicular $AE$ be dropped from the apex of a cone to the plane containing its base.

The line segment $AE$ is an altitude of the cone.",Definition:Cone (Geometry)/Altitude,"['Definitions/Altitudes (Geometry)', 'Definitions/Cones']",":

Let a perpendicular AE be dropped from the apex of a cone to the plane containing its base.

The line segment AE is an altitude of the cone."
Definition:Altitude,Altitude,":

An altitude of a cylinder is a line segment drawn perpendicular to the base and its opposite plane.


In the above diagram, $HJ$ is an altitude of the cylinder $ACBDFE$.",Definition:Altitude of Cylinder,"['Definitions/Altitudes (Geometry)', 'Definitions/Cylinders']",":

An altitude of a cylinder is a line segment drawn perpendicular to the base and its opposite plane.


In the above diagram, HJ is an altitude of the cylinder ACBDFE."
Definition:Altitude,Altitude,":

An altitude of a prism is a line which is perpendicular to the bases of the prism.

In the above diagram, a line of length $h$ is an altitude of the prism $AJ$.",Definition:Altitude of Prism,"['Definitions/Altitudes (Geometry)', 'Definitions/Prisms']",":

An altitude of a prism is a line which is perpendicular to the bases of the prism.

In the above diagram, a line of length h is an altitude of the prism AJ."
Definition:Altitude,Altitude,":

An altitude of a pyramid is a straight line perpendicular to the plane of the base to its apex.

In the above diagram, an altitude is a straight line length is $h$.",Definition:Altitude of Pyramid,"['Definitions/Altitudes (Geometry)', 'Definitions/Pyramids']",":

An altitude of a pyramid is a straight line perpendicular to the plane of the base to its apex.

In the above diagram, an altitude is a straight line length is h."
Definition:Altitude,Altitude,"Let $X$ be a point on the celestial sphere.

The (celestial) altitude of $X$ is defined as the angle subtended by the the arc of the vertical circle through $X$ between the celestial horizon and $X$ itself.


=== Symbol ===
",Definition:Celestial Altitude,"['Definitions/Celestial Altitude', 'Definitions/Celestial Sphere']","Let X be a point on the celestial sphere.

The (celestial) altitude of X is defined as the angle subtended by the the arc of the vertical circle through X between the celestial horizon and X itself.


=== Symbol ===
"
Definition:Amplitude,Amplitude,"Let $f: \R \to \R$ be a periodic real function.


The amplitude of $f$ is the maximum absolute difference of the value of $f$ from a reference level.",Definition:Periodic Real Function/Amplitude,['Definitions/Periodic Functions'],"Let f: → be a periodic real function.


The amplitude of f is the maximum absolute difference of the value of f from a reference level."
Definition:Amplitude,Amplitude,"Let $u = \map F {k, \phi}$ denote the incomplete elliptic integral of the first kind.

The parameter $\phi$ of $u = \map F {k, \phi}$ is called the amplitude of $u$.


=== Symbol ===
",Definition:Incomplete Elliptic Integral of the First Kind/Amplitude,['Definitions/Incomplete Elliptic Integral of the First Kind'],"Let u =  F k, ϕ denote the incomplete elliptic integral of the first kind.

The parameter ϕ of u =  F k, ϕ is called the amplitude of u.


=== Symbol ===
"
Definition:Amplitude,Amplitude,"Consider a physical system $S$ in a state of simple harmonic motion:
:$x = A \map \sin {\omega t + \phi}$


The parameter $A$ is known as the amplitude of the motion.
",Definition:Simple Harmonic Motion/Amplitude,['Definitions/Simple Harmonic Motion'],"Consider a physical system S in a state of simple harmonic motion:
:x = A sinω t + ϕ


The parameter A is known as the amplitude of the motion.
"
Definition:Archimedean,Archimedean,"An Archimedean polyhedron is a convex polyhedron with the following properties:
:$(1): \quad$ Each of its faces is a regular polygon
:$(2): \quad$ It is isogonal
:$(3): \quad$ The faces are not all congruent.
:$(4): \quad$ It is not a regular prism or a regular antiprism.",Definition:Archimedean Polyhedron,"['Definitions/Convex Polyhedra', 'Definitions/Archimedean Polyhedra']","An Archimedean polyhedron is a convex polyhedron with the following properties:
:(1): Each of its faces is a regular polygon
:(2): It is isogonal
:(3): The faces are not all congruent.
:(4): It is not a regular prism or a regular antiprism."
Definition:Archimedean,Archimedean,"The Archimedean spiral is the locus of the equation expressed in polar coordinates as:
:$r = a \theta$


:


=== Archimedes' Definition ===
",Definition:Archimedean Spiral,"['Definitions/Archimedean Spiral', 'Definitions/Spirals']","The Archimedean spiral is the locus of the equation expressed in polar coordinates as:
:r = a θ


:


=== Archimedes' Definition ===
"
Definition:Archimedean,Archimedean,"Let $\struct {S, \circ}$ be a closed algebraic structure.

Let $\cdot: \Z_{>0} \times S \to S$ be the operation defined as:
:$m \cdot a = \begin{cases}
a & : m = 1 \\
a \circ \paren {\paren {m - 1} \cdot a} & : m > 1 \end {cases}$


Let $n: S \to \R$ be a norm on $S$.



Then $n$ satisfies the Archimedean property on $S$ :
:$\forall a, b \in S: n \paren a < n \paren b \implies \exists m \in \N: n \paren {m \cdot a} > n \paren b$


Using the more common symbology for a norm:
:$\forall a, b \in S: \norm a < \norm b \implies \exists m \in \Z_{>0}: \norm {m \cdot a} > \norm b$


Category:Definitions/Abstract Algebra
Category:Definitions/Norm Theory
Let $\struct {S, \circ}$ be a closed algebraic structure.

Let $\cdot: \Z_{>0} \times S \to S$ be the operation defined as:
:$m \cdot a = \begin{cases}
a & : m = 1 \\
a \circ \paren {\paren {m - 1} \cdot a} & : m > 1 \end {cases}$


Let $\preceq$ be an ordering on $S$.


Then $\preceq$ satisfies the Archimedean property on $S$ :

:$\forall a, b \in S: a \prec b \implies \exists m \in \Z_{>0}: b \prec m \cdot a$


Category:Definitions/Abstract Algebra
Category:Definitions/Order Theory
",Definition:Archimedean Property,"['Definitions/Abstract Algebra', 'Definitions/Norm Theory']","Let S, ∘ be a closed algebraic structure.

Let ·: _>0× S → S be the operation defined as:
:m · a = 
a     : m = 1 

a ∘m - 1· a    : m > 1


Let n: S → be a norm on S.



Then n satisfies the Archimedean property on S :
:∀ a, b ∈ S: n  a < n  b ∃ m ∈: n m · a > n  b


Using the more common symbology for a norm:
:∀ a, b ∈ S:  a <  b ∃ m ∈_>0: m · a >  b


Category:Definitions/Abstract Algebra
Category:Definitions/Norm Theory
Let S, ∘ be a closed algebraic structure.

Let ·: _>0× S → S be the operation defined as:
:m · a = 
a     : m = 1 

a ∘m - 1· a    : m > 1


Let ≼ be an ordering on S.


Then ≼ satisfies the Archimedean property on S :

:∀ a, b ∈ S: a ≺ b ∃ m ∈_>0: b ≺ m · a


Category:Definitions/Abstract Algebra
Category:Definitions/Order Theory
"
Definition:Argument,Argument,"Let $R$ be the principal range of the complex numbers $\C$.

The unique value of $\theta$ in $R$ is known as the principal argument, of $z$.

This is denoted $\Arg z$.

Note the capital $A$.

The standard practice is for $R$ to be $\hointl {-\pi} \pi$.

This ensures that the principal argument is continuous on the real axis for positive numbers.

Thus, if $z$ is represented in the complex plane, the principal argument $\Arg z$ is intuitively defined as the angle which $z$ yields with the real ($y = 0$) axis.



",Definition:Argument of Complex Number,"['Definitions/Argument of Complex Number', 'Definitions/Complex Numbers', 'Definitions/Complex Analysis', 'Definitions/Polar Form of Complex Number']","Let R be the principal range of the complex numbers .

The unique value of θ in R is known as the principal argument, of z.

This is denoted z.

Note the capital A.

The standard practice is for R to be -ππ.

This ensures that the principal argument is continuous on the real axis for positive numbers.

Thus, if z is represented in the complex plane, the principal argument z is intuitively defined as the angle which z yields with the real (y = 0) axis.



"
Definition:Argument,Argument,"A logical argument (or just argument) is a process of creating a new statement from one or more existing statements.

An argument proceeds from a set of premises to a conclusion, by means of logical implication, via a procedure called logical inference.


An argument may have more than one premise, but only one conclusion.


While statements may be classified as either true or false, an argument may be classified as either valid or invalid.


Loosely speaking, a valid argument is one that leads unshakeably from true statements to other true statements, whereas an invalid argument is one that can lead you to, for example, a false statement from one which is true.


Thus:
:An argument may be valid, even though its premises are false.
:An argument may be invalid, even though its premises are true.
:An argument may be invalid and its premises false.

It is even possible for the conclusion of an argument to be true, even though the argument is invalid and its premises are false.


To be sure of the truth of a conclusion, it is necessary to make sure both that the premises are true and that the argument is valid.


However, while you may not actually know whether a statement is true or not, you can investigate the consequences of it being either true or false, and see what effect that has on the truth value of the proposition(s) of which it is a part. That, in short, is what the process of logical argument consists of.


An argument may be described symbolically by means of sequents, which specify the flow of an argument.


=== Finitary Argument ===

A valid argument is a logical argument in which the premises provide conclusive reasons for the conclusion.


When a proof is valid, we may say one of the following:
* The conclusion follows from the premises;
* The premises entail the conclusion;
* The conclusion is true on the strength of the premises;
* The conclusion is drawn from the premises;
* The conclusion is deduced from the premises;
* The conclusion is derived from the premises.


=== Proof ===

If all the premises of a valid argument are true, then the conclusion must also therefore be true.

It is not possible for the premises of a valid argument to be true, but for the conclusion to be false.


An invalid argument is a argument in which the premises do not provide conclusive reasons for the conclusion.
A logical argument (or just argument) is a process of creating a new statement from one or more existing statements.

An argument proceeds from a set of premises to a conclusion, by means of logical implication, via a procedure called logical inference.


An argument may have more than one premise, but only one conclusion.


While statements may be classified as either true or false, an argument may be classified as either valid or invalid.


Loosely speaking, a valid argument is one that leads unshakeably from true statements to other true statements, whereas an invalid argument is one that can lead you to, for example, a false statement from one which is true.


Thus:
:An argument may be valid, even though its premises are false.
:An argument may be invalid, even though its premises are true.
:An argument may be invalid and its premises false.

It is even possible for the conclusion of an argument to be true, even though the argument is invalid and its premises are false.


To be sure of the truth of a conclusion, it is necessary to make sure both that the premises are true and that the argument is valid.


However, while you may not actually know whether a statement is true or not, you can investigate the consequences of it being either true or false, and see what effect that has on the truth value of the proposition(s) of which it is a part. That, in short, is what the process of logical argument consists of.


An argument may be described symbolically by means of sequents, which specify the flow of an argument.


=== Finitary Argument ===

A logical argument (or just argument) is a process of creating a new statement from one or more existing statements.

An argument proceeds from a set of premises to a conclusion, by means of logical implication, via a procedure called logical inference.


An argument may have more than one premise, but only one conclusion.


While statements may be classified as either true or false, an argument may be classified as either valid or invalid.


Loosely speaking, a valid argument is one that leads unshakeably from true statements to other true statements, whereas an invalid argument is one that can lead you to, for example, a false statement from one which is true.


Thus:
:An argument may be valid, even though its premises are false.
:An argument may be invalid, even though its premises are true.
:An argument may be invalid and its premises false.

It is even possible for the conclusion of an argument to be true, even though the argument is invalid and its premises are false.


To be sure of the truth of a conclusion, it is necessary to make sure both that the premises are true and that the argument is valid.


However, while you may not actually know whether a statement is true or not, you can investigate the consequences of it being either true or false, and see what effect that has on the truth value of the proposition(s) of which it is a part. That, in short, is what the process of logical argument consists of.


An argument may be described symbolically by means of sequents, which specify the flow of an argument.


=== Finitary Argument ===

A logical argument (or just argument) is a process of creating a new statement from one or more existing statements.

An argument proceeds from a set of premises to a conclusion, by means of logical implication, via a procedure called logical inference.


An argument may have more than one premise, but only one conclusion.


While statements may be classified as either true or false, an argument may be classified as either valid or invalid.


Loosely speaking, a valid argument is one that leads unshakeably from true statements to other true statements, whereas an invalid argument is one that can lead you to, for example, a false statement from one which is true.


Thus:
:An argument may be valid, even though its premises are false.
:An argument may be invalid, even though its premises are true.
:An argument may be invalid and its premises false.

It is even possible for the conclusion of an argument to be true, even though the argument is invalid and its premises are false.


To be sure of the truth of a conclusion, it is necessary to make sure both that the premises are true and that the argument is valid.


However, while you may not actually know whether a statement is true or not, you can investigate the consequences of it being either true or false, and see what effect that has on the truth value of the proposition(s) of which it is a part. That, in short, is what the process of logical argument consists of.


An argument may be described symbolically by means of sequents, which specify the flow of an argument.


=== Finitary Argument ===

A logical argument (or just argument) is a process of creating a new statement from one or more existing statements.

An argument proceeds from a set of premises to a conclusion, by means of logical implication, via a procedure called logical inference.


An argument may have more than one premise, but only one conclusion.


While statements may be classified as either true or false, an argument may be classified as either valid or invalid.


Loosely speaking, a valid argument is one that leads unshakeably from true statements to other true statements, whereas an invalid argument is one that can lead you to, for example, a false statement from one which is true.


Thus:
:An argument may be valid, even though its premises are false.
:An argument may be invalid, even though its premises are true.
:An argument may be invalid and its premises false.

It is even possible for the conclusion of an argument to be true, even though the argument is invalid and its premises are false.


To be sure of the truth of a conclusion, it is necessary to make sure both that the premises are true and that the argument is valid.


However, while you may not actually know whether a statement is true or not, you can investigate the consequences of it being either true or false, and see what effect that has on the truth value of the proposition(s) of which it is a part. That, in short, is what the process of logical argument consists of.


An argument may be described symbolically by means of sequents, which specify the flow of an argument.


=== Finitary Argument ===

A logical argument (or just argument) is a process of creating a new statement from one or more existing statements.

An argument proceeds from a set of premises to a conclusion, by means of logical implication, via a procedure called logical inference.


An argument may have more than one premise, but only one conclusion.


While statements may be classified as either true or false, an argument may be classified as either valid or invalid.


Loosely speaking, a valid argument is one that leads unshakeably from true statements to other true statements, whereas an invalid argument is one that can lead you to, for example, a false statement from one which is true.


Thus:
:An argument may be valid, even though its premises are false.
:An argument may be invalid, even though its premises are true.
:An argument may be invalid and its premises false.

It is even possible for the conclusion of an argument to be true, even though the argument is invalid and its premises are false.


To be sure of the truth of a conclusion, it is necessary to make sure both that the premises are true and that the argument is valid.


However, while you may not actually know whether a statement is true or not, you can investigate the consequences of it being either true or false, and see what effect that has on the truth value of the proposition(s) of which it is a part. That, in short, is what the process of logical argument consists of.


An argument may be described symbolically by means of sequents, which specify the flow of an argument.


=== Finitary Argument ===

A logical argument (or just argument) is a process of creating a new statement from one or more existing statements.

An argument proceeds from a set of premises to a conclusion, by means of logical implication, via a procedure called logical inference.


An argument may have more than one premise, but only one conclusion.


While statements may be classified as either true or false, an argument may be classified as either valid or invalid.


Loosely speaking, a valid argument is one that leads unshakeably from true statements to other true statements, whereas an invalid argument is one that can lead you to, for example, a false statement from one which is true.


Thus:
:An argument may be valid, even though its premises are false.
:An argument may be invalid, even though its premises are true.
:An argument may be invalid and its premises false.

It is even possible for the conclusion of an argument to be true, even though the argument is invalid and its premises are false.


To be sure of the truth of a conclusion, it is necessary to make sure both that the premises are true and that the argument is valid.


However, while you may not actually know whether a statement is true or not, you can investigate the consequences of it being either true or false, and see what effect that has on the truth value of the proposition(s) of which it is a part. That, in short, is what the process of logical argument consists of.


An argument may be described symbolically by means of sequents, which specify the flow of an argument.


=== Finitary Argument ===

A logical argument (or just argument) is a process of creating a new statement from one or more existing statements.

An argument proceeds from a set of premises to a conclusion, by means of logical implication, via a procedure called logical inference.


An argument may have more than one premise, but only one conclusion.


While statements may be classified as either true or false, an argument may be classified as either valid or invalid.


Loosely speaking, a valid argument is one that leads unshakeably from true statements to other true statements, whereas an invalid argument is one that can lead you to, for example, a false statement from one which is true.


Thus:
:An argument may be valid, even though its premises are false.
:An argument may be invalid, even though its premises are true.
:An argument may be invalid and its premises false.

It is even possible for the conclusion of an argument to be true, even though the argument is invalid and its premises are false.


To be sure of the truth of a conclusion, it is necessary to make sure both that the premises are true and that the argument is valid.


However, while you may not actually know whether a statement is true or not, you can investigate the consequences of it being either true or false, and see what effect that has on the truth value of the proposition(s) of which it is a part. That, in short, is what the process of logical argument consists of.


An argument may be described symbolically by means of sequents, which specify the flow of an argument.


=== Finitary Argument ===

A logical argument (or just argument) is a process of creating a new statement from one or more existing statements.

An argument proceeds from a set of premises to a conclusion, by means of logical implication, via a procedure called logical inference.


An argument may have more than one premise, but only one conclusion.


While statements may be classified as either true or false, an argument may be classified as either valid or invalid.


Loosely speaking, a valid argument is one that leads unshakeably from true statements to other true statements, whereas an invalid argument is one that can lead you to, for example, a false statement from one which is true.


Thus:
:An argument may be valid, even though its premises are false.
:An argument may be invalid, even though its premises are true.
:An argument may be invalid and its premises false.

It is even possible for the conclusion of an argument to be true, even though the argument is invalid and its premises are false.


To be sure of the truth of a conclusion, it is necessary to make sure both that the premises are true and that the argument is valid.


However, while you may not actually know whether a statement is true or not, you can investigate the consequences of it being either true or false, and see what effect that has on the truth value of the proposition(s) of which it is a part. That, in short, is what the process of logical argument consists of.


An argument may be described symbolically by means of sequents, which specify the flow of an argument.


=== Finitary Argument ===

A logical argument (or just argument) is a process of creating a new statement from one or more existing statements.

An argument proceeds from a set of premises to a conclusion, by means of logical implication, via a procedure called logical inference.


An argument may have more than one premise, but only one conclusion.


While statements may be classified as either true or false, an argument may be classified as either valid or invalid.


Loosely speaking, a valid argument is one that leads unshakeably from true statements to other true statements, whereas an invalid argument is one that can lead you to, for example, a false statement from one which is true.


Thus:
:An argument may be valid, even though its premises are false.
:An argument may be invalid, even though its premises are true.
:An argument may be invalid and its premises false.

It is even possible for the conclusion of an argument to be true, even though the argument is invalid and its premises are false.


To be sure of the truth of a conclusion, it is necessary to make sure both that the premises are true and that the argument is valid.


However, while you may not actually know whether a statement is true or not, you can investigate the consequences of it being either true or false, and see what effect that has on the truth value of the proposition(s) of which it is a part. That, in short, is what the process of logical argument consists of.


An argument may be described symbolically by means of sequents, which specify the flow of an argument.


=== Finitary Argument ===

A finitary argument is a logical argument which starts with a finite number of axioms, and can be translated into a finite number of statements.
",Definition:Logical Argument,"['Definitions/Logical Arguments', 'Definitions/Logic']","A logical argument (or just argument) is a process of creating a new statement from one or more existing statements.

An argument proceeds from a set of premises to a conclusion, by means of logical implication, via a procedure called logical inference.


An argument may have more than one premise, but only one conclusion.


While statements may be classified as either true or false, an argument may be classified as either valid or invalid.


Loosely speaking, a valid argument is one that leads unshakeably from true statements to other true statements, whereas an invalid argument is one that can lead you to, for example, a false statement from one which is true.


Thus:
:An argument may be valid, even though its premises are false.
:An argument may be invalid, even though its premises are true.
:An argument may be invalid and its premises false.

It is even possible for the conclusion of an argument to be true, even though the argument is invalid and its premises are false.


To be sure of the truth of a conclusion, it is necessary to make sure both that the premises are true and that the argument is valid.


However, while you may not actually know whether a statement is true or not, you can investigate the consequences of it being either true or false, and see what effect that has on the truth value of the proposition(s) of which it is a part. That, in short, is what the process of logical argument consists of.


An argument may be described symbolically by means of sequents, which specify the flow of an argument.


=== Finitary Argument ===

A valid argument is a logical argument in which the premises provide conclusive reasons for the conclusion.


When a proof is valid, we may say one of the following:
* The conclusion follows from the premises;
* The premises entail the conclusion;
* The conclusion is true on the strength of the premises;
* The conclusion is drawn from the premises;
* The conclusion is deduced from the premises;
* The conclusion is derived from the premises.


=== Proof ===

If all the premises of a valid argument are true, then the conclusion must also therefore be true.

It is not possible for the premises of a valid argument to be true, but for the conclusion to be false.


An invalid argument is a argument in which the premises do not provide conclusive reasons for the conclusion.
A logical argument (or just argument) is a process of creating a new statement from one or more existing statements.

An argument proceeds from a set of premises to a conclusion, by means of logical implication, via a procedure called logical inference.


An argument may have more than one premise, but only one conclusion.


While statements may be classified as either true or false, an argument may be classified as either valid or invalid.


Loosely speaking, a valid argument is one that leads unshakeably from true statements to other true statements, whereas an invalid argument is one that can lead you to, for example, a false statement from one which is true.


Thus:
:An argument may be valid, even though its premises are false.
:An argument may be invalid, even though its premises are true.
:An argument may be invalid and its premises false.

It is even possible for the conclusion of an argument to be true, even though the argument is invalid and its premises are false.


To be sure of the truth of a conclusion, it is necessary to make sure both that the premises are true and that the argument is valid.


However, while you may not actually know whether a statement is true or not, you can investigate the consequences of it being either true or false, and see what effect that has on the truth value of the proposition(s) of which it is a part. That, in short, is what the process of logical argument consists of.


An argument may be described symbolically by means of sequents, which specify the flow of an argument.


=== Finitary Argument ===

A logical argument (or just argument) is a process of creating a new statement from one or more existing statements.

An argument proceeds from a set of premises to a conclusion, by means of logical implication, via a procedure called logical inference.


An argument may have more than one premise, but only one conclusion.


While statements may be classified as either true or false, an argument may be classified as either valid or invalid.


Loosely speaking, a valid argument is one that leads unshakeably from true statements to other true statements, whereas an invalid argument is one that can lead you to, for example, a false statement from one which is true.


Thus:
:An argument may be valid, even though its premises are false.
:An argument may be invalid, even though its premises are true.
:An argument may be invalid and its premises false.

It is even possible for the conclusion of an argument to be true, even though the argument is invalid and its premises are false.


To be sure of the truth of a conclusion, it is necessary to make sure both that the premises are true and that the argument is valid.


However, while you may not actually know whether a statement is true or not, you can investigate the consequences of it being either true or false, and see what effect that has on the truth value of the proposition(s) of which it is a part. That, in short, is what the process of logical argument consists of.


An argument may be described symbolically by means of sequents, which specify the flow of an argument.


=== Finitary Argument ===

A logical argument (or just argument) is a process of creating a new statement from one or more existing statements.

An argument proceeds from a set of premises to a conclusion, by means of logical implication, via a procedure called logical inference.


An argument may have more than one premise, but only one conclusion.


While statements may be classified as either true or false, an argument may be classified as either valid or invalid.


Loosely speaking, a valid argument is one that leads unshakeably from true statements to other true statements, whereas an invalid argument is one that can lead you to, for example, a false statement from one which is true.


Thus:
:An argument may be valid, even though its premises are false.
:An argument may be invalid, even though its premises are true.
:An argument may be invalid and its premises false.

It is even possible for the conclusion of an argument to be true, even though the argument is invalid and its premises are false.


To be sure of the truth of a conclusion, it is necessary to make sure both that the premises are true and that the argument is valid.


However, while you may not actually know whether a statement is true or not, you can investigate the consequences of it being either true or false, and see what effect that has on the truth value of the proposition(s) of which it is a part. That, in short, is what the process of logical argument consists of.


An argument may be described symbolically by means of sequents, which specify the flow of an argument.


=== Finitary Argument ===

A logical argument (or just argument) is a process of creating a new statement from one or more existing statements.

An argument proceeds from a set of premises to a conclusion, by means of logical implication, via a procedure called logical inference.


An argument may have more than one premise, but only one conclusion.


While statements may be classified as either true or false, an argument may be classified as either valid or invalid.


Loosely speaking, a valid argument is one that leads unshakeably from true statements to other true statements, whereas an invalid argument is one that can lead you to, for example, a false statement from one which is true.


Thus:
:An argument may be valid, even though its premises are false.
:An argument may be invalid, even though its premises are true.
:An argument may be invalid and its premises false.

It is even possible for the conclusion of an argument to be true, even though the argument is invalid and its premises are false.


To be sure of the truth of a conclusion, it is necessary to make sure both that the premises are true and that the argument is valid.


However, while you may not actually know whether a statement is true or not, you can investigate the consequences of it being either true or false, and see what effect that has on the truth value of the proposition(s) of which it is a part. That, in short, is what the process of logical argument consists of.


An argument may be described symbolically by means of sequents, which specify the flow of an argument.


=== Finitary Argument ===

A logical argument (or just argument) is a process of creating a new statement from one or more existing statements.

An argument proceeds from a set of premises to a conclusion, by means of logical implication, via a procedure called logical inference.


An argument may have more than one premise, but only one conclusion.


While statements may be classified as either true or false, an argument may be classified as either valid or invalid.


Loosely speaking, a valid argument is one that leads unshakeably from true statements to other true statements, whereas an invalid argument is one that can lead you to, for example, a false statement from one which is true.


Thus:
:An argument may be valid, even though its premises are false.
:An argument may be invalid, even though its premises are true.
:An argument may be invalid and its premises false.

It is even possible for the conclusion of an argument to be true, even though the argument is invalid and its premises are false.


To be sure of the truth of a conclusion, it is necessary to make sure both that the premises are true and that the argument is valid.


However, while you may not actually know whether a statement is true or not, you can investigate the consequences of it being either true or false, and see what effect that has on the truth value of the proposition(s) of which it is a part. That, in short, is what the process of logical argument consists of.


An argument may be described symbolically by means of sequents, which specify the flow of an argument.


=== Finitary Argument ===

A logical argument (or just argument) is a process of creating a new statement from one or more existing statements.

An argument proceeds from a set of premises to a conclusion, by means of logical implication, via a procedure called logical inference.


An argument may have more than one premise, but only one conclusion.


While statements may be classified as either true or false, an argument may be classified as either valid or invalid.


Loosely speaking, a valid argument is one that leads unshakeably from true statements to other true statements, whereas an invalid argument is one that can lead you to, for example, a false statement from one which is true.


Thus:
:An argument may be valid, even though its premises are false.
:An argument may be invalid, even though its premises are true.
:An argument may be invalid and its premises false.

It is even possible for the conclusion of an argument to be true, even though the argument is invalid and its premises are false.


To be sure of the truth of a conclusion, it is necessary to make sure both that the premises are true and that the argument is valid.


However, while you may not actually know whether a statement is true or not, you can investigate the consequences of it being either true or false, and see what effect that has on the truth value of the proposition(s) of which it is a part. That, in short, is what the process of logical argument consists of.


An argument may be described symbolically by means of sequents, which specify the flow of an argument.


=== Finitary Argument ===

A logical argument (or just argument) is a process of creating a new statement from one or more existing statements.

An argument proceeds from a set of premises to a conclusion, by means of logical implication, via a procedure called logical inference.


An argument may have more than one premise, but only one conclusion.


While statements may be classified as either true or false, an argument may be classified as either valid or invalid.


Loosely speaking, a valid argument is one that leads unshakeably from true statements to other true statements, whereas an invalid argument is one that can lead you to, for example, a false statement from one which is true.


Thus:
:An argument may be valid, even though its premises are false.
:An argument may be invalid, even though its premises are true.
:An argument may be invalid and its premises false.

It is even possible for the conclusion of an argument to be true, even though the argument is invalid and its premises are false.


To be sure of the truth of a conclusion, it is necessary to make sure both that the premises are true and that the argument is valid.


However, while you may not actually know whether a statement is true or not, you can investigate the consequences of it being either true or false, and see what effect that has on the truth value of the proposition(s) of which it is a part. That, in short, is what the process of logical argument consists of.


An argument may be described symbolically by means of sequents, which specify the flow of an argument.


=== Finitary Argument ===

A logical argument (or just argument) is a process of creating a new statement from one or more existing statements.

An argument proceeds from a set of premises to a conclusion, by means of logical implication, via a procedure called logical inference.


An argument may have more than one premise, but only one conclusion.


While statements may be classified as either true or false, an argument may be classified as either valid or invalid.


Loosely speaking, a valid argument is one that leads unshakeably from true statements to other true statements, whereas an invalid argument is one that can lead you to, for example, a false statement from one which is true.


Thus:
:An argument may be valid, even though its premises are false.
:An argument may be invalid, even though its premises are true.
:An argument may be invalid and its premises false.

It is even possible for the conclusion of an argument to be true, even though the argument is invalid and its premises are false.


To be sure of the truth of a conclusion, it is necessary to make sure both that the premises are true and that the argument is valid.


However, while you may not actually know whether a statement is true or not, you can investigate the consequences of it being either true or false, and see what effect that has on the truth value of the proposition(s) of which it is a part. That, in short, is what the process of logical argument consists of.


An argument may be described symbolically by means of sequents, which specify the flow of an argument.


=== Finitary Argument ===

A logical argument (or just argument) is a process of creating a new statement from one or more existing statements.

An argument proceeds from a set of premises to a conclusion, by means of logical implication, via a procedure called logical inference.


An argument may have more than one premise, but only one conclusion.


While statements may be classified as either true or false, an argument may be classified as either valid or invalid.


Loosely speaking, a valid argument is one that leads unshakeably from true statements to other true statements, whereas an invalid argument is one that can lead you to, for example, a false statement from one which is true.


Thus:
:An argument may be valid, even though its premises are false.
:An argument may be invalid, even though its premises are true.
:An argument may be invalid and its premises false.

It is even possible for the conclusion of an argument to be true, even though the argument is invalid and its premises are false.


To be sure of the truth of a conclusion, it is necessary to make sure both that the premises are true and that the argument is valid.


However, while you may not actually know whether a statement is true or not, you can investigate the consequences of it being either true or false, and see what effect that has on the truth value of the proposition(s) of which it is a part. That, in short, is what the process of logical argument consists of.


An argument may be described symbolically by means of sequents, which specify the flow of an argument.


=== Finitary Argument ===

A finitary argument is a logical argument which starts with a finite number of axioms, and can be translated into a finite number of statements.
"
Definition:Argument,Argument,"Let $f: S \to T$ be a mapping.

Let $f^{-1} \subseteq T \times S$ be the inverse of $f$, defined as:

:$f^{-1} = \set {\tuple {t, s}: \map f s = t}$


Every $s \in S$ such that $\map f s = t$ is called a preimage of $t$.


The preimage of an element $t \in T$ is defined as:

:$\map {f^{-1} } t := \set {s \in S: \map f s = t}$


This can also be expressed as:
:$\map {f^{-1} } t := \set {s \in \Img {f^{-1} }: \tuple {t, s} \in f^{-1} }$


That is, the preimage of $t$ under $f$ is the image of $t$ under $f^{-1}$.",Definition:Preimage/Mapping/Element,['Definitions/Preimages'],"Let f: S → T be a mapping.

Let f^-1⊆ T × S be the inverse of f, defined as:

:f^-1 = t, s:  f s = t


Every s ∈ S such that f s = t is called a preimage of t.


The preimage of an element t ∈ T is defined as:

:f^-1 t := s ∈ S:  f s = t


This can also be expressed as:
:f^-1 t := s ∈f^-1: t, s∈ f^-1


That is, the preimage of t under f is the image of t under f^-1."
Definition:Argument,Argument,"Let $P$ be a point in a system of polar coordinates where $O$ is the pole.


The angle measured anticlockwise from the polar axis to $OP$ is called the angular coordinate of $P$, and usually labelled $\theta$.

If the angle is measured clockwise from the polar axis to $OP$, its value is considered negative.",Definition:Polar Coordinates/Angular Coordinate,['Definitions/Polar Coordinates'],"Let P be a point in a system of polar coordinates where O is the pole.


The angle measured anticlockwise from the polar axis to OP is called the angular coordinate of P, and usually labelled θ.

If the angle is measured clockwise from the polar axis to OP, its value is considered negative."
Definition:Arrow,Arrow,"Let $\mathbf C$ be a metacategory.


A morphism of $\mathbf C$ is an object $f$, together with:

* A domain $\operatorname {dom} f$, which is an object of $\mathbf C$
* A codomain $\operatorname {cod} f$, also an object of $\mathbf C$


The collection of all morphisms of $\mathbf C$ is denoted $\mathbf C_1$.


If $A$ is the domain of $f$ and $B$ is its codomain, this is mostly represented by writing:

:$f: A \to B$ or $A \stackrel f \longrightarrow B$",Definition:Morphism,"['Definitions/Morphisms', 'Definitions/Category Theory']","Let 𝐂 be a metacategory.


A morphism of 𝐂 is an object f, together with:

* A domain dom f, which is an object of 𝐂
* A codomain cod f, also an object of 𝐂


The collection of all morphisms of 𝐂 is denoted 𝐂_1.


If A is the domain of f and B is its codomain, this is mostly represented by writing:

:f: A → B or A  f ⟶ B"
Definition:Arrow,Arrow,"Let $\mathbf C$ be a metacategory.


Its morphism category, denoted $\mathbf C^\to$, is defined as follows:




The morphisms of $\mathbf C^\to$ can be made more intuitive by the following diagram:

::$\begin{xy}
<-2em,0em>*+{f} = ""f"",
<2em,0em>*+{f'} = ""f2"",

""f"";""f2"" **@{-} ?>*@{>} ?*!/_1em/{\scriptstyle \tuple {g_1, g_2} },

<3em,0em>*{:},

<7em,2em>*+{A} = ""A"",
<7em,-2em>*+{B} = ""B"",
<11em,2em>*+{A'} = ""A2"",
<11em,-2em>*+{B'} = ""B2"",

""A"";""B"" **@{-} ?>*@{>} ?*!/^1em/{f},
""A"";""A2"" **@{-} ?>*@{>} ?*!/_1em/{g_1},
""A2"";""B2"" **@{-} ?>*@{>} ?*!/_1em/{f'},
""B"";""B2"" **@{-} ?>*@{>} ?*!/^1em/{g_2}
\end{xy}$

The composition likewise benefits from a diagrammatic representation:

::$\begin{xy}
<4em,5em>*{\tuple {h_1, h_2} \circ \tuple {g_1, g_2} },

<0em,2em>*+{A} = ""A"",
<0em,-2em>*+{B} = ""B"",
<4em,2em>*+{A'} = ""A2"",
<4em,-2em>*+{B'} = ""B2"",

""A"";""B"" **@{-} ?>*@{>} ?*!/^1em/{f},
""A"";""A2"" **@{-} ?>*@{>} ?*!/_1em/{g_1},
""A2"";""B2"" **@{-} ?>*@{>} ?*!/_1em/{f'},
""B"";""B2"" **@{-} ?>*@{>} ?*!/^1em/{g_2},

<8em,2em>*+{A} = ""A3"",
<8em,-2em>*+{B} = ""B3"",

""A2"";""A3"" **@{-} ?>*@{>} ?*!/_1em/{h_1},
""B2"";""B3"" **@{-} ?>*@{>} ?*!/^1em/{h_2},
""A3"";""B3"" **@{-} ?>*@{>} ?*!/_1em/{f},

<12em,5em>*{=},
<10em,0em>;<14em,0em> **@{~} ?>*@2{>},

<20em,5em>*+{\tuple {h_1 \circ g_1, h_2 \circ g_2} },

<16em,2em>*+{A} = ""AA"",
<16em,-2em>*+{B} = ""BB"",
<24em,2em>*+{A} = ""AA3"",
<24em,-2em>*+{B} = ""BB3"",

""AA"";""BB"" **@{-} ?>*@{>} ?*!/^1em/{f},
""AA"";""AA3"" **@{-} ?>*@{>} ?*!/_1em/{h_1 \circ g_1},
""AA3"";""BB3"" **@{-} ?>*@{>} ?*!/_1em/{f},
""BB"";""BB3"" **@{-} ?>*@{>} ?*!/^1em/{h_2 \circ g_2},
\end{xy}$",Definition:Morphism Category,"['Definitions/Examples of Categories', 'Definitions/Morphisms']","Let 𝐂 be a metacategory.


Its morphism category, denoted 𝐂^→, is defined as follows:




The morphisms of 𝐂^→ can be made more intuitive by the following diagram:

::<-2em,0em>*+f = ""f"",
<2em,0em>*+f' = ""f2"",

""f"";""f2"" **@- ?>*@> ?*!/_1em/g_1, g_2,

<3em,0em>*:,

<7em,2em>*+A = ""A"",
<7em,-2em>*+B = ""B"",
<11em,2em>*+A' = ""A2"",
<11em,-2em>*+B' = ""B2"",

""A"";""B"" **@- ?>*@> ?*!/^1em/f,
""A"";""A2"" **@- ?>*@> ?*!/_1em/g_1,
""A2"";""B2"" **@- ?>*@> ?*!/_1em/f',
""B"";""B2"" **@- ?>*@> ?*!/^1em/g_2

The composition likewise benefits from a diagrammatic representation:

::<4em,5em>*h_1, h_2∘g_1, g_2,

<0em,2em>*+A = ""A"",
<0em,-2em>*+B = ""B"",
<4em,2em>*+A' = ""A2"",
<4em,-2em>*+B' = ""B2"",

""A"";""B"" **@- ?>*@> ?*!/^1em/f,
""A"";""A2"" **@- ?>*@> ?*!/_1em/g_1,
""A2"";""B2"" **@- ?>*@> ?*!/_1em/f',
""B"";""B2"" **@- ?>*@> ?*!/^1em/g_2,

<8em,2em>*+A = ""A3"",
<8em,-2em>*+B = ""B3"",

""A2"";""A3"" **@- ?>*@> ?*!/_1em/h_1,
""B2"";""B3"" **@- ?>*@> ?*!/^1em/h_2,
""A3"";""B3"" **@- ?>*@> ?*!/_1em/f,

<12em,5em>*=,
<10em,0em>;<14em,0em> **@  ?>*@2>,

<20em,5em>*+h_1 ∘ g_1, h_2 ∘ g_2,

<16em,2em>*+A = ""AA"",
<16em,-2em>*+B = ""BB"",
<24em,2em>*+A = ""AA3"",
<24em,-2em>*+B = ""BB3"",

""AA"";""BB"" **@- ?>*@> ?*!/^1em/f,
""AA"";""AA3"" **@- ?>*@> ?*!/_1em/h_1 ∘ g_1,
""AA3"";""BB3"" **@- ?>*@> ?*!/_1em/f,
""BB"";""BB3"" **@- ?>*@> ?*!/^1em/h_2 ∘ g_2,"
Definition:Associative,Associative,"Let $S$ be a set.

Let $\circ : S \times S \to S$ be a binary operation.


Then $\circ$ is associative :

:$\forall x, y, z \in S: \paren {x \circ y} \circ z = x \circ \paren {y \circ z}$",Definition:Associative Operation,"['Definitions/Abstract Algebra', 'Definitions/Operations', 'Definitions/Associativity']","Let S be a set.

Let ∘ : S × S → S be a binary operation.


Then ∘ is associative :

:∀ x, y, z ∈ S: x ∘ y∘ z = x ∘y ∘ z"
Definition:Associative,Associative,"Let $\circ$ be a binary operation.

Then $\circ$ is defined as being power-associative on $S$ :

:$\forall x \in S: \paren {x \circ x} \circ x = x \circ \paren {x \circ x}$",Definition:Power-Associative Operation,"['Definitions/Abstract Algebra', 'Definitions/Operations']","Let ∘ be a binary operation.

Then ∘ is defined as being power-associative on S :

:∀ x ∈ S: x ∘ x∘ x = x ∘x ∘ x"
Definition:Associative,Associative,"Let $S$ be a set.

Let $\circ : S \times S \to S$ be a binary operation.


Then $\circ$ is associative :

:$\forall x, y, z \in S: \paren {x \circ y} \circ z = x \circ \paren {y \circ z}$
Let $S$ be a set.

Let $\circ : S \times S \to S$ be a binary operation.


Then $\circ$ is associative :

:$\forall x, y, z \in S: \paren {x \circ y} \circ z = x \circ \paren {y \circ z}$
",Definition:Semigroup,"['Definitions/Semigroups', 'Definitions/Algebraic Structures']","Let S be a set.

Let ∘ : S × S → S be a binary operation.


Then ∘ is associative :

:∀ x, y, z ∈ S: x ∘ y∘ z = x ∘y ∘ z
Let S be a set.

Let ∘ : S × S → S be a binary operation.


Then ∘ is associative :

:∀ x, y, z ∈ S: x ∘ y∘ z = x ∘y ∘ z
"
Definition:Associative,Associative,"Let $S$ be a set.

Let $\circ : S \times S \to S$ be a binary operation.


Then $\circ$ is associative :

:$\forall x, y, z \in S: \paren {x \circ y} \circ z = x \circ \paren {y \circ z}$
",Definition:Associative Algebra,"['Definitions/Associative Algebras', 'Definitions/Algebras', 'Definitions/Associativity']","Let S be a set.

Let ∘ : S × S → S be a binary operation.


Then ∘ is associative :

:∀ x, y, z ∈ S: x ∘ y∘ z = x ∘y ∘ z
"
Definition:Atom,Atom,"In a particular branch of logic, certain concepts are at such a basic level of simplicity they can not be broken down into anything simpler.

Those concepts are called atoms or described as atomic.


Different branches of logic admit different atoms.


=== Propositional Logic ===
",Definition:Atom (Logic),['Definitions/Logic'],"In a particular branch of logic, certain concepts are at such a basic level of simplicity they can not be broken down into anything simpler.

Those concepts are called atoms or described as atomic.


Different branches of logic admit different atoms.


=== Propositional Logic ===
"
Definition:Atom,Atom,"Let $\struct {X, \Sigma, \mu}$ be a measure space.


An element $x \in X$ is said to be an atom (of $\mu$) :

:$(1): \quad \set x \in \Sigma$
:$(2): \quad \map \mu {\set x} > 0$

",Definition:Atom of Measure,['Definitions/Measures'],"Let X, Σ, μ be a measure space.


An element x ∈ X is said to be an atom (of μ) :

:(1):    x ∈Σ
:(2):   μ x > 0

"
Definition:Atom,Atom,"Let $\struct {X, \Sigma}$ be a measurable space.

Let $E \in \Sigma$ be non-empty.


$E$ is said to be an atom (of $\Sigma$)  it satisfies:

:$\forall F \in \Sigma: F \subsetneq E \implies F = \O$


Thus, atoms are the minimal non-empty sets in $\Sigma$ with respect to the subset ordering.",Definition:Atom of Sigma-Algebra,['Definitions/Sigma-Algebras'],"Let X, Σ be a measurable space.

Let E ∈Σ be non-empty.


E is said to be an atom (of Σ)  it satisfies:

:∀ F ∈Σ: F ⊊ E  F = Ø


Thus, atoms are the minimal non-empty sets in Σ with respect to the subset ordering."
Definition:Atom,Atom,"Let $\struct {S, \vee, \wedge, \preceq}$ be a lattice.


An atom of $\struct {S, \vee, \wedge, \preceq}$ is an element $A \in S$ such that:
:$\forall B \in S: B \preceq A, B \ne A \implies B = \bot$
:$A \ne \bot$
where $\bot$ denotes the bottom of $\struct {S, \vee, \wedge, \preceq}$.",Definition:Atom of Lattice,"['Definitions/Atoms of Lattices', 'Definitions/Lattice Theory']","Let S, ∨, ∧, ≼ be a lattice.


An atom of S, ∨, ∧, ≼ is an element A ∈ S such that:
:∀ B ∈ S: B ≼ A, B  A  B =
:A
where  denotes the bottom of S, ∨, ∧, ≼."
Definition:Atom,Atom,"An atom (in the context of physics and chemistry) is the smallest piece of matter that can exist of a particular type of substance.


Atoms can be subdivided into smaller particles, but then it ceases to be that substance.


=== Diameter ===

An atom (in the context of physics and chemistry) is the smallest piece of matter that can exist of a particular type of substance.


Atoms can be subdivided into smaller particles, but then it ceases to be that substance.


=== Diameter ===

",Definition:Atom (Physics),"['Definitions/Atoms', 'Definitions/Atomic Physics', 'Definitions/Chemistry']","An atom (in the context of physics and chemistry) is the smallest piece of matter that can exist of a particular type of substance.


Atoms can be subdivided into smaller particles, but then it ceases to be that substance.


=== Diameter ===

An atom (in the context of physics and chemistry) is the smallest piece of matter that can exist of a particular type of substance.


Atoms can be subdivided into smaller particles, but then it ceases to be that substance.


=== Diameter ===

"
Definition:Auxiliary,Auxiliary,"Let:
:$(1): \quad y + p y' + q y = 0$
be a constant coefficient homogeneous linear second order ODE.


The auxiliary equation of $(1)$ is the quadratic equation:
:$m^2 + p m + q = 0$",Definition:Auxiliary Equation,"['Definitions/Auxiliary Equations', 'Definitions/Second Order ODEs']","Let:
:(1):    y + p y' + q y = 0
be a constant coefficient homogeneous linear second order ODE.


The auxiliary equation of (1) is the quadratic equation:
:m^2 + p m + q = 0"
Definition:Auxiliary,Auxiliary,"Let $L = \struct {S, \vee, \preceq}$ be a bounded below join semilattice.

Let $\RR \subseteq S \times S$ be a relation on $S$.


Then $\RR$ is an auxiliary relation  $\RR$ satisfies the auxiliary relation axioms:
",Definition:Auxiliary Relation,['Definitions/Order Theory'],"Let L = S, ∨, ≼ be a bounded below join semilattice.

Let ⊆ S × S be a relation on S.


Then  is an auxiliary relation   satisfies the auxiliary relation axioms:
"
Definition:Auxiliary,Auxiliary,"Consider the expression:

:$(1): \quad p \sin x + q \cos x$

where $x \in \R$.

Let $(1)$ be expressed in the form:
:$(2): \quad R \map \cos {x + \alpha}$

or:
:$(3): \quad R \map \sin {x + \alpha}$


The angle $\alpha$ is known as the auxiliary angle of either $(2)$ or $(3)$ as appropriate.
",Definition:Auxiliary Angle,"['Definitions/Angles', 'Definitions/Sine Function', 'Definitions/Cosine Function', 'Definitions/Trigonometry']","Consider the expression:

:(1):    p sin x + q cos x

where x ∈.

Let (1) be expressed in the form:
:(2):    R cosx + α

or:
:(3):    R sinx + α


The angle α is known as the auxiliary angle of either (2) or (3) as appropriate.
"
Definition:Bar,Bar,,Definition:Bar Chart,"['Definitions/Bar Charts', 'Definitions/Graphs (Statistics)']",
Definition:Bar,Bar,"The bar is a CGS unit of pressure.


It is defined as being:

:The amount of pressure equal to exactly $100 \, 000$ pascals.


=== Conversion Factors ===
",Definition:Bar (Unit),"['Definitions/Bar', 'Definitions/Units of Measurement', 'Definitions/CGS', 'Definitions/Pressure']","The bar is a CGS unit of pressure.


It is defined as being:

:The amount of pressure equal to exactly 100   000 pascals.


=== Conversion Factors ===
"
Definition:Bar,Bar,"A bar is a straight rigid body whose length in one dimension is such that its width in the other two dimensions is negligible.
",Definition:Bar (Mechanics),['Definitions/Ideals in Physics'],"A bar is a straight rigid body whose length in one dimension is such that its width in the other two dimensions is negligible.
"
Definition:Base,Base,"The base of a geometric figure is a specific part of that figure which is distinguished from the remainder of that figure and placed (actually or figuratively) at the bottom of a depiction or visualisation.

In some cases the base is truly qualitiatively different from the rest of the figure.

In other cases the base is selected arbitrarily as one of several parts of the figure which may equally well be so chosen.
The base of a geometric figure is a specific part of that figure which is distinguished from the remainder of that figure and placed (actually or figuratively) at the bottom of a depiction or visualisation.

In some cases the base is truly qualitiatively different from the rest of the figure.

In other cases the base is selected arbitrarily as one of several parts of the figure which may equally well be so chosen.
The base of a geometric figure is a specific part of that figure which is distinguished from the remainder of that figure and placed (actually or figuratively) at the bottom of a depiction or visualisation.

In some cases the base is truly qualitiatively different from the rest of the figure.

In other cases the base is selected arbitrarily as one of several parts of the figure which may equally well be so chosen.
",Definition:Base of Geometric Figure,['Definitions/Geometric Figures'],"The base of a geometric figure is a specific part of that figure which is distinguished from the remainder of that figure and placed (actually or figuratively) at the bottom of a depiction or visualisation.

In some cases the base is truly qualitiatively different from the rest of the figure.

In other cases the base is selected arbitrarily as one of several parts of the figure which may equally well be so chosen.
The base of a geometric figure is a specific part of that figure which is distinguished from the remainder of that figure and placed (actually or figuratively) at the bottom of a depiction or visualisation.

In some cases the base is truly qualitiatively different from the rest of the figure.

In other cases the base is selected arbitrarily as one of several parts of the figure which may equally well be so chosen.
The base of a geometric figure is a specific part of that figure which is distinguished from the remainder of that figure and placed (actually or figuratively) at the bottom of a depiction or visualisation.

In some cases the base is truly qualitiatively different from the rest of the figure.

In other cases the base is selected arbitrarily as one of several parts of the figure which may equally well be so chosen.
"
Definition:Base,Base,":


For a given triangle, one of the sides can be distinguished as being the base.

It is immaterial which is so chosen.

The usual practice is that the triangle is drawn so that the base is made horizontal, and at the bottom.

In the above diagram, it would be conventional for the side $AC$ to be identified as the base.",Definition:Triangle (Geometry)/Base,['Definitions/Triangles'],":


For a given triangle, one of the sides can be distinguished as being the base.

It is immaterial which is so chosen.

The usual practice is that the triangle is drawn so that the base is made horizontal, and at the bottom.

In the above diagram, it would be conventional for the side AC to be identified as the base."
Definition:Base,Base,":


The base of an isosceles triangle is specifically defined to be the side which is a different length from the other two.

In the above diagram, $BC$ is the base.
",Definition:Triangle (Geometry)/Isosceles/Base,['Definitions/Isosceles Triangles'],":


The base of an isosceles triangle is specifically defined to be the side which is a different length from the other two.

In the above diagram, BC is the base.
"
Definition:Base,Base,":

In a given parallelogram, one of the sides is distinguished as being the base.

It is immaterial which is so chosen, but usual practice is that it is one of the two longer sides.

In the parallelogram above, line $AB$ is considered to be the base.


Category:Definitions/Parallelograms",Definition:Quadrilateral/Parallelogram/Base,['Definitions/Parallelograms'],":

In a given parallelogram, one of the sides is distinguished as being the base.

It is immaterial which is so chosen, but usual practice is that it is one of the two longer sides.

In the parallelogram above, line AB is considered to be the base.


Category:Definitions/Parallelograms"
Definition:Base,Base,":

The base of a segment of a circle is the straight line forming one of the boundaries of the seqment.

In the above diagram, $AB$ is the base of the highlighted segment.


Category:Definitions/Segments of Circles
:

The base of a segment of a circle is the straight line forming one of the boundaries of the seqment.

In the above diagram, $AB$ is the base of the highlighted segment.


Category:Definitions/Segments of Circles
",Definition:Segment of Circle/Base,['Definitions/Segments of Circles'],":

The base of a segment of a circle is the straight line forming one of the boundaries of the seqment.

In the above diagram, AB is the base of the highlighted segment.


Category:Definitions/Segments of Circles
:

The base of a segment of a circle is the straight line forming one of the boundaries of the seqment.

In the above diagram, AB is the base of the highlighted segment.


Category:Definitions/Segments of Circles
"
Definition:Base,Base,"Consider a cone consisting of the set of all straight lines joining the boundary of a plane figure $PQR$ to a point $A$ not in the same plane of $PQR$:


:


The plane figure $PQR$ is called the base of the cone.",Definition:Cone (Geometry)/Base,['Definitions/Cones'],"Consider a cone consisting of the set of all straight lines joining the boundary of a plane figure PQR to a point A not in the same plane of PQR:


:


The plane figure PQR is called the base of the cone."
Definition:Base,Base,"

Let $\triangle AOB$ be a right-angled triangle such that $\angle AOB$ is the right angle.

Let $K$ be the right circular cone formed by the rotation of $\triangle AOB$ around $OB$.

Let $BC$ be the circle described by $B$.

The base of $K$ is the plane surface enclosed by the circle $BC$.



:



Category:Definitions/Right Circular Cones",Definition:Right Circular Cone/Base,['Definitions/Right Circular Cones'],"

Let AOB be a right-angled triangle such that ∠ AOB is the right angle.

Let K be the right circular cone formed by the rotation of AOB around OB.

Let BC be the circle described by B.

The base of K is the plane surface enclosed by the circle BC.



:



Category:Definitions/Right Circular Cones"
Definition:Base,Base,":


The polygon of a pyramid to whose vertices the apex is joined is called the base of the pyramid.

In the above diagram, $ABCDE$ is the base of the pyramid $ABCDEQ$.",Definition:Pyramid/Base,['Definitions/Pyramids'],":


The polygon of a pyramid to whose vertices the apex is joined is called the base of the pyramid.

In the above diagram, ABCDE is the base of the pyramid ABCDEQ."
Definition:Base,Base,"Let $\log_a$ denote the logarithm function on whatever domain: $\R$ or $\C$.

The constant $a$ is known as the base of the logarithm.
",Definition:Logarithm/Base,['Definitions/Logarithms'],"Let log_a denote the logarithm function on whatever domain:  or .

The constant a is known as the base of the logarithm.
"
Definition:Basis,Basis,,Definition:Basis (Linear Algebra),['Definitions/Linear Algebra'],
Definition:Basis,Basis,"Let $H$ be a Hilbert space.


A basis for $H$ is a maximal orthonormal subset of $H$.

Thus, $B$ is a basis for $H$  for all orthonormal subsets $B'$ of $H$:

:$B \subseteq B' \implies B = B'$",Definition:Basis (Hilbert Space),['Definitions/Hilbert Spaces'],"Let H be a Hilbert space.


A basis for H is a maximal orthonormal subset of H.

Thus, B is a basis for H  for all orthonormal subsets B' of H:

:B ⊆ B'  B = B'"
Definition:Below,Below,"In the context of numbers, below means less than.

Note that this applies to:
:the natural numbers $\N$
:the integers $\Z$
:the rational numbers $\Q$
:the real numbers $\R$

but specifically not the complex numbers $\C$ because the complex numbers do not have a usual ordering.
",Definition:Below (Number),['Definitions/Language Definitions'],"In the context of numbers, below means less than.

Note that this applies to:
:the natural numbers 
:the integers 
:the rational numbers 
:the real numbers 

but specifically not the complex numbers  because the complex numbers do not have a usual ordering.
"
Definition:Below,Below,"Let $a$ and $b$ be points in $3$-dimensional Euclidean space $\R^3$.

Let $P$ be an arbitrary plane embedded in $S$ be distinguished and defined as horizontal.

Then:
:$a$ is below $b$
:
:the height of $a$  $P$ is less than the height of $b$  $P$.
",Definition:Below (Solid Geometry),"['Definitions/Language Definitions', 'Definitions/Solid Geometry']","Let a and b be points in 3-dimensional Euclidean space ^3.

Let P be an arbitrary plane embedded in S be distinguished and defined as horizontal.

Then:
:a is below b
:
:the height of a  P is less than the height of b  P.
"
Definition:Below,Below,"Let $a$ and $b$ be points in the cartesian plane $\R^2$.

Then:
:$a$ is below $b$
:
:the $y$ coordinate of $a$ is less than the $y$ coordinate of $b$.
",Definition:Below (Plane Geometry),"['Definitions/Language Definitions', 'Definitions/Plane Geometry']","Let a and b be points in the cartesian plane ^2.

Then:
:a is below b
:
:the y coordinate of a is less than the y coordinate of b.
"
Definition:Bilinear Form,Bilinear Form,"Let $R$ be a ring.

Let $R_R$ denote the $R$-module $R$.

Let $M_R$ be an $R$-module.


A bilinear form on $M_R$ is a bilinear mapping $B : M_R \times M_R \to R_R$.
",Definition:Bilinear Form (Linear Algebra),"['Definitions/Bilinear Forms (Linear Algebra)', 'Definitions/Linear Forms (Linear Algebra)', 'Definitions/Linear Algebra', 'Definitions/Module Theory', 'Definitions/Vector Spaces', 'Definitions/Bilinear Forms']","Let R be a ring.

Let R_R denote the R-module R.

Let M_R be an R-module.


A bilinear form on M_R is a bilinear mapping B : M_R × M_R → R_R.
"
Definition:Bilinear Form,Bilinear Form,"A bilinear form is a linear form of order $2$.
",Definition:Bilinear Form (Polynomial Theory),"['Definitions/Bilinear Forms (Polynomial Theory)', 'Definitions/Linear Forms (Polynomial Theory)', 'Definitions/Bilinear Forms']","A bilinear form is a linear form of order 2.
"
Definition:Binomial,Binomial,"Let $a$ and $b$ be two (strictly) positive real numbers such that $a + b$ is a binomial.


Then $a + b$ is a first binomial :
: $(1): \quad a \in \Q$
: $(2): \quad \dfrac {\sqrt {a^2 - b^2} } a \in \Q$
where $\Q$ denotes the set of rational numbers.



:

Let $a$ and $b$ be two (strictly) positive real numbers such that $a + b$ is a binomial.


Then $a + b$ is a second binomial :
: $(1): \quad b \in \Q$
: $(2): \quad \dfrac {\sqrt {a^2 - b^2}} a \in \Q$
where $\Q$ denotes the set of rational numbers.



:

Let $a$ and $b$ be two (strictly) positive real numbers such that $a + b$ is a binomial.


Then $a + b$ is a third binomial :
: $(1): \quad a \notin \Q$
: $(2): \quad b \notin \Q$
: $(3): \quad \dfrac {\sqrt {a^2 - b^2}} a \in \Q$
where $\Q$ denotes the set of rational numbers.



:

Let $a$ and $b$ be two (strictly) positive real numbers such that $a + b$ is a binomial.


Then $a + b$ is a fourth binomial :
: $(1): \quad a \in \Q$
: $(2): \quad \dfrac {\sqrt {a^2 - b^2}} a \notin \Q$
where $\Q$ denotes the set of rational numbers.



:

Let $a$ and $b$ be two (strictly) positive real numbers such that $a + b$ is a binomial.


Then $a + b$ is a fifth binomial :
: $(1): \quad b \in \Q$
: $(2): \quad \dfrac {\sqrt {a^2 - b^2}} a \notin \Q$

where $\Q$ denotes the set of rational numbers.



:

Let $a$ and $b$ be two (strictly) positive real numbers such that $a + b$ is a binomial.


Then $a + b$ is a sixth binomial :
: $(1): \quad: a \notin \Q$
: $(2): \quad: b \notin \Q$
: $(3): \quad: \dfrac {\sqrt {a^2 - b^2}} a \notin \Q$
where $\Q$ denotes the set of rational numbers.



:

",Definition:Binomial (Euclidean),['Definitions/Euclidean Number Theory'],"Let a and b be two (strictly) positive real numbers such that a + b is a binomial.


Then a + b is a first binomial :
: (1):    a ∈
: (2):   √(a^2 - b^2) a ∈
where  denotes the set of rational numbers.



:

Let a and b be two (strictly) positive real numbers such that a + b is a binomial.


Then a + b is a second binomial :
: (1):    b ∈
: (2):   √(a^2 - b^2) a ∈
where  denotes the set of rational numbers.



:

Let a and b be two (strictly) positive real numbers such that a + b is a binomial.


Then a + b is a third binomial :
: (1):    a ∉
: (2):    b ∉
: (3):   √(a^2 - b^2) a ∈
where  denotes the set of rational numbers.



:

Let a and b be two (strictly) positive real numbers such that a + b is a binomial.


Then a + b is a fourth binomial :
: (1):    a ∈
: (2):   √(a^2 - b^2) a ∉
where  denotes the set of rational numbers.



:

Let a and b be two (strictly) positive real numbers such that a + b is a binomial.


Then a + b is a fifth binomial :
: (1):    b ∈
: (2):   √(a^2 - b^2) a ∉

where  denotes the set of rational numbers.



:

Let a and b be two (strictly) positive real numbers such that a + b is a binomial.


Then a + b is a sixth binomial :
: (1):   : a ∉
: (2):   : b ∉
: (3):   : √(a^2 - b^2) a ∉
where  denotes the set of rational numbers.



:

"
Definition:Binomial,Binomial,"A binomial is an expression which has $2$ terms.
",Definition:Binomial (Algebra),"['Definitions/Binomials (Algebra)', 'Definitions/Algebra']","A binomial is an expression which has 2 terms.
"
Definition:Binomial,Binomial,"Let $X$ be a discrete random variable on a probability space $\struct {\Omega, \Sigma, \Pr}$.


Then $X$ has the binomial distribution with parameters $n$ and $p$ :

:$\Img X = \set {0, 1, \ldots, n}$

:$\map \Pr {X = k} = \dbinom n k p^k \paren {1 - p}^{n - k}$

where $0 \le p \le 1$.


Note that the binomial distribution gives rise to a probability mass function satisfying $\map \Pr \Omega = 1$, because:
:$\ds \sum_{k \mathop \in \Z} \dbinom n k p^k \paren {1 - p}^{n - k} = \paren {p + \paren {1 - p} }^n = 1$

This is apparent from the Binomial Theorem.


It is written:
:$X \sim \Binomial n p$
Let $X$ be a discrete random variable on a probability space $\struct {\Omega, \Sigma, \Pr}$.


Then $X$ has the binomial distribution with parameters $n$ and $p$ :

:$\Img X = \set {0, 1, \ldots, n}$

:$\map \Pr {X = k} = \dbinom n k p^k \paren {1 - p}^{n - k}$

where $0 \le p \le 1$.


Note that the binomial distribution gives rise to a probability mass function satisfying $\map \Pr \Omega = 1$, because:
:$\ds \sum_{k \mathop \in \Z} \dbinom n k p^k \paren {1 - p}^{n - k} = \paren {p + \paren {1 - p} }^n = 1$

This is apparent from the Binomial Theorem.


It is written:
:$X \sim \Binomial n p$
",Definition:Binomial Distribution,"['Definitions/Binomial Distribution', 'Definitions/Examples of Probability Distributions']","Let X be a discrete random variable on a probability space Ω, Σ,.


Then X has the binomial distribution with parameters n and p :

:X = 0, 1, …, n

:X = k =  n k p^k 1 - p^n - k

where 0 ≤ p ≤ 1.


Note that the binomial distribution gives rise to a probability mass function satisfying Ω = 1, because:
:∑_k ∈ n k p^k 1 - p^n - k = p + 1 - p^n = 1

This is apparent from the Binomial Theorem.


It is written:
:X ∼ n p
Let X be a discrete random variable on a probability space Ω, Σ,.


Then X has the binomial distribution with parameters n and p :

:X = 0, 1, …, n

:X = k =  n k p^k 1 - p^n - k

where 0 ≤ p ≤ 1.


Note that the binomial distribution gives rise to a probability mass function satisfying Ω = 1, because:
:∑_k ∈ n k p^k 1 - p^n - k = p + 1 - p^n = 1

This is apparent from the Binomial Theorem.


It is written:
:X ∼ n p
"
Definition:Boundary,Boundary,"

For example, the endpoints of a line segment are its boundaries.


=== Containment ===
",Definition:Boundary (Geometry),['Definitions/Geometry'],"

For example, the endpoints of a line segment are its boundaries.


=== Containment ===
"
Definition:Boundary,Boundary,"=== Simple Graph ===
",Definition:Boundary (Graph Theory),"['Definitions/Graph Theory', 'Definitions/Boundaries (Graph Theory)']","=== Simple Graph ===
"
Definition:Bounded,Bounded,"Let $\struct {S, \preceq}$ be an ordered set.


A subset $T \subseteq S$ is bounded below (in $S$)  $T$ admits a lower bound (in $S$).


=== Subset of Real Numbers ===

The concept is usually encountered where $\struct {S, \preceq}$ is the set of real numbers under the usual ordering $\struct {\R, \le}$:


Let $\struct {S, \preceq}$ be an ordered set.


A subset $T \subseteq S$ is bounded above (in $S$)  $T$ admits an upper bound (in $S$).


=== Subset of Real Numbers ===

The concept is usually encountered where $\struct {S, \preceq}$ is the set of real numbers under the usual ordering $\struct {\R, \le}$:



=== Unbounded Above ===

",Definition:Bounded Set,"['Definitions/Bounded Sets', 'Definitions/Ordered Sets', 'Definitions/Boundedness']","Let S, ≼ be an ordered set.


A subset T ⊆ S is bounded below (in S)  T admits a lower bound (in S).


=== Subset of Real Numbers ===

The concept is usually encountered where S, ≼ is the set of real numbers under the usual ordering , ≤:


Let S, ≼ be an ordered set.


A subset T ⊆ S is bounded above (in S)  T admits an upper bound (in S).


=== Subset of Real Numbers ===

The concept is usually encountered where S, ≼ is the set of real numbers under the usual ordering , ≤:



=== Unbounded Above ===

"
Definition:Bounded,Bounded,"Let $B$ be a class.

Let $x$ be a set.

$B$ is bounded by $x$ :
:every element of $B$ is a subset of $x$.
Let $B$ be a class.

Let $B$ be a subclass of a class $A$.

Then $B$ is a bounded subset of $A$ :
:there exists a set $x \in A$ such that $B$ is bounded by $x$ 


That is,  every element of $B$ is a subset of $x$.
",Definition:Bounded Class,['Definitions/Bounded Classes'],"Let B be a class.

Let x be a set.

B is bounded by x :
:every element of B is a subset of x.
Let B be a class.

Let B be a subclass of a class A.

Then B is a bounded subset of A :
:there exists a set x ∈ A such that B is bounded by x 


That is,  every element of B is a subset of x.
"
Definition:Bounded,Bounded,"Let $\struct {S, \preceq}$ be an ordered set.

Let $T \subseteq S$ be both bounded below and bounded above in $S$.


Then $T$ is bounded in $S$.


=== Subset of Real Numbers ===

The concept is usually encountered where $\struct {S, \preceq}$ is the set of real numbers under the usual ordering $\struct {\R, \le}$:


Let $f: S \to T$ be a mapping whose codomain is an ordered set $\struct {T, \preceq}$.


Then $f$ is bounded above on $S$ by the upper bound $H$ :
:$\forall x \in S: \map f x \preceq H$


That is,  $f \sqbrk S = \set {\map f x: x \in S}$ is bounded above by $H$.


=== Real-Valued Function ===

The concept is usually encountered where $\struct {T, \preceq}$ is the set of real numbers under the usual ordering $\struct {\R, \le}$:


Let $f: S \to T$ be a mapping whose codomain is an ordered set $\struct {T, \preceq}$.


Then $f$ is said to be bounded below (in $T$) by the lower bound $L$ :
:$\forall x \in S: L \preceq \map f x$


That is, iff $f \sqbrk S = \set {\map f x: x \in S}$ is bounded below by $L$.


=== Real-Valued Function ===

The concept is usually encountered where $\struct {T, \preceq}$ is the set of real numbers under the usual ordering $\struct {\R, \le}$:


",Definition:Bounded Mapping,"['Definitions/Bounded Mappings', 'Definitions/Mappings', 'Definitions/Boundedness']","Let S, ≼ be an ordered set.

Let T ⊆ S be both bounded below and bounded above in S.


Then T is bounded in S.


=== Subset of Real Numbers ===

The concept is usually encountered where S, ≼ is the set of real numbers under the usual ordering , ≤:


Let f: S → T be a mapping whose codomain is an ordered set T, ≼.


Then f is bounded above on S by the upper bound H :
:∀ x ∈ S:  f x ≼ H


That is,  f  S =  f x: x ∈ S is bounded above by H.


=== Real-Valued Function ===

The concept is usually encountered where T, ≼ is the set of real numbers under the usual ordering , ≤:


Let f: S → T be a mapping whose codomain is an ordered set T, ≼.


Then f is said to be bounded below (in T) by the lower bound L :
:∀ x ∈ S: L ≼ f x


That is, iff f  S =  f x: x ∈ S is bounded below by L.


=== Real-Valued Function ===

The concept is usually encountered where T, ≼ is the set of real numbers under the usual ordering , ≤:


"
Definition:Bounded,Bounded,"Let $\sequence {x_n}$ be a real sequence.


Then $\sequence {x_n}$ is bounded above :
:$\exists M \in \R: \forall i \in \N: x_i \le M$
Let $\sequence {x_n}$ be a real sequence.


Then $\sequence {x_n}$ is bounded below :
:$\exists m \in \R: \forall i \in \N: m \le x_i$


=== Unbounded Below ===

Let $\sequence {x_n}$ be a real sequence.


$\sequence {x_n}$ is unbounded  it is not bounded.
",Definition:Bounded Sequence/Real,"['Definitions/Bounded Real Sequences', 'Definitions/Bounded Sequences', 'Definitions/Real Sequences']","Let x_n be a real sequence.


Then x_n is bounded above :
:∃ M ∈: ∀ i ∈: x_i ≤ M
Let x_n be a real sequence.


Then x_n is bounded below :
:∃ m ∈: ∀ i ∈: m ≤ x_i


=== Unbounded Below ===

Let x_n be a real sequence.


x_n is unbounded  it is not bounded.
"
Definition:Bounded,Bounded,"Let $D \subseteq \R^2$ be a subset of the plane.

$D$ is bounded  there exists a circle in the plane which completely encloses $D$.",Definition:Bounded Region of Plane,['Definitions/Geometry'],"Let D ⊆^2 be a subset of the plane.

D is bounded  there exists a circle in the plane which completely encloses D."
Definition:Bounded,Bounded,"Let $D$ be a subset of the complex plane $\C$.


Then $D$ is unbounded (in $\C$) :
: $\nexists M \in \R: \forall z \in D: \cmod z \le M$

That is, if $D$ is not bounded in $\C$.
",Definition:Bounded Metric Space/Complex,"['Definitions/Bounded Metric Spaces', 'Definitions/Complex Plane']","Let D be a subset of the complex plane .


Then D is unbounded (in ) :
: ∄ M ∈: ∀ z ∈ D:  z ≤ M

That is, if D is not bounded in .
"
Definition:Bounded,Bounded,"Let $\mathbb F \in \set {\R, \C}$.

Let $\struct {V, \tau}$ be a topological vector space over $\mathbb F$.


A subset $B \subseteq V$ is bounded :
:for each $U \in \tau$ such that $\mathbf 0_V \in U$ there is an $\epsilon \in \R_{>0}$ such that:
::$\epsilon B \subseteq U$

where:
:$\bf 0_V$ denotes the zero vector of $V$
:$\epsilon B$ denotes the dilation of $B$ by $\epsilon$",Definition:Bounded Subset of Topological Vector Space,['Definitions/Topological Vector Spaces'],"Let 𝔽∈,.

Let V, τ be a topological vector space over 𝔽.


A subset B ⊆ V is bounded :
:for each U ∈τ such that 0_V ∈ U there is an ϵ∈_>0 such that:
::ϵ B ⊆ U

where:
:0_V denotes the zero vector of V
:ϵ B denotes the dilation of B by ϵ"
Definition:Bounded,Bounded,,Definition:Bounded Above,['Definitions/Boundedness'],
Definition:Bounded Above,Bounded Above,"Let $\struct {S, \preceq}$ be an ordered set.


A subset $T \subseteq S$ is unbounded above (in $S$)  it is not bounded above.
",Definition:Bounded Above Set,"['Definitions/Bounded Above Sets', 'Definitions/Boundedness']","Let S, ≼ be an ordered set.


A subset T ⊆ S is unbounded above (in S)  it is not bounded above.
"
Definition:Bounded Above,Bounded Above,"Let $\R$ be the set of real numbers.

Let $T \subseteq \R$ be a subset of $\R$ .


$T \subseteq \R$ is unbounded above (in $\R$)  it is not bounded above.
",Definition:Bounded Above Set/Real Numbers,"['Definitions/Bounded Above Sets of Real Numbers', 'Definitions/Bounded Above Sets', 'Definitions/Real Numbers']","Let  be the set of real numbers.

Let T ⊆ be a subset of  .


T ⊆ is unbounded above (in )  it is not bounded above.
"
Definition:Bounded Above,Bounded Above,"Let $\struct {S, \preceq}$ be an ordered set.


A subset $T \subseteq S$ is bounded above (in $S$)  $T$ admits an upper bound (in $S$).


=== Subset of Real Numbers ===

The concept is usually encountered where $\struct {S, \preceq}$ is the set of real numbers under the usual ordering $\struct {\R, \le}$:



=== Unbounded Above ===

",Definition:Bounded Above Mapping,"['Definitions/Bounded Above Mappings', 'Definitions/Mappings', 'Definitions/Boundedness']","Let S, ≼ be an ordered set.


A subset T ⊆ S is bounded above (in S)  T admits an upper bound (in S).


=== Subset of Real Numbers ===

The concept is usually encountered where S, ≼ is the set of real numbers under the usual ordering , ≤:



=== Unbounded Above ===

"
Definition:Bounded Above,Bounded Above,"Let $\R$ be the set of real numbers.

A subset $T \subseteq \R$ is bounded above (in $\R$)  $T$ admits an upper bound (in $\R$).


=== Unbounded Above ===

",Definition:Bounded Above Mapping/Real-Valued,"['Definitions/Bounded Above Real-Valued Functions', 'Definitions/Real-Valued Functions', 'Definitions/Bounded Above Mappings']","Let  be the set of real numbers.

A subset T ⊆ is bounded above (in )  T admits an upper bound (in ).


=== Unbounded Above ===

"
Definition:Bounded Above,Bounded Above,"Let $\struct {T, \preceq}$ be an ordered set.

Let $\sequence {x_n}$ be a sequence in $T$.


Then $\sequence {x_n}$ is bounded above :
:$\exists M \in T: \forall i \in \N: x_i \preceq M$


=== Real Sequence ===

The concept is usually encountered where $\struct {T, \preceq}$ is the set of real numbers under the usual ordering $\struct {\R, \le}$:

",Definition:Bounded Above Sequence,"['Definitions/Bounded Above Sequences', 'Definitions/Boundedness', 'Definitions/Sequences']","Let T, ≼ be an ordered set.

Let x_n be a sequence in T.


Then x_n is bounded above :
:∃ M ∈ T: ∀ i ∈: x_i ≼ M


=== Real Sequence ===

The concept is usually encountered where T, ≼ is the set of real numbers under the usual ordering , ≤:

"
Definition:Bounded Above,Bounded Above,"Let $\sequence {x_n}$ be a real sequence.


Then $\sequence {x_n}$ is bounded above :
:$\exists M \in \R: \forall i \in \N: x_i \le M$",Definition:Bounded Above Sequence/Real,"['Definitions/Bounded Above Real Sequences', 'Definitions/Bounded Above Sequences', 'Definitions/Real Sequences']","Let x_n be a real sequence.


Then x_n is bounded above :
:∃ M ∈: ∀ i ∈: x_i ≤ M"
Definition:Bounded Above,Bounded Above,,Definition:Unbounded Above,['Definitions/Boundedness'],
Definition:Bounded Below,Bounded Below,"Let $\struct {S, \preceq}$ be an ordered set.


A subset $T \subseteq S$ is bounded below (in $S$)  $T$ admits a lower bound (in $S$).


=== Subset of Real Numbers ===

The concept is usually encountered where $\struct {S, \preceq}$ is the set of real numbers under the usual ordering $\struct {\R, \le}$:

",Definition:Bounded Below Set,"['Definitions/Bounded Below Sets', 'Definitions/Boundedness']","Let S, ≼ be an ordered set.


A subset T ⊆ S is bounded below (in S)  T admits a lower bound (in S).


=== Subset of Real Numbers ===

The concept is usually encountered where S, ≼ is the set of real numbers under the usual ordering , ≤:

"
Definition:Bounded Below,Bounded Below,"Let $\R$ be the set of real numbers.

Let $T \subseteq \R$ be a subset of $\R$ .


$T \subseteq \R$ is unbounded below (in $\R$)  it is not bounded below.
",Definition:Bounded Below Set/Real Numbers,"['Definitions/Bounded Below Sets of Real Numbers', 'Definitions/Bounded Below Sets', 'Definitions/Real Numbers']","Let  be the set of real numbers.

Let T ⊆ be a subset of  .


T ⊆ is unbounded below (in )  it is not bounded below.
"
Definition:Bounded Below,Bounded Below,"Let $\struct {S, \preceq}$ be an ordered set.


A subset $T \subseteq S$ is bounded below (in $S$)  $T$ admits a lower bound (in $S$).


=== Subset of Real Numbers ===

The concept is usually encountered where $\struct {S, \preceq}$ is the set of real numbers under the usual ordering $\struct {\R, \le}$:


",Definition:Bounded Below Mapping,"['Definitions/Bounded Below Mappings', 'Definitions/Mappings', 'Definitions/Boundedness']","Let S, ≼ be an ordered set.


A subset T ⊆ S is bounded below (in S)  T admits a lower bound (in S).


=== Subset of Real Numbers ===

The concept is usually encountered where S, ≼ is the set of real numbers under the usual ordering , ≤:


"
Definition:Bounded Below,Bounded Below,"Let $\R$ be the set of real numbers.

A subset $T \subseteq \R$ is bounded below (in $\R$)  $T$ admits a lower bound (in $\R$).


=== Unbounded Below ===

",Definition:Bounded Below Mapping/Real-Valued,"['Definitions/Bounded Below Real-Valued Functions', 'Definitions/Real-Valued Functions', 'Definitions/Bounded Below Mappings']","Let  be the set of real numbers.

A subset T ⊆ is bounded below (in )  T admits a lower bound (in ).


=== Unbounded Below ===

"
Definition:Bounded Below,Bounded Below,"Let $\struct {T, \preceq}$ be an ordered set.

Let $\sequence {x_n}$ be a sequence in $T$.


Then $\sequence {x_n}$ is bounded below :
:$\exists m \in T: \forall i \in \N: m \preceq x_i$


=== Real Sequence ===

The concept is usually encountered where $\struct {T, \preceq}$ is the set of real numbers under the usual ordering $\struct {\R, \le}$:

",Definition:Bounded Below Sequence,"['Definitions/Bounded Below Sequences', 'Definitions/Boundedness', 'Definitions/Sequences']","Let T, ≼ be an ordered set.

Let x_n be a sequence in T.


Then x_n is bounded below :
:∃ m ∈ T: ∀ i ∈: m ≼ x_i


=== Real Sequence ===

The concept is usually encountered where T, ≼ is the set of real numbers under the usual ordering , ≤:

"
Definition:Bounded Below,Bounded Below,"Let $\sequence {x_n}$ be a real sequence.


$\sequence {x_n}$ is unbounded below  there exists no $m$ in $\R$ such that:
:$\forall i \in \N: m \le x_i$


Category:Definitions/Unbounded Below Sequences
Category:Definitions/Real Sequences
",Definition:Bounded Below Sequence/Real,"['Definitions/Bounded Below Real Sequences', 'Definitions/Bounded Below Sequences', 'Definitions/Real Sequences']","Let x_n be a real sequence.


x_n is unbounded below  there exists no m in  such that:
:∀ i ∈: m ≤ x_i


Category:Definitions/Unbounded Below Sequences
Category:Definitions/Real Sequences
"
Definition:Bounded Below,Bounded Below,,Definition:Unbounded Below,['Definitions/Boundedness'],
Definition:Branch,Branch,"Let $\struct {T, \preceq}$ be a tree.

A branch of $\struct {T, \preceq}$ is a maximal chain in $\struct {T, \preceq}$.


Category:Definitions/Set Theory",Definition:Tree (Set Theory)/Branch,"['Definitions/Set Theory', 'Definitions/Set Theory']","Let T, ≼ be a tree.

A branch of T, ≼ is a maximal chain in T, ≼.


Category:Definitions/Set Theory"
Definition:Branch,Branch,"Let $U \subseteq \C$ be an open set.

Let $f : U \to \C$ be a complex multifunction.


A branch point of $f$ is a point $a$ in $U$ such that:

:$f$ has more than one value at one or more points in every neighborhood of $a$
:$f$ has exactly one value at $a$ itself.",Definition:Branch Point,"['Definitions/Branch Points', 'Definitions/Singular Points', 'Definitions/Complex Analysis']","Let U ⊆ be an open set.

Let f : U → be a complex multifunction.


A branch point of f is a point a in U such that:

:f has more than one value at one or more points in every neighborhood of a
:f has exactly one value at a itself."
Definition:Branch,Branch,A branch of a curve $\CC$ is a part of $\CC$ which is separated from another part of $\CC$ by a discontinuity or a singular point.,Definition:Branch of Curve,"['Definitions/Branches of Curves', 'Definitions/Analytic Geometry']",A branch of a curve  is a part of  which is separated from another part of  by a discontinuity or a singular point.
Definition:Bridge,Bridge,"Let $G = \struct {V, E}$ be a connected graph.

Let $e \in E$ be an edge of $G$ such that the edge deletion $G - e$ is disconnected.


Then $e$ is known as a bridge of $G$.",Definition:Bridge (Graph Theory),['Definitions/Graph Theory'],"Let G = V, E be a connected graph.

Let e ∈ E be an edge of G such that the edge deletion G - e is disconnected.


Then e is known as a bridge of G."
Definition:Bridge,Bridge,"Bridge is a game for $4$ players whose mechanism depends on the fact that each of the $4$ players are dealt a hand of $13$ cards from the standard deck of $52$ cards.


For the purposes of  at its current stage of evolution, details of the play of Bridge are not immediately relevant.

As and when a deeper analysis of Bridge becomes appropriate, further details can be incorporated.
Bridge is a game for $4$ players whose mechanism depends on the fact that each of the $4$ players are dealt a hand of $13$ cards from the standard deck of $52$ cards.


For the purposes of  at its current stage of evolution, details of the play of Bridge are not immediately relevant.

As and when a deeper analysis of Bridge becomes appropriate, further details can be incorporated.
Bridge is a game for $4$ players whose mechanism depends on the fact that each of the $4$ players are dealt a hand of $13$ cards from the standard deck of $52$ cards.


For the purposes of  at its current stage of evolution, details of the play of Bridge are not immediately relevant.

As and when a deeper analysis of Bridge becomes appropriate, further details can be incorporated.
",Definition:Bridge (Game),"['Definitions/Bridge (Game)', 'Definitions/Examples of Games']","Bridge is a game for 4 players whose mechanism depends on the fact that each of the 4 players are dealt a hand of 13 cards from the standard deck of 52 cards.


For the purposes of  at its current stage of evolution, details of the play of Bridge are not immediately relevant.

As and when a deeper analysis of Bridge becomes appropriate, further details can be incorporated.
Bridge is a game for 4 players whose mechanism depends on the fact that each of the 4 players are dealt a hand of 13 cards from the standard deck of 52 cards.


For the purposes of  at its current stage of evolution, details of the play of Bridge are not immediately relevant.

As and when a deeper analysis of Bridge becomes appropriate, further details can be incorporated.
Bridge is a game for 4 players whose mechanism depends on the fact that each of the 4 players are dealt a hand of 13 cards from the standard deck of 52 cards.


For the purposes of  at its current stage of evolution, details of the play of Bridge are not immediately relevant.

As and when a deeper analysis of Bridge becomes appropriate, further details can be incorporated.
"
Definition:Cancellable,Cancellable,"Let $\struct {S, \circ}$ be an algebraic structure.


The operation $\circ$ in $\struct {S, \circ}$ is left cancellable :
:$\forall a, b, c \in S: a \circ b = a \circ c \implies b = c$

That is,  all elements of $\struct {S, \circ}$ are left cancellable.
Let $\struct {S, \circ}$ be an algebraic structure.


The operation $\circ$ in $\struct {S, \circ}$ is right cancellable :
:$\forall a, b, c \in S: a \circ c = b \circ c \implies a = b$

That is,  all elements of $\struct {S, \circ}$ are right cancellable.
Let $\struct {S, \circ}$ be an algebraic structure.


The operation $\circ$ in $\struct {S, \circ}$ is left cancellable :
:$\forall a, b, c \in S: a \circ b = a \circ c \implies b = c$

That is,  all elements of $\struct {S, \circ}$ are left cancellable.
Let $\struct {S, \circ}$ be an algebraic structure.


The operation $\circ$ in $\struct {S, \circ}$ is right cancellable :
:$\forall a, b, c \in S: a \circ c = b \circ c \implies a = b$

That is,  all elements of $\struct {S, \circ}$ are right cancellable.
",Definition:Cancellable Operation,"['Definitions/Abstract Algebra', 'Definitions/Cancellability']","Let S, ∘ be an algebraic structure.


The operation ∘ in S, ∘ is left cancellable :
:∀ a, b, c ∈ S: a ∘ b = a ∘ c  b = c

That is,  all elements of S, ∘ are left cancellable.
Let S, ∘ be an algebraic structure.


The operation ∘ in S, ∘ is right cancellable :
:∀ a, b, c ∈ S: a ∘ c = b ∘ c  a = b

That is,  all elements of S, ∘ are right cancellable.
Let S, ∘ be an algebraic structure.


The operation ∘ in S, ∘ is left cancellable :
:∀ a, b, c ∈ S: a ∘ b = a ∘ c  b = c

That is,  all elements of S, ∘ are left cancellable.
Let S, ∘ be an algebraic structure.


The operation ∘ in S, ∘ is right cancellable :
:∀ a, b, c ∈ S: a ∘ c = b ∘ c  a = b

That is,  all elements of S, ∘ are right cancellable.
"
Definition:Cancellable,Cancellable,,Definition:Right Cancellable,['Definitions/Cancellability'],
Definition:Canonical,Canonical,"Let $r \in \Q$ be a rational number.

The canonical form of $r$ is the expression $\dfrac p q$, where:
:$r = \dfrac p q: p \in \Z, q \in \Z_{>0}, p \perp q$
where $p \perp q$ denotes that $p$ and $q$ have no common divisor except $1$.


That is, in its canonical form, $r$ is expressed as $\dfrac p q$ where:

:$p$ is an integer
:$q$ is a strictly positive integer
:$p$ and $q$ are coprime.",Definition:Rational Number/Canonical Form,"['Definitions/Canonical Form of Rational Number', 'Definitions/Rational Numbers', 'Definitions/Fractions', 'Definitions/Canonical Forms']","Let r ∈ be a rational number.

The canonical form of r is the expression p q, where:
:r =  p q: p ∈, q ∈_>0, p ⊥ q
where p ⊥ q denotes that p and q have no common divisor except 1.


That is, in its canonical form, r is expressed as p q where:

:p is an integer
:q is a strictly positive integer
:p and q are coprime."
Definition:Canonical,Canonical,"A canonical form of a mathematical object is a standard way of presenting that object as a mathematical expression.
",Definition:Quadratic Equation/Canonical Form,"['Definitions/Quadratic Equations', 'Definitions/Canonical Forms']","A canonical form of a mathematical object is a standard way of presenting that object as a mathematical expression.
"
Definition:Canonical,Canonical,,Definition:Canonical Injection,[],
Definition:Canonical,Canonical,A canonical form of a matrix is a form which all of a certain class of matrix can be reduced by transformations of a standard kind.,Definition:Canonical Form of Matrix,"['Definitions/Canonical Forms', 'Definitions/Matrices']",A canonical form of a matrix is a form which all of a certain class of matrix can be reduced by transformations of a standard kind.
Definition:Canonical,Canonical,A canonical form of a mathematical object is a standard way of presenting that object as a mathematical expression.,Definition:Canonical Form,"['Definitions/Canonical Forms', 'Definitions/Mathematics', 'Definitions/Computer Science']",A canonical form of a mathematical object is a standard way of presenting that object as a mathematical expression.
Definition:Chain,Chain,"Let $S$ be a set.

Let $\powerset S$ be its power set.

Let $N \subseteq \powerset S$ be a subset of $\powerset S$.


Then $N$ is a chain (of sets) :

:$\forall X, Y \in N: X \subseteq Y$ or $Y \subseteq X$
Let $\struct {S, \preceq}$ be an ordered set.


A chain in $S$ is a totally ordered subset of $S$.


Thus a totally ordered set is itself a chain in its own right.


=== Chain of Sets ===

An important special case of a chain is where the ordering in question is the subset relation:


",Definition:Chain (Order Theory),"['Definitions/Chains (Order Theory)', 'Definitions/Nests', 'Definitions/Order Theory']","Let S be a set.

Let S be its power set.

Let N ⊆ S be a subset of S.


Then N is a chain (of sets) :

:∀ X, Y ∈ N: X ⊆ Y or Y ⊆ X
Let S, ≼ be an ordered set.


A chain in S is a totally ordered subset of S.


Thus a totally ordered set is itself a chain in its own right.


=== Chain of Sets ===

An important special case of a chain is where the ordering in question is the subset relation:


"
Definition:Chain,Chain,"Let $S$ be a set.

Let $\powerset S$ be its power set.

Let $N \subseteq \powerset S$ be a subset of $\powerset S$.


Then $N$ is a chain (of sets) :

:$\forall X, Y \in N: X \subseteq Y$ or $Y \subseteq X$
",Definition:Chain (Order Theory)/Subset Relation,"['Definitions/Chains (Order Theory)', 'Definitions/Nests', 'Definitions/Order Theory']","Let S be a set.

Let S be its power set.

Let N ⊆ S be a subset of S.


Then N is a chain (of sets) :

:∀ X, Y ∈ N: X ⊆ Y or Y ⊆ X
"
Definition:Chain,Chain,"A chain is an inelastic thread whose stiffness and width are approximated to zero.

The mass of a chain is usually defined in terms of linear mass density.


Category:Definitions/Ideals in Physics
A chain is an inelastic thread whose stiffness and width are approximated to zero.

The mass of a chain is usually defined in terms of linear mass density.


Category:Definitions/Ideals in Physics
",Definition:Chain (Physics),['Definitions/Ideals in Physics'],"A chain is an inelastic thread whose stiffness and width are approximated to zero.

The mass of a chain is usually defined in terms of linear mass density.


Category:Definitions/Ideals in Physics
A chain is an inelastic thread whose stiffness and width are approximated to zero.

The mass of a chain is usually defined in terms of linear mass density.


Category:Definitions/Ideals in Physics
"
Definition:Chain,Chain,"The chain is an imperial unit of length.









",Definition:Imperial/Length/Chain,['Definitions/Chain (Linear Measure)'],"The chain is an imperial unit of length.









"
Definition:Chain,Chain,"Let $m$ be a positive integer.

Let $s \left({m}\right)$ be the aliquot sum of $m$.


Let a sequence $\left\langle{a_k}\right\rangle$ be a sociable chain.

The order of $a_k$ is the smallest $r \in \Z_{>0}$ such that
:$a_r = a_0$


Category:Definitions/Sociable Numbers
",Definition:Sociable Chain,"['Definitions/Number Theory', 'Definitions/Recreational Mathematics', 'Definitions/Aliquot Sequences', 'Definitions/Sociable Numbers']","Let m be a positive integer.

Let s (m) be the aliquot sum of m.


Let a sequence ⟨a_k⟩ be a sociable chain.

The order of a_k is the smallest r ∈_>0 such that
:a_r = a_0


Category:Definitions/Sociable Numbers
"
Definition:Character,Character,"Let $\struct {G, +}$ be a finite abelian group.

Let $\struct {\C_{\ne 0}, \times}$ be the multiplicative group of complex numbers.


A character of $G$ is a group homomorphism:

:$\chi: G \to \C_{\ne 0}$",Definition:Character (Number Theory),['Definitions/Analytic Number Theory'],"Let G, + be a finite abelian group.

Let _ 0, × be the multiplicative group of complex numbers.


A character of G is a group homomorphism:

:χ: G →_ 0"
Definition:Character,Character,"Let $q \in \Z_{>1}$.

Let $\paren {\Z / q \Z}$ denote the ring of integers modulo $q$.

Let $G = \paren {\Z / q \Z}^\times$ be the group of units of $\paren {\Z / q \Z}$.

Let $\C^\times$ be the group of units of $\C$.


Let $\chi_0$ be the trivial (Dirichlet) character modulo $q$.

Let $q^*$ be the least divisor of $q$ such that:
:$\chi = \chi_0 \chi^*$
where $\chi^*$ is some character modulo $q^*$.

If $q = q^*$ then $\chi$ is called primitive, otherwise $\chi$ is imprimitive.



Category:Definitions/Dirichlet Characters
",Definition:Dirichlet Character,"['Definitions/Analytic Number Theory', 'Definitions/Dirichlet Characters']","Let q ∈_>1.

Let / q denote the ring of integers modulo q.

Let G =  / q ^× be the group of units of / q.

Let ^× be the group of units of .


Let χ_0 be the trivial (Dirichlet) character modulo q.

Let q^* be the least divisor of q such that:
:χ = χ_0 χ^*
where χ^* is some character modulo q^*.

If q = q^* then χ is called primitive, otherwise χ is imprimitive.



Category:Definitions/Dirichlet Characters
"
Definition:Character,Character,"Let $p$ be an odd prime.

Let $a \in \Z$ be an integer such that $a \not \equiv 0 \pmod p$.


$a$ is either a quadratic residue or a quadratic non-residue of $p$.

Whether it is or not is known as the quadratic character of $a$ modulo $p$.
",Definition:Quadratic Residue/Character,['Definitions/Quadratic Residues'],"Let p be an odd prime.

Let a ∈ be an integer such that a ≢0  p.


a is either a quadratic residue or a quadratic non-residue of p.

Whether it is or not is known as the quadratic character of a modulo p.
"
Definition:Character,Character,"Let $\struct {G, \cdot}$ be a finite group.

Let $V$ be a finite dimensional  $k$-vector space.

Consider a linear representation $\rho: G \to \GL V$ of $G$.



Let $\map \tr {\map \rho g}$ denote the trace of $\map \rho g$.


The character associated with $\rho$ is defined as:
:$\chi: G \to k$
where $\map \chi g = \map \tr {\map \rho g}$, the trace of $\map \rho g$; which is a linear automorphism of $V$.

",Definition:Character (Representation Theory),['Definitions/Representation Theory'],"Let G, · be a finite group.

Let V be a finite dimensional  k-vector space.

Consider a linear representation ρ: G → V of G.



Let ρ g denote the trace of ρ g.


The character associated with ρ is defined as:
:χ: G → k
where χ g = ρ g, the trace of ρ g; which is a linear automorphism of V.

"
Definition:Character,Character,"Let $T$ be a topological space.

Let $x$ be a point of $T$.

Let $\map {\mathbb B} x$ be the set of all local bases at $x$.


The character of (the point) $x$ in $T$ is the smallest cardinality of the elements of $\map {\mathbb B} x$:
:$\map \chi {x, T} := \min \set {\card \BB: \BB \in \map {\mathbb B} x}$
",Definition:Character of Topological Space,['Definitions/Topology'],"Let T be a topological space.

Let x be a point of T.

Let 𝔹 x be the set of all local bases at x.


The character of (the point) x in T is the smallest cardinality of the elements of 𝔹 x:
:χx, T := min: ∈𝔹 x
"
Definition:Character,Character,"Let $T$ be a topological space.

Let $x$ be a point of $T$.

Let $\map {\mathbb B} x$ be the set of all local bases at $x$.


The character of (the point) $x$ in $T$ is the smallest cardinality of the elements of $\map {\mathbb B} x$:
:$\map \chi {x, T} := \min \set {\card \BB: \BB \in \map {\mathbb B} x}$",Definition:Character of Point in Topological Space,['Definitions/Topology'],"Let T be a topological space.

Let x be a point of T.

Let 𝔹 x be the set of all local bases at x.


The character of (the point) x in T is the smallest cardinality of the elements of 𝔹 x:
:χx, T := min: ∈𝔹 x"
Definition:Character,Character,"Let $\struct {A, \norm {\, \cdot \,} }$ be a Banach algebra over $\C$.

Let $\phi : A \to \C$ be a non-zero algebra homomorphism on $A$.


We say that $\phi$ is a character on $A$.",Definition:Character (Banach Algebra),['Definitions/Banach Algebras'],"Let A,  · be a Banach algebra over .

Let ϕ : A → be a non-zero algebra homomorphism on A.


We say that ϕ is a character on A."
Definition:Character,Character,,Definition:Characteristic,[],
Definition:Characteristic,Characteristic,"Let $G$ be a group.

Let $H$ be a subgroup such that:
:$\forall \phi \in \Aut G: \phi \sqbrk H = H$
where $\Aut G$ is the automorphism group of $G$.


Then $H$ is  characteristic (in $G$), or a characteristic subgroup of $G$.",Definition:Characteristic Subgroup,['Definitions/Subgroups'],"Let G be a group.

Let H be a subgroup such that:
:∀ϕ∈ G: ϕ H = H
where G is the automorphism group of G.


Then H is  characteristic (in G), or a characteristic subgroup of G."
Definition:Characteristic,Characteristic,"Let $R$ be a commutative ring with unity.

Let $\mathbf A$ be a square matrix over $R$ of order $n > 0$.

Let $\mathbf I_n$ be the $n \times n$ identity matrix.

Let $R \sqbrk x$ be the polynomial ring in one variable over $R$.


The characteristic matrix of $\mathbf A$ over $R \sqbrk x$ is the square matrix:
:$\mathbf I_n x - \mathbf A$",Definition:Characteristic Matrix,"['Definitions/Characteristic Matrices', 'Definitions/Matrices']","Let R be a commutative ring with unity.

Let 𝐀 be a square matrix over R of order n > 0.

Let 𝐈_n be the n × n identity matrix.

Let R  x be the polynomial ring in one variable over R.


The characteristic matrix of 𝐀 over R  x is the square matrix:
:𝐈_n x - 𝐀"
Definition:Characteristic,Characteristic,"
",Definition:Characteristic Polynomial,['Definitions/Characteristic Polynomials'],"
"
Definition:Characteristic,Characteristic,"Let $R$ be a commutative ring with unity.

Let $\mathbf A$ be a square matrix over $R$ of order $n > 0$.

Let $\mathbf I_n$ be the $n \times n$ identity matrix.

Let $R \sqbrk x$ be the polynomial ring in one variable over $R$.


The characteristic matrix of $\mathbf A$ over $R \sqbrk x$ is the square matrix:
:$\mathbf I_n x - \mathbf A$
Let $R$ be a commutative ring with unity.

Let $\mathbf A$ be a square matrix over $R$ of order $n > 0$.

Let $\mathbf I_n$ be the $n \times n$ identity matrix.

Let $R \sqbrk x$ be the polynomial ring in one variable over $R$.


The characteristic polynomial of $\mathbf A$ is the determinant of the characteristic matrix of $\mathbf A$ over $R \sqbrk x$:
:$\map {p_{\mathbf A} } x = \map \det {\mathbf I_n x - \mathbf A}$
Let $R$ be a commutative ring with unity.

Let $\mathbf A$ be a square matrix over $R$ of order $n > 0$.

Let $\mathbf I_n$ be the $n \times n$ identity matrix.

Let $R \sqbrk x$ be the polynomial ring in one variable over $R$.


The characteristic matrix of $\mathbf A$ over $R \sqbrk x$ is the square matrix:
:$\mathbf I_n x - \mathbf A$
",Definition:Characteristic Equation of Matrix,"['Definitions/Characteristic Equations', 'Definitions/Polynomial Theory', 'Definitions/Matrix Algebra', 'Definitions/Linear Algebra']","Let R be a commutative ring with unity.

Let 𝐀 be a square matrix over R of order n > 0.

Let 𝐈_n be the n × n identity matrix.

Let R  x be the polynomial ring in one variable over R.


The characteristic matrix of 𝐀 over R  x is the square matrix:
:𝐈_n x - 𝐀
Let R be a commutative ring with unity.

Let 𝐀 be a square matrix over R of order n > 0.

Let 𝐈_n be the n × n identity matrix.

Let R  x be the polynomial ring in one variable over R.


The characteristic polynomial of 𝐀 is the determinant of the characteristic matrix of 𝐀 over R  x:
:p_𝐀 x = 𝐈_n x - 𝐀
Let R be a commutative ring with unity.

Let 𝐀 be a square matrix over R of order n > 0.

Let 𝐈_n be the n × n identity matrix.

Let R  x be the polynomial ring in one variable over R.


The characteristic matrix of 𝐀 over R  x is the square matrix:
:𝐈_n x - 𝐀
"
Definition:Characteristic,Characteristic,"Let:
:$(1): \quad y + p y' + q y = 0$
be a constant coefficient homogeneous linear second order ODE.


The auxiliary equation of $(1)$ is the quadratic equation:
:$m^2 + p m + q = 0$",Definition:Auxiliary Equation,"['Definitions/Auxiliary Equations', 'Definitions/Second Order ODEs']","Let:
:(1):    y + p y' + q y = 0
be a constant coefficient homogeneous linear second order ODE.


The auxiliary equation of (1) is the quadratic equation:
:m^2 + p m + q = 0"
Definition:Characteristic,Characteristic,"Let $n \in \R$ be a positive real number such that $0 < n < 1$.

Let $n$ be presented (possibly approximated) in scientific notation as:
:$a \times 10^d$
where $d \in \Z$ is an integer.


Let $\log_{10} n$ be expressed in the form:
:$\log_{10} n = \begin {cases} c \cdotp m & : d \ge 0 \\ \overline c \cdotp m & : d < 0 \end {cases}$
where:
:$c = \size d$ is the absolute value of $d$
:$m := \log_{10} a$


$c$ is the characteristic of $\log_{10} n$.
",Definition:General Logarithm/Common/Characteristic,['Definitions/Logarithms'],"Let n ∈ be a positive real number such that 0 < n < 1.

Let n be presented (possibly approximated) in scientific notation as:
:a × 10^d
where d ∈ is an integer.


Let log_10 n be expressed in the form:
:log_10 n =  c  m     : d ≥ 0 
c m     : d < 0
where:
:c =  d is the absolute value of d
:m := log_10 a


c is the characteristic of log_10 n.
"
Definition:Characteristic Function,Characteristic Function,"Let $E \subseteq S$.

The characteristic function of $E$ is the function $\chi_E: S \to \set {0, 1}$ defined as:
:$\map {\chi_E} x = \begin {cases} 1 & : x \in E  \\  0 & : x \notin E \end {cases}$

That is:
:$\map {\chi_E} x = \begin {cases} 1 & : x \in E  \\ 0 & : x \in \relcomp S E \end {cases}$
where $\relcomp S E$ denotes the complement of $E$ relative to $S$.


=== Support ===

",Definition:Characteristic Function (Set Theory)/Set,"['Definitions/Characteristic Functions of Sets', 'Definitions/Set Theory']","Let E ⊆ S.

The characteristic function of E is the function χ_E: S →0, 1 defined as:
:χ_E x =  1     : x ∈ E  
  0     : x ∉ E

That is:
:χ_E x =  1     : x ∈ E  
 0     : x ∈ S E
where S E denotes the complement of E relative to S.


=== Support ===

"
Definition:Characteristic Function,Characteristic Function,"Let $E \subseteq S$.

The characteristic function of $E$ is the function $\chi_E: S \to \set {0, 1}$ defined as:
:$\map {\chi_E} x = \begin {cases} 1 & : x \in E  \\  0 & : x \notin E \end {cases}$

That is:
:$\map {\chi_E} x = \begin {cases} 1 & : x \in E  \\ 0 & : x \in \relcomp S E \end {cases}$
where $\relcomp S E$ denotes the complement of $E$ relative to $S$.


=== Support ===

The concept of a characteristic function of a subset carries over directly to relations.


Let $\RR \subseteq S \times T$ be a relation.

The characteristic function of $\RR$ is the mapping $\chi_\RR: S \times T \to \set {0, 1}$ defined as:
:$\map {\chi_\RR} {x, y} = \begin {cases} 1 & : \tuple {x, y} \in \RR \\ 0 & : \tuple {x, y} \notin \RR \end{cases}$


It can be expressed in Iverson bracket notation as:
:$\map {\chi_\RR} {x, y} = \sqbrk {\tuple {x, y} \in \RR}$


More generally, let $\ds \mathbb S = \prod_{i \mathop = 1}^n S_i = S_1 \times S_2 \times \ldots \times S_n$ be the cartesian product of $n$ sets $S_1, S_2, \ldots, S_n$.

Let $\RR \subseteq \mathbb S$ be an $n$-ary relation on $\mathbb S$.

The characteristic function of $\RR$ is the mapping $\chi_\RR: \mathbb S \to \set {0, 1}$ defined as:
:$\map {\chi_\RR} {s_1, s_2, \ldots, s_n} = \begin {cases} 1 & : \tuple {s_1, s_2, \ldots, s_n} \in \RR \\ 0 & : \tuple {s_1, s_2, \ldots, s_n} \notin \RR \end {cases}$


It can be expressed in Iverson bracket notation as:
:$\map {\chi_\RR} {s_1, s_2, \ldots, s_n} = \sqbrk {\tuple {s_1, s_2, \ldots, s_n} \in \RR}$
The concept of a characteristic function of a subset carries over directly to relations.


Let $\RR \subseteq S \times T$ be a relation.

The characteristic function of $\RR$ is the mapping $\chi_\RR: S \times T \to \set {0, 1}$ defined as:
:$\map {\chi_\RR} {x, y} = \begin {cases} 1 & : \tuple {x, y} \in \RR \\ 0 & : \tuple {x, y} \notin \RR \end{cases}$


It can be expressed in Iverson bracket notation as:
:$\map {\chi_\RR} {x, y} = \sqbrk {\tuple {x, y} \in \RR}$


More generally, let $\ds \mathbb S = \prod_{i \mathop = 1}^n S_i = S_1 \times S_2 \times \ldots \times S_n$ be the cartesian product of $n$ sets $S_1, S_2, \ldots, S_n$.

Let $\RR \subseteq \mathbb S$ be an $n$-ary relation on $\mathbb S$.

The characteristic function of $\RR$ is the mapping $\chi_\RR: \mathbb S \to \set {0, 1}$ defined as:
:$\map {\chi_\RR} {s_1, s_2, \ldots, s_n} = \begin {cases} 1 & : \tuple {s_1, s_2, \ldots, s_n} \in \RR \\ 0 & : \tuple {s_1, s_2, \ldots, s_n} \notin \RR \end {cases}$


It can be expressed in Iverson bracket notation as:
:$\map {\chi_\RR} {s_1, s_2, \ldots, s_n} = \sqbrk {\tuple {s_1, s_2, \ldots, s_n} \in \RR}$
",Definition:Characteristic Function (Set Theory)/Relation,"['Definitions/Characteristic Functions of Sets', 'Definitions/Relation Theory']","Let E ⊆ S.

The characteristic function of E is the function χ_E: S →0, 1 defined as:
:χ_E x =  1     : x ∈ E  
  0     : x ∉ E

That is:
:χ_E x =  1     : x ∈ E  
 0     : x ∈ S E
where S E denotes the complement of E relative to S.


=== Support ===

The concept of a characteristic function of a subset carries over directly to relations.


Let ⊆ S × T be a relation.

The characteristic function of  is the mapping χ_: S × T →0, 1 defined as:
:χ_x, y =  1     : x, y∈
 0     : x, y∉


It can be expressed in Iverson bracket notation as:
:χ_x, y = x, y∈


More generally, let 𝕊 = ∏_i  = 1^n S_i = S_1 × S_2 ×…× S_n be the cartesian product of n sets S_1, S_2, …, S_n.

Let ⊆𝕊 be an n-ary relation on 𝕊.

The characteristic function of  is the mapping χ_: 𝕊→0, 1 defined as:
:χ_s_1, s_2, …, s_n =  1     : s_1, s_2, …, s_n∈
 0     : s_1, s_2, …, s_n∉


It can be expressed in Iverson bracket notation as:
:χ_s_1, s_2, …, s_n = s_1, s_2, …, s_n∈
The concept of a characteristic function of a subset carries over directly to relations.


Let ⊆ S × T be a relation.

The characteristic function of  is the mapping χ_: S × T →0, 1 defined as:
:χ_x, y =  1     : x, y∈
 0     : x, y∉


It can be expressed in Iverson bracket notation as:
:χ_x, y = x, y∈


More generally, let 𝕊 = ∏_i  = 1^n S_i = S_1 × S_2 ×…× S_n be the cartesian product of n sets S_1, S_2, …, S_n.

Let ⊆𝕊 be an n-ary relation on 𝕊.

The characteristic function of  is the mapping χ_: 𝕊→0, 1 defined as:
:χ_s_1, s_2, …, s_n =  1     : s_1, s_2, …, s_n∈
 0     : s_1, s_2, …, s_n∉


It can be expressed in Iverson bracket notation as:
:χ_s_1, s_2, …, s_n = s_1, s_2, …, s_n∈
"
Definition:Characteristic Function,Characteristic Function,"Let $\struct {\Omega, \Sigma, \Pr}$ be a probability space.

Let $X$ be a real-valued random variable on $\struct {\Omega, \Sigma, \Pr}$.


The characteristic function of $X$ is the mapping $\phi: \R \to \C$ defined by:

:$\map \phi t = \expect {e^{i t X} }$

where:
:$i$ is the imaginary unit
:$\expect \cdot$ denotes expectation.",Definition:Characteristic Function of Random Variable,"['Definitions/Characteristic Functions of Random Variables', 'Definitions/Random Variables', 'Definitions/Probability Theory', 'Definitions/Characteristic Functions']","Let Ω, Σ, be a probability space.

Let X be a real-valued random variable on Ω, Σ,.


The characteristic function of X is the mapping ϕ: → defined by:

:ϕ t = e^i t X

where:
:i is the imaginary unit
:· denotes expectation."
Definition:Cipher,Cipher,"Let $n$ be a number expressed in a particular number base, $b$ for example.

Then $n$ can be expressed as:

:$\sqbrk {r_m r_{m - 1} \ldots r_2 r_1 r_0 . r_{-1} r_{-2} \ldots}_b$

where:
:$m$ is such that $b^m \le n < b^{m + 1}$;
:all the $r_i$ are such that $0 \le r_i < b$.

Each of the $r_i$ are known as the digits of $n$ (base $b$).


It is taken for granted that for base $10$ working, the digits are elements of the set of Arabic numerals: $\set {0, 1, 2, 3, 4, 5, 6, 7, 8, 9}$.",Definition:Digit,"['Definitions/Digits', 'Definitions/Numbers']","Let n be a number expressed in a particular number base, b for example.

Then n can be expressed as:

:r_m r_m - 1… r_2 r_1 r_0 . r_-1 r_-2…_b

where:
:m is such that b^m ≤ n < b^m + 1;
:all the r_i are such that 0 ≤ r_i < b.

Each of the r_i are known as the digits of n (base b).


It is taken for granted that for base 10 working, the digits are elements of the set of Arabic numerals: 0, 1, 2, 3, 4, 5, 6, 7, 8, 9."
Definition:Cipher,Cipher,"The word cipher is an archaic word meaning to calculate using the technique of algorism, that is, using digits as opposed to using an abacus.",Definition:Cipher (Algorism),['Definitions/Algorism'],"The word cipher is an archaic word meaning to calculate using the technique of algorism, that is, using digits as opposed to using an abacus."
Definition:Circuit,Circuit,"A circuit is a closed trail with at least one edge.


=== Subgraph ===

",Definition:Circuit (Graph Theory),"['Definitions/Circuits (Graph Theory)', 'Definitions/Graph Theory']","A circuit is a closed trail with at least one edge.


=== Subgraph ===

"
Definition:Circuit,Circuit,"Let $M = \struct {S, \mathscr I}$ be a matroid.


A circuit of $M$ is a dependent subset of $S$ which is a minimal dependent subset with respect to the subset ordering.
",Definition:Circuit (Matroid),['Definitions/Matroid Theory'],"Let M = S, ℐ be a matroid.


A circuit of M is a dependent subset of S which is a minimal dependent subset with respect to the subset ordering.
"
Definition:Circuit,Circuit,"An electric circuit is a configuration of electrical components whose collective properties can be modelled by means of a network each of whose edges corresponds to a specific component.

Its purpose is to guide and direct energy within a device.",Definition:Electric Circuit,['Definitions/Electronics'],"An electric circuit is a configuration of electrical components whose collective properties can be modelled by means of a network each of whose edges corresponds to a specific component.

Its purpose is to guide and direct energy within a device."
Definition:Circumference,Circumference,"The convex circumference of a circle $C$ is the circumference $C$ from a point outside $C$.
The concave circumference of a circle $C$ is the circumference $C$ from a point inside $C$.
",Definition:Circle/Circumference,"['Definitions/Circles', 'Definitions/Circumference of Geometric Figure']","The convex circumference of a circle C is the circumference C from a point outside C.
The concave circumference of a circle C is the circumference C from a point inside C.
"
Definition:Circumference,Circumference,"Let $G$ be a graph.

The circumference of $G$ is the longest length of any cycle in $G$.


An acyclic graph is defined as having a circumference of infinity.

Category:Definitions/Graph Theory",Definition:Circumference (Graph Theory),['Definitions/Graph Theory'],"Let G be a graph.

The circumference of G is the longest length of any cycle in G.


An acyclic graph is defined as having a circumference of infinity.

Category:Definitions/Graph Theory"
Definition:Class,Class,"A class is a collection of all sets such that a particular condition holds.


In class-builder notation, this is written as:

:$\set {x: \map p x}$

where $\map p x$ is a statement containing $x$ as a free variable.  

This is read:
:All $x$ such that $\map p x$ holds.
",Definition:Class Theory,"['Definitions/Branches of Mathematics', 'Definitions/Class Theory', 'Definitions/Set Theory']","A class is a collection of all sets such that a particular condition holds.


In class-builder notation, this is written as:

:x:  p x

where p x is a statement containing x as a free variable.  

This is read:
:All x such that p x holds.
"
Definition:Class,Class,"The class boundaries of a class interval are the endpoints of the integer interval or real interval which defines the class interval.
A class mark is a value within a class interval used to identify that class interval uniquely.

It is usual to use the midpoint.


=== Class Midpoint ===

A class interval is empty  it is of frequency zero.

",Definition:Class Interval,"['Definitions/Class Intervals', 'Definitions/Descriptive Statistics']","The class boundaries of a class interval are the endpoints of the integer interval or real interval which defines the class interval.
A class mark is a value within a class interval used to identify that class interval uniquely.

It is usual to use the midpoint.


=== Class Midpoint ===

A class interval is empty  it is of frequency zero.

"
Definition:Class,Class,"The class of continuous real functions is often denoted $C$.

Hence:
:$\map f x \in C$ at $a$
:
:$\ds \lim_{x \mathop \to a} \map f x = \map f a$
",Definition:Differentiability Class,"['Definitions/Differential Calculus', 'Definitions/Continuity', 'Definitions/Differentiability Classes']","The class of continuous real functions is often denoted C.

Hence:
:f x ∈ C at a
:
:lim_x → a f x =  f a
"
Definition:Class,Class,"Let $X$ and $Y$ be topological spaces.

Let $K \subseteq X$ be any subset.

Let $f : X \to Y$ be a continuous mapping.

The $K$-homotopy class of $f$ is the equivalence class of $f$ under the equivalence relation defined by homotopy relative to $K$.


=== Homotopy Class of Path ===

Let $S$ be a set.

Let $\RR \subseteq S \times S$ be an equivalence relation on $S$.

Let $x \in S$.


Then the equivalence class of $x$ under $\RR$ is the set:
:$\eqclass x \RR = \set {y \in S: \tuple {x, y} \in \RR}$


If $\RR$ is an equivalence on $S$, then each $t \in S$ that satisfies $\tuple {x, t} \in \RR$ (or $\tuple {t, x} \in \RR$) is called a $\RR$-relative of $x$.


That is, the equivalence class of $x$ under $\RR$ is the set of all $\RR$-relatives of $x$.


=== Representative of Equivalence Class ===

Let $T = \struct {S, \tau}$ be a topological space.

Let $f: \closedint 0 1 \to S$ be a path in $T$.


The homotopy class of the path $f$ is the homotopy class of $f$ relative to $\set {0, 1}$.


That is, the equivalence class of $f$ under the equivalence relation defined by path-homotopy.
",Definition:Homotopy Class,"['Definitions/Homotopy Classes', 'Definitions/Homotopy Theory', 'Definitions/Algebraic Topology']","Let X and Y be topological spaces.

Let K ⊆ X be any subset.

Let f : X → Y be a continuous mapping.

The K-homotopy class of f is the equivalence class of f under the equivalence relation defined by homotopy relative to K.


=== Homotopy Class of Path ===

Let S be a set.

Let ⊆ S × S be an equivalence relation on S.

Let x ∈ S.


Then the equivalence class of x under  is the set:
:x  = y ∈ S: x, y∈


If  is an equivalence on S, then each t ∈ S that satisfies x, t∈ (or t, x∈) is called a -relative of x.


That is, the equivalence class of x under  is the set of all -relatives of x.


=== Representative of Equivalence Class ===

Let T = S, τ be a topological space.

Let f:  0 1 → S be a path in T.


The homotopy class of the path f is the homotopy class of f relative to 0, 1.


That is, the equivalence class of f under the equivalence relation defined by path-homotopy.
"
Definition:Class,Class,"Let $S$ be a set.

Let $\RR \subseteq S \times S$ be an equivalence relation on $S$.

Let $x \in S$.


Then the equivalence class of $x$ under $\RR$ is the set:
:$\eqclass x \RR = \set {y \in S: \tuple {x, y} \in \RR}$


If $\RR$ is an equivalence on $S$, then each $t \in S$ that satisfies $\tuple {x, t} \in \RR$ (or $\tuple {t, x} \in \RR$) is called a $\RR$-relative of $x$.


That is, the equivalence class of $x$ under $\RR$ is the set of all $\RR$-relatives of $x$.


=== Representative of Equivalence Class ===

Let $S$ be a set.

Let $\RR \subseteq S \times S$ be an equivalence relation on $S$.

Let $x \in S$.


Then the equivalence class of $x$ under $\RR$ is the set:
:$\eqclass x \RR = \set {y \in S: \tuple {x, y} \in \RR}$


If $\RR$ is an equivalence on $S$, then each $t \in S$ that satisfies $\tuple {x, t} \in \RR$ (or $\tuple {t, x} \in \RR$) is called a $\RR$-relative of $x$.


That is, the equivalence class of $x$ under $\RR$ is the set of all $\RR$-relatives of $x$.


=== Representative of Equivalence Class ===

Let $S$ be a set.

Let $\RR \subseteq S \times S$ be an equivalence relation on $S$.

Let $x \in S$.


Let $\eqclass x \RR$ be the equivalence class of $x$ under $\RR$.

Let $y \in \eqclass x \RR$.

Then $y$ is a representative of $\eqclass x \RR$.
",Definition:Equivalence Class,"['Definitions/Equivalence Classes', 'Definitions/Equivalence Relations']","Let S be a set.

Let ⊆ S × S be an equivalence relation on S.

Let x ∈ S.


Then the equivalence class of x under  is the set:
:x  = y ∈ S: x, y∈


If  is an equivalence on S, then each t ∈ S that satisfies x, t∈ (or t, x∈) is called a -relative of x.


That is, the equivalence class of x under  is the set of all -relatives of x.


=== Representative of Equivalence Class ===

Let S be a set.

Let ⊆ S × S be an equivalence relation on S.

Let x ∈ S.


Then the equivalence class of x under  is the set:
:x  = y ∈ S: x, y∈


If  is an equivalence on S, then each t ∈ S that satisfies x, t∈ (or t, x∈) is called a -relative of x.


That is, the equivalence class of x under  is the set of all -relatives of x.


=== Representative of Equivalence Class ===

Let S be a set.

Let ⊆ S × S be an equivalence relation on S.

Let x ∈ S.


Let x be the equivalence class of x under .

Let y ∈ x.

Then y is a representative of x.
"
Definition:Class,Class,"Let $S$ be a set.

Let $\RR \subseteq S \times S$ be an equivalence relation on $S$.

Let $x \in S$.


Then the equivalence class of $x$ under $\RR$ is the set:
:$\eqclass x \RR = \set {y \in S: \tuple {x, y} \in \RR}$


If $\RR$ is an equivalence on $S$, then each $t \in S$ that satisfies $\tuple {x, t} \in \RR$ (or $\tuple {t, x} \in \RR$) is called a $\RR$-relative of $x$.


That is, the equivalence class of $x$ under $\RR$ is the set of all $\RR$-relatives of $x$.


=== Representative of Equivalence Class ===

",Definition:Conjugacy Class,['Definitions/Conjugacy'],"Let S be a set.

Let ⊆ S × S be an equivalence relation on S.

Let x ∈ S.


Then the equivalence class of x under  is the set:
:x  = y ∈ S: x, y∈


If  is an equivalence on S, then each t ∈ S that satisfies x, t∈ (or t, x∈) is called a -relative of x.


That is, the equivalence class of x under  is the set of all -relatives of x.


=== Representative of Equivalence Class ===

"
Definition:Class,Class,,Definition:NP Complexity Class,"['Definitions/Mathematical Logic', 'Definitions/Computability Theory']",
Definition:Closed,Closed,"Let $P$ be a statement.

$P$ is a closed statement  $P$ contains only bound occurrences of any variables that may appear in it.


That is, such that it contains no free occurrences of variables.",Definition:Closed Statement,['Definitions/Predicate Logic'],"Let P be a statement.

P is a closed statement  P contains only bound occurrences of any variables that may appear in it.


That is, such that it contains no free occurrences of variables."
Definition:Closed,Closed,"Let $T$ be a topological space.

Let $A \subseteq T$.


Then $A$ is regular closed in $T$ :
:$A = A^{\circ -}$

That is,  $A$ equals the closure of its interior.",Definition:Regular Closed Set,['Definitions/Topology'],"Let T be a topological space.

Let A ⊆ T.


Then A is regular closed in T :
:A = A^∘ -

That is,  A equals the closure of its interior."
Definition:Closed,Closed,"Let $T = \struct {S, \tau}$ be a topological space.

Let $p$ be a new element for $S$ such that $S^*_p := S \cup \set p$.


Let $\tau^*_p$ be the set defined as:
:$\tau^*_p := \set {U \cup \set p: U \in \tau} \cup \set \O$

That is, $\tau^*_p$ is the set of all sets formed by adding $p$ to all the open sets of $\tau$ and including the empty set.


Then:
:$\tau^*_p$ is the closed extension topology of $\tau$
and:
:$T^*_p := \struct {S^*_p, \tau^*_p}$ is the closed extension space of $T = \struct {S, \tau}$.
Let $T = \struct {S, \tau}$ be a topological space.

Let $p$ be a new element for $S$ such that $S^*_p := S \cup \set p$.


Let $\tau^*_p$ be the set defined as:
:$\tau^*_p := \set {U \cup \set p: U \in \tau} \cup \set \O$

That is, $\tau^*_p$ is the set of all sets formed by adding $p$ to all the open sets of $\tau$ and including the empty set.


Then:
:$\tau^*_p$ is the closed extension topology of $\tau$
and:
:$T^*_p := \struct {S^*_p, \tau^*_p}$ is the closed extension space of $T = \struct {S, \tau}$.
",Definition:Closed Extension Topology,['Definitions/Examples of Topologies'],"Let T = S, τ be a topological space.

Let p be a new element for S such that S^*_p := S ∪ p.


Let τ^*_p be the set defined as:
:τ^*_p := U ∪ p: U ∈τ∪Ø

That is, τ^*_p is the set of all sets formed by adding p to all the open sets of τ and including the empty set.


Then:
:τ^*_p is the closed extension topology of τ
and:
:T^*_p := S^*_p, τ^*_p is the closed extension space of T = S, τ.
Let T = S, τ be a topological space.

Let p be a new element for S such that S^*_p := S ∪ p.


Let τ^*_p be the set defined as:
:τ^*_p := U ∪ p: U ∈τ∪Ø

That is, τ^*_p is the set of all sets formed by adding p to all the open sets of τ and including the empty set.


Then:
:τ^*_p is the closed extension topology of τ
and:
:T^*_p := S^*_p, τ^*_p is the closed extension space of T = S, τ.
"
Definition:Closed,Closed,"A closed region is a region complete with its boundary.


Category:Definitions/Analytic Geometry
",Definition:Closed Region,"['Definitions/Complex Analysis', 'Definitions/Analytic Geometry']","A closed region is a region complete with its boundary.


Category:Definitions/Analytic Geometry
"
Definition:Closed,Closed,"Let $a, b \in \R$.

The closed (real) interval from $a$ to $b$ is defined as:

:$\closedint a b = \set {x \in \R: a \le x \le b}$
",Definition:Real Interval/Closed,['Definitions/Real Intervals'],"Let a, b ∈.

The closed (real) interval from a to b is defined as:

:a b = x ∈: a ≤ x ≤ b
"
Definition:Closed,Closed,"A closed walk is a walk whose first vertex is the same as the last.

That is, it is a walk which ends where it starts.",Definition:Walk (Graph Theory)/Closed,['Definitions/Walks'],"A closed walk is a walk whose first vertex is the same as the last.

That is, it is a walk which ends where it starts."
Definition:Closed,Closed,"A closed walk is a walk whose first vertex is the same as the last.

That is, it is a walk which ends where it starts.
",Definition:Circuit (Graph Theory),"['Definitions/Circuits (Graph Theory)', 'Definitions/Graph Theory']","A closed walk is a walk whose first vertex is the same as the last.

That is, it is a walk which ends where it starts.
"
Definition:Closed,Closed,"Let $\struct {S, \circ}$ be an algebraic structure.


Then $S$ has the property of closure under $\circ$ :

:$\forall \tuple {x, y} \in S \times S: x \circ y \in S$


$S$ is said to be closed under $\circ$, or just that $\struct {S, \circ}$ is closed.",Definition:Closure (Abstract Algebra)/Algebraic Structure,['Definitions/Algebraic Closure'],"Let S, ∘ be an algebraic structure.


Then S has the property of closure under ∘ :

:∀x, y∈ S × S: x ∘ y ∈ S


S is said to be closed under ∘, or just that S, ∘ is closed."
Definition:Closed,Closed,"Let $\struct {S, \circ}$ be an algebraic structure.


Then $S$ has the property of closure under $\circ$ :

:$\forall \tuple {x, y} \in S \times S: x \circ y \in S$


$S$ is said to be closed under $\circ$, or just that $\struct {S, \circ}$ is closed.
",Definition:Closure (Abstract Algebra)/Scalar Product,"['Definitions/Linear Algebra', 'Definitions/Vector Algebra']","Let S, ∘ be an algebraic structure.


Then S has the property of closure under ∘ :

:∀x, y∈ S × S: x ∘ y ∈ S


S is said to be closed under ∘, or just that S, ∘ is closed.
"
Definition:Closed,Closed,"Let $\struct {A, +, \circ}$ be a ring with unity $1_A$ and zero $0_A$.

Let $S \subseteq A$ be a subset.


Then $S$ is multiplicatively closed :

:$(1): \quad 1_A \in S$
:$(2): \quad x, y \in S \implies x \circ y \in S$",Definition:Multiplicatively Closed Subset of Ring,['Definitions/Localization of Rings'],"Let A, +, ∘ be a ring with unity 1_A and zero 0_A.

Let S ⊆ A be a subset.


Then S is multiplicatively closed :

:(1):    1_A ∈ S
:(2):    x, y ∈ S  x ∘ y ∈ S"
Definition:Closure,Closure,"Let $S$ be a set.

Let $\powerset S$ denote the power set of $S$.


A closure operator on $S$ is a mapping:
:$\cl: \powerset S \to \powerset S$
which satisfies the closure axioms as follows for all sets $X, Y \subseteq S$:

",Definition:Closure of Set under Closure Operator,['Definitions/Closure Operators'],"Let S be a set.

Let S denote the power set of S.


A closure operator on S is a mapping:
::  S → S
which satisfies the closure axioms as follows for all sets X, Y ⊆ S:

"
Definition:Closure,Closure,"Let $\struct {S, \preccurlyeq}$ be an ordered set.

Let $a \in S$.


The upper closure of $a$ (in $S$) is defined as:

:$a^\succcurlyeq := \set {b \in S: a \preccurlyeq b}$


That is, $a^\succcurlyeq$ is the set of all elements of $S$ that succeed $a$.
Let $\struct {S, \preceq}$ be an ordered set or preordered set.

Let $T \subseteq S$.


The upper closure of $T$ (in $S$) is defined as:

:$T^\succeq := \bigcup \set {t^\succeq: t \in T}$
where $t^\succeq$ denotes the upper closure of $t$ in $S$.

That is:
:$T^\succeq := \set {u \in S: \exists t \in T: t \preceq u}$
",Definition:Upper Closure,"['Definitions/Order Theory', 'Definitions/Upper Closures']","Let S, ≼ be an ordered set.

Let a ∈ S.


The upper closure of a (in S) is defined as:

:a^≽ := b ∈ S: a ≼ b


That is, a^≽ is the set of all elements of S that succeed a.
Let S, ≼ be an ordered set or preordered set.

Let T ⊆ S.


The upper closure of T (in S) is defined as:

:T^≽ := ⋃t^≽: t ∈ T
where t^≽ denotes the upper closure of t in S.

That is:
:T^≽ := u ∈ S: ∃ t ∈ T: t ≼ u
"
Definition:Closure,Closure,"Let $\struct {S, \preccurlyeq}$ be an ordered set.

Let $a \in S$.


The lower closure of $a$ (in $S$) is defined as:

:$a^\preccurlyeq := \set {b \in S: b \preccurlyeq a}$


That is, $a^\preccurlyeq$ is the set of all elements of $S$ that precede $a$.


=== Class Theory ===


Let $\struct {S, \preccurlyeq}$ be an ordered set or preordered set.

Let $T \subseteq S$.


The lower closure of $T$ (in $S$) is defined as:

:$T^\preccurlyeq := \bigcup \set {t^\preccurlyeq: t \in T}$
where $t^\preccurlyeq$ is the lower closure of $t$.

That is:
:$T^\preccurlyeq := \set {l \in S: \exists t \in T: l \preccurlyeq t}$
",Definition:Lower Closure,"['Definitions/Order Theory', 'Definitions/Lower Closures']","Let S, ≼ be an ordered set.

Let a ∈ S.


The lower closure of a (in S) is defined as:

:a^≼ := b ∈ S: b ≼ a


That is, a^≼ is the set of all elements of S that precede a.


=== Class Theory ===


Let S, ≼ be an ordered set or preordered set.

Let T ⊆ S.


The lower closure of T (in S) is defined as:

:T^≼ := ⋃t^≼: t ∈ T
where t^≼ is the lower closure of t.

That is:
:T^≼ := l ∈ S: ∃ t ∈ T: l ≼ t
"
Definition:Closure,Closure,"Let $\struct {S, \circ}$ be an algebraic structure.


Then $S$ has the property of closure under $\circ$ :

:$\forall \tuple {x, y} \in S \times S: x \circ y \in S$


$S$ is said to be closed under $\circ$, or just that $\struct {S, \circ}$ is closed.",Definition:Closure (Abstract Algebra)/Algebraic Structure,['Definitions/Algebraic Closure'],"Let S, ∘ be an algebraic structure.


Then S has the property of closure under ∘ :

:∀x, y∈ S × S: x ∘ y ∈ S


S is said to be closed under ∘, or just that S, ∘ is closed."
Definition:Closure,Closure,"Let $A$ be an extension of a commutative ring with unity $R$.


Let $C$ be the set of all elements of $A$ that are integral over $R$.

Then $C$ is called the integral closure of $R$ in $A$.",Definition:Integral Closure,"['Definitions/Algebraic Number Theory', 'Definitions/Commutative Algebra']","Let A be an extension of a commutative ring with unity R.


Let C be the set of all elements of A that are integral over R.

Then C is called the integral closure of R in A."
Definition:Closure,Closure,"Let $M = \struct {A, d}$ be a metric space.

Let $H \subseteq A$.

Let $H'$ be the set of limit points of $H$.

Let $H^i$ be the set of isolated points of $H$.


The closure of $H$ (in $M$) is the union of all isolated points of $H$ and all limit points of $H$:
:$H^- := H' \cup H^i$",Definition:Closure (Topology)/Metric Space,"['Definitions/Metric Spaces', 'Definitions/Set Closures']","Let M = A, d be a metric space.

Let H ⊆ A.

Let H' be the set of limit points of H.

Let H^i be the set of isolated points of H.


The closure of H (in M) is the union of all isolated points of H and all limit points of H:
:H^- := H' ∪ H^i"
Definition:Closure,Closure,"Let $M = \struct {X, \norm {\, \cdot \,} }$ be a normed vector space.

Let $S \subseteq X$.


The closure of $S$ (in $M$) is the union of $S$ and $S'$, the set of all limit points of $S$:
:$S^- := S \cup S'$",Definition:Closure/Normed Vector Space,"['Definitions/Set Closures', 'Definitions/Normed Vector Spaces']","Let M = X,  · be a normed vector space.

Let S ⊆ X.


The closure of S (in M) is the union of S and S', the set of all limit points of S:
:S^- := S ∪ S'"
Definition:Closure,Closure,"=== Definition 1 ===


The following is not equivalent to the above, but they are almost the same.

=== Definition 2 ===
",Definition:Transitive Closure (Set Theory),['Definitions/Relational Closures'],"=== Definition 1 ===


The following is not equivalent to the above, but they are almost the same.

=== Definition 2 ===
"
Definition:Column,Column,"A row of a truth table is one of the vertical lines headed by a statement form presenting all the truth values that the statement form takes.

Each entry in the column corresponds to one specific combination of truth values taken by the propositional variables that the statement form comprises.",Definition:Truth Table/Column,['Definitions/Truth Tables'],"A row of a truth table is one of the vertical lines headed by a statement form presenting all the truth values that the statement form takes.

Each entry in the column corresponds to one specific combination of truth values taken by the propositional variables that the statement form comprises."
Definition:Column,Column,"A column matrix is an $m \times 1$ matrix:

:$\mathbf C = \begin {bmatrix} c_{1 1} \\ c_{2 1} \\ \vdots \\ c_{m 1} \end {bmatrix}$


That is, it is a matrix with only one column.
",Definition:Matrix/Column,['Definitions/Matrices'],"A column matrix is an m × 1 matrix:

:𝐂 = [ c_1 1; c_2 1;     ⋮; c_m 1 ]


That is, it is a matrix with only one column.
"
Definition:Column,Column,"Let $\mathbf L$ be a Latin square.

The columns of $\mathbf L$ are the lines of elements reading down the page.",Definition:Latin Square/Column,['Definitions/Latin Squares'],"Let 𝐋 be a Latin square.

The columns of 𝐋 are the lines of elements reading down the page."
Definition:Common,Common,"Logarithms base $10$ are often referred to as common logarithms.


=== Notation for Negative Logarithm ===

",Definition:General Logarithm/Common,['Definitions/Logarithms'],"Logarithms base 10 are often referred to as common logarithms.


=== Notation for Negative Logarithm ===

"
Definition:Common,Common,"Let $x_1, x_2, \ldots, x_n \in \R$ be real numbers.

The arithmetic mean of $x_1, x_2, \ldots, x_n$ is defined as:

:$\ds A_n := \dfrac 1 n \sum_{k \mathop = 1}^n x_k$

That is, to find out the arithmetic mean of a set of numbers, add them all up and divide by how many there are.",Definition:Arithmetic Mean,"['Definitions/Arithmetic Mean', 'Definitions/Pythagorean Means', 'Definitions/Algebra', 'Definitions/Measures of Central Tendency']","Let x_1, x_2, …, x_n ∈ be real numbers.

The arithmetic mean of x_1, x_2, …, x_n is defined as:

:A_n :=  1 n ∑_k  = 1^n x_k

That is, to find out the arithmetic mean of a set of numbers, add them all up and divide by how many there are."
Definition:Common,Common,"Let $\dfrac a b$ and $\dfrac c d$ be fractions.

The lowest common denominator of $\dfrac a b$ and $\dfrac c d$ is the lowest common multiple of the denominators of $\dfrac a b$ and $\dfrac c d$:
:$\lcm \set {b, d}$
",Definition:Common Denominator,"['Definitions/Common Denominators', 'Definitions/Proof Techniques', 'Definitions/Arithmetic', 'Definitions/Algebra']","Let a b and c d be fractions.

The lowest common denominator of a b and c d is the lowest common multiple of the denominators of a b and c d:
:b, d
"
Definition:Common,Common,"Let $S$ be a finite set of real numbers, that is:

:$S = \set {x_1, x_2, \ldots, x_n: \forall k \in \N^*_n: x_k \in \R}$


Let $c \in \R$ such that $c$ divides all the elements of $S$, that is:

:$\forall x \in S: c \divides x$


Then $c$ is a common divisor of all the elements in $S$.",Definition:Common Divisor/Real Numbers,"['Definitions/Common Divisors', 'Definitions/Real Analysis']","Let S be a finite set of real numbers, that is:

:S = x_1, x_2, …, x_n: ∀ k ∈^*_n: x_k ∈


Let c ∈ such that c divides all the elements of S, that is:

:∀ x ∈ S: c  x


Then c is a common divisor of all the elements in S."
Definition:Common,Common,"Let $S$ be a finite set of real numbers, that is:

:$S = \set {x_1, x_2, \ldots, x_n: \forall k \in \N^*_n: x_k \in \R}$


Let $c \in \R$ such that $c$ divides all the elements of $S$, that is:

:$\forall x \in S: c \divides x$


Then $c$ is a common divisor of all the elements in $S$.
",Definition:Greatest Common Divisor/Real Numbers,"['Definitions/Euclidean Number Theory', 'Definitions/Real Analysis', 'Definitions/Greatest Common Divisor']","Let S be a finite set of real numbers, that is:

:S = x_1, x_2, …, x_n: ∀ k ∈^*_n: x_k ∈


Let c ∈ such that c divides all the elements of S, that is:

:∀ x ∈ S: c  x


Then c is a common divisor of all the elements in S.
"
Definition:Common,Common,"Let $S$ be a finite set of non-zero integers, that is:

:$S = \set {x_1, x_2, \ldots, x_n: \forall k \in \N^*_n: x_k \in \Z, x_k \ne 0}$


Let $m \in \Z$ such that all the elements of $S$ divide $m$, that is:

:$\forall x \in S: x \divides m$


Then $m$ is a common multiple of all the elements in $S$.

",Definition:Common Multiple,['Definitions/Number Theory'],"Let S be a finite set of non-zero integers, that is:

:S = x_1, x_2, …, x_n: ∀ k ∈^*_n: x_k ∈, x_k  0


Let m ∈ such that all the elements of S divide m, that is:

:∀ x ∈ S: x  m


Then m is a common multiple of all the elements in S.

"
Definition:Common,Common,"Let $\sequence {a_k}$ be the arithmetic sequence:

:$a_k = a_0 + k d$ for $k = 0, 1, 2, \ldots, n - 1$


The term $d$ is the common difference of $\sequence {a_k}$.",Definition:Arithmetic Sequence/Common Difference,['Definitions/Arithmetic Sequences'],"Let a_k be the arithmetic sequence:

:a_k = a_0 + k d for k = 0, 1, 2, …, n - 1


The term d is the common difference of a_k."
Definition:Common,Common,"Let $\sequence {x_n}$ be a geometric sequence in $\R$ defined as:
:$x_n = a r^n$ for $n = 0, 1, 2, 3, \ldots$


The parameter:
:$r \in \R: r \ne 0$
is called the common ratio of $\sequence {x_n}$.",Definition:Geometric Sequence/Common Ratio,['Definitions/Geometric Sequences'],"Let x_n be a geometric sequence in  defined as:
:x_n = a r^n for n = 0, 1, 2, 3, …


The parameter:
:r ∈: r  0
is called the common ratio of x_n."
Definition:Common,Common,A vulgar fraction is a fraction representing a rational number whose numerator and denominator are both integers.,Definition:Fraction/Vulgar,"['Definitions/Vulgar Fractions', 'Definitions/Fractions']",A vulgar fraction is a fraction representing a rational number whose numerator and denominator are both integers.
Definition:Common,Common,"Let $A$ and $B$ be planes.

The common section of $A$ and $B$ is the intersection of $A$ and $B$.",Definition:Common Section,['Definitions/Euclidean Geometry'],"Let A and B be planes.

The common section of A and B is the intersection of A and B."
Definition:Commutative Algebra,Commutative Algebra,"Let $R$ be a commutative ring.

Let $\struct {A_R, \oplus}$ be an algebra over $R$.


Then $\struct {A_R, \oplus}$ is a commutative algebra  $\oplus$ is a commutative operation.

That is:

:$\forall a, b \in A_R: a \oplus b = b \oplus a$
",Definition:Commutative Algebra (Abstract Algebra),"['Definitions/Commutative Algebras', 'Definitions/Algebras', 'Definitions/Commutativity']","Let R be a commutative ring.

Let A_R, ⊕ be an algebra over R.


Then A_R, ⊕ is a commutative algebra  ⊕ is a commutative operation.

That is:

:∀ a, b ∈ A_R: a ⊕ b = b ⊕ a
"
Definition:Commutative Algebra,Commutative Algebra,Commutative algebra is the branch of abstract algebra concerned with commutative and unitary rings.,Definition:Commutative Algebra (Mathematical Branch),"['Definitions/Branches of Mathematics', 'Definitions/Commutative Algebra', 'Definitions/Abstract Algebra', 'Definitions/Ring Theory', 'Definitions/Module Theory', 'Definitions/Algebraic Geometry', 'Definitions/Algebraic Number Theory', 'Definitions/Commutativity']",Commutative algebra is the branch of abstract algebra concerned with commutative and unitary rings.
Definition:Compact,Compact,"Let $D$ be a subset of the complex plane $\C$.


Then $D$ is compact (in $\C$) :
:$D$ is closed in $\C$
and
:$D$ is bounded in $\C$.",Definition:Compact Space/Metric Space/Complex,"['Definitions/Compact Spaces', 'Definitions/Complex Analysis']","Let D be a subset of the complex plane .


Then D is compact (in ) :
:D is closed in 
and
:D is bounded in ."
Definition:Compact,Compact,"Let $\R^n$ denote Euclidean $n$-space.

Let $H \subseteq \R^n$.


Then $H$ is compact in $\R^n$  $H$ is closed and bounded.


=== Real Analysis ===

The same definition applies when $n = 1$, that is, for the real number line:

Let $\R$ be the real number line considered as a topological space under the Euclidean topology.

Let $H \subseteq \R$.



=== Complex Analysis ===
",Definition:Compact Space/Euclidean Space,"['Definitions/Euclidean Space', 'Definitions/Compact Spaces']","Let ^n denote Euclidean n-space.

Let H ⊆^n.


Then H is compact in ^n  H is closed and bounded.


=== Real Analysis ===

The same definition applies when n = 1, that is, for the real number line:

Let  be the real number line considered as a topological space under the Euclidean topology.

Let H ⊆.



=== Complex Analysis ===
"
Definition:Compact,Compact,"Let $M = \struct{X, \norm {\,\cdot\,}}$ be a normed vector space.

Let $K \subseteq X$ be a subset of $X$.


The normed vector subspace $M_K = \struct {K, \norm {\,\cdot\,}_K}$ is compact in $M$  $M_K$ is itself a compact normed vector space.
",Definition:Compact Space/Normed Vector Space,['Definitions/Normed Vector Spaces'],"Let M = X,  · be a normed vector space.

Let K ⊆ X be a subset of X.


The normed vector subspace M_K = K,  · _K is compact in M  M_K is itself a compact normed vector space.
"
Definition:Compact,Compact,,Definition:Compact Linear Transformation,"['Definitions/Linear Transformations', 'Definitions/Compact Linear Transformations']",
Definition:Compact,Compact,"Let $\struct {S, \preceq}$ be an ordered set.

Let $x \in S$.


Then $x$ is compact (element)  $x \ll x$

where $\ll$ denotes the way below relation.",Definition:Compact Element,['Definitions/Order Theory'],"Let S, ≼ be an ordered set.

Let x ∈ S.


Then x is compact (element)  x ≪ x

where ≪ denotes the way below relation."
Definition:Compatible,Compatible,"Let $\struct {S, \circ}$ be a closed algebraic structure.

Let $\RR$ be a relation on $S$.


Then $\RR$ is compatible with $\circ$ :

:$\forall x, y, z \in S: x \mathrel \RR y \implies \paren {x \circ z} \mathrel \RR \paren {y \circ z}$

:$\forall x, y, z \in S: x \mathrel \RR y \implies \paren {z \circ x} \mathrel \RR \paren {z \circ y}$",Definition:Relation Compatible with Operation,['Definitions/Compatible Relations'],"Let S, ∘ be a closed algebraic structure.

Let  be a relation on S.


Then  is compatible with ∘ :

:∀ x, y, z ∈ S: x  y x ∘ zy ∘ z

:∀ x, y, z ∈ S: x  y z ∘ xz ∘ y"
Definition:Compatible,Compatible,"Let $\struct {S, \circ}$ be a closed algebraic structure.

Let $\RR$ be a relation on $S$.


Then $\RR$ is compatible with $\circ$ :

:$\forall x, y, z \in S: x \mathrel \RR y \implies \paren {x \circ z} \mathrel \RR \paren {y \circ z}$

:$\forall x, y, z \in S: x \mathrel \RR y \implies \paren {z \circ x} \mathrel \RR \paren {z \circ y}$
Let $\struct {S, \circ}$ be a closed algebraic structure.

Let $\RR$ be a relation in $S$.


Then $\RR$ is conversely compatible with $\circ$ :

:$\forall x, y, z \in S: \paren {x \circ z} \mathrel \RR \paren {y \circ z} \implies x \mathrel \RR y$

:$\forall x, y, z \in S: \paren {z \circ x} \mathrel \RR \paren {z \circ y} \implies x \mathrel \RR y$
",Definition:Relation Strongly Compatible with Operation,['Definitions/Compatible Relations'],"Let S, ∘ be a closed algebraic structure.

Let  be a relation on S.


Then  is compatible with ∘ :

:∀ x, y, z ∈ S: x  y x ∘ zy ∘ z

:∀ x, y, z ∈ S: x  y z ∘ xz ∘ y
Let S, ∘ be a closed algebraic structure.

Let  be a relation in S.


Then  is conversely compatible with ∘ :

:∀ x, y, z ∈ S: x ∘ zy ∘ z x  y

:∀ x, y, z ∈ S: z ∘ xz ∘ y x  y
"
Definition:Compatible,Compatible,"Let $F$ be a (unary) operation which can be applied to sets.


Then $F$ is compatible with set equivalence :

:$F \sqbrk A = F \sqbrk B \iff A \sim B$

where:
:$A$ and $B$ are arbitrary sets
:$F \sqbrk A$ denotes the image of $A$ under $F$
:$\sim$ denotes set equivalence.",Definition:Operation Compatible with Set Equivalence,['Definitions/Set Theory'],"Let F be a (unary) operation which can be applied to sets.


Then F is compatible with set equivalence :

:F  A = F  B  A ∼ B

where:
:A and B are arbitrary sets
:F  A denotes the image of A under F
:∼ denotes set equivalence."
Definition:Compatible,Compatible,"Let $\struct {R, +, \circ}$ be a ring whose zero is $0_R$.


An ordering $\preccurlyeq$ on $R$ is compatible with the ring structure of $R$  $\preccurlyeq$ satisies the ring compatible ordering axioms:
",Definition:Ordering Compatible with Ring Structure,"['Definitions/Ordered Rings', 'Definitions/Ordered Integral Domains']","Let R, +, ∘ be a ring whose zero is 0_R.


An ordering ≼ on R is compatible with the ring structure of R  ≼ satisies the ring compatible ordering axioms:
"
Definition:Compatible,Compatible,"Let $\struct {M, \cdot}$ be a semigroup.

Let $\struct {R, +, \circ}$ be a ring.

Let $\sequence {R_n}_{n \mathop \in M}$ be a gradation of type $M$ on the additive group of $R$.


The gradation is compatible with the ring structure 
:$\forall m, n \in M : \forall x \in S_m, y \in S_n: x \circ y \in S_{m \cdot n}$

and so:

:$S_m S_n \subseteq S_ {m \cdot n}$",Definition:Gradation Compatible with Ring Structure,['Definitions/Ring Theory'],"Let M, · be a semigroup.

Let R, +, ∘ be a ring.

Let R_n_n ∈ M be a gradation of type M on the additive group of R.


The gradation is compatible with the ring structure 
:∀ m, n ∈ M : ∀ x ∈ S_m, y ∈ S_n: x ∘ y ∈ S_m · n

and so:

:S_m S_n ⊆ S_ m · n"
Definition:Compatible,Compatible,"Let $\struct {S, \circ}$ be a closed algebraic structure.

Let $\RR$ be a relation in $S$.


Then $\RR$ is conversely compatible with $\circ$ :

:$\forall x, y, z \in S: \paren {x \circ z} \mathrel \RR \paren {y \circ z} \implies x \mathrel \RR y$

:$\forall x, y, z \in S: \paren {z \circ x} \mathrel \RR \paren {z \circ y} \implies x \mathrel \RR y$",Definition:Relation Conversely Compatible with Operation,['Definitions/Compatible Relations'],"Let S, ∘ be a closed algebraic structure.

Let  be a relation in S.


Then  is conversely compatible with ∘ :

:∀ x, y, z ∈ S: x ∘ zy ∘ z x  y

:∀ x, y, z ∈ S: z ∘ xz ∘ y x  y"
Definition:Compatible,Compatible,"Let $\struct {S, \circ}$ be a closed algebraic structure.

Let $\RR$ be a relation on $S$.


Then $\RR$ is compatible with $\circ$ :

:$\forall x, y, z \in S: x \mathrel \RR y \implies \paren {x \circ z} \mathrel \RR \paren {y \circ z}$

:$\forall x, y, z \in S: x \mathrel \RR y \implies \paren {z \circ x} \mathrel \RR \paren {z \circ y}$
",Definition:Universally Compatible Relation,['Definitions/Abstract Algebra'],"Let S, ∘ be a closed algebraic structure.

Let  be a relation on S.


Then  is compatible with ∘ :

:∀ x, y, z ∈ S: x  y x ∘ zy ∘ z

:∀ x, y, z ∈ S: x  y z ∘ xz ∘ y
"
Definition:Compatible,Compatible,"Let $\struct {S, \circ}$ be an algebraic structure.

Let $\RR$ be an equivalence relation on $S$.


Then $\RR$ is a congruence relation for $\circ$ :

:$\forall x_1, x_2, y_1, y_2 \in S: \paren {x_1 \mathrel \RR x_2} \land \paren {y_1 \mathrel \RR y_2} \implies \paren {x_1 \circ y_1} \mathrel \RR \paren {x_2 \circ y_2}$",Definition:Congruence Relation,"['Definitions/Abstract Algebra', 'Definitions/Equivalence Relations']","Let S, ∘ be an algebraic structure.

Let  be an equivalence relation on S.


Then  is a congruence relation for ∘ :

:∀ x_1, x_2, y_1, y_2 ∈ S: x_1  x_2y_1  y_2x_1 ∘ y_1x_2 ∘ y_2"
Definition:Compatible,Compatible,"Let $\UU_1$ and $\UU_2$ be quasiuniformities on a set $S$.

Let $\struct {\struct {S, \UU_1}, \tau_1}$ and $\struct {\struct {S, \UU_2}, \tau_2}$ be the quasiuniform spaces generated by $\UU_1$ and $\UU_2$.


Then $\UU_1$ and $\UU_2$ are compatible (with each other)  their topologies are equal.

That is,  $\tau_1 = \tau_2$.",Definition:Compatible Quasiuniformities,['Definitions/Uniformities'],"Let _1 and _2 be quasiuniformities on a set S.

Let S, _1, τ_1 and S, _2, τ_2 be the quasiuniform spaces generated by _1 and _2.


Then _1 and _2 are compatible (with each other)  their topologies are equal.

That is,  τ_1 = τ_2."
Definition:Compatible,Compatible,"Let $M$ be a topological space.

Let $d$ be a natural number.

Let $\struct {U, \phi}$ and $\struct {V, \psi}$ be $d$-dimensional charts of $M$.


$\struct {U, \phi}$ and $\struct {V, \psi}$ are smoothly compatible  their transition mapping:
:$\psi \circ \phi^{-1} : \map \phi {U \cap V} \to \map \psi {U \cap V}$
is of class $C^\infty$.


Category:Definitions/Manifolds
",Definition:Compatible Charts,['Definitions/Manifolds'],"Let M be a topological space.

Let d be a natural number.

Let U, ϕ and V, ψ be d-dimensional charts of M.


U, ϕ and V, ψ are smoothly compatible  their transition mapping:
:ψ∘ϕ^-1 : ϕU ∩ V→ψU ∩ V
is of class C^∞.


Category:Definitions/Manifolds
"
Definition:Compatible,Compatible,"Let $M$ be a topological space.

Let $d$ be a natural number.

Let $\struct {U, \phi}$ and $\struct {V, \psi}$ be $d$-dimensional charts of $M$.


$\struct {U, \phi}$ and $\struct {V, \psi}$ are smoothly compatible  their transition mapping:
:$\psi \circ \phi^{-1} : \map \phi {U \cap V} \to \map \psi {U \cap V}$
is of class $C^\infty$.


Category:Definitions/Manifolds",Definition:Compatible Charts/Smooth,['Definitions/Manifolds'],"Let M be a topological space.

Let d be a natural number.

Let U, ϕ and V, ψ be d-dimensional charts of M.


U, ϕ and V, ψ are smoothly compatible  their transition mapping:
:ψ∘ϕ^-1 : ϕU ∩ V→ψU ∩ V
is of class C^∞.


Category:Definitions/Manifolds"
Definition:Compatible,Compatible,"Let $M$ be a topological space.

Let $d$ be a natural number.

Let $\struct {U, \phi}$ and $\struct {V, \psi}$ be $d$-dimensional charts of $M$.


Then $\struct {U, \phi}$ and $\struct {V, \psi}$ are $C^k$-compatible  their transition mapping:
:$\psi \circ \phi^{-1}: \map \phi {U \cap V} \to \map \psi {U \cap V}$
is of class $C^k$.


=== Smoothly Compatible Charts ===

",Definition:Chart Compatible with Atlas,['Definitions/Manifolds'],"Let M be a topological space.

Let d be a natural number.

Let U, ϕ and V, ψ be d-dimensional charts of M.


Then U, ϕ and V, ψ are C^k-compatible  their transition mapping:
:ψ∘ϕ^-1: ϕU ∩ V→ψU ∩ V
is of class C^k.


=== Smoothly Compatible Charts ===

"
Definition:Complement,Complement,":

Let $\angle BAC$ be a right angle.

Let $\angle BAD + \angle DAC = \angle BAC$.

That is:
:$\angle DAC = \angle BAC - \angle BAD$


Then $\angle DAC$ is the complement of $\angle BAD$.


Hence, for any angle $\alpha$ (whether less than a right angle or not), the complement of $\alpha$ is $\dfrac \pi 2 - \alpha$.

Measured in degrees, the complement of $\alpha$ is $90^\circ - \alpha$.


If $\alpha$ is the complement of $\beta$, then it follows that $\beta$ is the complement of $\alpha$.

Hence we can say that $\alpha$ and $\beta$ are complementary.


It can be seen from this that the complement of an angle greater than a right angle is negative.


Thus complementary angles are two angles whose measures add up to the measure of a right angle.

That is, their measurements add up to $90$ degrees or $\dfrac \pi 2$ radians.
:

Let $\angle BAC$ be a right angle.

Let $\angle BAD + \angle DAC = \angle BAC$.

That is:
:$\angle DAC = \angle BAC - \angle BAD$


Then $\angle DAC$ is the complement of $\angle BAD$.


Hence, for any angle $\alpha$ (whether less than a right angle or not), the complement of $\alpha$ is $\dfrac \pi 2 - \alpha$.

Measured in degrees, the complement of $\alpha$ is $90^\circ - \alpha$.


If $\alpha$ is the complement of $\beta$, then it follows that $\beta$ is the complement of $\alpha$.

Hence we can say that $\alpha$ and $\beta$ are complementary.


It can be seen from this that the complement of an angle greater than a right angle is negative.


Thus complementary angles are two angles whose measures add up to the measure of a right angle.

That is, their measurements add up to $90$ degrees or $\dfrac \pi 2$ radians.
:

Let $\angle BAC$ be a right angle.

Let $\angle BAD + \angle DAC = \angle BAC$.

That is:
:$\angle DAC = \angle BAC - \angle BAD$


Then $\angle DAC$ is the complement of $\angle BAD$.


Hence, for any angle $\alpha$ (whether less than a right angle or not), the complement of $\alpha$ is $\dfrac \pi 2 - \alpha$.

Measured in degrees, the complement of $\alpha$ is $90^\circ - \alpha$.


If $\alpha$ is the complement of $\beta$, then it follows that $\beta$ is the complement of $\alpha$.

Hence we can say that $\alpha$ and $\beta$ are complementary.


It can be seen from this that the complement of an angle greater than a right angle is negative.


Thus complementary angles are two angles whose measures add up to the measure of a right angle.

That is, their measurements add up to $90$ degrees or $\dfrac \pi 2$ radians.
:

Let $\angle BAC$ be a right angle.

Let $\angle BAD + \angle DAC = \angle BAC$.

That is:
:$\angle DAC = \angle BAC - \angle BAD$


Then $\angle DAC$ is the complement of $\angle BAD$.


Hence, for any angle $\alpha$ (whether less than a right angle or not), the complement of $\alpha$ is $\dfrac \pi 2 - \alpha$.

Measured in degrees, the complement of $\alpha$ is $90^\circ - \alpha$.


If $\alpha$ is the complement of $\beta$, then it follows that $\beta$ is the complement of $\alpha$.

Hence we can say that $\alpha$ and $\beta$ are complementary.


It can be seen from this that the complement of an angle greater than a right angle is negative.


Thus complementary angles are two angles whose measures add up to the measure of a right angle.

That is, their measurements add up to $90$ degrees or $\dfrac \pi 2$ radians.
:

Let $\angle BAC$ be a right angle.

Let $\angle BAD + \angle DAC = \angle BAC$.

That is:
:$\angle DAC = \angle BAC - \angle BAD$


Then $\angle DAC$ is the complement of $\angle BAD$.


Hence, for any angle $\alpha$ (whether less than a right angle or not), the complement of $\alpha$ is $\dfrac \pi 2 - \alpha$.

Measured in degrees, the complement of $\alpha$ is $90^\circ - \alpha$.


If $\alpha$ is the complement of $\beta$, then it follows that $\beta$ is the complement of $\alpha$.

Hence we can say that $\alpha$ and $\beta$ are complementary.


It can be seen from this that the complement of an angle greater than a right angle is negative.


Thus complementary angles are two angles whose measures add up to the measure of a right angle.

That is, their measurements add up to $90$ degrees or $\dfrac \pi 2$ radians.
:

Let $\angle BAC$ be a right angle.

Let $\angle BAD + \angle DAC = \angle BAC$.

That is:
:$\angle DAC = \angle BAC - \angle BAD$


Then $\angle DAC$ is the complement of $\angle BAD$.


Hence, for any angle $\alpha$ (whether less than a right angle or not), the complement of $\alpha$ is $\dfrac \pi 2 - \alpha$.

Measured in degrees, the complement of $\alpha$ is $90^\circ - \alpha$.


If $\alpha$ is the complement of $\beta$, then it follows that $\beta$ is the complement of $\alpha$.

Hence we can say that $\alpha$ and $\beta$ are complementary.


It can be seen from this that the complement of an angle greater than a right angle is negative.


Thus complementary angles are two angles whose measures add up to the measure of a right angle.

That is, their measurements add up to $90$ degrees or $\dfrac \pi 2$ radians.
:

Let $\angle BAC$ be a right angle.

Let $\angle BAD + \angle DAC = \angle BAC$.

That is:
:$\angle DAC = \angle BAC - \angle BAD$


Then $\angle DAC$ is the complement of $\angle BAD$.


Hence, for any angle $\alpha$ (whether less than a right angle or not), the complement of $\alpha$ is $\dfrac \pi 2 - \alpha$.

Measured in degrees, the complement of $\alpha$ is $90^\circ - \alpha$.


If $\alpha$ is the complement of $\beta$, then it follows that $\beta$ is the complement of $\alpha$.

Hence we can say that $\alpha$ and $\beta$ are complementary.


It can be seen from this that the complement of an angle greater than a right angle is negative.


Thus complementary angles are two angles whose measures add up to the measure of a right angle.

That is, their measurements add up to $90$ degrees or $\dfrac \pi 2$ radians.
:

Let $\angle BAC$ be a right angle.

Let $\angle BAD + \angle DAC = \angle BAC$.

That is:
:$\angle DAC = \angle BAC - \angle BAD$


Then $\angle DAC$ is the complement of $\angle BAD$.


Hence, for any angle $\alpha$ (whether less than a right angle or not), the complement of $\alpha$ is $\dfrac \pi 2 - \alpha$.

Measured in degrees, the complement of $\alpha$ is $90^\circ - \alpha$.


If $\alpha$ is the complement of $\beta$, then it follows that $\beta$ is the complement of $\alpha$.

Hence we can say that $\alpha$ and $\beta$ are complementary.


It can be seen from this that the complement of an angle greater than a right angle is negative.


Thus complementary angles are two angles whose measures add up to the measure of a right angle.

That is, their measurements add up to $90$ degrees or $\dfrac \pi 2$ radians.
",Definition:Complementary Angles,['Definitions/Angles'],":

Let ∠ BAC be a right angle.

Let ∠ BAD + ∠ DAC = ∠ BAC.

That is:
:∠ DAC = ∠ BAC - ∠ BAD


Then ∠ DAC is the complement of ∠ BAD.


Hence, for any angle α (whether less than a right angle or not), the complement of α is π 2 - α.

Measured in degrees, the complement of α is 90^∘ - α.


If α is the complement of β, then it follows that β is the complement of α.

Hence we can say that α and β are complementary.


It can be seen from this that the complement of an angle greater than a right angle is negative.


Thus complementary angles are two angles whose measures add up to the measure of a right angle.

That is, their measurements add up to 90 degrees or π 2 radians.
:

Let ∠ BAC be a right angle.

Let ∠ BAD + ∠ DAC = ∠ BAC.

That is:
:∠ DAC = ∠ BAC - ∠ BAD


Then ∠ DAC is the complement of ∠ BAD.


Hence, for any angle α (whether less than a right angle or not), the complement of α is π 2 - α.

Measured in degrees, the complement of α is 90^∘ - α.


If α is the complement of β, then it follows that β is the complement of α.

Hence we can say that α and β are complementary.


It can be seen from this that the complement of an angle greater than a right angle is negative.


Thus complementary angles are two angles whose measures add up to the measure of a right angle.

That is, their measurements add up to 90 degrees or π 2 radians.
:

Let ∠ BAC be a right angle.

Let ∠ BAD + ∠ DAC = ∠ BAC.

That is:
:∠ DAC = ∠ BAC - ∠ BAD


Then ∠ DAC is the complement of ∠ BAD.


Hence, for any angle α (whether less than a right angle or not), the complement of α is π 2 - α.

Measured in degrees, the complement of α is 90^∘ - α.


If α is the complement of β, then it follows that β is the complement of α.

Hence we can say that α and β are complementary.


It can be seen from this that the complement of an angle greater than a right angle is negative.


Thus complementary angles are two angles whose measures add up to the measure of a right angle.

That is, their measurements add up to 90 degrees or π 2 radians.
:

Let ∠ BAC be a right angle.

Let ∠ BAD + ∠ DAC = ∠ BAC.

That is:
:∠ DAC = ∠ BAC - ∠ BAD


Then ∠ DAC is the complement of ∠ BAD.


Hence, for any angle α (whether less than a right angle or not), the complement of α is π 2 - α.

Measured in degrees, the complement of α is 90^∘ - α.


If α is the complement of β, then it follows that β is the complement of α.

Hence we can say that α and β are complementary.


It can be seen from this that the complement of an angle greater than a right angle is negative.


Thus complementary angles are two angles whose measures add up to the measure of a right angle.

That is, their measurements add up to 90 degrees or π 2 radians.
:

Let ∠ BAC be a right angle.

Let ∠ BAD + ∠ DAC = ∠ BAC.

That is:
:∠ DAC = ∠ BAC - ∠ BAD


Then ∠ DAC is the complement of ∠ BAD.


Hence, for any angle α (whether less than a right angle or not), the complement of α is π 2 - α.

Measured in degrees, the complement of α is 90^∘ - α.


If α is the complement of β, then it follows that β is the complement of α.

Hence we can say that α and β are complementary.


It can be seen from this that the complement of an angle greater than a right angle is negative.


Thus complementary angles are two angles whose measures add up to the measure of a right angle.

That is, their measurements add up to 90 degrees or π 2 radians.
:

Let ∠ BAC be a right angle.

Let ∠ BAD + ∠ DAC = ∠ BAC.

That is:
:∠ DAC = ∠ BAC - ∠ BAD


Then ∠ DAC is the complement of ∠ BAD.


Hence, for any angle α (whether less than a right angle or not), the complement of α is π 2 - α.

Measured in degrees, the complement of α is 90^∘ - α.


If α is the complement of β, then it follows that β is the complement of α.

Hence we can say that α and β are complementary.


It can be seen from this that the complement of an angle greater than a right angle is negative.


Thus complementary angles are two angles whose measures add up to the measure of a right angle.

That is, their measurements add up to 90 degrees or π 2 radians.
:

Let ∠ BAC be a right angle.

Let ∠ BAD + ∠ DAC = ∠ BAC.

That is:
:∠ DAC = ∠ BAC - ∠ BAD


Then ∠ DAC is the complement of ∠ BAD.


Hence, for any angle α (whether less than a right angle or not), the complement of α is π 2 - α.

Measured in degrees, the complement of α is 90^∘ - α.


If α is the complement of β, then it follows that β is the complement of α.

Hence we can say that α and β are complementary.


It can be seen from this that the complement of an angle greater than a right angle is negative.


Thus complementary angles are two angles whose measures add up to the measure of a right angle.

That is, their measurements add up to 90 degrees or π 2 radians.
:

Let ∠ BAC be a right angle.

Let ∠ BAD + ∠ DAC = ∠ BAC.

That is:
:∠ DAC = ∠ BAC - ∠ BAD


Then ∠ DAC is the complement of ∠ BAD.


Hence, for any angle α (whether less than a right angle or not), the complement of α is π 2 - α.

Measured in degrees, the complement of α is 90^∘ - α.


If α is the complement of β, then it follows that β is the complement of α.

Hence we can say that α and β are complementary.


It can be seen from this that the complement of an angle greater than a right angle is negative.


Thus complementary angles are two angles whose measures add up to the measure of a right angle.

That is, their measurements add up to 90 degrees or π 2 radians.
"
Definition:Complement,Complement,"Let $ABDC$ and $EFHG$ be two parallelograms with the same angles, which share a diagonal, such that $ABDC \cap EFHG \ne \O$.

:

Then the two parallelograms $CIGK$ and $BJHL$ are known as the complements of the parallelograms $ABDC$ and $EFHG$.",Definition:Complements of Parallelograms,['Definitions/Parallelograms'],"Let ABDC and EFHG be two parallelograms with the same angles, which share a diagonal, such that ABDC ∩ EFHG Ø.

:

Then the two parallelograms CIGK and BJHL are known as the complements of the parallelograms ABDC and EFHG."
Definition:Complement,Complement,"Let $S$ be a set, and let $T \subseteq S$, that is: let $T$ be a subset of $S$.

Then the set difference $S \setminus T$ can be written $\relcomp S T$, and is called the relative complement of $T$ in $S$, or the complement of $T$ relative to $S$.

Thus:
:$\relcomp S T = \set {x \in S : x \notin T}$
",Definition:Complement of Relation,['Definitions/Relation Theory'],"Let S be a set, and let T ⊆ S, that is: let T be a subset of S.

Then the set difference S ∖ T can be written S T, and is called the relative complement of T in S, or the complement of T relative to S.

Thus:
:S T = x ∈ S : x ∉ T
"
Definition:Complement,Complement,"The set complement (or, when the context is established, just complement) of a set $S$ in a universe $\mathbb U$ is defined as:

:$\map \complement S = \relcomp {\mathbb U} S = \mathbb U \setminus S$

See the definition of Relative Complement for the definition of $\relcomp {\mathbb U} S$.


Thus the complement of a set $S$ is the relative complement of $S$ in the universe, or the complement of $S$ relative to the universe.

A common alternative to the symbology $\map \complement S$, which we will sometimes use, is $\overline S$.
The set complement (or, when the context is established, just complement) of a set $S$ in a universe $\mathbb U$ is defined as:

:$\map \complement S = \relcomp {\mathbb U} S = \mathbb U \setminus S$

See the definition of Relative Complement for the definition of $\relcomp {\mathbb U} S$.


Thus the complement of a set $S$ is the relative complement of $S$ in the universe, or the complement of $S$ relative to the universe.

A common alternative to the symbology $\map \complement S$, which we will sometimes use, is $\overline S$.
Let $S$ be a set, and let $T \subseteq S$, that is: let $T$ be a subset of $S$.

Then the set difference $S \setminus T$ can be written $\relcomp S T$, and is called the relative complement of $T$ in $S$, or the complement of $T$ relative to $S$.

Thus:
:$\relcomp S T = \set {x \in S : x \notin T}$
The set complement (or, when the context is established, just complement) of a set $S$ in a universe $\mathbb U$ is defined as:

:$\map \complement S = \relcomp {\mathbb U} S = \mathbb U \setminus S$

See the definition of Relative Complement for the definition of $\relcomp {\mathbb U} S$.


Thus the complement of a set $S$ is the relative complement of $S$ in the universe, or the complement of $S$ relative to the universe.

A common alternative to the symbology $\map \complement S$, which we will sometimes use, is $\overline S$.
Let $S$ be a set, and let $T \subseteq S$, that is: let $T$ be a subset of $S$.

Then the set difference $S \setminus T$ can be written $\relcomp S T$, and is called the relative complement of $T$ in $S$, or the complement of $T$ relative to $S$.

Thus:
:$\relcomp S T = \set {x \in S : x \notin T}$
Let $S$ be a set, and let $T \subseteq S$, that is: let $T$ be a subset of $S$.

Then the set difference $S \setminus T$ can be written $\relcomp S T$, and is called the relative complement of $T$ in $S$, or the complement of $T$ relative to $S$.

Thus:
:$\relcomp S T = \set {x \in S : x \notin T}$
",Definition:Set Complement,"['Definitions/Set Complement', 'Definitions/Set Theory']","The set complement (or, when the context is established, just complement) of a set S in a universe 𝕌 is defined as:

:∁ S = 𝕌 S = 𝕌∖ S

See the definition of Relative Complement for the definition of 𝕌 S.


Thus the complement of a set S is the relative complement of S in the universe, or the complement of S relative to the universe.

A common alternative to the symbology ∁ S, which we will sometimes use, is S.
The set complement (or, when the context is established, just complement) of a set S in a universe 𝕌 is defined as:

:∁ S = 𝕌 S = 𝕌∖ S

See the definition of Relative Complement for the definition of 𝕌 S.


Thus the complement of a set S is the relative complement of S in the universe, or the complement of S relative to the universe.

A common alternative to the symbology ∁ S, which we will sometimes use, is S.
Let S be a set, and let T ⊆ S, that is: let T be a subset of S.

Then the set difference S ∖ T can be written S T, and is called the relative complement of T in S, or the complement of T relative to S.

Thus:
:S T = x ∈ S : x ∉ T
The set complement (or, when the context is established, just complement) of a set S in a universe 𝕌 is defined as:

:∁ S = 𝕌 S = 𝕌∖ S

See the definition of Relative Complement for the definition of 𝕌 S.


Thus the complement of a set S is the relative complement of S in the universe, or the complement of S relative to the universe.

A common alternative to the symbology ∁ S, which we will sometimes use, is S.
Let S be a set, and let T ⊆ S, that is: let T be a subset of S.

Then the set difference S ∖ T can be written S T, and is called the relative complement of T in S, or the complement of T relative to S.

Thus:
:S T = x ∈ S : x ∉ T
Let S be a set, and let T ⊆ S, that is: let T be a subset of S.

Then the set difference S ∖ T can be written S T, and is called the relative complement of T in S, or the complement of T relative to S.

Thus:
:S T = x ∈ S : x ∉ T
"
Definition:Complement,Complement,"Let $S$ be a set, and let $T \subseteq S$, that is: let $T$ be a subset of $S$.

Then the set difference $S \setminus T$ can be written $\relcomp S T$, and is called the relative complement of $T$ in $S$, or the complement of $T$ relative to $S$.

Thus:
:$\relcomp S T = \set {x \in S : x \notin T}$",Definition:Relative Complement,"['Definitions/Set Theory', 'Definitions/Relative Complement']","Let S be a set, and let T ⊆ S, that is: let T be a subset of S.

Then the set difference S ∖ T can be written S T, and is called the relative complement of T in S, or the complement of T relative to S.

Thus:
:S T = x ∈ S : x ∉ T"
Definition:Complement,Complement,"For any propositional formula $\mathbf A$, the set $\left\{{\mathbf A, \neg \mathbf A}\right\}$ is called a complementary pair of formulas.
",Definition:Logical Complement,['Definitions/Propositional Logic'],"For any propositional formula 𝐀, the set {𝐀, 𝐀} is called a complementary pair of formulas.
"
Definition:Complement,Complement,"Let $\struct {S, \vee, \wedge, \preceq}$ be a bounded lattice.

Suppose that every $a \in S$ admits a complement.


Then $\struct {S, \vee, \wedge, \preceq}$ is called a complemented lattice.
",Definition:Complement (Lattice Theory),['Definitions/Lattice Theory'],"Let S, ∨, ∧, ≼ be a bounded lattice.

Suppose that every a ∈ S admits a complement.


Then S, ∨, ∧, ≼ is called a complemented lattice.
"
Definition:Complete,Complete,"Let $\struct {S, \preceq}$ be an ordered set.

Then $\struct {S, \preceq}$ has the Dedekind completeness property  every non-empty subset of $S$ that is bounded above admits a supremum (in $S$).
",Definition:Dedekind Completeness Property,"['Definitions/Dedekind Completeness Property', 'Definitions/Order Theory']","Let S, ≼ be an ordered set.

Then S, ≼ has the Dedekind completeness property  every non-empty subset of S that is bounded above admits a supremum (in S).
"
Definition:Complete,Complete,An inductive ordered set is an ordered set in which every chain has an upper bound.,Definition:Inductive Ordered Set,"['Definitions/Set Theory', 'Definitions/Order Theory']",An inductive ordered set is an ordered set in which every chain has an upper bound.
Definition:Complete,Complete,"Let $S$ be a set of truth functions.


Then $S$ is functionally complete  all possible truth functions are definable from $S$.",Definition:Functionally Complete,['Definitions/Truth Functions'],"Let S be a set of truth functions.


Then S is functionally complete  all possible truth functions are definable from S."
Definition:Complete,Complete,"Let $\LL$ be a logical language.

Let $\mathscr P$ be a proof system for $\LL$, and let $\mathscr M$ be a formal semantics for $\LL$.


$\mathscr P$ is strongly complete for $\mathscr M$ :

:Every $\mathscr M$-semantic consequence is a $\mathscr P$-provable consequence.

Symbolically, this can be expressed as the statement that, for every collection $\FF$ of logical formulas, and every logical formula $\phi$ of $\LL$:

:$\FF \models_{\mathscr M} \phi$ implies $\FF \vdash_{\mathscr P} \phi$
",Definition:Complete Proof System,"['Definitions/Complete Proof Systems', 'Definitions/Proof Systems']","Let  be a logical language.

Let 𝒫 be a proof system for , and let ℳ be a formal semantics for .


𝒫 is strongly complete for ℳ :

:Every ℳ-semantic consequence is a 𝒫-provable consequence.

Symbolically, this can be expressed as the statement that, for every collection  of logical formulas, and every logical formula ϕ of :

:_ℳϕ implies ⊢_𝒫ϕ
"
Definition:Complete,Complete,"Let $\LL$ be a language.

Let $\mathscr M$ be a formal semantics for $\LL$.

Let $T$ be an $\LL$-theory.


$T$ is complete (with respect to $\LL$ and $\mathscr M$) :
:$T$ is satisfiable for $\mathscr M$
:for every $\LL$-sentence $\phi$, either $T \models_{\mathscr M} \phi$ or $T \models_{\mathscr M} \neg \phi$
where $T \models_{\mathscr M} \phi$ denotes semantic entailment.",Definition:Complete Theory,"['Definitions/Model Theory', 'Definitions/Formal Semantics']","Let  be a language.

Let ℳ be a formal semantics for .

Let T be an -theory.


T is complete (with respect to  and ℳ) :
:T is satisfiable for ℳ
:for every -sentence ϕ, either T _ℳϕ or T _ℳϕ
where T _ℳϕ denotes semantic entailment."
Definition:Complete,Complete,"Let $G = \struct {V, E}$ be a simple graph such that every vertex is adjacent to every other vertex.

Then $G$ is called complete.


The complete graph of order $p$ is denoted $K_p$.
Let $G = \struct {V, E}$ be a simple graph such that every vertex is adjacent to every other vertex.

Then $G$ is called complete.


The complete graph of order $p$ is denoted $K_p$.
",Definition:Complete Graph,['Definitions/Complete Graphs'],"Let G = V, E be a simple graph such that every vertex is adjacent to every other vertex.

Then G is called complete.


The complete graph of order p is denoted K_p.
Let G = V, E be a simple graph such that every vertex is adjacent to every other vertex.

Then G is called complete.


The complete graph of order p is denoted K_p.
"
Definition:Complete,Complete,"A complete bipartite graph is a bipartite graph $G = \struct {A \mid B, E}$ in which every vertex in $A$ is adjacent to every vertex in $B$.

The complete bipartite graph where $A$ has $m$ vertices and $B$ has $n$ vertices is denoted $K_{m, n}$.",Definition:Complete Bipartite Graph,"['Definitions/Complete Bipartite Graphs', 'Definitions/Bipartite Graphs', 'Definitions/Graph Theory']","A complete bipartite graph is a bipartite graph G = A | B, E in which every vertex in A is adjacent to every vertex in B.

The complete bipartite graph where A has m vertices and B has n vertices is denoted K_m, n."
Definition:Completion,Completion,"Let $\struct {S, \preceq_S}$ be an ordered set.


An ordered set $\struct {T, \preceq_T}$ is an order completion of $S$ :

:$(1):\quad S \subseteq T$

:$(2):\quad {\preceq_T \restriction_S} = {\preceq_S}$, where $\restriction$ denotes restriction

:$(3):\quad \struct {T, \preceq_T}$ is a complete ordered set

:$(4):\quad$ For all ordered sets $\struct {T', \preceq_{T'} }$ satisfying $(1), (2)$ and $(3)$, there is a unique increasing injection $\phi: T' \to T$",Definition:Order Completion,['Definitions/Order Theory'],"Let S, ≼_S be an ordered set.


An ordered set T, ≼_T is an order completion of S :

:(1):   S ⊆ T

:(2):  ≼_T _S = ≼_S, where  denotes restriction

:(3):  T, ≼_T is a complete ordered set

:(4): For all ordered sets T', ≼_T' satisfying (1), (2) and (3), there is a unique increasing injection ϕ: T' → T"
Definition:Completion,Completion,"Let $\struct {X, \Sigma, \mu}, \struct {\tilde X, \Sigma^*, \bar \mu}$ be measure spaces.

Then:
:$\struct {\tilde X, \Sigma^*, \bar \mu}$ is a completion of $\struct {X, \Sigma, \mu}$
or:
:$\struct {\tilde X, \Sigma^*, \bar \mu}$ completes $\struct {X, \Sigma, \mu}$

 the following conditions hold:

:$(1): \quad \struct {\tilde X, \Sigma^*, \bar \mu}$ is a complete measure space
:$(2): \quad \tilde X = X$
:$(3): \quad \Sigma$ is a sub-$\sigma$-algebra of $\Sigma^*$
:$(4): \quad \forall E \in \Sigma: \map {\bar \mu} E = \map \mu E$, that is: $\bar \mu \restriction_\Sigma = \mu$",Definition:Completion (Measure Space),['Definitions/Measure Theory'],"Let X, Σ, μ, X̃, Σ^*, μ̅ be measure spaces.

Then:
:X̃, Σ^*, μ̅ is a completion of X, Σ, μ
or:
:X̃, Σ^*, μ̅ completes X, Σ, μ

 the following conditions hold:

:(1):   X̃, Σ^*, μ̅ is a complete measure space
:(2):   X̃ = X
:(3):   Σ is a sub-σ-algebra of Σ^*
:(4):   ∀ E ∈Σ: μ̅ E = μ E, that is: μ̅_Σ = μ"
Definition:Completion,Completion,"Let $M_1 = \struct {A, d}$ and $M_2 = \struct {\tilde A, \tilde d}$ be metric spaces.

Then $M_2$ is a completion of $M_1$, or $M_2$ completes $M_1$, :
:$(1): \quad M_2$ is a complete metric space
:$(2): \quad A \subseteq \tilde A$
:$(3): \quad A$ is dense in $M_2$
:$(4): \quad \forall x, y \in A : \map {\tilde d} {x, y} = \map d {x, y}$. In terms of restriction of functions, this says that $\map {\tilde d {\restriction_A} } = d$.


It is immediate from this definition that a completion of a metric space $M_1$ consists of:
:A complete metric space $M_2$
:An isometry $\phi : A \to \tilde A$
such that $\map \phi A = \set {\map \phi x: x \in A}$ is dense in $M_2$.

An isometry is often required to be bijective, so here one should consider $\phi$ as a mapping from $A$ to the image of $\phi$.

Therefore to insist that $\phi$ be an isometry, in this context, is to say that $\phi$ must be an injection that preserves the metric of $M_1$.",Definition:Completion (Metric Space),['Definitions/Complete Metric Spaces'],"Let M_1 = A, d and M_2 = Ã, d̃ be metric spaces.

Then M_2 is a completion of M_1, or M_2 completes M_1, :
:(1):    M_2 is a complete metric space
:(2):    A ⊆Ã
:(3):    A is dense in M_2
:(4):   ∀ x, y ∈ A : d̃x, y =  d x, y. In terms of restriction of functions, this says that d̃_A = d.


It is immediate from this definition that a completion of a metric space M_1 consists of:
:A complete metric space M_2
:An isometry ϕ : A →Ã
such that ϕ A = ϕ x: x ∈ A is dense in M_2.

An isometry is often required to be bijective, so here one should consider ϕ as a mapping from A to the image of ϕ.

Therefore to insist that ϕ be an isometry, in this context, is to say that ϕ must be an injection that preserves the metric of M_1."
Definition:Completion,Completion,"Let $M_1 = \struct {A, d}$ and $M_2 = \struct {\tilde A, \tilde d}$ be metric spaces.

Then $M_2$ is a completion of $M_1$, or $M_2$ completes $M_1$, :
:$(1): \quad M_2$ is a complete metric space
:$(2): \quad A \subseteq \tilde A$
:$(3): \quad A$ is dense in $M_2$
:$(4): \quad \forall x, y \in A : \map {\tilde d} {x, y} = \map d {x, y}$. In terms of restriction of functions, this says that $\map {\tilde d {\restriction_A} } = d$.


It is immediate from this definition that a completion of a metric space $M_1$ consists of:
:A complete metric space $M_2$
:An isometry $\phi : A \to \tilde A$
such that $\map \phi A = \set {\map \phi x: x \in A}$ is dense in $M_2$.

An isometry is often required to be bijective, so here one should consider $\phi$ as a mapping from $A$ to the image of $\phi$.

Therefore to insist that $\phi$ be an isometry, in this context, is to say that $\phi$ must be an injection that preserves the metric of $M_1$.
",Definition:Completion (Normed Division Ring),['Definitions/Normed Division Rings'],"Let M_1 = A, d and M_2 = Ã, d̃ be metric spaces.

Then M_2 is a completion of M_1, or M_2 completes M_1, :
:(1):    M_2 is a complete metric space
:(2):    A ⊆Ã
:(3):    A is dense in M_2
:(4):   ∀ x, y ∈ A : d̃x, y =  d x, y. In terms of restriction of functions, this says that d̃_A = d.


It is immediate from this definition that a completion of a metric space M_1 consists of:
:A complete metric space M_2
:An isometry ϕ : A →Ã
such that ϕ A = ϕ x: x ∈ A is dense in M_2.

An isometry is often required to be bijective, so here one should consider ϕ as a mapping from A to the image of ϕ.

Therefore to insist that ϕ be an isometry, in this context, is to say that ϕ must be an injection that preserves the metric of M_1.
"
Definition:Complex,Complex,"Let $\struct {S, \circ}$ be an algebraic structure.


We can define an operation on the power set $\powerset S$ as follows:

:$\forall A, B \in \powerset S: A \circ_\PP B = \set {a \circ b: a \in A, b \in B}$


This is called the operation induced on $\powerset S$ by $\circ$, and $A \circ_\PP B$ is called the subset product of $A$ and $B$.


It is usual to write $A \circ B$ for $A \circ_\PP B$.


=== Subset Product with Singleton ===

When one of the subsets in a subset product is a singleton, we can (and often do) dispose of the set braces. Thus:
",Definition:Subset Product,"['Definitions/Abstract Algebra', 'Definitions/Group Theory', 'Definitions/Subset Products']","Let S, ∘ be an algebraic structure.


We can define an operation on the power set S as follows:

:∀ A, B ∈ S: A ∘_ B = a ∘ b: a ∈ A, b ∈ B


This is called the operation induced on S by ∘, and A ∘_ B is called the subset product of A and B.


It is usual to write A ∘ B for A ∘_ B.


=== Subset Product with Singleton ===

When one of the subsets in a subset product is a singleton, we can (and often do) dispose of the set braces. Thus:
"
Definition:Complex,Complex,"Let $G$ be a group.

Let $K \subseteq G$ be a subset of $G$.


Then $K$ is referred to by some sources as a complex of elements of $G$.",Definition:Complex (Group Theory),"['Definitions/Group Theory', 'Definitions/Subsets', 'Definitions/Complexes of Groups']","Let G be a group.

Let K ⊆ G be a subset of G.


Then K is referred to by some sources as a complex of elements of G."
Definition:Complex,Complex,"Let $R$ be a commutative ring with unity.

Let $\ds M = \bigoplus_{n \mathop \in \Z} M^n$ be a $\Z$-graded $R$-module that is also a differential module with differential $\d$.


Then $M$ is a differential complex if $\d$ satisfies:

:$\map \d {M^n} \subseteq M^{n + 1}$

for all $n \in \Z$.


The notation $\d_n := \d \restriction_{M_n}$ is often seen.",Definition:Differential Complex,['Definitions/Homological Algebra'],"Let R be a commutative ring with unity.

Let M = ⊕_n ∈ M^n be a -graded R-module that is also a differential module with differential $̣.


ThenMis a differential complex if$̣ satisfies:

:Ṃ^̣ṇ⊆ M^n + 1

for all n ∈.


The notation _̣n := _M_n is often seen."
Definition:Complex,Complex,"A complex fraction is a fraction such that the numerator or denominator or both are themselves fractions.
",Definition:Fraction/Complex,"['Definitions/Complex Fractions', 'Definitions/Fractions']","A complex fraction is a fraction such that the numerator or denominator or both are themselves fractions.
"
Definition:Component,Component,A substatement of a compound statement is one of the statements that comprise it.,Definition:Compound Statement/Substatement,['Definitions/Compound Statements'],A substatement of a compound statement is one of the statements that comprise it.
Definition:Component,Component,"Let $S$ be a set.

Let $\mathbb S = \set {S_1 \mid S_2 \mid \cdots}$ be a partition of $S$.


The elements $S_1, S_2, \ldots \in \mathbb S$ are known as the components of the partition.


Category:Definitions/Set Partitions",Definition:Set Partition/Component,['Definitions/Set Partitions'],"Let S be a set.

Let 𝕊 = S_1 | S_2 |⋯ be a partition of S.


The elements S_1, S_2, …∈𝕊 are known as the components of the partition.


Category:Definitions/Set Partitions"
Definition:Component,Component,"Let $T$ be a topological space.

Let us define the relation $\sim$ on $T$ as follows:

:$x \sim y \iff x$ and $y$ are arc-connected.


We have that $\sim $ is an equivalence relation, so from the Fundamental Theorem on Equivalence Relations, the points in $T$ can be partitioned into equivalence classes.

These equivalence classes are called the arc components of $T$.


If $x \in T$, then the arc component of $T$ containing $x$ (that is, the set of points $y \in T$ with $x \sim y$) can be denoted by $\map {\operatorname {AC}_x} T$.",Definition:Arc Component,['Definitions/Arc-Connected Spaces'],"Let T be a topological space.

Let us define the relation ∼ on T as follows:

:x ∼ y  x and y are arc-connected.


We have that ∼ is an equivalence relation, so from the Fundamental Theorem on Equivalence Relations, the points in T can be partitioned into equivalence classes.

These equivalence classes are called the arc components of T.


If x ∈ T, then the arc component of T containing x (that is, the set of points y ∈ T with x ∼ y) can be denoted by AC_x T."
Definition:Component,Component,"Let $T = \struct {S, \tau}$ be a topological space.


A subset $Y \subseteq S$ is an irreducible component of $T$ :
:$Y$ is irreducible
:$Y$ is not a proper subset of an irreducible subset of $S$.

That is, :
:$Y$ is maximal in the ordered set of irreducible subsets of $S$, ordered by the subset relation.",Definition:Irreducible Component,['Definitions/Irreducible Spaces'],"Let T = S, τ be a topological space.


A subset Y ⊆ S is an irreducible component of T :
:Y is irreducible
:Y is not a proper subset of an irreducible subset of S.

That is, :
:Y is maximal in the ordered set of irreducible subsets of S, ordered by the subset relation."
Definition:Component,Component,"Let $G$ be a graph.

Let $H$ be a subgraph of $G$ such that:

:$H$ is connected

:$H$ is not contained in any connected subgraph of $G$ which has more vertices or edges than $H$ has.


Then $H$ is a component of $G$.",Definition:Component of Graph,['Definitions/Graph Theory'],"Let G be a graph.

Let H be a subgraph of G such that:

:H is connected

:H is not contained in any connected subgraph of G which has more vertices or edges than H has.


Then H is a component of G."
Definition:Component,Component,An electrical component is a device whose purpose is to control electricity in a specific manner.,Definition:Electrical Component,"['Definitions/Electrical Components', 'Definitions/Electronics']",An electrical component is a device whose purpose is to control electricity in a specific manner.
Definition:Composition,Composition,"Let $\sqbrk {S \to S}$ be the set of all mappings from a set $S$ to itself.

Then the concept of composite mapping defines a binary operation on $\sqbrk {S \to S}$:

:$\forall f, g \in \sqbrk {S \to S}: g \circ f = \set {\tuple {s, t}: s \in S, \tuple {f \paren s, t} \in g} \in \sqbrk {S \to S}$


Thus, for every pair $\tuple {f, g}$ of mappings in $\sqbrk {S \to S}$, the composition $g \circ f$ is another element of $\sqbrk {S \to S}$.
",Definition:Composition of Mappings,"['Definitions/Mapping Theory', 'Definitions/Composite Mappings']","Let S → S be the set of all mappings from a set S to itself.

Then the concept of composite mapping defines a binary operation on S → S:

:∀ f, g ∈S → S: g ∘ f = s, t: s ∈ S, f  s, t∈ g∈S → S


Thus, for every pair f, g of mappings in S → S, the composition g ∘ f is another element of S → S.
"
Definition:Composition,Composition,"Let $\RR_1 \subseteq S_1 \times T_1$ and $\RR_2 \subseteq S_2 \times T_2$ be relations.


Then the composite of $\RR_1$ and $\RR_2$ is defined and denoted as:

:$\RR_2 \circ \RR_1 := \set {\tuple {x, z} \in S_1 \times T_2: \exists y \in S_2 \cap T_1: \tuple {x, y} \in \RR_1 \land \tuple {y, z} \in \RR_2}$






It is clear that the composite relation $\RR_2 \circ \RR_1$ can also be defined as:

:$\map {\RR_2 \circ \RR_1} {S_1} = \map {\RR_2} {\map {\RR_1} {S_1} }$


Note that:
:$(1): \quad \RR_2 \circ \RR_1 \subseteq S_1 \times T_2$
:$(2): \quad$ The domain of $\RR_2 \circ \RR_1$ equals the domain of $\RR_1$, that is, $S_1$
:$(3): \quad$ The codomain of $\RR_2 \circ \RR_1$ equals the codomain of $\RR_2$, that is, $T_2$.",Definition:Composition of Relations,['Definitions/Relation Theory'],"Let _1 ⊆ S_1 × T_1 and _2 ⊆ S_2 × T_2 be relations.


Then the composite of _1 and _2 is defined and denoted as:

:_2 ∘_1 := x, z∈ S_1 × T_2: ∃ y ∈ S_2 ∩ T_1: x, y∈_1 y, z∈_2






It is clear that the composite relation _2 ∘_1 can also be defined as:

:_2 ∘_1S_1 = _2_1S_1


Note that:
:(1):   _2 ∘_1 ⊆ S_1 × T_2
:(2): The domain of _2 ∘_1 equals the domain of _1, that is, S_1
:(3): The codomain of _2 ∘_1 equals the codomain of _2, that is, T_2."
Definition:Composition,Composition,,Definition:Composition Series,"['Definitions/Composition Series', 'Definitions/Normal Series', 'Definitions/Finite Groups']",
Definition:Composition,Composition,"Let $\mathbf C, \mathbf D$ and $\mathbf E$ be metacategories.

Let $F: \mathbf C \to \mathbf D$ and $G: \mathbf D \to \mathbf E$ be (covariant) functors.


The composition of $G$ with $F$ is the functor $GF: \mathbf C \to \mathbf E$ defined by:

:For all objects $C$ of $\mathbf C$: $\hskip{2.9cm} GF \left({C}\right) := G \left({FC}\right)$
:For all morphisms $f: C_1 \to C_2$ of $\mathbf C$: $\quad GF \left({f}\right) := G \left({Ff}\right)$

$GF$ is said to be a composite functor.",Definition:Composition of Functors,['Definitions/Category Theory'],"Let 𝐂, 𝐃 and 𝐄 be metacategories.

Let F: 𝐂→𝐃 and G: 𝐃→𝐄 be (covariant) functors.


The composition of G with F is the functor GF: 𝐂→𝐄 defined by:

:For all objects C of 𝐂: 2.9cm GF (C) := G (FC)
:For all morphisms f: C_1 → C_2 of 𝐂: GF (f) := G (Ff)

GF is said to be a composite functor."
Definition:Composition,Composition,"Let $\mathbf C, \mathbf D$ and $\mathbf E$ be metacategories.

Let $F: \mathbf C \to \mathbf D$ and $G: \mathbf D \to \mathbf E$ be (covariant) functors.


The composition of $G$ with $F$ is the functor $GF: \mathbf C \to \mathbf E$ defined by:

:For all objects $C$ of $\mathbf C$: $\hskip{2.9cm} GF \left({C}\right) := G \left({FC}\right)$
:For all morphisms $f: C_1 \to C_2$ of $\mathbf C$: $\quad GF \left({f}\right) := G \left({Ff}\right)$

$GF$ is said to be a composite functor.
",Definition:Composition Functor on Slice Categories,"['Definitions/Slice Categories', 'Definitions/Category Theory']","Let 𝐂, 𝐃 and 𝐄 be metacategories.

Let F: 𝐂→𝐃 and G: 𝐃→𝐄 be (covariant) functors.


The composition of G with F is the functor GF: 𝐂→𝐄 defined by:

:For all objects C of 𝐂: 2.9cm GF (C) := G (FC)
:For all morphisms f: C_1 → C_2 of 𝐂: GF (f) := G (Ff)

GF is said to be a composite functor.
"
Definition:Composition,Composition,"A $k$-composition of a (strictly) positive integer $n \in \Z_{> 0}$ is an ordered $k$-tuple:
:$c = \tuple {c_1, c_2, \ldots, c_k}$
such that:
:$(1): \quad c_1 + c_2 + \cdots + c_k = n$
:$(2): \quad \forall i \in \closedint 1 k: c_i \in \Z_{>0}$, that is, all the $c_i$ are strictly positive integers.

Category:Definitions/Combinatorics",Definition:Composition (Combinatorics),['Definitions/Combinatorics'],"A k-composition of a (strictly) positive integer n ∈_> 0 is an ordered k-tuple:
:c = c_1, c_2, …, c_k
such that:
:(1):    c_1 + c_2 + ⋯ + c_k = n
:(2):   ∀ i ∈ 1 k: c_i ∈_>0, that is, all the c_i are strictly positive integers.

Category:Definitions/Combinatorics"
Definition:Composition,Composition,"Let $R = a : b$ be a ratio.

Then the composition of $R$ is the ratio $a + b : b$.



:


Category:Definitions/Euclidean Algebra",Definition:Composition of Ratio,['Definitions/Euclidean Algebra'],"Let R = a : b be a ratio.

Then the composition of R is the ratio a + b : b.



:


Category:Definitions/Euclidean Algebra"
Definition:Cone,Cone,"A cone is a three-dimensional geometric figure which consists of the set of all straight lines joining the boundary of a plane figure $PQR$ to a point $A$ not in the same plane of $PQR$:


:",Definition:Cone (Geometry),['Definitions/Cones'],"A cone is a three-dimensional geometric figure which consists of the set of all straight lines joining the boundary of a plane figure PQR to a point A not in the same plane of PQR:


:"
Definition:Cone,Cone,"Let $\mathbf C$ be a metacategory.

Let $D: \mathbf J \to \mathbf C$ be a $\mathbf J$-diagram in $\mathbf C$.


A cone to $D$ comprises an object $C$ of $\mathbf C$, and a morphism:

:$c_j: C \to D_j$

for each object of $\mathbf J$, such that for each morphism $\alpha: i \to j$ of $\mathbf J$:

::$\begin{xy}\xymatrix@+0.5em@L+2px{
 C
  \ar[d]_*+{c_i}
  \ar[dr]^*+{c_j}

\\
 D_i
  \ar[r]_*+{D_\alpha}
&
 D_j
}\end{xy}$

is a commutative diagram.

",Definition:Cone (Category Theory),['Definitions/Category Theory'],"Let 𝐂 be a metacategory.

Let D: 𝐉→𝐂 be a 𝐉-diagram in 𝐂.


A cone to D comprises an object C of 𝐂, and a morphism:

:c_j: C → D_j

for each object of 𝐉, such that for each morphism α: i → j of 𝐉:

::@+0.5em@L+2px
 C
  [d]_*+c_i[dr]^*+c_j

 D_i
  [r]_*+D_α   
 D_j

is a commutative diagram.

"
Definition:Congruence,Congruence,"In the field of Euclidean geometry, two geometric figures are congruent :

:they are, informally speaking, both ""the same size and shape""

:they differ only in position in space

:one figure can be overlaid on the other figure with a series of rotations, translations, and reflections.


Specifically:
:all corresponding angles of the congruent figures must have the same measurement
:all corresponding sides of the congruent figures must be be the same length.",Definition:Congruence (Geometry),"['Definitions/Congruence (Geometry)', 'Definitions/Geometry']","In the field of Euclidean geometry, two geometric figures are congruent :

:they are, informally speaking, both ""the same size and shape""

:they differ only in position in space

:one figure can be overlaid on the other figure with a series of rotations, translations, and reflections.


Specifically:
:all corresponding angles of the congruent figures must have the same measurement
:all corresponding sides of the congruent figures must be be the same length."
Definition:Congruence,Congruence,"Let $\struct {X, d}$ be a metric space.

Two subsets $A, B \subseteq X$ of $X$ are said to be congruent  there exists an isometry $f: X \to X$ such that $\map {f^\to} A = B$.

Such an isometry is called a congruence.",Definition:Congruence (Metric Spaces),['Definitions/Metric Spaces'],"Let X, d be a metric space.

Two subsets A, B ⊆ X of X are said to be congruent  there exists an isometry f: X → X such that f^→ A = B.

Such an isometry is called a congruence."
Definition:Congruence,Congruence,"Let $\struct {R, +, \circ}$ be a ring, and let $J$ be an ideal of $R$.


The notation:

:$a \equiv b \pmod J$

is used to mean:

:$a + \paren {-b} \in J$",Definition:Congruence Modulo an Ideal,['Definitions/Ideal Theory'],"Let R, +, ∘ be a ring, and let J be an ideal of R.


The notation:

:a ≡ b  J

is used to mean:

:a + -b∈ J"
Definition:Congruence,Congruence,"Let $\struct {S, \circ}$ be an algebraic structure.

Let $\RR$ be an equivalence relation on $S$.


Then $\RR$ is a congruence relation for $\circ$ :

:$\forall x_1, x_2, y_1, y_2 \in S: \paren {x_1 \mathrel \RR x_2} \land \paren {y_1 \mathrel \RR y_2} \implies \paren {x_1 \circ y_1} \mathrel \RR \paren {x_2 \circ y_2}$",Definition:Congruence Relation,"['Definitions/Abstract Algebra', 'Definitions/Equivalence Relations']","Let S, ∘ be an algebraic structure.

Let  be an equivalence relation on S.


Then  is a congruence relation for ∘ :

:∀ x_1, x_2, y_1, y_2 ∈ S: x_1  x_2y_1  y_2x_1 ∘ y_1x_2 ∘ y_2"
Definition:Congruence,Congruence,"Let $R$ be a commutative ring with unity.

Let $n$ be a positive integer.

Let $\mathbf A$ and $\mathbf B$ be square matrices of order $n$ over $R$.


Then:
:$\mathbf A$ and $\mathbf B$ are congruent
:
:there exists an invertible matrix $\mathbf P \in R^{n \times n}$ such that $\mathbf B = \mathbf P^\intercal \mathbf A \mathbf P$
where $\mathbf P^\intercal$ denotes the transpose of $\mathbf P$.",Definition:Matrix Congruence,"['Definitions/Matrix Congruence', 'Definitions/Matrix Equivalence', 'Definitions/Matrix Algebra', 'Definitions/Linear Algebra']","Let R be a commutative ring with unity.

Let n be a positive integer.

Let 𝐀 and 𝐁 be square matrices of order n over R.


Then:
:𝐀 and 𝐁 are congruent
:
:there exists an invertible matrix 𝐏∈ R^n × n such that 𝐁 = 𝐏^⊺𝐀𝐏
where 𝐏^⊺ denotes the transpose of 𝐏."
Definition:Conjugate,Conjugate,"The conjugate of an angle $\theta$ is the angle $\phi$ such that:
:$\theta + \phi = 2 \pi$
where $\theta$ and $\pi$ are expressed in radians.

That is, it is the angle that makes the given angle equal to a full angle.


Equivalently, the conjugate of an angle $\theta$ is the angle $\phi$ such that:
:$\theta + \phi = 360 \degrees$
where $\theta$ and $\pi$ are expressed in degrees.


Thus, conjugate angles are two angles whose measures add up to the measure of $4$ right angles.

That is, their measurements add up to $360$ degrees or $2 \pi$ radians.",Definition:Conjugate Angles,"['Definitions/Conjugate Angles', 'Definitions/Angles']","The conjugate of an angle θ is the angle ϕ such that:
:θ + ϕ = 2 π
where θ and π are expressed in radians.

That is, it is the angle that makes the given angle equal to a full angle.


Equivalently, the conjugate of an angle θ is the angle ϕ such that:
:θ + ϕ = 360
where θ and π are expressed in degrees.


Thus, conjugate angles are two angles whose measures add up to the measure of 4 right angles.

That is, their measurements add up to 360 degrees or 2 π radians."
Definition:Conjugate,Conjugate," 

:


Consider a hyperbola $K$ whose foci are $F_1$ and $F_2$.


Let $PQ$ and $RS$ be line segments constructed through the vertices of $K$ parallel to the minor axis of $K$ and intersecting the asymptotes of $K$ at $P$, $Q$, $R$ and $S$ as above.

Construct the line segments $PR$ and $QS$.

Let $C_1$ and $C_2$ be the points of intersection of $PR$ and $QS$ with the minor axis of $K$.


The conjugate axis of $K$ is the line segment $C_1 C_2$.",Definition:Hyperbola/Conjugate Axis,['Definitions/Hyperbolas']," 

:


Consider a hyperbola K whose foci are F_1 and F_2.


Let PQ and RS be line segments constructed through the vertices of K parallel to the minor axis of K and intersecting the asymptotes of K at P, Q, R and S as above.

Construct the line segments PR and QS.

Let C_1 and C_2 be the points of intersection of PR and QS with the minor axis of K.


The conjugate axis of K is the line segment C_1 C_2."
Definition:Conjugate,Conjugate,"Let $\CC$ be a circle.

Let $P$ and $Q$ be points in the plane of $\CC$.

Let:
:$P$ lie on the polar of $Q$
:$Q$ lie on the polar of $P$.


$P$ and $Q$ are known as conjugate points with respect to $\CC$.
",Definition:Conjugate Points (Geometry),"['Definitions/Conjugate Points', 'Definitions/Polars of Points']","Let  be a circle.

Let P and Q be points in the plane of .

Let:
:P lie on the polar of Q
:Q lie on the polar of P.


P and Q are known as conjugate points with respect to .
"
Definition:Conjugate,Conjugate,"Let $\CC$ be a circle.

Let $\PP$ and $\QQ$ be the straight lines in the plane of $\CC$.


Let $P$ and $Q$ be the poles of $\PP$ and $\QQ$ with respect to $\CC$ respectively.

Let $P$ and $Q$ be such that $P$ lies on $\QQ$ and $Q$ lies on $\PP$.

Then $\PP$ and $\QQ$ are known as conjugate lines with respect to $\CC$.",Definition:Conjugate Lines,"['Definitions/Conjugate Lines', 'Definitions/Polars of Points', 'Definitions/Conic Sections']","Let  be a circle.

Let  and  be the straight lines in the plane of .


Let P and Q be the poles of  and  with respect to  respectively.

Let P and Q be such that P lies on  and Q lies on .

Then  and  are known as conjugate lines with respect to ."
Definition:Conjugate,Conjugate,"Let $K$ be a conic section.

Let $D_1$ and $D_2$ be diameters of $K$ such that:
:$D_1$ belongs to the system of parallel chords whose midpoints define $D_2$
and:
:$D_2$ belongs to the system of parallel chords whose midpoints define $D_1$.

Then $D_1$ and $D_2$ are known as conjugate diameters.",Definition:Conjugate Diameters,"['Definitions/Conjugate Diameters', 'Definitions/Conic Sections']","Let K be a conic section.

Let D_1 and D_2 be diameters of K such that:
:D_1 belongs to the system of parallel chords whose midpoints define D_2
and:
:D_2 belongs to the system of parallel chords whose midpoints define D_1.

Then D_1 and D_2 are known as conjugate diameters."
Definition:Conjugate,Conjugate,"Let $A = \struct {A_F, \oplus}$ be an algebra over a field $F$.

Let $C: A_F \to A_F$ be a conjugation on $A$.

Let $a \in A$.


Then $\map C a$ is called the conjugate of $a$.",Definition:Conjugation on Algebra/Conjugate,['Definitions/Conjugations on Algebras'],"Let A = A_F, ⊕ be an algebra over a field F.

Let C: A_F → A_F be a conjugation on A.

Let a ∈ A.


Then C a is called the conjugate of a."
Definition:Conjugate,Conjugate,"Let $\alpha = r + s \sqrt n$ be a quadratic irrational.


Then its conjugate is defined as:
:$\tilde \alpha = r - s \sqrt n$


Thus $\alpha$ and $\tilde \alpha$ are known as conjugate quadratic irrationals.


Notation may vary.",Definition:Conjugate of Quadratic Irrational,['Definitions/Quadratic Irrationals'],"Let α = r + s √(n) be a quadratic irrational.


Then its conjugate is defined as:
:α̃= r - s √(n)


Thus α and α̃ are known as conjugate quadratic irrationals.


Notation may vary."
Definition:Conjugate,Conjugate,"Let $z = a + i b$ be a complex number.


Then the (complex) conjugate of $z$ is denoted $\overline z$ and is defined as:

:$\overline z := a - i b$


That is, you get the complex conjugate of a complex number by negating its imaginary part.


=== Complex Conjugation ===
",Definition:Complex Conjugate,"['Definitions/Complex Conjugates', 'Definitions/Complex Numbers', 'Definitions/Complex Analysis']","Let z = a + i b be a complex number.


Then the (complex) conjugate of z is denoted z and is defined as:

:z := a - i b


That is, you get the complex conjugate of a complex number by negating its imaginary part.


=== Complex Conjugation ===
"
Definition:Connected,Connected,"Let $\RR \subseteq S \times S$ be a relation on a set $S$.


Then $\RR$ is connected :
:$\forall a, b \in S: a \ne b \implies \tuple {a, b} \in \RR \lor \tuple {b, a} \in \RR$


That is,  every pair of distinct elements is comparable.",Definition:Connected Relation,"['Definitions/Connected Relations', 'Definitions/Relation Theory']","Let ⊆ S × S be a relation on a set S.


Then  is connected :
:∀ a, b ∈ S: a  b a, b∈b, a∈


That is,  every pair of distinct elements is comparable."
Definition:Connected,Connected,"The connected sum of two manifolds $A^n, B^n$ of dimension $n$ is defined as follows:

Let $\Bbb D^n$ be a closed n-disk.

Let $\alpha: \Bbb D^n \to A^n$ be a continuous (or, in the case of smooth manifolds, a smooth) injection.

Let $\beta: \Bbb D^n \to B^n$ be a similar function.  


Define the set:
:$S = \paren {A^n \setminus \map \alpha {\paren {\Bbb D^n}^\circ} } \cup \paren {B^n \setminus \map \beta {\paren {\Bbb D^n}^\circ} }$
where:
:$\setminus$ denotes set difference
:$\paren {\Bbb D^n}^\circ$ denotes the interior of $\Bbb  D^n$.


Define an equivalence relation $\sim$ on $S$ as:
:$x \sim y \iff \paren {\paren {x = y} \lor \paren {\map {\alpha^{-1} } x = \map {\beta^{-1} } y} }$


Since the interiors of the disks were removed from the manifolds, it necessarily follows that:
:$\map {\alpha^{-1} } x, \map {\beta^{-1} } y \in \partial \Bbb D^n$


The connected sum $A^n \# B^n$ is defined as the quotient space of $S$ under $\sim$.

Category:Definitions/Topology",Definition:Connected Sum,['Definitions/Topology'],"The connected sum of two manifolds A^n, B^n of dimension n is defined as follows:

Let D^n be a closed n-disk.

Let α:  D^n → A^n be a continuous (or, in the case of smooth manifolds, a smooth) injection.

Let β:  D^n → B^n be a similar function.  


Define the set:
:S = A^n ∖α D^n^∘∪B^n ∖β D^n^∘
where:
:∖ denotes set difference
:D^n^∘ denotes the interior of D^n.


Define an equivalence relation ∼ on S as:
:x ∼ y x = yα^-1 x = β^-1 y


Since the interiors of the disks were removed from the manifolds, it necessarily follows that:
:α^-1 x, β^-1 y ∈∂ D^n


The connected sum A^n # B^n is defined as the quotient space of S under ∼.

Category:Definitions/Topology"
Definition:Connected,Connected,"Let $G$ be a graph.

Two vertices $u, v \in G$ are connected  either:

:$(1): \quad u = v$
:$(2): \quad u \ne v$, and there exists a walk between them.
Let $G = \struct {V, E}$ be a graph.

Then $G$ is disconnected  it is not connected.

That is,  there exists (at least) two vertices $u, v \in V$ such that $u$ and $v$ are not connected.
",Definition:Connected (Graph Theory)/Graph,['Definitions/Connectedness (Graph Theory)'],"Let G be a graph.

Two vertices u, v ∈ G are connected  either:

:(1):    u = v
:(2):    u  v, and there exists a walk between them.
Let G = V, E be a graph.

Then G is disconnected  it is not connected.

That is,  there exists (at least) two vertices u, v ∈ V such that u and v are not connected.
"
Definition:Connected,Connected,"Let $G$ be a graph.

Two vertices $u, v \in G$ are connected  either:

:$(1): \quad u = v$
:$(2): \quad u \ne v$, and there exists a walk between them.",Definition:Connected (Graph Theory)/Vertices,"['Definitions/Connectedness (Graph Theory)', 'Definitions/Vertices of Graphs']","Let G be a graph.

Two vertices u, v ∈ G are connected  either:

:(1):    u = v
:(2):    u  v, and there exists a walk between them."
Definition:Connected,Connected,,Definition:Disconnected,[],
Definition:Consequence,Consequence,"Let $\mathscr P$ be a proof system for a formal language $\LL$.

Let $\FF$ be a collection of WFFs of $\LL$.


Denote with $\map {\mathscr P} \FF$ the proof system obtained from $\mathscr P$ by adding all the WFFs from $\FF$ as axioms.

Let $\phi$ be a theorem of $\map {\mathscr P} \FF$.


Then $\phi$ is called a provable consequence of $\FF$, and this is denoted as:

:$\FF \vdash_{\mathscr P} \phi$


Note in particular that for $\FF = \O$, this notation agrees with the notation for a $\mathscr P$-theorem:

:$\vdash_{\mathscr P} \phi$",Definition:Provable Consequence,"['Definitions/Provable Consequences', 'Definitions/Logical Implication', 'Definitions/Proof Systems']","Let 𝒫 be a proof system for a formal language .

Let  be a collection of WFFs of .


Denote with 𝒫 the proof system obtained from 𝒫 by adding all the WFFs from  as axioms.

Let ϕ be a theorem of 𝒫.


Then ϕ is called a provable consequence of , and this is denoted as:

:⊢_𝒫ϕ


Note in particular that for = Ø, this notation agrees with the notation for a 𝒫-theorem:

:⊢_𝒫ϕ"
Definition:Consequence,Consequence,"Let $\mathscr M$ be a formal semantics for a formal language $\LL$.

Let $\FF$ be a collection of WFFs of $\LL$.


Let $\map {\mathscr M} \FF$ be the formal semantics obtained from $\mathscr M$ by retaining only the structures of $\mathscr M$ that are models of $\FF$.

Let $\phi$ be a tautology for $\map {\mathscr M} \FF$.


Then $\phi$ is called a semantic consequence of $\FF$, and this is denoted as:

:$\FF \models_{\mathscr M} \phi$


That is to say, $\phi$ is a semantic consequence of $\FF$ , for each $\mathscr M$-structure $\MM$:

:$\MM \models_{\mathscr M} \FF$ implies $\MM \models_{\mathscr M} \phi$

where $\models_{\mathscr M}$ is the models relation.


Note in particular that for $\FF = \O$, the notation agrees with the notation for a $\mathscr M$-tautology:

:$\models_{\mathscr M} \phi$


The concept naturally generalises to sets of formulas $\GG$ on the :

:$\FF \models_{\mathscr M} \GG$

 $\FF \models_{\mathscr M} \phi$ for every $\phi \in \GG$.





",Definition:Semantic Consequence,"['Definitions/Semantic Consequences', 'Definitions/Formal Semantics', 'Definitions/Logical Implication']","Let ℳ be a formal semantics for a formal language .

Let  be a collection of WFFs of .


Let ℳ be the formal semantics obtained from ℳ by retaining only the structures of ℳ that are models of .

Let ϕ be a tautology for ℳ.


Then ϕ is called a semantic consequence of , and this is denoted as:

:_ℳϕ


That is to say, ϕ is a semantic consequence of  , for each ℳ-structure :

:_ℳ implies _ℳϕ

where _ℳ is the models relation.


Note in particular that for = Ø, the notation agrees with the notation for a ℳ-tautology:

:_ℳϕ


The concept naturally generalises to sets of formulas  on the :

:_ℳ

 _ℳϕ for every ϕ∈.





"
Definition:Consequence,Consequence,"A consequence is a state in a game which results from a move made by a player in that game made according to the rules.
",Definition:Consequence (Game Theory),['Definitions/Game Theory'],"A consequence is a state in a game which results from a move made by a player in that game made according to the rules.
"
Definition:Consequence,Consequence,"A consequence is a state in a game which results from a move made by a player in that game made according to the rules.
Let $G$ be a game.

Let $P$ be a player of $G$.

Let $A$ be the set of moves available to $P$.

Let $C$ be the set of consequences of those moves.


A consequence function for $P$ is a mapping from the set $A$ to the set $C$:
:$g: A \to C$
",Definition:Consequence Function,['Definitions/Game Theory'],"A consequence is a state in a game which results from a move made by a player in that game made according to the rules.
Let G be a game.

Let P be a player of G.

Let A be the set of moves available to P.

Let C be the set of consequences of those moves.


A consequence function for P is a mapping from the set A to the set C:
:g: A → C
"
Definition:Consistent,Consistent,"A system of simultaneous equations is referred to as consistent  it has at least one solution.

That is,  there exists a set of values for its variables such that all the equations are satisfied.


=== Inconsistent ===

A set of equations is described as inconsistent  they are not consistent

That is, there exists no set of values for its variables such that all the equations are satisfied.
",Definition:Consistent Simultaneous Equations,"['Definitions/Consistent Simultaneous Equations', 'Definitions/Simultaneous Equations']","A system of simultaneous equations is referred to as consistent  it has at least one solution.

That is,  there exists a set of values for its variables such that all the equations are satisfied.


=== Inconsistent ===

A set of equations is described as inconsistent  they are not consistent

That is, there exists no set of values for its variables such that all the equations are satisfied.
"
Definition:Consistent,Consistent,"Let $X_1, X_2, \ldots, X_n$ be random variables.

Let the joint distribution of $X_1, X_2, \ldots, X_n$ be indexed by a population parameter $\theta$.

Let $\hat \theta$ be an estimator of $\theta$.

Then $\hat \theta$ is consistent :
:$\ds \lim_{n \mathop \to \infty} \map \Pr {\size {\hat \theta - \theta} \ge \epsilon} = 0$
for all $\epsilon > 0$.",Definition:Consistent Estimator,"['Definitions/Consistent Estimators', 'Definitions/Estimators']","Let X_1, X_2, …, X_n be random variables.

Let the joint distribution of X_1, X_2, …, X_n be indexed by a population parameter θ.

Let θ̂ be an estimator of θ.

Then θ̂ is consistent :
:lim_n →∞θ̂- θ≥ϵ = 0
for all ϵ > 0."
Definition:Content,Content,"Let $f \in \Z \sqbrk X$ be a polynomial with integer coefficients.

Then the content of $f$, denoted $\cont f$, is the greatest common divisor of the coefficients of $f$.",Definition:Content of Polynomial/Integer,"['Definitions/Polynomial Theory', 'Definitions/Content of Polynomial']","Let f ∈ X be a polynomial with integer coefficients.

Then the content of f, denoted f, is the greatest common divisor of the coefficients of f."
Definition:Content,Content,"Let $f \in \Q \sqbrk X$ be a polynomial with rational coefficients.


The content of $f$ is defined as:
:$\cont f := \dfrac {\cont {n f} } n$
where $n \in \N$ is such that $n f \in \Z \sqbrk X$.",Definition:Content of Polynomial/Rational,['Definitions/Content of Polynomial'],"Let f ∈ X be a polynomial with rational coefficients.


The content of f is defined as:
:f := n f n
where n ∈ is such that n f ∈ X."
Definition:Continuous,Continuous,"The concept of continuity makes precise the intuitive notion that a function has no ""jumps"" at a given point.

Loosely speaking, in the case of a real function, continuity at a point is defined as the property that the graph of the function does not have a ""break"" at the point.

Thus, a small change in the independent variable causes a similar small change in the dependent variable

This concept appears throughout mathematics and correspondingly has many variations and generalizations.",Definition:Continuous Mapping,"['Definitions/Continuous Mappings', 'Definitions/Mappings', 'Definitions/Continuity']","The concept of continuity makes precise the intuitive notion that a function has no ""jumps"" at a given point.

Loosely speaking, in the case of a real function, continuity at a point is defined as the property that the graph of the function does not have a ""break"" at the point.

Thus, a small change in the independent variable causes a similar small change in the dependent variable

This concept appears throughout mathematics and correspondingly has many variations and generalizations."
Definition:Continuous,Continuous,"Continuous geometry is a branch of the projective geometry which investigates spaces whose dimension can range over the continuous interval $\closedint 0 1$.
Let $a, b \in \R$.

The closed (real) interval from $a$ to $b$ is defined as:

:$\closedint a b = \set {x \in \R: a \le x \le b}$
",Definition:Continuous Geometry,"['Definitions/Continuous Geometry', 'Definitions/Projective Geometry', 'Definitions/Branches of Mathematics']","Continuous geometry is a branch of the projective geometry which investigates spaces whose dimension can range over the continuous interval 0 1.
Let a, b ∈.

The closed (real) interval from a to b is defined as:

:a b = x ∈: a ≤ x ≤ b
"
Definition:Continuous,Continuous,"Let $L = \left({X, \preceq}\right)$ be an ordered set.

Let $S = \left({Y, \preceq'}\right)$ be an ordered subset of $L$.


Then $S$ is continuous lattice subframe of $L$ 
:$S$ inherits infima and directed suprema.",Definition:Continuous Lattice Subframe,['Definitions/Order Theory'],"Let L = (X, ≼) be an ordered set.

Let S = (Y, ≼') be an ordered subset of L.


Then S is continuous lattice subframe of L 
:S inherits infima and directed suprema."
Definition:Continuous,Continuous,"A continuous variable is a variable which can take any value between two given values.
Data which can be described with a continuous variable obtained by the process of measurement are known as continuous data.
",Definition:Sample Statistic/Continuous,['Definitions/Sample Statistics'],"A continuous variable is a variable which can take any value between two given values.
Data which can be described with a continuous variable obtained by the process of measurement are known as continuous data.
"
Definition:Continuous,Continuous,"Let $f: \R \to \R$ be a real function.


Then $f$ is everywhere continuous  $f$ is continuous at every point in $\R$.
",Definition:Continuous Real Function,"['Definitions/Continuous Real Functions', 'Definitions/Continuous Functions', 'Definitions/Continuous Mappings', 'Definitions/Real Functions']","Let f: → be a real function.


Then f is everywhere continuous  f is continuous at every point in .
"
Definition:Continuous,Continuous,"Let $A \subseteq \R$ be an open subset of the real numbers $\R$.

Let $f: A \to \R$ be a real function.


Let $x_0 \in A$. 

Then $f$ is said to be left-continuous at $x_0$  the limit from the left of $\map f x$ as $x \to x_0$ exists and:

:$\ds \lim_{\substack {x \mathop \to x_0^- \\ x_0 \mathop \in A} } \map f x = \map f {x_0}$

where $\ds \lim_{x \mathop \to x_0^-}$ is a limit from the left.


Furthermore, $f$ is said to be left-continuous :

:$\forall x_0 \in A$, $f$ is left-continuous at $x_0$",Definition:Continuous Real Function/Left-Continuous,['Definitions/Continuous Real Functions'],"Let A ⊆ be an open subset of the real numbers .

Let f: A → be a real function.


Let x_0 ∈ A. 

Then f is said to be left-continuous at x_0  the limit from the left of f x as x → x_0 exists and:

:lim_x → x_0^- 
 x_0 ∈ A f x =  f x_0

where lim_x → x_0^- is a limit from the left.


Furthermore, f is said to be left-continuous :

:∀ x_0 ∈ A, f is left-continuous at x_0"
Definition:Continuous,Continuous,"Let $S \subseteq \R$ be an open subset of the real numbers $\R$.

Let $f: S \to \R$ be a real function.


Let $x_0 \in S$. 

Then $f$ is said to be right-continuous at $x_0$  the limit from the right of $\map f x$ as $x \to x_0$ exists and:

:$\ds \lim_{\substack {x \mathop \to x_0^+ \\ x_0 \mathop \in A}} \map f x = \map f {x_0}$

where $\ds \lim_{x \mathop \to x_0^+}$ is a limit from the right.


Furthermore, $f$ is said to be right-continuous :

:$\forall x_0 \in S$, $f$ is right-continuous at $x_0$",Definition:Continuous Real Function/Right-Continuous,['Definitions/Continuous Real Functions'],"Let S ⊆ be an open subset of the real numbers .

Let f: S → be a real function.


Let x_0 ∈ S. 

Then f is said to be right-continuous at x_0  the limit from the right of f x as x → x_0 exists and:

:lim_x → x_0^+ 
 x_0 ∈ A f x =  f x_0

where lim_x → x_0^+ is a limit from the right.


Furthermore, f is said to be right-continuous :

:∀ x_0 ∈ S, f is right-continuous at x_0"
Definition:Continuous,Continuous,"Let $\R^n$ be the cartesian $n$-space.

Let $f: \R^n \to \R$ be a real-valued function on $\R^n$.


Then $f$ is continuous on $\R^n$ :
:$\forall a \in \R^n: \forall \epsilon \in \R_{>0}: \exists \delta \in \R_{>0}: \forall x \in \R^n: \map d {x, a} < \delta \implies \size {\map f x - \map f a} < \epsilon$
where $\map d {x, a}$ is the distance function on $\R^n$:

:$\ds d: \R^n \to \R: \map d {x, y} := \sqrt {\sum_{i \mathop = 1}^n \paren {x_i - y_i}^2}$

where $x = \tuple {x_1, x_2, \ldots, x_n}, y = \tuple {y_1, y_2, \ldots, y_n}$ are general elements of $\R^n$.",Definition:Continuous Real-Valued Vector Function,"['Definitions/Continuity', 'Definitions/Real-Valued Functions']","Let ^n be the cartesian n-space.

Let f: ^n → be a real-valued function on ^n.


Then f is continuous on ^n :
:∀ a ∈^n: ∀ϵ∈_>0: ∃δ∈_>0: ∀ x ∈^n:  d x, a < δ f x -  f a < ϵ
where d x, a is the distance function on ^n:

:d: ^n →:  d x, y := √(∑_i  = 1^n x_i - y_i^2)

where x = x_1, x_2, …, x_n, y = y_1, y_2, …, y_n are general elements of ^n."
Definition:Continuous,Continuous,"
",Definition:Continuity/Functional,['Definitions/Calculus of Variations'],"
"
Definition:Continuous,Continuous,"Let $T_1 = \struct {S_1, \tau_1}$ and $T_2 = \struct {S_2, \tau_2}$ be topological spaces.

Let $f: S_1 \to S_2$ be a mapping from $S_1$ to $S_2$.
Let $T_1 = \struct {S_1, \tau_1}$ and $T_2 = \struct {S_2, \tau_2}$ be topological spaces.

Let $f: S_1 \to S_2$ be a mapping from $S_1$ to $S_2$.
Let $T_1 = \struct {S_1, \tau_1}$ and $T_2 = \struct {S_2, \tau_2}$ be topological spaces.

Let $f: S_1 \to S_2$ be a mapping from $S_1$ to $S_2$.
",Definition:Continuous Extension,['Definitions/Continuity'],"Let T_1 = S_1, τ_1 and T_2 = S_2, τ_2 be topological spaces.

Let f: S_1 → S_2 be a mapping from S_1 to S_2.
Let T_1 = S_1, τ_1 and T_2 = S_2, τ_2 be topological spaces.

Let f: S_1 → S_2 be a mapping from S_1 to S_2.
Let T_1 = S_1, τ_1 and T_2 = S_2, τ_2 be topological spaces.

Let f: S_1 → S_2 be a mapping from S_1 to S_2.
"
Definition:Continuous,Continuous,"Let $T_1 = \struct {S_1, \tau_1}$ and $T_2 = \struct {S_2, \tau_2}$ be topological spaces.

Let $f: S_1 \to S_2$ be a mapping from $S_1$ to $S_2$.",Definition:Continuous Mapping (Topology),"['Definitions/Continuous Mappings (Topology)', 'Definitions/Continuous Mappings', 'Definitions/Continuity', 'Definitions/Topology']","Let T_1 = S_1, τ_1 and T_2 = S_2, τ_2 be topological spaces.

Let f: S_1 → S_2 be a mapping from S_1 to S_2."
Definition:Continuous,Continuous,"Let $\mathbf C$, $\mathbf D$ be metacategories.

Let $F: \mathbf C \to \mathbf D$ be a functor.


Then $F$ is continuous  for all diagrams $D: \mathbf J \to \mathbf C$ with limit ${\varprojlim \,}_j \, D_j$:

:$\map F {{\varprojlim \,}_j \, D_j} \cong {\varprojlim \,}_j \, F D_j$

where $F D: \mathbf J \to \mathbf D$ is the diagram obtained by composition of $F$ with $D$, and $\mathbf J$ is an arbitrary metacategory.",Definition:Continuous Functor,"['Definitions/Category Theory', 'Definitions/Limits and Colimits']","Let 𝐂, 𝐃 be metacategories.

Let F: 𝐂→𝐃 be a functor.


Then F is continuous  for all diagrams D: 𝐉→𝐂 with limit _j   D_j:

:F  _j   D_j≅ _j   F D_j

where F D: 𝐉→𝐃 is the diagram obtained by composition of F with D, and 𝐉 is an arbitrary metacategory."
Definition:Continuous,Continuous,,Definition:Discontinuous,[],
Definition:Convolution,Convolution,,Definition:Convolution of Real Sequences,['Definitions/Real Analysis'],
Definition:Convolution,Convolution,"Let $\BB^n$ be the Borel $\sigma$-algebra on $\R^n$, and let $\lambda^n$ be Lebesgue measure on $\R^n$.

Let $f, g: \R^n \to \R$ be $\BB^n$-measurable functions such that for all $x \in \R^n$:

:$\ds \int_{\R^n} \map f {x - y} \map g y \rd \map {\lambda^n} y$

is finite.


The convolution of $f$ and $g$, denoted $f * g$, is the mapping defined by:

:$\ds f * g: \R^n \to \R, \map {f * g} x := \int_{\R^n} \map f {x - y} \map g y \rd \map {\lambda^n} y$
Let $\mu$ be a measure on the Borel $\sigma$-algebra $\BB^n$ on $\R^n$.

Let $f: \R^n \to \R$ be a $\BB^n$-measurable function such that for all $x \in \R^n$:

:$\ds \int_{\R^n} \map f {x - y} \rd \map \mu y$

is finite.


The convolution of $f$ and $\mu$ is the mapping $f * \mu: \R^n \to \R$ defined as:

:$\ds \forall x \in \R^n: \map {f * \mu} x := \int_{\R^n} \map f {x - y} \rd \map \mu y$
Let $\mu$ and $\nu$ be measures on the Borel $\sigma$-algebra $\BB^n$ on $\R^n$.


The convolution of $\mu$ and $\nu$, denoted $\mu * \nu$, is the measure defined by:

:$\ds \mu * \nu: \BB^n \to \overline \R, \map {\mu * \nu} B := \int \map {\chi_B} {x + y} \map {\rd \mu} x \map {\rd \nu} y$
where $\chi_B$ is the characteristic function of $B$.
",Definition:Convolution (Measure Theory),['Definitions/Measure Theory'],"Let ^n be the Borel σ-algebra on ^n, and let λ^n be Lebesgue measure on ^n.

Let f, g: ^n → be ^n-measurable functions such that for all x ∈^n:

:∫_^n f x - y g y λ^n y

is finite.


The convolution of f and g, denoted f * g, is the mapping defined by:

:f * g: ^n →, f * g x := ∫_^n f x - y g y λ^n y
Let μ be a measure on the Borel σ-algebra ^n on ^n.

Let f: ^n → be a ^n-measurable function such that for all x ∈^n:

:∫_^n f x - yμ y

is finite.


The convolution of f and μ is the mapping f * μ: ^n → defined as:

:∀ x ∈^n: f * μ x := ∫_^n f x - yμ y
Let μ and ν be measures on the Borel σ-algebra ^n on ^n.


The convolution of μ and ν, denoted μ * ν, is the measure defined by:

:μ * ν: ^n →, μ * ν B := ∫χ_Bx + yμ x ν y
where χ_B is the characteristic function of B.
"
Definition:Convolution,Convolution,"Let $\BB^n$ be the Borel $\sigma$-algebra on $\R^n$, and let $\lambda^n$ be Lebesgue measure on $\R^n$.

Let $f, g: \R^n \to \R$ be $\BB^n$-measurable functions such that for all $x \in \R^n$:

:$\ds \int_{\R^n} \map f {x - y} \map g y \rd \map {\lambda^n} y$

is finite.


The convolution of $f$ and $g$, denoted $f * g$, is the mapping defined by:

:$\ds f * g: \R^n \to \R, \map {f * g} x := \int_{\R^n} \map f {x - y} \map g y \rd \map {\lambda^n} y$",Definition:Convolution of Measurable Functions,['Definitions/Measure Theory'],"Let ^n be the Borel σ-algebra on ^n, and let λ^n be Lebesgue measure on ^n.

Let f, g: ^n → be ^n-measurable functions such that for all x ∈^n:

:∫_^n f x - y g y λ^n y

is finite.


The convolution of f and g, denoted f * g, is the mapping defined by:

:f * g: ^n →, f * g x := ∫_^n f x - y g y λ^n y"
Definition:Convolution,Convolution,"Let $\mu$ be a measure on the Borel $\sigma$-algebra $\BB^n$ on $\R^n$.

Let $f: \R^n \to \R$ be a $\BB^n$-measurable function such that for all $x \in \R^n$:

:$\ds \int_{\R^n} \map f {x - y} \rd \map \mu y$

is finite.


The convolution of $f$ and $\mu$ is the mapping $f * \mu: \R^n \to \R$ defined as:

:$\ds \forall x \in \R^n: \map {f * \mu} x := \int_{\R^n} \map f {x - y} \rd \map \mu y$",Definition:Convolution of Measurable Function and Measure,['Definitions/Measure Theory'],"Let μ be a measure on the Borel σ-algebra ^n on ^n.

Let f: ^n → be a ^n-measurable function such that for all x ∈^n:

:∫_^n f x - yμ y

is finite.


The convolution of f and μ is the mapping f * μ: ^n → defined as:

:∀ x ∈^n: f * μ x := ∫_^n f x - yμ y"
Definition:Convolution,Convolution,"Let $\mu$ and $\nu$ be measures on the Borel $\sigma$-algebra $\BB^n$ on $\R^n$.


The convolution of $\mu$ and $\nu$, denoted $\mu * \nu$, is the measure defined by:

:$\ds \mu * \nu: \BB^n \to \overline \R, \map {\mu * \nu} B := \int \map {\chi_B} {x + y} \map {\rd \mu} x \map {\rd \nu} y$
where $\chi_B$ is the characteristic function of $B$.",Definition:Convolution of Measures,['Definitions/Measure Theory'],"Let μ and ν be measures on the Borel σ-algebra ^n on ^n.


The convolution of μ and ν, denoted μ * ν, is the measure defined by:

:μ * ν: ^n →, μ * ν B := ∫χ_Bx + yμ x ν y
where χ_B is the characteristic function of B."
Definition:Convolution,Convolution,"Let $\struct {M, \cdot}$ be a divisor-finite monoid.

Let $\struct {R, +, \times}$ be a non-associative ring.

Let $f, g : M \to R$ be mappings.


The convolution of $f$ and $g$ is the mapping $f * g: M \to R$ defined as:
:$\forall m \in M: \map {\paren {f * g} } m := \ds \sum_{x y \mathop = m} \map f x \times \map g y$
where the summation is over the finite set $\set {\tuple {x, y} \in M^2: x y = m}$.",Definition:Convolution of Mappings on Divisor-Finite Monoid,['Definitions/Monoids'],"Let M, · be a divisor-finite monoid.

Let R, +, × be a non-associative ring.

Let f, g : M → R be mappings.


The convolution of f and g is the mapping f * g: M → R defined as:
:∀ m ∈ M: f * g m := ∑_x y  = m f x × g y
where the summation is over the finite set x, y∈ M^2: x y = m."
Definition:Coterminal,Coterminal,"Coterminal angles are angles which are rotations between the same $2$ lines.

That is, they are angles with the same arms.
",Definition:Coterminal Angles,"['Definitions/Coterminal Angles', 'Definitions/Angles']","Coterminal angles are angles which are rotations between the same 2 lines.

That is, they are angles with the same arms.
"
Definition:Coterminal,Coterminal,Two sides of a polygon that meet at the same vertex are adjacent to each other.,Definition:Polygon/Adjacent/Sides,['Definitions/Adjacent (Polygons)'],Two sides of a polygon that meet at the same vertex are adjacent to each other.
Definition:Coterminal,Coterminal,Two edges of a polyhedron that intersect at a particular vertex are referred to as adjacent to each other.,Definition:Polyhedron/Adjacent/Edge to Edge,['Definitions/Adjacent (Polyhedra)'],Two edges of a polyhedron that intersect at a particular vertex are referred to as adjacent to each other.
Definition:Couple,Couple,"A couple, in the context of mechanics, is a system of $2$ forces that are:
:equal in magnitude
:exactly opposite in direction
:with different lines of action.",Definition:Couple (Mechanics),"['Definitions/Couples (Mechanics)', 'Definitions/Mechanics']","A couple, in the context of mechanics, is a system of 2 forces that are:
:equal in magnitude
:exactly opposite in direction
:with different lines of action."
Definition:Couple,Couple,"A Tusi couple is a hypocycloid with $2$ cusps.


:",Definition:Tusi Couple,['Definitions/Hypocycloids'],"A Tusi couple is a hypocycloid with 2 cusps.


:"
Definition:Critical Point,Critical Point,"Let $f: X \to Y$ be a smooth map of manifolds.


A point $x \in X$ is called a critical point of $f$  $\d f_x: T_x \sqbrk X \to T_y \sqbrk Y$ is not surjective at $x$.



Category:Definitions/Topology",Definition:Critical Point (Topology),['Definitions/Topology'],"Let f: X → Y be a smooth map of manifolds.


A point x ∈ X is called a critical point of f  f̣_x: T_x  X → T_y  Y is not surjective at x.



Category:Definitions/Topology"
Definition:Critical Point,Critical Point,"Let $M$ be a smooth manifold.

Let $f \in \map {\CC^\infty} M : M \to \R$ be a smooth real-valued function.

Let $p \in M$ be a base point in $M$.

Let $\rd f_p$ be the differential of $f$ at $p$.

Suppose $\rd f_p = 0$.


Then $p$ is called a critical point of $f$.
",Definition:Critical Point/Smooth Manifold,['Definitions/Smooth Manifolds'],"Let M be a smooth manifold.

Let f ∈^∞ M : M → be a smooth real-valued function.

Let p ∈ M be a base point in M.

Let f_p be the differential of f at p.

Suppose f_p = 0.


Then p is called a critical point of f.
"
Definition:Critical Point,Critical Point,"Let $\struct {M, g}$ be a Riemannian manifold.

Let $I = \closedint a b$ be a closed real interval.

Let $J \subseteq \R$ be an open real interval.

Let $\gamma : I \to M$ be an admissible curve.

Let $L_g$ be the Riemannian length of some admissible curve.

Let $\Gamma : J \times I \to M$ be the proper variation of $\gamma$ such that:

:$\forall s \in J, \forall t \in I : \tuple {s, t} \stackrel {\Gamma}{\mapsto} \map {\Gamma_s} t$

Suppose:

:$\forall \Gamma : \ds \dfrac d {d s} \map {L_g} {\Gamma_s} = 0$


Then $\gamma$ is called the critical point of $L_g$.",Definition:Critical Point of Riemannian Length,"['Definitions/Riemannian Manifolds', 'Definitions/Curves']","Let M, g be a Riemannian manifold.

Let I =  a b be a closed real interval.

Let J ⊆ be an open real interval.

Let γ : I → M be an admissible curve.

Let L_g be the Riemannian length of some admissible curve.

Let Γ : J × I → M be the proper variation of γ such that:

:∀ s ∈ J, ∀ t ∈ I : s, tΓ↦Γ_s t

Suppose:

:∀Γ :  d d sL_gΓ_s = 0


Then γ is called the critical point of L_g."
Definition:Critical Point,Critical Point,"A critical point is a point on a graph where the curve has a vertical tangent.

That is, where limit of the derivative tends to infinity.
",Definition:Critical Point (Analysis),"['Definitions/Critical Points (Analysis)', 'Definitions/Real Analysis']","A critical point is a point on a graph where the curve has a vertical tangent.

That is, where limit of the derivative tends to infinity.
"
Definition:Cubic,Cubic,"Cubic is an adjective which means in the shape of a cube.
",Definition:Cubic (Geometry),['Definitions/Cubes'],"Cubic is an adjective which means in the shape of a cube.
"
Definition:Cubic,Cubic,"A cubic polynomial is a polynomial of degree $3$.
",Definition:Cubic Polynomial,['Definitions/Polynomial Theory'],"A cubic polynomial is a polynomial of degree 3.
"
Definition:Cubic,Cubic,"A cubic graph is a $3$-regular graph, that is, a graph whose vertices all have degree $3$.",Definition:Cubic Graph,"['Definitions/Cubic Graphs', 'Definitions/Regular Graphs', 'Definitions/Graph Theory']","A cubic graph is a 3-regular graph, that is, a graph whose vertices all have degree 3."
Definition:Cut,Cut,"Let $r \in \Q$ be a rational number.

Let $\alpha$ be the cut consisting of all rational numbers $p$ such that $p < r$.


Then $\alpha$ is referred to as a rational cut.


To express the fact that $\alpha$ is a rational cut, the notation $\alpha = r^*$ can be used.
",Definition:Cut (Analysis),"['Definitions/Cuts', 'Definitions/Real Analysis', 'Definitions/Order Theory']","Let r ∈ be a rational number.

Let α be the cut consisting of all rational numbers p such that p < r.


Then α is referred to as a rational cut.


To express the fact that α is a rational cut, the notation α = r^* can be used.
"
Definition:Cut,Cut,"Let $G = \struct {V, E}$ be a graph.


A vertex cut of $G$ is a set of vertices $W \subseteq \map V G$ such that the vertex deletion $G \setminus W$ is disconnected.
Let $G = \struct {V, E}$ be a graph.


A vertex cut of $G$ is a set of vertices $W \subseteq \map V G$ such that the vertex deletion $G \setminus W$ is disconnected.
",Definition:Cut-Vertex,['Definitions/Vertices of Graphs'],"Let G = V, E be a graph.


A vertex cut of G is a set of vertices W ⊆ V G such that the vertex deletion G ∖ W is disconnected.
Let G = V, E be a graph.


A vertex cut of G is a set of vertices W ⊆ V G such that the vertex deletion G ∖ W is disconnected.
"
Definition:Cut,Cut,"Let $T = \struct {S, \tau}$ be a topological space.

Let $H \subseteq S$ be a connected set in $T$ and let $p \in H$.

Let $p \in H$ such that $H \setminus \set p$ is disconnected, where $\setminus$ denotes set difference.


Then $p$ is a cut point of $H$.
",Definition:Cut Point,"['Definitions/Connected Spaces', 'Definitions/Disconnected Sets']","Let T = S, τ be a topological space.

Let H ⊆ S be a connected set in T and let p ∈ H.

Let p ∈ H such that H ∖ p is disconnected, where ∖ denotes set difference.


Then p is a cut point of H.
"
Definition:Cut,Cut,"Let $G = \struct {V, E}$ be a graph.


A vertex cut of $G$ is a set of vertices $W \subseteq \map V G$ such that the vertex deletion $G \setminus W$ is disconnected.",Definition:Vertex Cut,['Definitions/Vertices of Graphs'],"Let G = V, E be a graph.


A vertex cut of G is a set of vertices W ⊆ V G such that the vertex deletion G ∖ W is disconnected."
Definition:Cut,Cut,"Let $G$ be a graph.


An edge cut of $G$ is a set of edges $W \subseteq \map E G$ such that the edge deletion $G \setminus W$ is disconnected.",Definition:Edge Cut,['Definitions/Edges of Graphs'],"Let G be a graph.


An edge cut of G is a set of edges W ⊆ E G such that the edge deletion G ∖ W is disconnected."
Definition:Cycle,Cycle,"Let $S_n$ denote the symmetric group on $n$ letters.

Let $\rho \in S_n$ be a permutation on $S$.


Then $\rho$ is a cyclic permutation of length $k$  there exists $k \in \Z: k > 0$ and $i \in \Z$ such that:
:$(1): \quad k$ is the smallest such that $\map {\rho^k} i = i$

:$(2): \quad \rho$ fixes each $j$ not in $\set {i, \map \rho i, \ldots, \map {\rho^{k - 1} } i}$.


$\rho$ is usually denoted using cycle notation as:
:$\begin{pmatrix} i & \map \rho i & \ldots & \map {\rho^{k - 1} } i \end{pmatrix}$

but some sources introduce it using two-row notation:

:$\begin{pmatrix} a_1 & a_2 & \cdots & a_k & \cdots & i & \cdots \\ a_2 & a_3 & \cdots & a_1 & \cdots & i & \cdots \end{pmatrix}$
Let $S_n$ denote the symmetric group on $n$ letters.

Let $\rho \in S_n$ be a permutation on $S$.


Then $\rho$ is a cyclic permutation of length $k$  there exists $k \in \Z: k > 0$ and $i \in \Z$ such that:
:$(1): \quad k$ is the smallest such that $\map {\rho^k} i = i$

:$(2): \quad \rho$ fixes each $j$ not in $\set {i, \map \rho i, \ldots, \map {\rho^{k - 1} } i}$.


$\rho$ is usually denoted using cycle notation as:
:$\begin{pmatrix} i & \map \rho i & \ldots & \map {\rho^{k - 1} } i \end{pmatrix}$

but some sources introduce it using two-row notation:

:$\begin{pmatrix} a_1 & a_2 & \cdots & a_k & \cdots & i & \cdots \\ a_2 & a_3 & \cdots & a_1 & \cdots & i & \cdots \end{pmatrix}$
Let $S_n$ denote the symmetric group on $n$ letters.

Let $\rho \in S_n$ be a permutation on $S$.


Then $\rho$ is a cyclic permutation of length $k$  there exists $k \in \Z: k > 0$ and $i \in \Z$ such that:
:$(1): \quad k$ is the smallest such that $\map {\rho^k} i = i$

:$(2): \quad \rho$ fixes each $j$ not in $\set {i, \map \rho i, \ldots, \map {\rho^{k - 1} } i}$.


$\rho$ is usually denoted using cycle notation as:
:$\begin{pmatrix} i & \map \rho i & \ldots & \map {\rho^{k - 1} } i \end{pmatrix}$

but some sources introduce it using two-row notation:

:$\begin{pmatrix} a_1 & a_2 & \cdots & a_k & \cdots & i & \cdots \\ a_2 & a_3 & \cdots & a_1 & \cdots & i & \cdots \end{pmatrix}$
Let $S_n$ denote the symmetric group on $n$ letters.

Let $\rho \in S_n$ be a permutation on $S$.


Then $\rho$ is a cyclic permutation of length $k$  there exists $k \in \Z: k > 0$ and $i \in \Z$ such that:
:$(1): \quad k$ is the smallest such that $\map {\rho^k} i = i$

:$(2): \quad \rho$ fixes each $j$ not in $\set {i, \map \rho i, \ldots, \map {\rho^{k - 1} } i}$.


$\rho$ is usually denoted using cycle notation as:
:$\begin{pmatrix} i & \map \rho i & \ldots & \map {\rho^{k - 1} } i \end{pmatrix}$

but some sources introduce it using two-row notation:

:$\begin{pmatrix} a_1 & a_2 & \cdots & a_k & \cdots & i & \cdots \\ a_2 & a_3 & \cdots & a_1 & \cdots & i & \cdots \end{pmatrix}$
Let $S_n$ denote the symmetric group on $n$ letters.

Let $\rho \in S_n$ be a permutation on $S$.


Then $\rho$ is a cyclic permutation of length $k$  there exists $k \in \Z: k > 0$ and $i \in \Z$ such that:
:$(1): \quad k$ is the smallest such that $\map {\rho^k} i = i$

:$(2): \quad \rho$ fixes each $j$ not in $\set {i, \map \rho i, \ldots, \map {\rho^{k - 1} } i}$.


$\rho$ is usually denoted using cycle notation as:
:$\begin{pmatrix} i & \map \rho i & \ldots & \map {\rho^{k - 1} } i \end{pmatrix}$

but some sources introduce it using two-row notation:

:$\begin{pmatrix} a_1 & a_2 & \cdots & a_k & \cdots & i & \cdots \\ a_2 & a_3 & \cdots & a_1 & \cdots & i & \cdots \end{pmatrix}$
",Definition:Permutation on n Letters/Cycle Notation,['Definitions/Permutation Theory'],"Let S_n denote the symmetric group on n letters.

Let ρ∈ S_n be a permutation on S.


Then ρ is a cyclic permutation of length k  there exists k ∈: k > 0 and i ∈ such that:
:(1):    k is the smallest such that ρ^k i = i

:(2):   ρ fixes each j not in i, ρ i, …, ρ^k - 1 i.


ρ is usually denoted using cycle notation as:
:[         i       ρ i         … ρ^k - 1 i ]

but some sources introduce it using two-row notation:

:[ a_1 a_2   ⋯ a_k   ⋯   i   ⋯; a_2 a_3   ⋯ a_1   ⋯   i   ⋯ ]
Let S_n denote the symmetric group on n letters.

Let ρ∈ S_n be a permutation on S.


Then ρ is a cyclic permutation of length k  there exists k ∈: k > 0 and i ∈ such that:
:(1):    k is the smallest such that ρ^k i = i

:(2):   ρ fixes each j not in i, ρ i, …, ρ^k - 1 i.


ρ is usually denoted using cycle notation as:
:[         i       ρ i         … ρ^k - 1 i ]

but some sources introduce it using two-row notation:

:[ a_1 a_2   ⋯ a_k   ⋯   i   ⋯; a_2 a_3   ⋯ a_1   ⋯   i   ⋯ ]
Let S_n denote the symmetric group on n letters.

Let ρ∈ S_n be a permutation on S.


Then ρ is a cyclic permutation of length k  there exists k ∈: k > 0 and i ∈ such that:
:(1):    k is the smallest such that ρ^k i = i

:(2):   ρ fixes each j not in i, ρ i, …, ρ^k - 1 i.


ρ is usually denoted using cycle notation as:
:[         i       ρ i         … ρ^k - 1 i ]

but some sources introduce it using two-row notation:

:[ a_1 a_2   ⋯ a_k   ⋯   i   ⋯; a_2 a_3   ⋯ a_1   ⋯   i   ⋯ ]
Let S_n denote the symmetric group on n letters.

Let ρ∈ S_n be a permutation on S.


Then ρ is a cyclic permutation of length k  there exists k ∈: k > 0 and i ∈ such that:
:(1):    k is the smallest such that ρ^k i = i

:(2):   ρ fixes each j not in i, ρ i, …, ρ^k - 1 i.


ρ is usually denoted using cycle notation as:
:[         i       ρ i         … ρ^k - 1 i ]

but some sources introduce it using two-row notation:

:[ a_1 a_2   ⋯ a_k   ⋯   i   ⋯; a_2 a_3   ⋯ a_1   ⋯   i   ⋯ ]
Let S_n denote the symmetric group on n letters.

Let ρ∈ S_n be a permutation on S.


Then ρ is a cyclic permutation of length k  there exists k ∈: k > 0 and i ∈ such that:
:(1):    k is the smallest such that ρ^k i = i

:(2):   ρ fixes each j not in i, ρ i, …, ρ^k - 1 i.


ρ is usually denoted using cycle notation as:
:[         i       ρ i         … ρ^k - 1 i ]

but some sources introduce it using two-row notation:

:[ a_1 a_2   ⋯ a_k   ⋯   i   ⋯; a_2 a_3   ⋯ a_1   ⋯   i   ⋯ ]
"
Definition:Cycle,Cycle,"An odd cycle is a cycle with odd length, that is, with an odd number of edges.
An even cycle is a cycle with even length, that is, with an even number of edges.
",Definition:Cycle (Graph Theory),"['Definitions/Cycles (Graph Theory)', 'Definitions/Circuits (Graph Theory)', 'Definitions/Paths (Graph Theory)', 'Definitions/Graph Theory']","An odd cycle is a cycle with odd length, that is, with an odd number of edges.
An even cycle is a cycle with even length, that is, with an even number of edges.
"
Definition:Cycle,Cycle,"A cycle, or periodic solution, is a solution of a differential equation which is a periodic function.

Category:Definitions/Differential Equations",Definition:Cycle (Periodic Solution),['Definitions/Differential Equations'],"A cycle, or periodic solution, is a solution of a differential equation which is a periodic function.

Category:Definitions/Differential Equations"
Definition:Cycle,Cycle,"The number of partial denominators in a cycle of a periodic continued fraction is called the cycle length.
",Definition:Periodic Continued Fraction/Cycle,['Definitions/Continued Fractions'],"The number of partial denominators in a cycle of a periodic continued fraction is called the cycle length.
"
Definition:Cyclic,Cyclic,"A cyclic polygon is a polygon $P$ which can be inscribed in a circle.
",Definition:Cyclic Polygon,['Definitions/Polygons'],"A cyclic polygon is a polygon P which can be inscribed in a circle.
"
Definition:Cyclic,Cyclic,"A cyclic quadrilateral is a quadrilateral which can be circumscribed:

:


This is an example of a quadrilateral which cannot be circumscribed, and so therefore is not cyclic:

:
",Definition:Cyclic Quadrilateral,"['Definitions/Quadrilaterals', 'Definitions/Circumscribe']","A cyclic quadrilateral is a quadrilateral which can be circumscribed:

:


This is an example of a quadrilateral which cannot be circumscribed, and so therefore is not cyclic:

:
"
Definition:Decomposition,Decomposition,"A set $S \subseteq \R^n$ is decomposable in $m$ sets $A_1, \ldots, A_m \subset \R^n$  there exist isometries $\phi_1, \ldots, \phi_m: \R^n \to \R^n$ such that:

:$(1):\quad \ds S = \bigcup_{k \mathop = 1}^m \map {\phi_k} {A_k}$ 
:$(2):\quad \forall i \ne j: \map {\phi_i} {A_i} \cap \map {\phi_j} {A_j} = \O$

Such a union is known as a decomposition.

",Definition:Decomposable Set,['Definitions/Topology'],"A set S ⊆^n is decomposable in m sets A_1, …, A_m ⊂^n  there exist isometries ϕ_1, …, ϕ_m: ^n →^n such that:

:(1):   S = ⋃_k  = 1^m ϕ_kA_k 
:(2):  ∀ i  j: ϕ_iA_i∩ϕ_jA_j = Ø

Such a union is known as a decomposition.

"
Definition:Decomposition,Decomposition,"Let $\struct {S_1, \circ {\restriction_{S_1} } }, \struct {S_2, \circ {\restriction_{S_2} } }, \ldots, \struct {S_n, \circ {\restriction_{S_n} } }$ be closed algebraic substructures of an algebraic structure $\struct {S, \circ}$

where $\circ {\restriction_{S_1} }, \circ {\restriction_{S_2} }, \ldots, \circ {\restriction_{S_n} }$ are the operations induced by the restrictions of $\circ$ to $S_1, S_2, \ldots, S_n$ respectively.

Let $\struct {S, \circ}$ be the internal direct product of $S_1$, $S_2, \ldots, S_n$.


The set of algebraic substructures $\struct {S_1, \circ {\restriction_{S_1} } }, \struct {S_2, \circ {\restriction_{S_2} } }, \ldots, \struct {S_n, \circ {\restriction_{S_n} } }$ whose (external) direct product is isomorphic with $\struct {S, \circ}$ is called a decomposition of $S$.",Definition:Internal Direct Product/Decomposition,['Definitions/Internal Direct Products'],"Let S_1, ∘_S_1, S_2, ∘_S_2, …, S_n, ∘_S_n be closed algebraic substructures of an algebraic structure S, ∘

where ∘_S_1, ∘_S_2, …, ∘_S_n are the operations induced by the restrictions of ∘ to S_1, S_2, …, S_n respectively.

Let S, ∘ be the internal direct product of S_1, S_2, …, S_n.


The set of algebraic substructures S_1, ∘_S_1, S_2, ∘_S_2, …, S_n, ∘_S_n whose (external) direct product is isomorphic with S, ∘ is called a decomposition of S."
Definition:Decomposition,Decomposition,"Let $\struct {H_1, \circ {\restriction_{H_1} } }, \struct {H_2, \circ {\restriction_{H_2} } }, \ldots, \struct {H_n, \circ {\restriction_{H_n} } }$ be subgroups of a group $\struct {G, \circ}$

where $\circ {\restriction_{H_1} }, \circ {\restriction_{H_2} }, \ldots, \circ {\restriction_{H_n} }$ are the operations induced by the restrictions of $\circ$ to $H_1, H_2, \ldots, H_n$ respectively.

Let $\struct {G, \circ}$ be the internal group direct product of $H_1$, $H_2, \ldots, H_n$.


The set of subgroups $\struct {H_1, \circ {\restriction_{H_1} } }, \struct {H_2, \circ {\restriction_{H_2} } }, \ldots, \struct {H_n, \circ {\restriction_{H_n} } }$ whose group direct product is isomorphic with $\struct {G, \circ}$ is called a decomposition of $G$.",Definition:Internal Group Direct Product/Decomposition,['Definitions/Internal Group Direct Products'],"Let H_1, ∘_H_1, H_2, ∘_H_2, …, H_n, ∘_H_n be subgroups of a group G, ∘

where ∘_H_1, ∘_H_2, …, ∘_H_n are the operations induced by the restrictions of ∘ to H_1, H_2, …, H_n respectively.

Let G, ∘ be the internal group direct product of H_1, H_2, …, H_n.


The set of subgroups H_1, ∘_H_1, H_2, ∘_H_2, …, H_n, ∘_H_n whose group direct product is isomorphic with G, ∘ is called a decomposition of G."
Definition:Decomposition,Decomposition,"Let $n > 1 \in \Z$.


From the Fundamental Theorem of Arithmetic, $n$ has a unique factorization of the form:






where:
:$p_1 < p_2 < \cdots < p_r$ are distinct primes
:$k_1, k_2, \ldots, k_r$ are (strictly) positive integers.


This unique expression is known as the prime decomposition of $n$.


=== Multiplicity ===

",Definition:Prime Decomposition,"['Definitions/Prime Decompositions', 'Definitions/Prime Numbers', 'Definitions/Factorization', 'Definitions/Number Theory']","Let n > 1 ∈.


From the Fundamental Theorem of Arithmetic, n has a unique factorization of the form:






where:
:p_1 < p_2 < ⋯ < p_r are distinct primes
:k_1, k_2, …, k_r are (strictly) positive integers.


This unique expression is known as the prime decomposition of n.


=== Multiplicity ===

"
Definition:Decomposition,Decomposition,"Let $\map R x = \dfrac {\map P x} {\map Q x}$ be a rational function, where $\map P x$ and $\map Q x$ are expressible as polynomial functions.

Let $\map Q x$ be expressible as:
:$\map Q x = \ds \prod_{k \mathop = 1}^n \map {q_k} x$
where the $\map {q_k} x$ are themselves polynomial functions of degree at least $1$.


Let $\map R x$ be expressible as:
:$\map R x = \map r x \ds \sum_{k \mathop = 0}^n \dfrac {\map {p_k} x} {\map {q_k} x}$
where:
:$\map r x$ is a polynomial function which may or may not be the null polynomial, or be of degree $0$ (that is, a constant)
:each of the $\map {p_k} x$ are polynomial functions
:the degree of $\map {p_k} x$ is strictly less than the degree of $\map {q_k} x$ for all $k$.


Then $\map r x \ds \sum_{k \mathop = 0}^n \dfrac {\map {p_k} x} {\map {q_k} x}$ is a partial fractions expansion of $\map R x$.",Definition:Partial Fractions Expansion,"['Definitions/Partial Fractions Expansions', 'Definitions/Algebra', 'Definitions/Real Analysis', 'Definitions/Analysis']","Let R x =  P x Q x be a rational function, where P x and Q x are expressible as polynomial functions.

Let Q x be expressible as:
:Q x = ∏_k  = 1^n q_k x
where the q_k x are themselves polynomial functions of degree at least 1.


Let R x be expressible as:
:R x =  r x ∑_k  = 0^n p_k xq_k x
where:
:r x is a polynomial function which may or may not be the null polynomial, or be of degree 0 (that is, a constant)
:each of the p_k x are polynomial functions
:the degree of p_k x is strictly less than the degree of q_k x for all k.


Then r x ∑_k  = 0^n p_k xq_k x is a partial fractions expansion of R x."
Definition:Definite,Definite,,Definition:Positive Definite,['Definitions/Abstract Algebra'],
Definition:Definite,Definite,"Let $\C$ be the field of complex numbers.

Let $\F$ be a subfield of $\C$.

Let $V$ be a vector space over $\F$

Let $\innerprod \cdot \cdot: V \times V \to \mathbb F$ be a mapping.


Then $\innerprod \cdot \cdot: V \times V \to \mathbb F$ is non-negative definite :

:$\forall x \in V: \innerprod x x \in \R_{\ge 0}$


That is, the image of $\innerprod x x$ is always a non-negative real number.",Definition:Non-Negative Definite Mapping,['Definitions/Hilbert Spaces'],"Let  be the field of complex numbers.

Let  be a subfield of .

Let V be a vector space over 

Let ··: V × V →𝔽 be a mapping.


Then ··: V × V →𝔽 is non-negative definite :

:∀ x ∈ V:  x x ∈_≥ 0


That is, the image of x x is always a non-negative real number."
Definition:Degenerate,Degenerate,"A degenerate case is a specific manifestation of a particular type of object being included in another, usually simpler, type of object.",Definition:Degenerate Case,"['Definitions/Language Definitions', 'Definitions/Degenerate Cases']","A degenerate case is a specific manifestation of a particular type of object being included in another, usually simpler, type of object."
Definition:Degenerate,Degenerate,"A point-circle is the locus in the Cartesian plane of an equation of the form:

:$(1): \quad \paren {x - a}^2 + \paren {y - b}^2 = 0$

where $a$ and $b$ are real constants.


There is only one point in the Cartesian plane which satisfies $(1)$, and that is the point $\tuple {a, b}$.

It can be considered to be a circle whose radius is equal to zero.",Definition:Point-Circle,"['Definitions/Point-Circles', 'Definitions/Degenerate Conics', 'Definitions/Circles']","A point-circle is the locus in the Cartesian plane of an equation of the form:

:(1):   x - a^2 + y - b^2 = 0

where a and b are real constants.


There is only one point in the Cartesian plane which satisfies (1), and that is the point a, b.

It can be considered to be a circle whose radius is equal to zero."
Definition:Degenerate,Degenerate,"A degenerate case is a specific manifestation of a particular type of object being included in another, usually simpler, type of object.
",Definition:Conic Section/Intersection with Cone/Degenerate Hyperbola,"['Definitions/Degenerate Conics', 'Definitions/Hyperbolas', 'Definitions/Examples of Degenerate Cases']","A degenerate case is a specific manifestation of a particular type of object being included in another, usually simpler, type of object.
"
Definition:Degenerate,Degenerate,"A degenerate parabola is the conic section whose slicing plane passes through the apex of the cone and is thus tangent to the cone

Hence it consists of a single straight line.",Definition:Degenerate Parabola,"['Definitions/Degenerate Conics', 'Definitions/Parabolas', 'Definitions/Examples of Degenerate Cases']","A degenerate parabola is the conic section whose slicing plane passes through the apex of the cone and is thus tangent to the cone

Hence it consists of a single straight line."
Definition:Degenerate,Degenerate,"Let $\mathbb K$ be a field.

Let $V$ be a vector space over $\mathbb K$.


A bilinear form on $V$ which is not degenerate is nondegenerate.
",Definition:Degenerate Bilinear Form,"['Definitions/Bilinear Forms (Linear Algebra)', 'Definitions/Examples of Degenerate Cases']","Let 𝕂 be a field.

Let V be a vector space over 𝕂.


A bilinear form on V which is not degenerate is nondegenerate.
"
Definition:Degenerate,Degenerate,"Let $\struct {S, \vee, \wedge, \neg}$ be a Boolean algebra.


Then $\struct {S, \vee, \wedge, \neg}$ is said to be degenerate  $S$ is a singleton.",Definition:Degenerate Boolean Algebra,"['Definitions/Boolean Algebras', 'Definitions/Examples of Degenerate Cases']","Let S, ∨, ∧, be a Boolean algebra.


Then S, ∨, ∧, is said to be degenerate  S is a singleton."
Definition:Degenerate,Degenerate,"Let $T = \left({S, \tau}\right)$ be a topological space.


A non-degenerate connected set of $T$ is a connected set of $T$ containing more than one element.


Category:Definitions/Connected Sets
Let $T = \struct {S, \tau}$ be a topological space.


$T$ is a degenerate connected space  it contains exactly one element.
",Definition:Degenerate Connected Set,"['Definitions/Connected Sets', 'Definitions/Examples of Degenerate Cases']","Let T = (S, τ) be a topological space.


A non-degenerate connected set of T is a connected set of T containing more than one element.


Category:Definitions/Connected Sets
Let T = S, τ be a topological space.


T is a degenerate connected space  it contains exactly one element.
"
Definition:Degenerate,Degenerate,"Let $T = \struct {S, \tau}$ be a topological space.


A non-degenerate continuum of $T$ is a continuum in $T$ containing more than one element.
",Definition:Degenerate Continuum,['Definitions/Continua (Topology)'],"Let T = S, τ be a topological space.


A non-degenerate continuum of T is a continuum in T containing more than one element.
"
Definition:Degenerate,Degenerate,"Let $X$ be a discrete random variable on a probability space.


Then $X$ has a degenerate distribution with parameter $r$ :

:$\Omega_X = \set r$

:$\map \Pr {X = k} = \begin {cases}
1 & : k = r \\
0 & : k \ne r
\end {cases}$

That is, there is only value that $X$ can take, namely $r$, which it takes with certainty.




It trivially gives rise to a probability mass function satisfying $\map \Pr \Omega = 1$.

Equally trivially, it has an expectation of $r$ and a variance of $0$.",Definition:Degenerate Distribution,"['Definitions/Probability Theory', 'Definitions/Examples of Degenerate Cases']","Let X be a discrete random variable on a probability space.


Then X has a degenerate distribution with parameter r :

:Ω_X =  r

:X = k = 
1     : k = r 

0     : k  r

That is, there is only value that X can take, namely r, which it takes with certainty.




It trivially gives rise to a probability mass function satisfying Ω = 1.

Equally trivially, it has an expectation of r and a variance of 0."
Definition:Degree,Degree,,Definition:Degree of Polynomial,"['Definitions/Degree of Polynomial', 'Definitions/Polynomial Theory']",
Definition:Degree,Degree,"The degree of a monomial is defined as:
:$\ds \sum_{j \mathop \in J} k_j$
that is, the modulus of the corresponding multiindex.


Category:Definitions/Monomials",Definition:Monomial of Free Commutative Monoid/Degree,['Definitions/Monomials'],"The degree of a monomial is defined as:
:∑_j ∈ J k_j
that is, the modulus of the corresponding multiindex.


Category:Definitions/Monomials"
Definition:Degree,Degree,"
",Definition:Algebraic Number/Degree,['Definitions/Algebraic Numbers'],"
"
Definition:Degree,Degree,"Let $f: \R^2 \to \R$ be a homogeneous function of two variables:

:$\exists n \in \Z: \forall t \in \R: \map f {t x, t y} = t^n \map f {x, y}$


The integer $n$ is known as the degree of $f$.",Definition:Homogeneous Function/Real Space/Degree,['Definitions/Homogeneous Functions'],"Let f: ^2 → be a homogeneous function of two variables:

:∃ n ∈: ∀ t ∈:  f t x, t y = t^n  f x, y


The integer n is known as the degree of f."
Definition:Degree,Degree,"Let $G$ be an abelian group.

Let $\Delta$ be a set.


A gradation of type $\Delta$ on $G$ is a family of subgroups $\family {G_\lambda}_{\lambda \mathop \in \Delta}$ of which $G$ is the internal direct sum.",Definition:Gradation on Abelian Group,['Definitions/Group Theory'],"Let G be an abelian group.

Let Δ be a set.


A gradation of type Δ on G is a family of subgroups G_λ_λ∈Δ of which G is the internal direct sum."
Definition:Degree,Degree,"Let $K$ be a field, and let $L/K$ be a field extension of $K$.

The transcendence degree of $L/K$ is the largest cardinality of an algebraically independent subset $A \subseteq L$.

Category:Definitions/Field Extensions",Definition:Transcendence Degree,['Definitions/Field Extensions'],"Let K be a field, and let L/K be a field extension of K.

The transcendence degree of L/K is the largest cardinality of an algebraically independent subset A ⊆ L.

Category:Definitions/Field Extensions"
Definition:Degree,Degree,"The degree (of arc) is a unit of measurement of the length of an arc of a circle.

It is defined as the length of the arc which subtends $1$ degree (of angle) at the center of the circle.
The degree (of angle) is a measurement of plane angles, symbolized by $\degrees$.









=== Value of Degree in Radians ===

",Definition:Degree of Arc,"['Definitions/Degrees of Arc', 'Definitions/Arc Length', 'Definitions/Units of Measurement']","The degree (of arc) is a unit of measurement of the length of an arc of a circle.

It is defined as the length of the arc which subtends 1 degree (of angle) at the center of the circle.
The degree (of angle) is a measurement of plane angles, symbolized by .









=== Value of Degree in Radians ===

"
Definition:Deltoid,Deltoid,"A deltoid is a dart such that:
:the $2$ sides which are adjacent to the reflex angle are equal to each other
:the other $2$ sides are also equal to each other.


:

In the above diagram, the figure on the right is a deltoid.
",Definition:Quadrilateral/Dart/Deltoid,['Definitions/Quadrilaterals'],"A deltoid is a dart such that:
:the 2 sides which are adjacent to the reflex angle are equal to each other
:the other 2 sides are also equal to each other.


:

In the above diagram, the figure on the right is a deltoid.
"
Definition:Deltoid,Deltoid,"A deltoid is a hypocycloid with $3$ cusps.


:",Definition:Deltoid (Hypocycloid),['Definitions/Hypocycloids'],"A deltoid is a hypocycloid with 3 cusps.


:"
Definition:Dense,Dense,"Let $T = \struct {S, \tau}$ be a topological space.

Let $H \subseteq S$.


Then $H$ is dense-in-itself  it contains no isolated points.
",Definition:Dense-in-itself,"['Definitions/Topology', 'Definitions/Denseness']","Let T = S, τ be a topological space.

Let H ⊆ S.


Then H is dense-in-itself  it contains no isolated points.
"
Definition:Dense,Dense,"Let $\struct {S, \preceq}$ be an ordered set.


A subset $T \subseteq S$ is said to be densely ordered in $\struct {S, \preceq}$ :
:$\forall a, b \in S: a \prec b \implies \exists c \in T: a \prec c \prec b$
",Definition:Densely Ordered,"['Definitions/Order Theory', 'Definitions/Densely Ordered']","Let S, ≼ be an ordered set.


A subset T ⊆ S is said to be densely ordered in S, ≼ :
:∀ a, b ∈ S: a ≺ b ∃ c ∈ T: a ≺ c ≺ b
"
Definition:Dense,Dense,"Let $L = \struct {S, \wedge, \preceq}$ be a bounded below meet semilattice.

Let $x \in S$.


Then $x$ is dense 
:$\forall y \in S: y \ne \bot \implies x \wedge y \ne \bot$

where $\bot$ denotes the smallest element in $L$.
Let $L = \struct {S, \wedge, \preceq}$ be a bounded below meet semilattice.

Let $A$ be a subset of $S$.


Then $A$ is dense  it includes only dense elements.

That means that  $\forall x \in A: x$ is a dense element.
",Definition:Dense (Lattice Theory),['Definitions/Lattice Theory'],"Let L = S, ∧, ≼ be a bounded below meet semilattice.

Let x ∈ S.


Then x is dense 
:∀ y ∈ S: y  x ∧ y

where  denotes the smallest element in L.
Let L = S, ∧, ≼ be a bounded below meet semilattice.

Let A be a subset of S.


Then A is dense  it includes only dense elements.

That means that  ∀ x ∈ A: x is a dense element.
"
Definition:Diagonal,Diagonal,"A diagonal of a polyhedron $P$ is a straight line connecting $2$ vertices of $P$ which are not adjacent to the same face.
",Definition:Diagonal of Polyhedron,"['Definitions/Diagonals of Polyhedra', 'Definitions/Polyhedra']","A diagonal of a polyhedron P is a straight line connecting 2 vertices of P which are not adjacent to the same face.
"
Definition:Diagonal,Diagonal,"Let $ABCD$ be a parallelogram:

:

The diameters of $ABCD$ are the lines $AC$ and $BD$ joining their opposite vertices.",Definition:Diameter of Parallelogram,['Definitions/Parallelograms'],"Let ABCD be a parallelogram:

:

The diameters of ABCD are the lines AC and BD joining their opposite vertices."
Definition:Diagonal,Diagonal,"Let $S$ be a set.

The diagonal relation on $S$ is the relation $\Delta_S$ on $S$ defined as:

:$\Delta_S = \set {\tuple {x, x}: x \in S} \subseteq S \times S$

Alternatively:

:$\Delta_S = \set {\tuple {x, y}: x, y \in S: x = y}$


=== Class Theory ===


",Definition:Diagonal Relation,"['Definitions/Diagonal Relation', 'Definitions/Examples of Equivalence Relations', 'Definitions/Examples of Relations']","Let S be a set.

The diagonal relation on S is the relation Δ_S on S defined as:

:Δ_S = x, x: x ∈ S⊆ S × S

Alternatively:

:Δ_S = x, y: x, y ∈ S: x = y


=== Class Theory ===


"
Definition:Diagonal,Diagonal,"Let $S$ be a set.

Let $S \times S$ be the Cartesian product of $S$ with itself.


Then the diagonal mapping on $S$ is defined as $\Delta: S \to S \times S$:
:$\forall x \in S: \Delta \left({x}\right) = \left({x, x}\right)$


Clearly $\Delta$ is an injection, and is not a surjection unless $S$ is a singleton.

",Definition:Diagonal Mapping,['Definitions/Mapping Theory'],"Let S be a set.

Let S × S be the Cartesian product of S with itself.


Then the diagonal mapping on S is defined as Δ: S → S × S:
:∀ x ∈ S: Δ(x) = (x, x)


Clearly Δ is an injection, and is not a surjection unless S is a singleton.

"
Definition:Diagonal,Diagonal,"Let $\mathbf A = \sqbrk a_{m n}$ be a matrix.

The elements $a_{j j}: j \in \closedint 1 {\min \set {m, n} }$ constitute the main diagonal of $\mathbf A$.

That is, the main diagonal of $\mathbf A$ is the diagonal of $\mathbf A$ from the top left corner, that is, the element $a_{1 1}$, running towards the lower right corner.


=== Diagonal Elements ===

The elements of the main diagonal of a matrix or a determinant are called the diagonal elements.
",Definition:Main Diagonal/Diagonal Elements,"['Definitions/Diagonal Elements', 'Definitions/Main Diagonal', 'Definitions/Matrices']","Let 𝐀 =  a_m n be a matrix.

The elements a_j j: j ∈ 1 minm, n constitute the main diagonal of 𝐀.

That is, the main diagonal of 𝐀 is the diagonal of 𝐀 from the top left corner, that is, the element a_1 1, running towards the lower right corner.


=== Diagonal Elements ===

The elements of the main diagonal of a matrix or a determinant are called the diagonal elements.
"
Definition:Diagonal,Diagonal,"Let $\mathbf A = \begin{bmatrix}
a_{11} & a_{12} & \cdots & a_{1n} \\
a_{21} & a_{22} & \cdots & a_{2n} \\
\vdots & \vdots & \ddots & \vdots \\
a_{n1} & a_{n2} & \cdots & a_{nn} \\
\end{bmatrix}$ be a square matrix of order $n$.

Then $\mathbf A$ is a diagonal matrix  all elements of $\mathbf A$ are zero except for possibly its diagonal elements.


Thus $\mathbf A = \begin{bmatrix}
a_{11} & 0 & \cdots & 0 \\
0 & a_{22} & \cdots & 0 \\
\vdots & \vdots & \ddots & \vdots \\
0 & 0 & \cdots & a_{nn} \\
\end{bmatrix}$.


It follows by the definition of triangular matrix that a diagonal matrix is both an upper triangular matrix and a lower triangular matrix.
The elements of the main diagonal of a matrix or a determinant are called the diagonal elements.
Let $\mathbf A = \begin{bmatrix}
a_{11} & a_{12} & \cdots & a_{1n} \\
a_{21} & a_{22} & \cdots & a_{2n} \\
\vdots & \vdots & \ddots & \vdots \\
a_{n1} & a_{n2} & \cdots & a_{nn} \\
\end{bmatrix}$ be a square matrix of order $n$.

Then $\mathbf A$ is a diagonal matrix  all elements of $\mathbf A$ are zero except for possibly its diagonal elements.


Thus $\mathbf A = \begin{bmatrix}
a_{11} & 0 & \cdots & 0 \\
0 & a_{22} & \cdots & 0 \\
\vdots & \vdots & \ddots & \vdots \\
0 & 0 & \cdots & a_{nn} \\
\end{bmatrix}$.


It follows by the definition of triangular matrix that a diagonal matrix is both an upper triangular matrix and a lower triangular matrix.
",Definition:Diagonal Matrix,"['Definitions/Diagonal Matrices', 'Definitions/Square Matrices', 'Definitions/Matrices']","Let 𝐀 = [ a_11 a_12    ⋯ a_1n; a_21 a_22    ⋯ a_2n;    ⋮    ⋮    ⋱    ⋮; a_n1 a_n2    ⋯ a_nn;      ] be a square matrix of order n.

Then 𝐀 is a diagonal matrix  all elements of 𝐀 are zero except for possibly its diagonal elements.


Thus 𝐀 = [ a_11    0    ⋯    0;    0 a_22    ⋯    0;    ⋮    ⋮    ⋱    ⋮;    0    0    ⋯ a_nn;      ].


It follows by the definition of triangular matrix that a diagonal matrix is both an upper triangular matrix and a lower triangular matrix.
The elements of the main diagonal of a matrix or a determinant are called the diagonal elements.
Let 𝐀 = [ a_11 a_12    ⋯ a_1n; a_21 a_22    ⋯ a_2n;    ⋮    ⋮    ⋱    ⋮; a_n1 a_n2    ⋯ a_nn;      ] be a square matrix of order n.

Then 𝐀 is a diagonal matrix  all elements of 𝐀 are zero except for possibly its diagonal elements.


Thus 𝐀 = [ a_11    0    ⋯    0;    0 a_22    ⋯    0;    ⋮    ⋮    ⋱    ⋮;    0    0    ⋯ a_nn;      ].


It follows by the definition of triangular matrix that a diagonal matrix is both an upper triangular matrix and a lower triangular matrix.
"
Definition:Diagonal,Diagonal,"A diagonalizable matrix $\mathbf A$ is a square matrix which is similar to a diagonal matrix.

That is, $\mathbf A$ is diagonalizable  there exists an invertible matrix $\mathbf X$ such that $\mathbf X^-1 \mathbf A \mathbf X$ is a diagonal matrix.
Let $\mathbf A = \begin{bmatrix}
a_{11} & a_{12} & \cdots & a_{1n} \\
a_{21} & a_{22} & \cdots & a_{2n} \\
\vdots & \vdots & \ddots & \vdots \\
a_{n1} & a_{n2} & \cdots & a_{nn} \\
\end{bmatrix}$ be a square matrix of order $n$.

Then $\mathbf A$ is a diagonal matrix  all elements of $\mathbf A$ are zero except for possibly its diagonal elements.


Thus $\mathbf A = \begin{bmatrix}
a_{11} & 0 & \cdots & 0 \\
0 & a_{22} & \cdots & 0 \\
\vdots & \vdots & \ddots & \vdots \\
0 & 0 & \cdots & a_{nn} \\
\end{bmatrix}$.


It follows by the definition of triangular matrix that a diagonal matrix is both an upper triangular matrix and a lower triangular matrix.
A diagonalizable matrix $\mathbf A$ is a square matrix which is similar to a diagonal matrix.

That is, $\mathbf A$ is diagonalizable  there exists an invertible matrix $\mathbf X$ such that $\mathbf X^-1 \mathbf A \mathbf X$ is a diagonal matrix.
Let $\mathbf A = \begin{bmatrix}
a_{11} & a_{12} & \cdots & a_{1n} \\
a_{21} & a_{22} & \cdots & a_{2n} \\
\vdots & \vdots & \ddots & \vdots \\
a_{n1} & a_{n2} & \cdots & a_{nn} \\
\end{bmatrix}$ be a square matrix of order $n$.

Then $\mathbf A$ is a diagonal matrix  all elements of $\mathbf A$ are zero except for possibly its diagonal elements.


Thus $\mathbf A = \begin{bmatrix}
a_{11} & 0 & \cdots & 0 \\
0 & a_{22} & \cdots & 0 \\
\vdots & \vdots & \ddots & \vdots \\
0 & 0 & \cdots & a_{nn} \\
\end{bmatrix}$.


It follows by the definition of triangular matrix that a diagonal matrix is both an upper triangular matrix and a lower triangular matrix.
",Definition:Diagonalizable Matrix,"['Definitions/Diagonalizable Matrices', 'Definitions/Matrices']","A diagonalizable matrix 𝐀 is a square matrix which is similar to a diagonal matrix.

That is, 𝐀 is diagonalizable  there exists an invertible matrix 𝐗 such that 𝐗^-1 𝐀𝐗 is a diagonal matrix.
Let 𝐀 = [ a_11 a_12    ⋯ a_1n; a_21 a_22    ⋯ a_2n;    ⋮    ⋮    ⋱    ⋮; a_n1 a_n2    ⋯ a_nn;      ] be a square matrix of order n.

Then 𝐀 is a diagonal matrix  all elements of 𝐀 are zero except for possibly its diagonal elements.


Thus 𝐀 = [ a_11    0    ⋯    0;    0 a_22    ⋯    0;    ⋮    ⋮    ⋱    ⋮;    0    0    ⋯ a_nn;      ].


It follows by the definition of triangular matrix that a diagonal matrix is both an upper triangular matrix and a lower triangular matrix.
A diagonalizable matrix 𝐀 is a square matrix which is similar to a diagonal matrix.

That is, 𝐀 is diagonalizable  there exists an invertible matrix 𝐗 such that 𝐗^-1 𝐀𝐗 is a diagonal matrix.
Let 𝐀 = [ a_11 a_12    ⋯ a_1n; a_21 a_22    ⋯ a_2n;    ⋮    ⋮    ⋱    ⋮; a_n1 a_n2    ⋯ a_nn;      ] be a square matrix of order n.

Then 𝐀 is a diagonal matrix  all elements of 𝐀 are zero except for possibly its diagonal elements.


Thus 𝐀 = [ a_11    0    ⋯    0;    0 a_22    ⋯    0;    ⋮    ⋮    ⋱    ⋮;    0    0    ⋯ a_nn;      ].


It follows by the definition of triangular matrix that a diagonal matrix is both an upper triangular matrix and a lower triangular matrix.
"
Definition:Diagram,Diagram,"A diagram is a graphical technique for illustrating a concept in picture form.

It is generally considered that its use should be limited to that of an aid to understanding, and should not be used in order to prove something.
",Definition:Diagram (Graphical Technique),"['Definitions/Diagrams', 'Definitions/Proof Techniques']","A diagram is a graphical technique for illustrating a concept in picture form.

It is generally considered that its use should be limited to that of an aid to understanding, and should not be used in order to prove something.
"
Definition:Diagram,Diagram,"Let $\mathbf J$ and $\mathbf C$ be metacategories.


A diagram of type $\mathbf J$ in $\mathbf C$ is a functor $D: \mathbf J \to \mathbf C$.


=== Index Category ===

In this context, $\mathbf J$ is referred to as the index category.

Its objects are typically denoted by lowercase letters, $i, j$ etc.


Furthermore, one writes $D_i$ in place of the formally more correct $D \left({i}\right)$.

Similarly, for $\alpha: i \to j$ a morphism one writes $D_\alpha$ in place of $D \left({\alpha}\right)$.",Definition:Diagram (Category Theory),['Definitions/Category Theory'],"Let 𝐉 and 𝐂 be metacategories.


A diagram of type 𝐉 in 𝐂 is a functor D: 𝐉→𝐂.


=== Index Category ===

In this context, 𝐉 is referred to as the index category.

Its objects are typically denoted by lowercase letters, i, j etc.


Furthermore, one writes D_i in place of the formally more correct D (i).

Similarly, for α: i → j a morphism one writes D_α in place of D (α)."
Definition:Diameter,Diameter,":


:


In the above diagram, the line $CD$ is a diameter.
",Definition:Circle/Diameter,"['Definitions/Circles', 'Definitions/Diameters of Conic Sections']",":


:


In the above diagram, the line CD is a diameter.
"
Definition:Diameter,Diameter,"Let $\KK$ be an ellipse.

A diameter of $\KK$ is the locus of the midpoints of a system of parallel chords of $\KK$.


:",Definition:Diameter of Ellipse,"['Definitions/Ellipses', 'Definitions/Diameters of Conic Sections']","Let  be an ellipse.

A diameter of  is the locus of the midpoints of a system of parallel chords of .


:"
Definition:Diameter,Diameter,"Let $\KK$ be a hyperbola.

A diameter of $\KK$ is the locus of the midpoints of a system of parallel chords of $\KK$.


:",Definition:Diameter of Hyperbola,"['Definitions/Hyperbolas', 'Definitions/Diameters of Conic Sections']","Let  be a hyperbola.

A diameter of  is the locus of the midpoints of a system of parallel chords of .


:"
Definition:Diameter,Diameter,"Let $\KK$ be a parabola.

A diameter of $\KK$ is the locus of the midpoints of a system of parallel chords of $\KK$.


:",Definition:Diameter of Parabola,"['Definitions/Diameters of Conic Sections', 'Definitions/Parabolas']","Let  be a parabola.

A diameter of  is the locus of the midpoints of a system of parallel chords of .


:"
Definition:Diameter,Diameter,"By the definition of a sphere, there is one point inside it such that the distance between that point and any given point on the surface of the sphere are equal, and that point is called the center of the sphere.


The diameter of a sphere is the length of any straight line drawn from a point on the surface to another point on the surface through the center.



:

",Definition:Sphere/Geometry/Diameter,['Definitions/Spheres'],"By the definition of a sphere, there is one point inside it such that the distance between that point and any given point on the surface of the sphere are equal, and that point is called the center of the sphere.


The diameter of a sphere is the length of any straight line drawn from a point on the surface to another point on the surface through the center.



:

"
Definition:Diameter,Diameter,"Let $ABCD$ be a parallelogram:

:

The diameters of $ABCD$ are the lines $AC$ and $BD$ joining their opposite vertices.",Definition:Diameter of Parallelogram,['Definitions/Parallelograms'],"Let ABCD be a parallelogram:

:

The diameters of ABCD are the lines AC and BD joining their opposite vertices."
Definition:Diameter,Diameter,"The diameter of a geometric figure is the greatest length that can be formed between two opposite parallel straight lines that can be drawn tangent to its boundary.
",Definition:Geometric Figure/Diameter,['Definitions/Geometric Figures'],"The diameter of a geometric figure is the greatest length that can be formed between two opposite parallel straight lines that can be drawn tangent to its boundary.
"
Definition:Diameter,Diameter,"Let $M = \struct {A, d}$ be a metric space.

Let $S \subseteq A$ be subset of $A$.


Then the diameter of $S$ is the extended real number defined by:

:$\map \diam S := \begin {cases} \sup \set {\map d {x, y}: x, y \in S} & : \text {if this quantity is finite} \\ + \infty & : \text {otherwise} \end {cases}$


Thus, by the definition of the supremum, the diameter is the smallest real number $D$ such that any two points of $S$ are at most a distance $D$ apart.

If $d: S^2 \to \R$ does not admit a supremum, then $\map \diam S$ is infinite.",Definition:Diameter of Subset of Metric Space,['Definitions/Metric Spaces'],"Let M = A, d be a metric space.

Let S ⊆ A be subset of A.


Then the diameter of S is the extended real number defined by:

:S := sup d x, y: x, y ∈ S    : if this quantity is finite
 + ∞    : otherwise


Thus, by the definition of the supremum, the diameter is the smallest real number D such that any two points of S are at most a distance D apart.

If d: S^2 → does not admit a supremum, then S is infinite."
Definition:Difference,Difference,"The (set) difference between two sets $S$ and $T$ is written $S \setminus T$, and means the set that consists of the elements of $S$ which are not elements of $T$:
:$x \in S \setminus T \iff x \in S \land x \notin T$


It can also be defined as:
:$S \setminus T = \set {x \in S: x \notin T}$
:$S \setminus T = \set {x: x \in S \land x \notin T}$",Definition:Set Difference,"['Definitions/Set Theory', 'Definitions/Set Difference']","The (set) difference between two sets S and T is written S ∖ T, and means the set that consists of the elements of S which are not elements of T:
:x ∈ S ∖ T  x ∈ S  x ∉ T


It can also be defined as:
:S ∖ T = x ∈ S: x ∉ T
:S ∖ T = x: x ∈ S  x ∉ T"
Definition:Difference,Difference,"Let $a$ and $b$ be real numbers.

The absolute difference between $a$ and $b$ is defined and denoted as:
:$\size {a - b}$
where $\size {\, \cdot \,}$ is the absolute value function.",Definition:Absolute Difference,['Definitions/Subtraction'],"Let a and b be real numbers.

The absolute difference between a and b is defined and denoted as:
:a - b
where · is the absolute value function."
Definition:Differentiable,Differentiable,"Let $M$ be a topological space.

Let $d$ be a natural number.

Let $k \ge 1$ be a natural number.



A $d$-dimensional differentiable structure of class $\CC^k$ on $M$ is a non-empty equivalence class of the set of $d$-dimensional $\CC^k$-atlases on $M$ under the equivalence relation of compatibility.
",Definition:Topological Manifold/Differentiable Manifold,"['Definitions/Topological Manifolds', 'Definitions/Differentiable Manifolds']","Let M be a topological space.

Let d be a natural number.

Let k ≥ 1 be a natural number.



A d-dimensional differentiable structure of class ^k on M is a non-empty equivalence class of the set of d-dimensional ^k-atlases on M under the equivalence relation of compatibility.
"
Definition:Differentiable,Differentiable,"Let $M$ and $N$ be differentiable manifolds.

Let $f : M \to N$ be continuous.

=== Definition 1 ===

$f$  is differentiable  for every pair of charts $(U, \phi)$ and $(V,\psi)$ of $M$ and $N$:
:$\psi\circ f\circ \phi^{-1} : \phi ( U \cap f^{-1}(V)) \to \psi(V)$
is differentiable.


=== Definition 2 ===


$f$  is differentiable  $f$ is  differentiable at every point of $M$.


=== At a Point ===

",Definition:Differentiable Mapping,['Definitions/Differential Calculus'],"Let M and N be differentiable manifolds.

Let f : M → N be continuous.

=== Definition 1 ===

f  is differentiable  for every pair of charts (U, ϕ) and (V,ψ) of M and N:
:ψ∘ f∘ϕ^-1 : ϕ ( U ∩ f^-1(V)) →ψ(V)
is differentiable.


=== Definition 2 ===


f  is differentiable  f is  differentiable at every point of M.


=== At a Point ===

"
Definition:Differentiable,Differentiable,"Let $S$ be a normed linear space of mappings.

Let $y, h \in S: \R \to \R$ be real functions.

Let $J \sqbrk y$, $\phi \sqbrk {y; h}$ be functionals.

Let $\Delta J \sqbrk {y; h}$ be an increment of the functional $J$ such that:

:$\Delta J \sqbrk {y; h} = \phi \sqbrk {y;h} + \epsilon \norm h$

where $\epsilon = \epsilon \sqbrk {y; h}$ is a functional, and $\norm h$ is the norm of $S$.

Suppose $\phi \sqbrk {y; h}$ is a linear  $h$ and:

:$\ds \lim_{\norm h \mathop \to 0} \epsilon = 0$


Then the functional $J \sqbrk y $ is said to be differentiable.",Definition:Differentiable Functional,['Definitions/Calculus of Variations'],"Let S be a normed linear space of mappings.

Let y, h ∈ S: → be real functions.

Let J  y, ϕy; h be functionals.

Let Δ J y; h be an increment of the functional J such that:

:Δ J y; h = ϕy;h + ϵ h

where ϵ = ϵy; h is a functional, and h is the norm of S.

Suppose ϕy; h is a linear  h and:

:lim_ h → 0ϵ = 0


Then the functional J  y is said to be differentiable."
Definition:Dihedral,Dihedral,"A dihedral is a configuration in solid geometry formed by two half-planes meeting at a common straight line.


=== Dihedral Angle ===

A dihedral angle is the angle contained by two straight lines drawn perpendicular to the common section at the same point, one in each of the two half-planes forming a dihedral.
",Definition:Dihedral (Geometry),"['Definitions/Dihedrals (Geometry)', 'Definitions/Solid Geometry']","A dihedral is a configuration in solid geometry formed by two half-planes meeting at a common straight line.


=== Dihedral Angle ===

A dihedral angle is the angle contained by two straight lines drawn perpendicular to the common section at the same point, one in each of the two half-planes forming a dihedral.
"
Definition:Dihedral,Dihedral,"A dihedral angle is the angle contained by two straight lines drawn perpendicular to the common section at the same point, one in each of the two half-planes forming a dihedral.
A dihedral is a configuration in solid geometry formed by two half-planes meeting at a common straight line.


=== Dihedral Angle ===

",Definition:Dihedral (Geometry)/Angle,"['Definitions/Dihedral Angles', 'Definitions/Dihedrals (Geometry)']","A dihedral angle is the angle contained by two straight lines drawn perpendicular to the common section at the same point, one in each of the two half-planes forming a dihedral.
A dihedral is a configuration in solid geometry formed by two half-planes meeting at a common straight line.


=== Dihedral Angle ===

"
Definition:Dihedral,Dihedral,"The dihedral group $D_n$ of order $2 n$ is the group of symmetries of the regular $n$-gon.


=== Even Polygon ===


:

=== Odd Polygon ===


:",Definition:Dihedral Group,"['Definitions/Dihedral Groups', 'Definitions/Examples of Groups', 'Definitions/Examples of Symmetry Groups']","The dihedral group D_n of order 2 n is the group of symmetries of the regular n-gon.


=== Even Polygon ===


:

=== Odd Polygon ===


:"
Definition:Dimension,Dimension,"The dimension of a (geometrical) space is the minimum number of coordinates needed to specify a point in it.
",Definition:Dimension (Geometry),"['Definitions/Dimensions (Geometry)', 'Definitions/Geometry']","The dimension of a (geometrical) space is the minimum number of coordinates needed to specify a point in it.
"
Definition:Dimension,Dimension,"Let $H$ be a Hilbert space, and let $E$ be a basis of $H$.


Then the dimension $\dim H$ of $H$ is defined as $\card E$, the cardinality of $E$.",Definition:Dimension (Hilbert Space),['Definitions/Hilbert Spaces'],"Let H be a Hilbert space, and let E be a basis of H.


Then the dimension H of H is defined as E, the cardinality of E."
Definition:Dimension,Dimension,,Definition:Affine Dimension,['Definitions/Affine Geometry'],
Definition:Dimension,Dimension,"Let $M$ be a locally Euclidean space. 

Let $\struct {U, \kappa}$ be a coordinate chart such that: 
:$\kappa: U \to \map \kappa U \subseteq \R^n$
for some $n \in \N$.


Then the natural number $n$ is called the dimension of $M$.
",Definition:Dimension (Topology)/Locally Euclidean Space,['Definitions/Topology'],"Let M be a locally Euclidean space. 

Let U, κ be a coordinate chart such that: 
:κ: U →κ U ⊆^n
for some n ∈.


Then the natural number n is called the dimension of M.
"
Definition:Dimension,Dimension,"Let $\struct {R, +, \circ}$ be a commutative ring with unity.


The Krull dimension of $R$ is the supremum of lengths of chains of prime ideals, ordered by the subset relation:




where:
:$\map {\mathrm {ht} } {\mathfrak p}$ is the height of $\mathfrak p$
:$\Spec R$ is the prime spectrum of $R$



In particular, the Krull dimension is $\infty$ if there exist arbitrarily long chains.
Let $T$ be a topological space.


Its Krull dimension $\map {\dim_{\mathrm {Krull} } } T$ is the supremum of lengths of chains of closed irreducible sets of $T$, ordered by the subset relation.

Thus, the Krull dimension is $\infty$ if there exist arbitrarily long chains.
",Definition:Krull Dimension,['Definitions/Algebraic Geometry'],"Let R, +, ∘ be a commutative ring with unity.


The Krull dimension of R is the supremum of lengths of chains of prime ideals, ordered by the subset relation:




where:
:ht𝔭 is the height of 𝔭
:R is the prime spectrum of R



In particular, the Krull dimension is ∞ if there exist arbitrarily long chains.
Let T be a topological space.


Its Krull dimension _Krull T is the supremum of lengths of chains of closed irreducible sets of T, ordered by the subset relation.

Thus, the Krull dimension is ∞ if there exist arbitrarily long chains.
"
Definition:Dimension,Dimension,"Let $\struct {R, +, \circ}$ be a commutative ring with unity.


The Krull dimension of $R$ is the supremum of lengths of chains of prime ideals, ordered by the subset relation:




where:
:$\map {\mathrm {ht} } {\mathfrak p}$ is the height of $\mathfrak p$
:$\Spec R$ is the prime spectrum of $R$



In particular, the Krull dimension is $\infty$ if there exist arbitrarily long chains.",Definition:Krull Dimension of Ring,"['Definitions/Ring Theory', 'Definitions/Ideal Theory', 'Definitions/Commutative Algebra']","Let R, +, ∘ be a commutative ring with unity.


The Krull dimension of R is the supremum of lengths of chains of prime ideals, ordered by the subset relation:




where:
:ht𝔭 is the height of 𝔭
:R is the prime spectrum of R



In particular, the Krull dimension is ∞ if there exist arbitrarily long chains."
Definition:Dimension,Dimension,"Let $T$ be a topological space.


Its Krull dimension $\map {\dim_{\mathrm {Krull} } } T$ is the supremum of lengths of chains of closed irreducible sets of $T$, ordered by the subset relation.

Thus, the Krull dimension is $\infty$ if there exist arbitrarily long chains.",Definition:Krull Dimension of Topological Space,['Definitions/Irreducible Spaces'],"Let T be a topological space.


Its Krull dimension _Krull T is the supremum of lengths of chains of closed irreducible sets of T, ordered by the subset relation.

Thus, the Krull dimension is ∞ if there exist arbitrarily long chains."
Definition:Dimension,Dimension,The order of a differential equation is defined as being the order of the highest order derivative that is present in the equation.,Definition:Differential Equation/Order,"['Definitions/Order of Differential Equation', 'Definitions/Differential Equations']",The order of a differential equation is defined as being the order of the highest order derivative that is present in the equation.
Definition:Dimension,Dimension,The dimension of a configuration space $S$ is the number of degrees of freedom of the system defined by $S$.,Definition:Configuration Space/Dimension,['Definitions/Configuration Spaces'],The dimension of a configuration space S is the number of degrees of freedom of the system defined by S.
Definition:Direct Image,Direct Image,"Let $S$ and $T$ be sets.

Let $\powerset S$ and $\powerset T$ be their power sets.

Let $f \subseteq S \times T$ be a mapping from $S$ to $T$.


The direct image mapping of $f$ is the mapping $f^\to: \powerset S \to \powerset T$ that sends a subset $X \subseteq S$ to its image under $f$:
:$\forall X \in \powerset S: \map {f^\to} X = \begin {cases} \set {t \in T: \exists s \in X: \map f s = t} & : X \ne \O \\ \O & : X = \O \end {cases}$


=== Direct Image Mapping as Set of Images of Subsets ===



The direct image mapping of $f$ can be seen to be the set of images of all the subsets of the domain of $f$:

:$\forall X \subseteq S: f \sqbrk X = \map {f^\to} X$


Both approaches to this concept are used in .
",Definition:Direct Image Mapping/Mapping,['Definitions/Direct Image Mappings'],"Let S and T be sets.

Let S and T be their power sets.

Let f ⊆ S × T be a mapping from S to T.


The direct image mapping of f is the mapping f^→:  S → T that sends a subset X ⊆ S to its image under f:
:∀ X ∈ S: f^→ X = t ∈ T: ∃ s ∈ X:  f s = t    : X Ø
Ø    : X = Ø


=== Direct Image Mapping as Set of Images of Subsets ===



The direct image mapping of f can be seen to be the set of images of all the subsets of the domain of f:

:∀ X ⊆ S: f  X = f^→ X


Both approaches to this concept are used in .
"
Definition:Direct Image,Direct Image,"Let $S$ and $T$ be sets.

Let $\powerset S$ and $\powerset T$ be their power sets.

Let $\RR \subseteq S \times T$ be a relation on $S \times T$.


The direct image mapping of $\RR$ is the mapping $\RR^\to: \powerset S \to \powerset T$ that sends a subset $X \subseteq T$ to its image under $\RR$:

:$\forall X \in \powerset S: \map {\RR^\to} X = \begin {cases} \set {t \in T: \exists s \in X: \tuple {s, t} \in \RR} & : X \ne \O \\ \O & : X = \O \end {cases}$",Definition:Direct Image Mapping/Relation,['Definitions/Direct Image Mappings'],"Let S and T be sets.

Let S and T be their power sets.

Let ⊆ S × T be a relation on S × T.


The direct image mapping of  is the mapping ^→:  S → T that sends a subset X ⊆ T to its image under :

:∀ X ∈ S: ^→ X = t ∈ T: ∃ s ∈ X: s, t∈    : X Ø
Ø    : X = Ø"
Definition:Directed,Directed,"Let $\struct {S, \precsim}$ be a preordered set.


Then $\struct {S, \precsim}$ is a directed preordering  every pair of elements of $S$ has an upper bound in $S$:
:$\forall x, y \in S: \exists z \in S: x \precsim z$ and $y \precsim z$",Definition:Directed Preordering,['Definitions/Preorder Theory'],"Let S, ≾ be a preordered set.


Then S, ≾ is a directed preordering  every pair of elements of S has an upper bound in S:
:∀ x, y ∈ S: ∃ z ∈ S: x ≾ z and y ≾ z"
Definition:Directed,Directed,"Let $\struct {S, \precsim}$ be a preordered set.

Let $H$ be a non-empty subset of $S$.

Then $H$ is a directed subset of $S$ :

:$\forall x, y \in H: \exists z \in H: x \precsim z$ and $y \precsim z$",Definition:Directed Subset,['Definitions/Preorder Theory'],"Let S, ≾ be a preordered set.

Let H be a non-empty subset of S.

Then H is a directed subset of S :

:∀ x, y ∈ H: ∃ z ∈ H: x ≾ z and y ≾ z"
Definition:Directed,Directed,"A directed line segment is a line segment endowed with the additional property of direction.

It is often used in the context of applied mathematics to represent a vector quantity.





",Definition:Directed Line Segment,"['Definitions/Directed Line Segments', 'Definitions/Straight Lines', 'Definitions/Analytic Geometry', 'Definitions/Vectors', 'Definitions/Applied Mathematics']","A directed line segment is a line segment endowed with the additional property of direction.

It is often used in the context of applied mathematics to represent a vector quantity.





"
Definition:Directed,Directed,"Let $G = \struct {V, A}$ be a digraph.


A directed walk in $G$ is a finite or infinite sequence $\sequence {x_k}$ such that:

:$\forall k \in \N: k + 1 \in \Dom {\sequence {x_k} }: \tuple {x_k, x_{k + 1} } \in A$
",Definition:Directed Walk,"['Definitions/Digraphs', 'Definitions/Walks']","Let G = V, A be a digraph.


A directed walk in G is a finite or infinite sequence x_k such that:

:∀ k ∈: k + 1 ∈x_k: x_k, x_k + 1∈ A
"
Definition:Directed,Directed,"A directed network is a network whose underlying graph is a digraph:


:",Definition:Network/Directed,['Definitions/Network Theory'],"A directed network is a network whose underlying graph is a digraph:


:"
Definition:Directed,Directed,,Definition:Direct Limit,['Definitions/Examples of Limits and Colimits'],
Definition:Directrix,Directrix,"
Let $K$ be a cone.

Let $B$ be the base of $K$.


The boundary of $B$ is known as the directrix of $K$.
The directrix of a cylindrical surface $S$ is the curve $C$ through which pass all the parallel straight lines forming $S$.
",Definition:Directrix of Ruled Surface,"['Definitions/Directrices of Ruled Surfaces', 'Definitions/Directrices', 'Definitions/Analytic Geometry', 'Definitions/Geometry']","
Let K be a cone.

Let B be the base of K.


The boundary of B is known as the directrix of K.
The directrix of a cylindrical surface S is the curve C through which pass all the parallel straight lines forming S.
"
Definition:Directrix,Directrix,"Let $K$ be a conic section specified in terms of:
:a given straight line $D$
:a given point $F$
:a given constant $e$

where $K$ is the locus of points $P$ such that the distance $p$ from $P$ to $D$ and the distance $q$ from $P$ to $F$ are related by the condition:
:$q = e p$


The line $D$ is known as the directrix of the conic section.
",Definition:Conic Section/Directrix,"['Definitions/Directrices', 'Definitions/Conic Sections']","Let K be a conic section specified in terms of:
:a given straight line D
:a given point F
:a given constant e

where K is the locus of points P such that the distance p from P to D and the distance q from P to F are related by the condition:
:q = e p


The line D is known as the directrix of the conic section.
"
Definition:Directrix,Directrix,"Let $\KK$ be a conchoid.

Let $\CC$ be the curve  which the conchoid is constructed.


$\CC$ is known as the directrix of $\KK$.


=== Also known as ===

",Definition:Conchoid/Directrix,"['Definitions/Conchoids', 'Definitions/Directrices']","Let  be a conchoid.

Let  be the curve  which the conchoid is constructed.


 is known as the directrix of .


=== Also known as ===

"
Definition:Discontinuous,Discontinuous,"Let $f$ be a real function.

Then $f$ is discontinuous  there exists at least one $a \in \R$ at which $f$ is discontinuous.


=== At a Point ===

=== At a Point ===



",Definition:Discontinuous Mapping,"['Definitions/Discontinuous Mappings', 'Definitions/Continuity']","Let f be a real function.

Then f is discontinuous  there exists at least one a ∈ at which f is discontinuous.


=== At a Point ===

=== At a Point ===



"
Definition:Discontinuous,Discontinuous,"Let $f$ be a real function.

Then $f$ is discontinuous  there exists at least one $a \in \R$ at which $f$ is discontinuous.


=== At a Point ===

Let $A \subseteq \R$ be a subset of the real numbers.

Let $f : A \to \R$ be a real function.

Let $a\in A$.


Then $f$ is discontinuous at $a$  $f$ is not continuous at $a$.
",Definition:Discontinuous Mapping/Real Function,['Definitions/Discontinuous Mappings'],"Let f be a real function.

Then f is discontinuous  there exists at least one a ∈ at which f is discontinuous.


=== At a Point ===

Let A ⊆ be a subset of the real numbers.

Let f : A → be a real function.

Let a∈ A.


Then f is discontinuous at a  f is not continuous at a.
"
Definition:Discontinuous,Discontinuous,"Let $A \subseteq \R$ be a subset of the real numbers.

Let $f : A \to \R$ be a real function.

Let $a\in A$.


Then $f$ is discontinuous at $a$  $f$ is not continuous at $a$.
",Definition:Discontinuous Mapping/Real Function/Point,"['Definitions/Discontinuous Mappings', 'Definitions/Discontinuities (Real Analysis)']","Let A ⊆ be a subset of the real numbers.

Let f : A → be a real function.

Let a∈ A.


Then f is discontinuous at a  f is not continuous at a.
"
Definition:Discrete,Discrete,"

Discrete is a word used to define an object which is not continuous.
",Definition:Discrete Mathematics,"['Definitions/Branches of Mathematics', 'Definitions/Discrete Mathematics']","

Discrete is a word used to define an object which is not continuous.
"
Definition:Discrete,Discrete,"Let $\CC$ be a metacategory.


Then $\CC$ is said to be discrete  it comprises only identity morphisms.

If the collection $\CC$ constitutes the objects of $\mathbf C$, then $\mathbf C$ may also be denoted $\map {\mathbf {Dis} } \CC$.",Definition:Discrete Category,['Definitions/Examples of Categories'],"Let  be a metacategory.


Then  is said to be discrete  it comprises only identity morphisms.

If the collection  constitutes the objects of 𝐂, then 𝐂 may also be denoted 𝐃𝐢𝐬."
Definition:Discrete,Discrete,"Let $G$ be a subgroup of the additive group of real numbers.


Then $G$ is discrete :
:$\forall g \in G: \exists \epsilon \in \R_{>0}: \openint {g - \epsilon} {g + \epsilon} \cap G = \set g$

That is, there exists a neighborhood of $g$ which contains no other elements of $G$.


Category:Definitions/Subgroups
Category:Definitions/Topological Groups
Category:Definitions/Real Numbers",Definition:Discrete Subgroup/Real Numbers,"['Definitions/Subgroups', 'Definitions/Topological Groups', 'Definitions/Real Numbers']","Let G be a subgroup of the additive group of real numbers.


Then G is discrete :
:∀ g ∈ G: ∃ϵ∈_>0: g - ϵg + ϵ∩ G =  g

That is, there exists a neighborhood of g which contains no other elements of G.


Category:Definitions/Subgroups
Category:Definitions/Topological Groups
Category:Definitions/Real Numbers"
Definition:Discrete,Discrete,"The standard discrete metric on a set $S$ is the metric satisfying:

:$\map d {x, y} = \begin {cases} 0 & : x = y \\ 1 & : x \ne y \end {cases}$


This can be expressed using the Kronecker delta notation as:
:$\map d {x, y} = 1 - \delta_{x y}$


The resulting metric space $M = \struct {S, d}$ is the standard discrete metric space on $S$.
The standard discrete metric on a set $S$ is the metric satisfying:

:$\map d {x, y} = \begin {cases} 0 & : x = y \\ 1 & : x \ne y \end {cases}$


This can be expressed using the Kronecker delta notation as:
:$\map d {x, y} = 1 - \delta_{x y}$


The resulting metric space $M = \struct {S, d}$ is the standard discrete metric space on $S$.
",Definition:Standard Discrete Metric,"['Definitions/Standard Discrete Metric', 'Definitions/Examples of Metric Spaces']","The standard discrete metric on a set S is the metric satisfying:

:d x, y =  0     : x = y 
 1     : x  y


This can be expressed using the Kronecker delta notation as:
:d x, y = 1 - δ_x y


The resulting metric space M = S, d is the standard discrete metric space on S.
The standard discrete metric on a set S is the metric satisfying:

:d x, y =  0     : x = y 
 1     : x  y


This can be expressed using the Kronecker delta notation as:
:d x, y = 1 - δ_x y


The resulting metric space M = S, d is the standard discrete metric space on S.
"
Definition:Discrete,Discrete,"Let $T = \struct {S, \tau}$ be a topological space.

Let $\FF$ be a set of subsets of $S$.


Then $\FF$ is discrete  each element of $S$ has a neighborhood which intersects at most one of the sets in $\FF$.",Definition:Discrete Set of Subsets,['Definitions/Topology'],"Let T = S, τ be a topological space.

Let  be a set of subsets of S.


Then  is discrete  each element of S has a neighborhood which intersects at most one of the sets in ."
Definition:Discrete,Discrete,Discrete geometry is a branch of geometry that studies constructive methods of discrete geometric objects.,Definition:Discrete Geometry,"['Definitions/Branches of Mathematics', 'Definitions/Discrete Geometry', 'Definitions/Geometry']",Discrete geometry is a branch of geometry that studies constructive methods of discrete geometric objects.
Definition:Discrete,Discrete,"Let $\struct {X, \Sigma, \mu}$ be a measure space.


Then $\mu$ is said to be a discrete measure  it is a series of Dirac measures.

That is,  there exist:
:a sequence $\sequence {x_n}_{n \mathop \in \N}$ in $X$
and:
:a sequence $\sequence {\lambda_n}_{n \mathop \in \N}$ in $\R$

such that:

:$(1):\quad \forall E \in \Sigma: \map \mu E = \ds \sum_{n \mathop \in \N} \lambda_n \, \map {\delta_{x_n} } E$

where $\delta_{x_n}$ denotes the Dirac measure at $x_n$.


By Series of Measures is Measure, defining $\mu$ by $(1)$ yields a measure.",Definition:Discrete Measure,"['Definitions/Measure Theory', 'Definitions/Measures', 'Definitions/Measures']","Let X, Σ, μ be a measure space.


Then μ is said to be a discrete measure  it is a series of Dirac measures.

That is,  there exist:
:a sequence x_n_n ∈ in X
and:
:a sequence λ_n_n ∈ in 

such that:

:(1):  ∀ E ∈Σ: μ E = ∑_n ∈λ_n  δ_x_n E

where δ_x_n denotes the Dirac measure at x_n.


By Series of Measures is Measure, defining μ by (1) yields a measure."
Definition:Discrete,Discrete,"Let $\EE$ be an experiment.

Let $\Omega$ denote the sample space of $\EE$.


If $\Omega$ is a countable set, whether finite or infinite, then it is known as a discrete sample space.",Definition:Sample Space/Discrete,['Definitions/Probability Theory'],"Let  be an experiment.

Let Ω denote the sample space of .


If Ω is a countable set, whether finite or infinite, then it is known as a discrete sample space."
Definition:Discrete,Discrete,A discrete variable is a variable which is not continuous.,Definition:Variable/Discrete,"['Definitions/Descriptive Statistics', 'Definitions/Variables']",A discrete variable is a variable which is not continuous.
Definition:Discrete,Discrete,"A discrete variable is a variable which is not continuous.
Data which can be described with a discrete variable are known as discrete data.
",Definition:Sample Statistic/Discrete,['Definitions/Sample Statistics'],"A discrete variable is a variable which is not continuous.
Data which can be described with a discrete variable are known as discrete data.
"
Definition:Discriminant,Discriminant,"


Let $k$ be a field.

Let $\map f X \in k \sqbrk X$ be a polynomial of degree $n$.

Let $\overline k$ be an algebraic closure of $k$.

Let the roots of $f$ in $\overline k$ be $\alpha_1, \alpha_2, \ldots, \alpha_n$.


Then the discriminant $\map \Delta f$ of $f$ is defined as:

:$\ds \map \Delta f := \prod_{1 \mathop \le i \mathop < j \mathop \le n} \paren {\alpha_i - \alpha_j}^2$


=== Quadratic Equation ===

The concept is usually encountered in the context of a quadratic equation $a x^2 + b x + c = 0$:


=== Cubic Equation ===

In the context of a cubic equation $a x^3 + b x^2 + c x + d = 0$:

",Definition:Discriminant of Polynomial,"['Definitions/Discriminants of Polynomials', 'Definitions/Discriminants', 'Definitions/Polynomial Theory']","


Let k be a field.

Let f X ∈ k  X be a polynomial of degree n.

Let k be an algebraic closure of k.

Let the roots of f in k be α_1, α_2, …, α_n.


Then the discriminant Δ f of f is defined as:

:Δ f := ∏_1 ≤ i  < j ≤ nα_i - α_j^2


=== Quadratic Equation ===

The concept is usually encountered in the context of a quadratic equation a x^2 + b x + c = 0:


=== Cubic Equation ===

In the context of a cubic equation a x^3 + b x^2 + c x + d = 0:

"
Definition:Discriminant,Discriminant,"Let $\mathbb K$ be a field.

Let $V$ be a vector space over $\mathbb K$ of finite dimension $n>0$.

Let $b : V\times V \to \mathbb K$ be a bilinear form on $V$.

Let $A$ be the matrix of $b$ relative to an ordered basis of $V$.


If $b$ is nondegenerate, its discriminant is the equivalence class of the determinant $\det A$ in the quotient group $\dfrac {\mathbb K^\times} {\paren {\mathbb K^\times}^2}$.

If $b$ is degenerate, its discriminant is $0$.
Let $\mathbb K$ be a field.

Let $V$ be a vector space over $\mathbb K$ of finite dimension $n>0$.

Let $b : V\times V \to \mathbb K$ be a bilinear form on $V$.

Let $A$ be the matrix of $b$ relative to an ordered basis of $V$.


If $b$ is nondegenerate, its discriminant is the equivalence class of the determinant $\det A$ in the quotient group $\dfrac {\mathbb K^\times} {\paren {\mathbb K^\times}^2}$.

If $b$ is degenerate, its discriminant is $0$.
",Definition:Discriminant of Bilinear Form,"['Definitions/Bilinear Forms (Linear Algebra)', 'Definitions/Discriminants']","Let 𝕂 be a field.

Let V be a vector space over 𝕂 of finite dimension n>0.

Let b : V× V →𝕂 be a bilinear form on V.

Let A be the matrix of b relative to an ordered basis of V.


If b is nondegenerate, its discriminant is the equivalence class of the determinant A in the quotient group 𝕂^×𝕂^×^2.

If b is degenerate, its discriminant is 0.
Let 𝕂 be a field.

Let V be a vector space over 𝕂 of finite dimension n>0.

Let b : V× V →𝕂 be a bilinear form on V.

Let A be the matrix of b relative to an ordered basis of V.


If b is nondegenerate, its discriminant is the equivalence class of the determinant A in the quotient group 𝕂^×𝕂^×^2.

If b is degenerate, its discriminant is 0.
"
Definition:Discriminant,Discriminant,"Let $K$ be a conic section embedded in a Cartesian plane with the general equation:
:$a x^2 + 2 h x y + b y^2 + 2 g x + 2 f y + c = 0$
where $a, b, c, f, g, h \in \R$.


The discriminant of $K$ is defined as the determinant calculated as:
:$\Delta = \begin {vmatrix} a & h & g \\ h & b & f \\ g & f & c \end {vmatrix}$
",Definition:Discriminant of Conic Section,"['Definitions/Discriminants of Conic Sections', 'Definitions/Discriminants', 'Definitions/Conic Sections']","Let K be a conic section embedded in a Cartesian plane with the general equation:
:a x^2 + 2 h x y + b y^2 + 2 g x + 2 f y + c = 0
where a, b, c, f, g, h ∈.


The discriminant of K is defined as the determinant calculated as:
:Δ =  a     h     g 
 h     b     f 
 g     f     c
"
Definition:Discriminant,Discriminant,"A discriminant function is a function which assigns a given individual to one of a number of populations according to the data appertaining to that individual.

It is based on measurements on individuals for whom the population to which each one belongs is known.

It is chosen to minimize the probabilities or costs of misclassification.
",Definition:Discriminant Function,"['Definitions/Discriminant Functions', 'Definitions/Descriptive Statistics', 'Definitions/Discriminants']","A discriminant function is a function which assigns a given individual to one of a number of populations according to the data appertaining to that individual.

It is based on measurements on individuals for whom the population to which each one belongs is known.

It is chosen to minimize the probabilities or costs of misclassification.
"
Definition:Distance,Distance,"Let $x, y \in \C$ be complex numbers.

Let $\cmod {x - y}$ be the complex modulus of $x - y$.


Then the function $d: \C^2 \to \R$:
:$\map d {x, y} = \cmod {x - y}$
is called the distance between $x$ and $y$.",Definition:Distance/Points/Complex Numbers,['Definitions/Complex Analysis'],"Let x, y ∈ be complex numbers.

Let x - y be the complex modulus of x - y.


Then the function d: ^2 →:
:d x, y = x - y
is called the distance between x and y."
Definition:Distance,Distance,"Let $\struct {A, d}$ be a metric space.

The mapping $d: A \times A \to \R$ is referred to as a distance function on $A$.


Here, $d: A \times A \to \R$ is a real-valued function satisfying the metric space axioms:

",Definition:Metric Space/Distance Function,"['Definitions/Distance Functions', 'Definitions/Metric Spaces']","Let A, d be a metric space.

The mapping d: A × A → is referred to as a distance function on A.


Here, d: A × A → is a real-valued function satisfying the metric space axioms:

"
Definition:Distance,Distance,"Let $M = \struct {A, d}$ be a metric space.

Let $x \in A$.

Let $S, T$ be subsets of $A$.


The distance between $x$ and $S$ is defined and annotated $\ds \map d {x, S} = \inf_{y \mathop \in S} \paren {\map d {x, y} }$.

The distance between $S$ and $T$ is defined and annotated $\ds \map d {S, T} = \inf_{\substack {x \mathop \in S \\ y \mathop \in T} } \paren {\map d {x, y} }$.",Definition:Distance/Sets/Metric Spaces,"['Definitions/Metric Spaces', 'Definitions/Distance Function']","Let M = A, d be a metric space.

Let x ∈ A.

Let S, T be subsets of A.


The distance between x and S is defined and annotated d x, S = inf_y ∈ S d x, y.

The distance between S and T is defined and annotated d S, T = inf_x ∈ S 
 y ∈ T d x, y."
Definition:Distance,Distance,The distance between two points $A$ and $B$ in space is defined as the length of a straight line that would be drawn from $A$ to $B$.,Definition:Linear Measure/Distance,['Definitions/Length'],The distance between two points A and B in space is defined as the length of a straight line that would be drawn from A to B.
Definition:Distance,Distance,"The angular distance between two points $A$ and $B$  a reference point $P$ is the angle between the straight lines $AP$ and $BP$, that is:
:$\angle APB$",Definition:Angular Distance,['Definitions/Angles'],"The angular distance between two points A and B  a reference point P is the angle between the straight lines AP and BP, that is:
:∠ APB"
Definition:Distance,Distance,"Let $u$ and $v$ be two codewords of a linear code.

The Hamming distance between $u$ and $v$ is the number of corresponding terms at which $u$ and $v$ are different.
",Definition:Hamming Distance,"['Definitions/Hamming Distance', 'Definitions/Linear Codes']","Let u and v be two codewords of a linear code.

The Hamming distance between u and v is the number of corresponding terms at which u and v are different.
"
Definition:Divergent,Divergent,"A sequence which is not convergent is divergent.



=== Divergent Real Sequence ===

A real sequence which is not convergent is divergent.
",Definition:Divergent Sequence,"['Definitions/Divergent Sequences', 'Definitions/Divergence', 'Definitions/Sequences']","A sequence which is not convergent is divergent.



=== Divergent Real Sequence ===

A real sequence which is not convergent is divergent.
"
Definition:Divergent,Divergent,A series which is not convergent is divergent.,Definition:Divergent Series,"['Definitions/Divergent Series', 'Definitions/Divergence', 'Definitions/Series']",A series which is not convergent is divergent.
Definition:Divergent,Divergent,A function which is not convergent is divergent.,Definition:Divergent Function,['Definitions/Real Analysis'],A function which is not convergent is divergent.
Definition:Divergent,Divergent,"An improper integral of a real function $f$ is said to diverge if any of the following hold:

:$(1): \quad f$ is continuous on $\hointr a \to$ and the limit $\ds \lim_{b \mathop \to +\infty} \int_a^b \map f x \rd x$ does not exist

:$(2): \quad f$ is continuous on $\hointl \gets b$ and the limit $\ds \lim_{a \mathop \to -\infty} \int_a^b \map f x \rd x$ does not exist

:$(3): \quad f$ is continuous on $\hointr a b$, has an infinite discontinuity at $b$, and the limit $\ds \lim_{c \mathop \to b^-} \int_a^c \map f x \rd x$ does not exist

:$(4): \quad f$ is continuous on $\hointl a b$, has an infinite discontinuity at $a$, and the limit $\ds \lim_{c \mathop \to a^+} \int_c^b \map f x \rd x$ does not exist.",Definition:Divergent Improper Integral,"['Definitions/Divergent Improper Integrals', 'Definitions/Improper Integrals', 'Definitions/Integral Calculus']","An improper integral of a real function f is said to diverge if any of the following hold:

:(1):    f is continuous on a → and the limit lim_b → +∞∫_a^b  f x  x does not exist

:(2):    f is continuous on b and the limit lim_a → -∞∫_a^b  f x  x does not exist

:(3):    f is continuous on a b, has an infinite discontinuity at b, and the limit lim_c → b^-∫_a^c  f x  x does not exist

:(4):    f is continuous on a b, has an infinite discontinuity at a, and the limit lim_c → a^+∫_c^b  f x  x does not exist."
Definition:Divergent,Divergent,"An infinite product which is not convergent is divergent.




=== Divergence to zero ===

An infinite product which is not convergent is divergent.




=== Divergence to zero ===

An infinite product which is not convergent is divergent.




=== Divergence to zero ===

",Definition:Divergent Product,"['Definitions/Divergent Products', 'Definitions/Infinite Products', 'Definitions/Divergence']","An infinite product which is not convergent is divergent.




=== Divergence to zero ===

An infinite product which is not convergent is divergent.




=== Divergence to zero ===

An infinite product which is not convergent is divergent.




=== Divergence to zero ===

"
Definition:Divisor,Divisor,"Let $\struct {R, +, \circ}$ be an ring with unity whose zero is $0_R$ and whose unity is $1_R$.

Let $x, y \in D$.

We define the term $x$ divides $y$ in $R$ as follows:
:$x \mathrel {\divides_R} y \iff \exists t \in R: y = t \circ x$


When no ambiguity results, the subscript is usually dropped, and $x$ divides $y$ in $R$ is just written $x \divides y$.",Definition:Divisor (Algebra)/Ring with Unity,"['Definitions/Divisibility', 'Definitions/Factorization']","Let R, +, ∘ be an ring with unity whose zero is 0_R and whose unity is 1_R.

Let x, y ∈ D.

We define the term x divides y in R as follows:
:x _R y ∃ t ∈ R: y = t ∘ x


When no ambiguity results, the subscript is usually dropped, and x divides y in R is just written x  y."
Definition:Divisor,Divisor,"Let $D$ be an integral domain.

Let $D \sqbrk x$ be the polynomial ring in one variable over $D$.

Let $f, g \in D \sqbrk x$ be polynomials.


Then:
:$f$ divides $g$
:$f$ is a divisor of $g$
:$g$ is divisible by $f$
:
:$\exists h \in D \sqbrk x : g = f h$


This is denoted:
:$f \divides g$


=== Notation ===

",Definition:Divisor of Polynomial,"['Definitions/Divisors of Polynomials', 'Definitions/Divisors', 'Definitions/Polynomial Theory']","Let D be an integral domain.

Let D  x be the polynomial ring in one variable over D.

Let f, g ∈ D  x be polynomials.


Then:
:f divides g
:f is a divisor of g
:g is divisible by f
:
:∃ h ∈ D  x : g = f h


This is denoted:
:f  g


=== Notation ===

"
Definition:Divisor,Divisor,"Let $c = a / b$ denote the division operation on two elements $a$ and $b$ of a field or a Euclidean domain.

The element $b$ is the divisor of $a$.
",Definition:Division/Divisor,['Definitions/Division'],"Let c = a / b denote the division operation on two elements a and b of a field or a Euclidean domain.

The element b is the divisor of a.
"
Definition:Domain,Domain,"The collection of all possible objects that a variable may refer to has to be specified.

This collection is the domain of the variable.",Definition:Variable/Domain,"['Definitions/Algebra', 'Definitions/Variables']","The collection of all possible objects that a variable may refer to has to be specified.

This collection is the domain of the variable."
Definition:Domain,Domain,"Let $f: X \to Y$ be a morphism.

Then the domain of $f$ is defined to be the object $X$.

This is usually denoted $X = \Dom f$ or $X = \map D f$.
",Definition:Domain (Category Theory),['Definitions/Morphisms'],"Let f: X → Y be a morphism.

Then the domain of f is defined to be the object X.

This is usually denoted X =  f or X =  D f.
"
Definition:Domain,Domain,,Definition:Codomain,[],
Definition:Dominate,Dominate,"Let $S, T$ be sets.


$S$ is strictly dominated by set $T$  $S \preccurlyeq T$ but $\neg T \preccurlyeq S$.

This can be written $S \prec T$ or $S < T$.


Category:Definitions/Set Theory
",Definition:Dominate (Set Theory),['Definitions/Set Theory'],"Let S, T be sets.


S is strictly dominated by set T  S ≼ T but T ≼ S.

This can be written S ≺ T or S < T.


Category:Definitions/Set Theory
"
Definition:Dominate,Dominate,"Let $\sequence {a_n}$ be a sequence in $\R$.

Let $\sequence {z_n}$ be a sequence in $\C$.


Then $\sequence {a_n}$ dominates $\sequence {z_n}$ :
:$\forall n \in \N: \cmod {z_n} \le a_n$

Category:Definitions/Analysis
Category:Definitions/Complex Analysis",Definition:Dominate (Analysis),"['Definitions/Analysis', 'Definitions/Complex Analysis']","Let a_n be a sequence in .

Let z_n be a sequence in .


Then a_n dominates z_n :
:∀ n ∈: z_n≤ a_n

Category:Definitions/Analysis
Category:Definitions/Complex Analysis"
Definition:Dram,Dram,"The dram is an avoirdupois unit of mass.

=== Conversion Factors ===
",Definition:Avoirdupois/Dram,['Definitions/Dram (Avoirdupois)'],"The dram is an avoirdupois unit of mass.

=== Conversion Factors ===
"
Definition:Dram,Dram,"The drachm is an apothecaries' unit of mass.

=== Conversion Factors ===
",Definition:Apothecaries' Weights and Measures/Mass/Drachm,['Definitions/Drachm'],"The drachm is an apothecaries' unit of mass.

=== Conversion Factors ===
"
Definition:Dual,Dual,"Let $\struct {S, \preceq}$ be an ordered set.

Let $\succeq$ be the dual ordering of $\preceq$.


The ordered set $\struct {S, \succeq}$ is called the dual ordered set (or just dual) of $\struct {S, \preceq}$.


That it indeed is an ordered set is a consequence of Dual Ordering is Ordering.
To denote the dual of an ordering, the conventional technique is to reverse the symbol.

Thus:
:$\succeq$ denotes $\preceq^{-1}$
:$\succcurlyeq$ denotes $\preccurlyeq^{-1}$
:$\curlyeqsucc$ denotes $\curlyeqprec^{-1}$

and so:
:$a \preceq b \iff b \succeq a$
:$a \preccurlyeq b \iff b \succcurlyeq a$
:$a \curlyeqprec b \iff b \curlyeqsucc a$


Similarly for the standard symbols used to denote an ordering on numbers:
:$\ge$ denotes $\le^{-1}$
:$\geqslant$ denotes $\leqslant^{-1}$
:$\eqslantgtr$ denotes $\eqslantless^{-1}$

and so on.
To denote the dual of an strict ordering, the conventional technique is to reverse the symbol.

Thus:
:$\succ$ denotes $\prec^{-1}$

and so:
:$a \prec b \iff b \succ a$


Similarly for the standard symbol used to denote a strict ordering on numbers:
:$>$ denotes $<^{-1}$

and so on.
",Definition:Dual Ordering,"['Definitions/Dual Orderings', 'Definitions/Order Theory']","Let S, ≼ be an ordered set.

Let ≽ be the dual ordering of ≼.


The ordered set S, ≽ is called the dual ordered set (or just dual) of S, ≼.


That it indeed is an ordered set is a consequence of Dual Ordering is Ordering.
To denote the dual of an ordering, the conventional technique is to reverse the symbol.

Thus:
:≽ denotes ≼^-1
:≽ denotes ≼^-1
:⋟ denotes ⋞^-1

and so:
:a ≼ b  b ≽ a
:a ≼ b  b ≽ a
:a ⋞ b  b ⋟ a


Similarly for the standard symbols used to denote an ordering on numbers:
:≥ denotes ≤^-1
:⩾ denotes ⩽^-1
:⪖ denotes ⪕^-1

and so on.
To denote the dual of an strict ordering, the conventional technique is to reverse the symbol.

Thus:
:≻ denotes ≺^-1

and so:
:a ≺ b  b ≻ a


Similarly for the standard symbol used to denote a strict ordering on numbers:
:> denotes <^-1

and so on.
"
Definition:Dual,Dual,"Let $\struct {S, \preceq_S}$ and $\struct {T, \preceq_T}$ be ordered sets.

Let $\phi: S \to T$ be a bijection.


Then $\phi$ is a dual isomorphism between $\struct {S, \preceq_S}$ and $\struct {T, \preceq_T}$  $\phi$ and $\phi^{-1}$ are decreasing mappings.


If there is a dual isomorphism between $\struct {S, \preceq_S}$ and $\struct {T, \preceq_T}$, then $\struct {S, \preceq_S}$ is dual to $\struct {T, \preceq_T}$.

Equivalently, $\struct {S, \preceq_S}$ is dual to $\struct {T, \preceq_T}$  $S$ with the dual ordering is isomorphic to $T$.
",Definition:Self-Dual (Order Theory),['Definitions/Order Theory'],"Let S, ≼_S and T, ≼_T be ordered sets.

Let ϕ: S → T be a bijection.


Then ϕ is a dual isomorphism between S, ≼_S and T, ≼_T  ϕ and ϕ^-1 are decreasing mappings.


If there is a dual isomorphism between S, ≼_S and T, ≼_T, then S, ≼_S is dual to T, ≼_T.

Equivalently, S, ≼_S is dual to T, ≼_T  S with the dual ordering is isomorphic to T.
"
Definition:Dual,Dual,"Let $R$ be a commutative ring.

Let $G$ be an $R$-module.


The double dual $G^{**}$ of $G$ is the dual of its dual $G^*$.


Category:Definitions/Algebraic Duals
",Definition:Algebraic Dual,"['Definitions/Linear Forms (Linear Algebra)', 'Definitions/Algebraic Duals']","Let R be a commutative ring.

Let G be an R-module.


The double dual G^** of G is the dual of its dual G^*.


Category:Definitions/Algebraic Duals
"
Definition:Dual,Dual,"Let $\struct {R, +, \times}$ be a commutative ring.

Let $\struct {G, +_G, \circ}_R$ be an $R$-module.

Let $\struct {R, +_R, \circ}_R$ denote the $R$-module $R$.


The $R$-module $\map {\LL_R} {G, R}$ of all linear forms on $G$ is usually denoted $G^*$ and is called the algebraic dual of $G$.


=== Double Dual ===

Let $\struct {R, +, \times}$ be a commutative ring.

Let $\struct {G, +_G, \circ}_R$ be an $R$-module.

Let $\struct {R, +_R, \circ}_R$ denote the $R$-module $R$.


The $R$-module $\map {\LL_R} {G, R}$ of all linear forms on $G$ is usually denoted $G^*$ and is called the algebraic dual of $G$.


=== Double Dual ===

",Definition:Algebraic Dual/Double Dual,['Definitions/Algebraic Duals'],"Let R, +, × be a commutative ring.

Let G, +_G, ∘_R be an R-module.

Let R, +_R, ∘_R denote the R-module R.


The R-module _RG, R of all linear forms on G is usually denoted G^* and is called the algebraic dual of G.


=== Double Dual ===

Let R, +, × be a commutative ring.

Let G, +_G, ∘_R be an R-module.

Let R, +_R, ∘_R denote the R-module R.


The R-module _RG, R of all linear forms on G is usually denoted G^* and is called the algebraic dual of G.


=== Double Dual ===

"
Definition:Dual,Dual,"Let $\struct {R, +, \times}$ be a commutative ring.

Let $\struct {G, +_G, \circ}_R$ be an $R$-module.

Let $\struct {R, +_R, \circ}_R$ denote the $R$-module $R$.


The $R$-module $\map {\LL_R} {G, R}$ of all linear forms on $G$ is usually denoted $G^*$ and is called the algebraic dual of $G$.


=== Double Dual ===

",Definition:Ordered Dual Basis,['Definitions/Linear Algebra'],"Let R, +, × be a commutative ring.

Let G, +_G, ∘_R be an R-module.

Let R, +_R, ∘_R denote the R-module R.


The R-module _RG, R of all linear forms on G is usually denoted G^* and is called the algebraic dual of G.


=== Double Dual ===

"
Definition:Dual,Dual,"Let $V$ be a vector space.

Let $\phi: V \to \R$ be a linear mapping.


The set of all $\phi$ is called a dual space (of $V$) and is denoted by $V^*$.",Definition:Dual Vector Space,"['Definitions/Vector Spaces', 'Definitions/Algebraic Duals']","Let V be a vector space.

Let ϕ: V → be a linear mapping.


The set of all ϕ is called a dual space (of V) and is denoted by V^*."
Definition:Dual,Dual,"Let $\struct {X, \norm \cdot_X}$ be a normed vector space.

Let $X^\ast$ be the vector space of bounded linear functionals on $X$. 

Let $\norm \cdot_{X^\ast}$ be the norm on bounded linear functionals.


We say that $\struct {X^\ast, \norm \cdot_{X^\ast} }$ is the normed dual space of $X$.",Definition:Normed Dual Space,"['Definitions/Normed Vector Spaces', 'Definitions/Functional Analysis', 'Definitions/Normed Dual Spaces']","Let X, ·_X be a normed vector space.

Let X^∗ be the vector space of bounded linear functionals on X. 

Let ·_X^∗ be the norm on bounded linear functionals.


We say that X^∗, ·_X^∗ is the normed dual space of X."
Definition:Dual,Dual,"Let $\struct {X, \norm \cdot_X}$ be a normed vector space.

Let $X^\ast$ be the vector space of bounded linear functionals on $X$. 

Let $\norm \cdot_{X^\ast}$ be the norm on bounded linear functionals.


We say that $\struct {X^\ast, \norm \cdot_{X^\ast} }$ is the normed dual space of $X$.
Let $\struct {X, \norm \cdot_X}$ be a normed vector space.

Let $X^\ast$ be the vector space of bounded linear functionals on $X$. 

Let $\norm \cdot_{X^\ast}$ be the norm on bounded linear functionals.


We say that $\struct {X^\ast, \norm \cdot_{X^\ast} }$ is the normed dual space of $X$.
",Definition:Second Normed Dual,"['Definitions/Normed Dual Spaces', 'Definitions/Second Normed Duals']","Let X, ·_X be a normed vector space.

Let X^∗ be the vector space of bounded linear functionals on X. 

Let ·_X^∗ be the norm on bounded linear functionals.


We say that X^∗, ·_X^∗ is the normed dual space of X.
Let X, ·_X be a normed vector space.

Let X^∗ be the vector space of bounded linear functionals on X. 

Let ·_X^∗ be the norm on bounded linear functionals.


We say that X^∗, ·_X^∗ is the normed dual space of X.
"
Definition:Dual,Dual,"Let $\mathbf C$ be a metacategory.


Its dual category, denoted $\mathbf C^{\text{op} }$, is defined as follows:



It can be seen that this comes down to the metacategory obtained by reversing the direction of all morphisms of $\mathbf C$.",Definition:Dual Category,"['Definitions/Category Theory', 'Definitions/Examples of Categories']","Let 𝐂 be a metacategory.


Its dual category, denoted 𝐂^op, is defined as follows:



It can be seen that this comes down to the metacategory obtained by reversing the direction of all morphisms of 𝐂."
Definition:Dual,Dual,"=== Morphisms-Only Category Theory ===

Let $\Sigma$ be a statement in the language of category theory.

The dual statement $\Sigma^*$ of $\Sigma$ is the statement obtained from substituting:








=== Object Category Theory ===

In the more convenient description of metacategories by using objects, the dual statement $\Sigma^*$ of $\Sigma$ then becomes the statement obtained from substituting:









=== Example ===

For example, if $\Sigma$ is the statement:

:$\exists g: g \circ f = \operatorname{id}_{\Dom f}$

describing that $f$ is a split mono, then $\Sigma^*$ becomes:

:$\exists g: f \circ g = \operatorname{id}_{\Cdm f}$

which precisely expresses $f$ to be a split epi.


For a set $\EE$ of statements, write:

:$\EE^* := \set {\Sigma^*: \Sigma \in \EE}$ 

for the set comprising of the dual statement of those in $\EE$.",Definition:Dual Statement (Category Theory),['Definitions/Category Theory'],"=== Morphisms-Only Category Theory ===

Let Σ be a statement in the language of category theory.

The dual statement Σ^* of Σ is the statement obtained from substituting:








=== Object Category Theory ===

In the more convenient description of metacategories by using objects, the dual statement Σ^* of Σ then becomes the statement obtained from substituting:









=== Example ===

For example, if Σ is the statement:

:∃ g: g ∘ f = id_ f

describing that f is a split mono, then Σ^* becomes:

:∃ g: f ∘ g = id_ f

which precisely expresses f to be a split epi.


For a set  of statements, write:

:^* := Σ^*: Σ∈ 

for the set comprising of the dual statement of those in ."
Definition:Dual,Dual,"Let $P$ be a polyhedron.

The dual polyhedron $D$ of $P$ is the polyhedron which can be constructed as follows:

:$(1): \quad$ The vertices of $D$ are the centroids of the faces of $P$.

:$(2): \quad$ For each edge of $P$ which is adjacent to two faces $F_1$ and $F_2$ of $P$, an edge of $D$ is constructed which is adjacent to the vertices of $D$ forming the centroids of $F_1$ and $F_2$.",Definition:Dual Polyhedron,['Definitions/Polyhedra'],"Let P be a polyhedron.

The dual polyhedron D of P is the polyhedron which can be constructed as follows:

:(1): The vertices of D are the centroids of the faces of P.

:(2): For each edge of P which is adjacent to two faces F_1 and F_2 of P, an edge of D is constructed which is adjacent to the vertices of D forming the centroids of F_1 and F_2."
Definition:Dual,Dual,,Definition:Self-Dual,[],
Definition:Eccentricity,Eccentricity,"Let $K$ be a conic section specified in terms of:
:a given straight line $D$
:a given point $F$
:a given constant $e$

where $K$ is the locus of points $P$ such that the distance $p$ from $P$ to $D$ and the distance $q$ from $P$ to $F$ are related by the condition:
:$q = e p$


The constant $e$ is known as the eccentricity of the conic section.
",Definition:Conic Section/Eccentricity,"['Definitions/Eccentricity of Conic Section', 'Definitions/Conic Sections']","Let K be a conic section specified in terms of:
:a given straight line D
:a given point F
:a given constant e

where K is the locus of points P such that the distance p from P to D and the distance q from P to F are related by the condition:
:q = e p


The constant e is known as the eccentricity of the conic section.
"
Definition:Eccentricity,Eccentricity,"Let $G = \struct {V, E}$ be a graph.

Let $v \in V$ be a vertex of $G$.


The eccentricity of $v$ is the maximum distance from $v$ to another vertex of $G$:


That is:
:$\map E v = \ds \max_{u \mathop \in V} \map D {v, u}$

where $\map D {v, u}$ denotes the distance from $v$ to $u$.",Definition:Eccentricity of Vertex,['Definitions/Graph Theory'],"Let G = V, E be a graph.

Let v ∈ V be a vertex of G.


The eccentricity of v is the maximum distance from v to another vertex of G:


That is:
:E v = max_u ∈ V D v, u

where D v, u denotes the distance from v to u."
Definition:Edge,Edge,"


Let $G = \struct {V, E}$ be a graph.

The edges are the elements of $E$.


In the above, the edges are $AB, AE, BE, CD, CE, CF, DE, DF, FG$.


=== Join ===




Let $G = \struct {V, E}$ be a graph.

The edges are the elements of $E$.


In the above, the edges are $AB, AE, BE, CD, CE, CF, DE, DF, FG$.


=== Join ===

",Definition:Graph (Graph Theory)/Edge,"['Definitions/Edges of Graphs', 'Definitions/Edges']","


Let G = V, E be a graph.

The edges are the elements of E.


In the above, the edges are AB, AE, BE, CD, CE, CF, DE, DF, FG.


=== Join ===




Let G = V, E be a graph.

The edges are the elements of E.


In the above, the edges are AB, AE, BE, CD, CE, CF, DE, DF, FG.


=== Join ===

"
Definition:Edge,Edge,"The edges of a polyhedron are the sides of the polygons which constitute its faces.
",Definition:Polyhedron/Edge,"['Definitions/Edges of Polyhedra', 'Definitions/Polyhedra', 'Definitions/Edges']","The edges of a polyhedron are the sides of the polygons which constitute its faces.
"
Definition:Edge,Edge,"Each of the half-lines that form a polyhedral angle is known as an edge of the polyhedral angle.
",Definition:Polyhedral Angle/Edge,"['Definitions/Polyhedral Angles', 'Definitions/Edges']","Each of the half-lines that form a polyhedral angle is known as an edge of the polyhedral angle.
"
Definition:Edge,Edge,"Let $\PP$ denote the plane.

Let $\LL$ denote an infinite straight line in $\PP$.

Let $\HH$ denote one of the half-planes into which $\LL$ divides $\PP$.

Then $\LL$ is called the edge of $\HH$.
",Definition:Half-Plane/Edge,"['Definitions/Half-Planes', 'Definitions/Edges']","Let  denote the plane.

Let  denote an infinite straight line in .

Let  denote one of the half-planes into which  divides .

Then  is called the edge of .
"
Definition:Edge,Edge,"Let $\mathbf C$ be a metacategory.


A morphism of $\mathbf C$ is an object $f$, together with:

* A domain $\operatorname {dom} f$, which is an object of $\mathbf C$
* A codomain $\operatorname {cod} f$, also an object of $\mathbf C$


The collection of all morphisms of $\mathbf C$ is denoted $\mathbf C_1$.


If $A$ is the domain of $f$ and $B$ is its codomain, this is mostly represented by writing:

:$f: A \to B$ or $A \stackrel f \longrightarrow B$",Definition:Morphism,"['Definitions/Morphisms', 'Definitions/Category Theory']","Let 𝐂 be a metacategory.


A morphism of 𝐂 is an object f, together with:

* A domain dom f, which is an object of 𝐂
* A codomain cod f, also an object of 𝐂


The collection of all morphisms of 𝐂 is denoted 𝐂_1.


If A is the domain of f and B is its codomain, this is mostly represented by writing:

:f: A → B or A  f ⟶ B"
Definition:Efficiency,Efficiency,"Let $T_0$ and $T_1$ both be statistics used as estimators.

Efficiency is a comparison of the variances of $T_0$ and $T_1$.


Thus $T_0$ is of higher efficiency than $T_1$  $T_0$ has a smaller variance than $T_1$.
Let $T_0$ and $T_1$ both be statistics used as estimators.

Efficiency is a comparison of the variances of $T_0$ and $T_1$.


Thus $T_0$ is of higher efficiency than $T_1$  $T_0$ has a smaller variance than $T_1$.
",Definition:Efficiency (Statistics),"['Definitions/Efficiency (Statistics)', 'Definitions/Statistics', 'Definitions/Efficiency']","Let T_0 and T_1 both be statistics used as estimators.

Efficiency is a comparison of the variances of T_0 and T_1.


Thus T_0 is of higher efficiency than T_1  T_0 has a smaller variance than T_1.
Let T_0 and T_1 both be statistics used as estimators.

Efficiency is a comparison of the variances of T_0 and T_1.


Thus T_0 is of higher efficiency than T_1  T_0 has a smaller variance than T_1.
"
Definition:Efficiency,Efficiency,One design is more efficient than another if the same precision can be achieved with the same resources.,Definition:Efficiency (Experimental Design),"['Definitions/Efficiency (Experimental Design)', 'Definitions/Experimental Designs', 'Definitions/Efficiency']",One design is more efficient than another if the same precision can be achieved with the same resources.
Definition:Elevation,Elevation,"The elevation of a point $P$ is the height of $P$ above some reference horizontal baseline or plane.
",Definition:Elevation of Point,"['Definitions/Elevation of Point', 'Definitions/Elevation']","The elevation of a point P is the height of P above some reference horizontal baseline or plane.
"
Definition:Elevation,Elevation,"Let $A$ and $B$ be points in space such that $A$ is higher than $B$.

The angle of elevation of $A$ from $B$ is the angle between the line $AB$ and the horizontal.
",Definition:Angle of Elevation,"['Definitions/Angles of Elevation', 'Definitions/Angles', 'Definitions/Elevation']","Let A and B be points in space such that A is higher than B.

The angle of elevation of A from B is the angle between the line AB and the horizontal.
"
Definition:Embedding,Embedding,"Let $A, B$ be topological spaces.

Let $f: A \to B$ be a mapping.

Let the image of $f$ be given the subspace topology.

Let the restriction $f {\restriction_{A \times f\sqbrk A }}$ of $f$ to its image be a homeomorphism.


Then $f$ is an embedding (of $A$ into $B$).
",Definition:Embedding (Topology),"['Definitions/Embeddings (Topology)', 'Definitions/Homeomorphisms (Topological Spaces)', 'Definitions/Topology']","Let A, B be topological spaces.

Let f: A → B be a mapping.

Let the image of f be given the subspace topology.

Let the restriction f _A × f A of f to its image be a homeomorphism.


Then f is an embedding (of A into B).
"
Definition:Embedding,Embedding,"Let $\MM$ and $\NN$ be $\LL$-structures with universes $M$ and $N$ respectively.


An $\LL$-embedding $j:\MM \to \NN$ is an elementary embedding  it preserves truth; that is:

:$\MM \models \map \phi {a_1, \ldots, a_n} \iff \NN \models \map \phi {\map j {a_1}, \ldots, \map j {a_n} }$

holds for all $n \in \N$, all $\LL$-formulas $\phi$ with $n$ free variables, and for all $a_1, \ldots, a_n \in M$.


=== Partial Elementary Embedding ===




Category:Definitions/Model Theory for Predicate Logic
",Definition:Embedding (Model Theory),"['Definitions/Model Theory for Predicate Logic', 'Definitions/Automorphisms']","Let  and  be -structures with universes M and N respectively.


An -embedding j:→ is an elementary embedding  it preserves truth; that is:

:ϕa_1, …, a_nϕ j a_1, …,  j a_n

holds for all n ∈, all -formulas ϕ with n free variables, and for all a_1, …, a_n ∈ M.


=== Partial Elementary Embedding ===




Category:Definitions/Model Theory for Predicate Logic
"
Definition:Embedding,Embedding,"Let $K$ and $L$ be fields.

A (field) monomorphism $\phi: K \to L$ is called an embedding of $K$ in $L$.
",Definition:Embedding (Galois Theory),"['Definitions/Galois Theory', 'Definitions/Field Theory']","Let K and L be fields.

A (field) monomorphism ϕ: K → L is called an embedding of K in L.
"
Definition:Embedding,Embedding,"Let $\struct {R, +, \circ}$ and $\struct {S, \oplus, *}$ be rings.

Let $\phi: R \to S$ be a (ring) homomorphism.


Then $\phi$ is a ring monomorphism  $\phi$ is an injection.",Definition:Ring Monomorphism,"['Definitions/Monomorphisms (Abstract Algebra)', 'Definitions/Ring Homomorphisms']","Let R, +, ∘ and S, ⊕, * be rings.

Let ϕ: R → S be a (ring) homomorphism.


Then ϕ is a ring monomorphism  ϕ is an injection."
Definition:Embedding,Embedding,"Let $C$ and $D$ be categories.

Let $F : C \to D$ be a functor.


=== Definition 1 ===

The functor $F$ is an embedding  it is:
* injective on objects
* faithful


=== Definition 2 ===

The functor $F$ is an embedding  it is injective on morphisms.


=== Definition 3 ===

The functor $F$ is an embedding  it is a monomorphisms in the category of categories.",Definition:Embedding of Categories,['Definitions/Category Theory'],"Let C and D be categories.

Let F : C → D be a functor.


=== Definition 1 ===

The functor F is an embedding  it is:
* injective on objects
* faithful


=== Definition 2 ===

The functor F is an embedding  it is injective on morphisms.


=== Definition 3 ===

The functor F is an embedding  it is a monomorphisms in the category of categories."
Definition:Empty,Empty,"The empty set is a set which has no elements.

That is, $x \in \O$ is false, whatever $x$ is.


It is usually denoted by some variant of a zero with a line through it, for example $\O$ or $\emptyset$, and can always be represented as $\set {}$.
",Definition:Empty Set,"['Definitions/Empty Set', 'Definitions/Set Theory']","The empty set is a set which has no elements.

That is, x ∈Ø is false, whatever x is.


It is usually denoted by some variant of a zero with a line through it, for example Ø or ∅, and can always be represented as .
"
Definition:Empty,Empty,"A class is defined as being empty  it has no elements.

That is:
:$\forall x: x \notin A$
or:
:$\neg \exists x: x \in A$


The empty class is usually denoted $\O$ or $\emptyset$.

On  the preferred symbol is $\O$.
A class is defined as being empty  it has no elements.

That is:
:$\forall x: x \notin A$
or:
:$\neg \exists x: x \in A$


The empty class is usually denoted $\O$ or $\emptyset$.

On  the preferred symbol is $\O$.
",Definition:Empty Class (Class Theory),['Definitions/Class Theory'],"A class is defined as being empty  it has no elements.

That is:
:∀ x: x ∉ A
or:
:∃ x: x ∈ A


The empty class is usually denoted Ø or ∅.

On  the preferred symbol is Ø.
A class is defined as being empty  it has no elements.

That is:
:∀ x: x ∉ A
or:
:∃ x: x ∈ A


The empty class is usually denoted Ø or ∅.

On  the preferred symbol is Ø.
"
Definition:Empty,Empty,"An edgeless graph is a graph with no edges.

That is, an edgeless graph is a graph of size zero.

Equivalently, an edgeless graph is a graph whose vertices are all isolated.


The edgeless graph of order $n$ is denoted $N_n$ and can be referred to as the $n$-edgeless graph.",Definition:Edgeless Graph,"['Definitions/Edgeless Graphs', 'Definitions/Graph Theory']","An edgeless graph is a graph with no edges.

That is, an edgeless graph is a graph of size zero.

Equivalently, an edgeless graph is a graph whose vertices are all isolated.


The edgeless graph of order n is denoted N_n and can be referred to as the n-edgeless graph."
Definition:Empty,Empty,"Take the summation:
:$\ds \sum_{\map \Phi j} a_j$
where $\map \Phi j$ is a propositional function of $j$.

Suppose that there are no values of $j$ for which $\map \Phi j$ is true.

Then $\ds \sum_{\map \Phi j} a_j$ is defined as being $0$.

This summation is called a vacuous summation.


This is because:
:$\forall a: a + 0 = a$
where $a$ is a number.

Hence for all $j$ for which $\map \Phi j$ is false, the sum is unaffected.


This is most frequently seen in the form:
:$\ds \sum_{j \mathop = m}^n a_j = 0$
where $m > n$.

In this case, $j$ can not at the same time be both greater than or equal to $m$ and less than or equal to $n$.


Some sources consider such a treatment as abuse of notation.",Definition:Summation/Vacuous Summation,['Definitions/Summations'],"Take the summation:
:∑_Φ j a_j
where Φ j is a propositional function of j.

Suppose that there are no values of j for which Φ j is true.

Then ∑_Φ j a_j is defined as being 0.

This summation is called a vacuous summation.


This is because:
:∀ a: a + 0 = a
where a is a number.

Hence for all j for which Φ j is false, the sum is unaffected.


This is most frequently seen in the form:
:∑_j  = m^n a_j = 0
where m > n.

In this case, j can not at the same time be both greater than or equal to m and less than or equal to n.


Some sources consider such a treatment as abuse of notation."
Definition:Empty,Empty,A class interval is empty  it is of frequency zero.,Definition:Class Interval/Empty,['Definitions/Class Intervals'],A class interval is empty  it is of frequency zero.
Definition:Empty,Empty,"The category $\mathbf 0$, zero, is the empty category:


:$\qquad$


with:
:no objects
and consequently:
:no morphisms.",Definition:Zero (Category),['Definitions/Examples of Categories'],"The category 0, zero, is the empty category:


:


with:
:no objects
and consequently:
:no morphisms."
Definition:Empty Class,Empty Class,"A class is defined as being empty  it has no elements.

That is:
:$\forall x: x \notin A$
or:
:$\neg \exists x: x \in A$


The empty class is usually denoted $\O$ or $\emptyset$.

On  the preferred symbol is $\O$.
",Definition:Empty Class (Class Theory),['Definitions/Class Theory'],"A class is defined as being empty  it has no elements.

That is:
:∀ x: x ∉ A
or:
:∃ x: x ∈ A


The empty class is usually denoted Ø or ∅.

On  the preferred symbol is Ø.
"
Definition:Empty Class,Empty Class,A class interval is empty  it is of frequency zero.,Definition:Class Interval/Empty,['Definitions/Class Intervals'],A class interval is empty  it is of frequency zero.
Definition:Entropy,Entropy,"Let $X$ be a discrete random variable.

Let $X$ take a finite number of values with probabilities $p_1, p_2, \dotsc, p_n$.


The uncertainty of $X$ is defined as:

:$\map H X = \ds -\sum_k p_k \lg p_k$

where:
:$\lg$ denotes logarithm base $2$
:the summation is over those $k$ where $p_k > 0$.


=== Units ===
",Definition:Uncertainty,"['Definitions/Uncertainty', 'Definitions/Information Theory', 'Definitions/Probability Theory']","Let X be a discrete random variable.

Let X take a finite number of values with probabilities p_1, p_2, …, p_n.


The uncertainty of X is defined as:

:H X =  -∑_k p_k  p_k

where:
: denotes logarithm base 2
:the summation is over those k where p_k > 0.


=== Units ===
"
Definition:Entropy,Entropy,"Let $X$ be a discrete random variable.

Let $X$ take a finite number of values with probabilities $p_1, p_2, \dotsc, p_n$.


The uncertainty of $X$ is defined as:

:$\map H X = \ds -\sum_k p_k \lg p_k$

where:
:$\lg$ denotes logarithm base $2$
:the summation is over those $k$ where $p_k > 0$.


=== Units ===

",Definition:Differential Entropy,['Definitions/Probability Theory'],"Let X be a discrete random variable.

Let X take a finite number of values with probabilities p_1, p_2, …, p_n.


The uncertainty of X is defined as:

:H X =  -∑_k p_k  p_k

where:
: denotes logarithm base 2
:the summation is over those k where p_k > 0.


=== Units ===

"
Definition:Entropy,Entropy,"Entropy is a property of a thermodynamic system.

It quantifies the number $\Omega$ of microstates that are consistent with the macroscopic quantities that characterize the system.


The entropy of a system is equal to the expectation of the value:
:$k \ln P$
where:
:$k$ is a constant which relates the mean kinetic energy and absolute temperature of the system
:$P$ is the coefficient of probability of the system.



",Definition:Entropy (Physics),"['Definitions/Physics', 'Definitions/Physical Quantities', 'Definitions/Examples of Scalar Quantities']","Entropy is a property of a thermodynamic system.

It quantifies the number Ω of microstates that are consistent with the macroscopic quantities that characterize the system.


The entropy of a system is equal to the expectation of the value:
:k ln P
where:
:k is a constant which relates the mean kinetic energy and absolute temperature of the system
:P is the coefficient of probability of the system.



"
Definition:Epicycle,Epicycle,"An epicycle is the orbit described by a body moving in a uniform circular motion around a point which is itself moving in a uniform circular motion around another point.

:

That point may itself also be moving in a uniform circular motion around yet another point.
",Definition:Epicycle (Ptolemaic Astronomy),"['Definitions/Geometry', 'Definitions/Circles', 'Definitions/Epicycles']","An epicycle is the orbit described by a body moving in a uniform circular motion around a point which is itself moving in a uniform circular motion around another point.

:

That point may itself also be moving in a uniform circular motion around yet another point.
"
Definition:Epicycle,Epicycle,"Let an epicycloid be generated by rolling a circle $C_1$ around the outside of another circle $C_2$.


:


The circle $C_1$ can be referred to as the epicycle of the epicycloid.
",Definition:Epicycloid/Generator/Epicycle,['Definitions/Epicycloids'],"Let an epicycloid be generated by rolling a circle C_1 around the outside of another circle C_2.


:


The circle C_1 can be referred to as the epicycle of the epicycloid.
"
Definition:Epicycle,Epicycle,"Let a hypocycloid be generated by rolling circle $C_1$ around the inside of another (larger) circle $C_2$.


:


The circle $C_1$ can be referred to as the epicycle of the hypocycloid.
",Definition:Hypocycloid/Generator/Epicycle,"['Definitions/Hypocycloids', 'Definitions/Epicycles']","Let a hypocycloid be generated by rolling circle C_1 around the inside of another (larger) circle C_2.


:


The circle C_1 can be referred to as the epicycle of the hypocycloid.
"
Definition:Equator,Equator,"The (geographical) equator is the great circle described on the surface of Earth whose plane is perpendicular to Earth's axis of rotation.


:
",Definition:Geographical Equator,"['Definitions/Geographical Equator', 'Definitions/Geographical Coordinates', 'Definitions/Equator']","The (geographical) equator is the great circle described on the surface of Earth whose plane is perpendicular to Earth's axis of rotation.


:
"
Definition:Equator,Equator,"Consider the celestial sphere with observer $O$.

Let $P$ be the north celestial pole.


The great circle whose plane is perpendicular to $OP$ is known as the celestial equator.


:",Definition:Celestial Equator,"['Definitions/Celestial Equator', 'Definitions/Celestial Sphere', 'Definitions/Equator']","Consider the celestial sphere with observer O.

Let P be the north celestial pole.


The great circle whose plane is perpendicular to OP is known as the celestial equator.


:"
Definition:Equiangular,Equiangular,An equiangular polygon is a polygon in which all the vertices have the same angle.,Definition:Polygon/Equiangular,"['Definitions/Equiangular Polygons', 'Definitions/Equiangular', 'Definitions/Polygons']",An equiangular polygon is a polygon in which all the vertices have the same angle.
Definition:Equiangular,Equiangular,"Two geometric figures are equiangular (with each other) when the angles of each pair of their corresponding vertices are equal.
",Definition:Equiangular Geometric Figures,"['Definitions/Geometric Figures', 'Definitions/Equiangular']","Two geometric figures are equiangular (with each other) when the angles of each pair of their corresponding vertices are equal.
"
Definition:Equiangular,Equiangular,"The logarithmic spiral is the locus of the equation expressed in Polar coordinates as:
:$r = a e^{b \theta}$


:",Definition:Logarithmic Spiral,"['Definitions/Logarithmic Spiral', 'Definitions/Spirals']","The logarithmic spiral is the locus of the equation expressed in Polar coordinates as:
:r = a e^b θ


:"
Definition:Equiangular,Equiangular,"A rectangular hyperbola is a hyperbola whose transverse axis is equal to its conjugate axis.


=== Standard Form ===
",Definition:Rectangular Hyperbola,"['Definitions/Rectangular Hyperbolas', 'Definitions/Hyperbolas']","A rectangular hyperbola is a hyperbola whose transverse axis is equal to its conjugate axis.


=== Standard Form ===
"
Definition:Equiangular,Equiangular,"Let $T$ be a transformation of the plane.

Let $T$ have the property that:
:for all pairs of curves $\CC_1$ and $\CC_2$ which intersect at angle $\theta$, the images of $\CC_1$ and $\CC_2$ under $T$ also intersect at angle $\theta$.

Then $T$ is defined as being a conformal transformation.",Definition:Conformal Transformation,"['Definitions/Conformal Transformations', 'Definitions/Analytic Geometry', 'Definitions/Mapping Theory']","Let T be a transformation of the plane.

Let T have the property that:
:for all pairs of curves _1 and _2 which intersect at angle θ, the images of _1 and _2 under T also intersect at angle θ.

Then T is defined as being a conformal transformation."
Definition:Equilibrium,Equilibrium,"Let $S$ be a stochastic process.

Suppose that the observations of the time series to which $S$ gives rise have a constant mean level.

Then $S$ is said to be in (statistical) equilibrium.",Definition:Statistical Equilibrium,['Definitions/Stochastic Processes'],"Let S be a stochastic process.

Suppose that the observations of the time series to which S gives rise have a constant mean level.

Then S is said to be in (statistical) equilibrium."
Definition:Equilibrium,Equilibrium,"The equilibrium position of a body $B$ attached to a spring $S$ is the position it occupies when $S$ is exerting no force upon $B$.

For an ideal spring obeying Hooke's Law $\mathbf F = -k \mathbf x$, the equilibrium position is set to be the point $\mathbf x = \bszero$.",Definition:Spring/Equilibrium Position,['Definitions/Mechanics'],"The equilibrium position of a body B attached to a spring S is the position it occupies when S is exerting no force upon B.

For an ideal spring obeying Hooke's Law 𝐅 = -k 𝐱, the equilibrium position is set to be the point 𝐱 =."
Definition:Equilibrium,Equilibrium,An equilibrium point is a stable outcome of a game associated with a particular set of strategies.,Definition:Equilibrium Point,['Definitions/Game Theory'],An equilibrium point is a stable outcome of a game associated with a particular set of strategies.
Definition:Equilibrium,Equilibrium,"A system of strategies, one for each player, is in equilibrium  they result in an equilibrium point.

Such strategies are known as equilibrium strategies.


In the singular, an equilibrium strategy is one that contributes to an equilibrium point.
An equilibrium point is a stable outcome of a game associated with a particular set of strategies.
A system of strategies, one for each player, is in equilibrium  they result in an equilibrium point.

Such strategies are known as equilibrium strategies.


In the singular, an equilibrium strategy is one that contributes to an equilibrium point.
A system of strategies, one for each player, is in equilibrium  they result in an equilibrium point.

Such strategies are known as equilibrium strategies.


In the singular, an equilibrium strategy is one that contributes to an equilibrium point.
An equilibrium point is a stable outcome of a game associated with a particular set of strategies.
",Definition:Equilibrium Strategy,"['Definitions/Equilibrium Strategies', 'Definitions/Strategies']","A system of strategies, one for each player, is in equilibrium  they result in an equilibrium point.

Such strategies are known as equilibrium strategies.


In the singular, an equilibrium strategy is one that contributes to an equilibrium point.
An equilibrium point is a stable outcome of a game associated with a particular set of strategies.
A system of strategies, one for each player, is in equilibrium  they result in an equilibrium point.

Such strategies are known as equilibrium strategies.


In the singular, an equilibrium strategy is one that contributes to an equilibrium point.
A system of strategies, one for each player, is in equilibrium  they result in an equilibrium point.

Such strategies are known as equilibrium strategies.


In the singular, an equilibrium strategy is one that contributes to an equilibrium point.
An equilibrium point is a stable outcome of a game associated with a particular set of strategies.
"
Definition:Equilibrium,Equilibrium,"Let a strategic game $G$ be modelled by:
:$G = \stratgame N {A_i} {\succsim_i}$


A Nash equilibrium of $G$ is a profile $a^* \in A$ of moves which has the property that:
:$\forall i \in N: \forall a_i \in A_i: \tuple {a^*_{-i}, a^*_i} \succsim_i \tuple {a^*_{-i}, a_i}$


Thus, for $a^*$ to be a Nash equilibrium, no player $i$ has a move yielding a preferable outcome to that when $a^*_i$ is chosen, given that every other player $j$ has chosen his own equilibrium move.

That is, no player can profitably deviate, if no other player also deviates.


Let a strategic game $G$ be modelled by:
:$G = \stratgame N {A_i} {\succsim_i}$


A Nash equilibrium of $G$ is a profile $a^* \in A$ of moves which has the property that:
:$\forall i \in N: \forall a_i \in A_i: \tuple {a^*_{-i}, a^*_i} \succsim_i \tuple {a^*_{-i}, a_i}$


Thus, for $a^*$ to be a Nash equilibrium, no player $i$ has a move yielding a preferable outcome to that when $a^*_i$ is chosen, given that every other player $j$ has chosen his own equilibrium move.

That is, no player can profitably deviate, if no other player also deviates.


",Definition:Nash Equilibrium,['Definitions/Game Theory'],"Let a strategic game G be modelled by:
:G =  N A_i≿_i


A Nash equilibrium of G is a profile a^* ∈ A of moves which has the property that:
:∀ i ∈ N: ∀ a_i ∈ A_i: a^*_-i, a^*_i≿_i a^*_-i, a_i


Thus, for a^* to be a Nash equilibrium, no player i has a move yielding a preferable outcome to that when a^*_i is chosen, given that every other player j has chosen his own equilibrium move.

That is, no player can profitably deviate, if no other player also deviates.


Let a strategic game G be modelled by:
:G =  N A_i≿_i


A Nash equilibrium of G is a profile a^* ∈ A of moves which has the property that:
:∀ i ∈ N: ∀ a_i ∈ A_i: a^*_-i, a^*_i≿_i a^*_-i, a_i


Thus, for a^* to be a Nash equilibrium, no player i has a move yielding a preferable outcome to that when a^*_i is chosen, given that every other player j has chosen his own equilibrium move.

That is, no player can profitably deviate, if no other player also deviates.


"
Definition:Equivalence,Equivalence,"Let $\mathscr M$ be a formal semantics for a formal language $\LL$.

Let $\phi, \psi$ be $\LL$-WFFs.


Then $\phi$ and $\psi$ are $\mathscr M$-semantically equivalent :

:$\phi \models_{\mathscr M} \psi$ and $\psi \models_{\mathscr M} \phi$

that is,  they are $\mathscr M$-semantic consequences of one another.


Equivalently, $\phi$ and $\psi$ are $\mathscr M$-semantically equivalent , for each $\mathscr M$-structure $\MM$:

:$\MM \models_{\mathscr M} \phi$  $\MM \models_{\mathscr M} \psi$




",Definition:Semantic Equivalence,['Definitions/Formal Semantics'],"Let ℳ be a formal semantics for a formal language .

Let ϕ, ψ be -WFFs.


Then ϕ and ψ are ℳ-semantically equivalent :

:ϕ_ℳψ and ψ_ℳϕ

that is,  they are ℳ-semantic consequences of one another.


Equivalently, ϕ and ψ are ℳ-semantically equivalent , for each ℳ-structure :

:_ℳϕ  _ℳψ




"
Definition:Equivalence,Equivalence,"Let $\mathscr P$ be a proof system for a formal language $\LL$.

Let $\phi, \psi$ be $\LL$-WFFs.


Then $\phi$ and $\psi$ are $\mathscr P$-provably equivalent :

:$\phi \vdash_{\mathscr P} \psi$ and $\psi \vdash_{\mathscr P} \phi$

that is,  they are $\mathscr P$-provable consequences of one another.


The provable equivalence of $\phi$ and $\psi$ can be denoted by:

:$\phi \dashv \vdash_{\mathscr P} \psi$",Definition:Provable Equivalence,['Definitions/Proof Systems'],"Let 𝒫 be a proof system for a formal language .

Let ϕ, ψ be -WFFs.


Then ϕ and ψ are 𝒫-provably equivalent :

:ϕ⊢_𝒫ψ and ψ⊢_𝒫ϕ

that is,  they are 𝒫-provable consequences of one another.


The provable equivalence of ϕ and ψ can be denoted by:

:ϕ⊣⊢_𝒫ψ"
Definition:Equivalence,Equivalence,"Let $S$ and $T$ be sets.

Then $S$ and $T$ are equivalent :
:there exists a bijection $f: S \to T$ between the elements of $S$ and those of $T$.

That is,  they have the same cardinality.


This can be written $S \sim T$.


If $S$ and $T$ are not equivalent we write $S \nsim T$.",Definition:Set Equivalence,"['Definitions/Set Equivalence', 'Definitions/Set Theory']","Let S and T be sets.

Then S and T are equivalent :
:there exists a bijection f: S → T between the elements of S and those of T.

That is,  they have the same cardinality.


This can be written S ∼ T.


If S and T are not equivalent we write S  T."
Definition:Equivalence,Equivalence,"Two matrices $\mathbf A = \sqbrk a_{m n}, \mathbf B = \sqbrk b_{m n}$ are row equivalent if one can be obtained from the other by a finite sequence of elementary row operations.

This relationship can be denoted $\mathbf A \sim \mathbf B$.",Definition:Row Equivalence,"['Definitions/Matrix Theory', 'Definitions/Elementary Row Operations']","Two matrices 𝐀 =  a_m n, 𝐁 =  b_m n are row equivalent if one can be obtained from the other by a finite sequence of elementary row operations.

This relationship can be denoted 𝐀∼𝐁."
Definition:Equivalence,Equivalence,,Definition:Equivalence,[],
Definition:Euclidean,Euclidean,"Euclidean geometry is the branch of geometry in which the parallel postulate applies.

An assumption which is currently under question is whether or not ordinary space is itself Euclidean.


Euclidean geometry adheres to Euclid's postulates.
Let $S$ be one of the standard number fields $\Q$, $\R$, $\C$.

Let $S^n$ be a cartesian space for $n \in \N_{\ge 1}$.

Let $d: S^n \times S^n \to \R$ be the usual (Euclidean) metric on $S^n$.

Then $\tuple {S^n, d}$ is a Euclidean space.
Euclidean geometry is the branch of geometry in which the parallel postulate applies.

An assumption which is currently under question is whether or not ordinary space is itself Euclidean.


Euclidean geometry adheres to Euclid's postulates.
",Definition:Euclidean Geometry,"['Definitions/Euclidean Geometry', 'Definitions/Geometry', 'Definitions/Pure Mathematics', 'Definitions/Branches of Mathematics']","Euclidean geometry is the branch of geometry in which the parallel postulate applies.

An assumption which is currently under question is whether or not ordinary space is itself Euclidean.


Euclidean geometry adheres to Euclid's postulates.
Let S be one of the standard number fields , , .

Let S^n be a cartesian space for n ∈_≥ 1.

Let d: S^n × S^n → be the usual (Euclidean) metric on S^n.

Then S^n, d is a Euclidean space.
Euclidean geometry is the branch of geometry in which the parallel postulate applies.

An assumption which is currently under question is whether or not ordinary space is itself Euclidean.


Euclidean geometry adheres to Euclid's postulates.
"
Definition:Euclidean,Euclidean,"Let $\R^n$ be an $n$-dimensional real vector space.

Let the Euclidean metric $d$ be applied to $\R^n$.

Then $\struct {\R^n, d}$ is a (real) Euclidean $n$-space.
Let $\mathbf v = \tuple {v_1, v_2, \ldots, v_n}$ be a vector in the real Euclidean $n$-space $\R^n$.


The Euclidean norm of $\mathbf v$ is defined as:
:$\ds \norm {\mathbf v} = \paren {\sum_{k \mathop = 1}^n v_k^2}^{1/2}$
",Definition:Euclidean Norm,"['Definitions/Euclidean Norms', 'Definitions/Normed Spaces', 'Definitions/Euclidean Space']","Let ^n be an n-dimensional real vector space.

Let the Euclidean metric d be applied to ^n.

Then ^n, d is a (real) Euclidean n-space.
Let 𝐯 = v_1, v_2, …, v_n be a vector in the real Euclidean n-space ^n.


The Euclidean norm of 𝐯 is defined as:
:𝐯 = ∑_k  = 1^n v_k^2^1/2
"
Definition:Euclidean,Euclidean,"Let $\struct {D, +, \circ}$ be an integral domain with zero $0_D$.

Let there exist a mapping $\nu: D \setminus \set {0_D} \to \N$ such that for all $a \in D, b \in D_{\ne 0_D}$:






Then $\nu$ is a Euclidean valuation on $D$.
Let $\struct {D, +, \circ}$ be an integral domain.

Let there exist a Euclidean valuation on $D$.

Then $D$ is called a Euclidean domain.


=== Euclidean Valuation ===

Let $\struct {D, +, \circ}$ be an integral domain with zero $0_D$.

Let there exist a mapping $\nu: D \setminus \set {0_D} \to \N$ such that for all $a \in D, b \in D_{\ne 0_D}$:






Then $\nu$ is a Euclidean valuation on $D$.
",Definition:Euclidean Domain,"['Definitions/Euclidean Domains', 'Definitions/Integral Domains']","Let D, +, ∘ be an integral domain with zero 0_D.

Let there exist a mapping ν: D ∖0_D→ such that for all a ∈ D, b ∈ D_ 0_D:






Then ν is a Euclidean valuation on D.
Let D, +, ∘ be an integral domain.

Let there exist a Euclidean valuation on D.

Then D is called a Euclidean domain.


=== Euclidean Valuation ===

Let D, +, ∘ be an integral domain with zero 0_D.

Let there exist a mapping ν: D ∖0_D→ such that for all a ∈ D, b ∈ D_ 0_D:






Then ν is a Euclidean valuation on D.
"
Definition:Euclidean,Euclidean,"Let $\RR \subseteq S \times S$ be a relation in $S$.


$\RR$ is left-Euclidean :

:$\tuple {x, z} \in \RR \land \tuple {y, z} \in \RR \implies \tuple {x, y} \in \RR$
Let $\RR \subseteq S \times S$ be a relation in $S$.


$\RR$ is right-Euclidean :

:$\tuple {x, y} \in \RR \land \tuple {x, z} \in \RR \implies \tuple {y, z} \in \RR$
Let $\RR \subseteq S \times S$ be a relation in $S$.


$\RR$ is left-Euclidean :

:$\tuple {x, z} \in \RR \land \tuple {y, z} \in \RR \implies \tuple {x, y} \in \RR$
Let $\RR \subseteq S \times S$ be a relation in $S$.


$\RR$ is right-Euclidean :

:$\tuple {x, y} \in \RR \land \tuple {x, z} \in \RR \implies \tuple {y, z} \in \RR$
",Definition:Euclidean Relation,"['Definitions/Euclidean Relations', 'Definitions/Relation Theory']","Let ⊆ S × S be a relation in S.


 is left-Euclidean :

:x, z∈y, z∈x, y∈
Let ⊆ S × S be a relation in S.


 is right-Euclidean :

:x, y∈x, z∈y, z∈
Let ⊆ S × S be a relation in S.


 is left-Euclidean :

:x, z∈y, z∈x, y∈
Let ⊆ S × S be a relation in S.


 is right-Euclidean :

:x, y∈x, z∈y, z∈
"
Definition:Euler Characteristic,Euler Characteristic,"Let $G = \struct {V, E}$ be a finite graph.

Let $G$ be embedded in a surface.


The Euler characteristic of $G$ is written $\map \chi G$ and is defined as:
:$\map \chi G = v - e + f$
where:
:$v = \size V$ is the number of vertices
:$e = \size E$ is the number of edges
:$f$ is the number of faces.
",Definition:Euler Characteristic of Finite Graph,"['Definitions/Euler Characteristic of Finite Graph', 'Definitions/Euler Characteristic', 'Definitions/Graph Theory']","Let G = V, E be a finite graph.

Let G be embedded in a surface.


The Euler characteristic of G is written χ G and is defined as:
:χ G = v - e + f
where:
:v =  V is the number of vertices
:e =  E is the number of edges
:f is the number of faces.
"
Definition:Euler Characteristic,Euler Characteristic,"Let $S$ be a surface.

Let $T$ be a triangulation of $S$.

The Euler characteristic of $S$ is written $\map \chi S$ and is defined as:
:$\map \chi S = v - e + f$
where:
:$v = \size V$ is the number of vertices of $T$
:$e = \size E$ is the number of edges of $T$
:$f$ is the number of faces of $T$.
",Definition:Euler Characteristic of Surface,"['Definitions/Euler Characteristic of Surface', 'Definitions/Euler Characteristic', 'Definitions/Graph Theory']","Let S be a surface.

Let T be a triangulation of S.

The Euler characteristic of S is written χ S and is defined as:
:χ S = v - e + f
where:
:v =  V is the number of vertices of T
:e =  E is the number of edges of T
:f is the number of faces of T.
"
Definition:Expansion,Expansion,"Let $S$ be a set.

Let $\tau_1$ and $\tau_2$ be topologies on $S$ such that $\tau_1 \subseteq \tau_2$.


Then $\tau_2$ is an expansion of $\tau_1$.",Definition:Expansion of Topology,['Definitions/Topology'],"Let S be a set.

Let τ_1 and τ_2 be topologies on S such that τ_1 ⊆τ_2.


Then τ_2 is an expansion of τ_1."
Definition:Expansion,Expansion,"An (algebraic) expansion is a algebraic expression written as the sum of terms.

The word is also used for the actual process of converting the power of a multinomial, in order to obtain such an expression.
",Definition:Algebraic Expansion,"['Definitions/Algebraic Expansions', 'Definitions/Algebra']","An (algebraic) expansion is a algebraic expression written as the sum of terms.

The word is also used for the actual process of converting the power of a multinomial, in order to obtain such an expression.
"
Definition:Expansion,Expansion,,Definition:Continued Fraction Expansion,['Definitions/Continued Fractions'],
Definition:Expansion,Expansion,"Let $S$ be a set.

Let $\Bbb S = \set {S_1, S_2, \ldots, S_n}$ form a partition of $S$.


Then the representation by such a partition $\ds \bigcup_{k \mathop = 1}^n S_k = S$ is also called a finite expansion of $S$.


The notations:
:$S = S_1 \mid S_2 \mid \cdots \mid S_n$
or:
:$\Bbb S = \set {S_1 \mid S_2 \mid \cdots \mid S_n}$
are sometimes seen.


Category:Definitions/Set Partitions",Definition:Set Partition/Finite Expansion,['Definitions/Set Partitions'],"Let S be a set.

Let S = S_1, S_2, …, S_n form a partition of S.


Then the representation by such a partition ⋃_k  = 1^n S_k = S is also called a finite expansion of S.


The notations:
:S = S_1 | S_2 |⋯| S_n
or:
:S = S_1 | S_2 |⋯| S_n
are sometimes seen.


Category:Definitions/Set Partitions"
Definition:Extension,Extension,"Let:

:$\RR_1 \subseteq X \times Y$ be a relation on $X \times Y$
:$\RR_2 \subseteq S \times T$ be a relation on $S \times T$
:$X \subseteq S$
:$Y \subseteq T$
:$\RR_2 \restriction_{X \times Y}$ be the restriction of $\RR_2$ to $X \times Y$.


Let $\RR_2 \restriction_{X \times Y} = \RR_1$.


Then $\RR_2$ extends or is an extension of $\RR_1$.",Definition:Extension of Relation,['Definitions/Relation Theory'],"Let:

:_1 ⊆ X × Y be a relation on X × Y
:_2 ⊆ S × T be a relation on S × T
:X ⊆ S
:Y ⊆ T
:_2 _X × Y be the restriction of _2 to X × Y.


Let _2 _X × Y = _1.


Then _2 extends or is an extension of _1."
Definition:Extension,Extension,"Let:

:$\RR_1 \subseteq X \times Y$ be a relation on $X \times Y$
:$\RR_2 \subseteq S \times T$ be a relation on $S \times T$
:$X \subseteq S$
:$Y \subseteq T$
:$\RR_2 \restriction_{X \times Y}$ be the restriction of $\RR_2$ to $X \times Y$.


Let $\RR_2 \restriction_{X \times Y} = \RR_1$.


Then $\RR_2$ extends or is an extension of $\RR_1$.
",Definition:Extension of Mapping,"['Definitions/Mapping Theory', 'Definitions/Restrictions']","Let:

:_1 ⊆ X × Y be a relation on X × Y
:_2 ⊆ S × T be a relation on S × T
:X ⊆ S
:Y ⊆ T
:_2 _X × Y be the restriction of _2 to X × Y.


Let _2 _X × Y = _1.


Then _2 extends or is an extension of _1.
"
Definition:Extension,Extension,"Let $\left({S, \circ}\right)$ be a magma.

Let $\left({T, \circ \restriction_T}\right)$ be a submagma of $\left({S, \circ}\right)$, where $\circ \restriction_T$ denotes the restriction of $\circ$ to $T$.


Then:
: $\left({S, \circ}\right)$ is an extension of $\left({T, \circ \restriction_T}\right)$
or
: $\left({S, \circ}\right)$ extends $\left({T, \circ \restriction_T}\right)$


We can use the term directly to the operation itself and say:
: $\circ$ is an extension of $\circ \restriction_T$
or:
: $\circ$ extends $\circ \restriction_T$",Definition:Extension of Operation,['Definitions/Abstract Algebra'],"Let (S, ∘) be a magma.

Let (T, ∘_T) be a submagma of (S, ∘), where ∘_T denotes the restriction of ∘ to T.


Then:
: (S, ∘) is an extension of (T, ∘_T)
or
: (S, ∘) extends (T, ∘_T)


We can use the term directly to the operation itself and say:
: ∘ is an extension of ∘_T
or:
: ∘ extends ∘_T"
Definition:Extension,Extension,"As a mapping is, by definition, also a relation, the definition of an extension of a mapping is the same as that for an extension of a relation:

Let:

:$f_1 \subseteq X \times Y$ be a mapping on $X \times Y$
:$f_2 \subseteq S \times T$ be a mapping on $S \times T$
:$X \subseteq S$
:$Y \subseteq T$
:$f_2 \restriction_{X \times Y}$ be the restriction of $f_2$ to $X \times Y$.


Let $f_2 \restriction_{X \times Y} = f_1$.

That is, let $f_1$ be a subset of $f_2$.


Then $f_2$ extends or is an extension of $f_1$.
",Definition:Extension of Sequence,['Definitions/Sequences'],"As a mapping is, by definition, also a relation, the definition of an extension of a mapping is the same as that for an extension of a relation:

Let:

:f_1 ⊆ X × Y be a mapping on X × Y
:f_2 ⊆ S × T be a mapping on S × T
:X ⊆ S
:Y ⊆ T
:f_2 _X × Y be the restriction of f_2 to X × Y.


Let f_2 _X × Y = f_1.

That is, let f_1 be a subset of f_2.


Then f_2 extends or is an extension of f_1.
"
Definition:Extension,Extension,"Let $A$ and $B$ be classes.

Let $B$ be an extension of $A$.

$B$ is an immediate extension of $A$  $B$ contains exactly one more element than $A$.
",Definition:Extension of Class,"['Definitions/Class Theory', 'Definitions/Subclasses', 'Definitions/Class Extensions']","Let A and B be classes.

Let B be an extension of A.

B is an immediate extension of A  B contains exactly one more element than A.
"
Definition:Extension,Extension,"Let $A$ and $B$ be classes.

$B$ is an extension of $A$ :
:$A \subseteq B$


=== Immediate Extension ===

",Definition:Extension of Class/Immediate,['Definitions/Class Extensions'],"Let A and B be classes.

B is an extension of A :
:A ⊆ B


=== Immediate Extension ===

"
Definition:Extension,Extension,"Let $F$ be a field.


A field extension over $F$ is a field $E$ where $F \subseteq E$.

That is, such that $F$ is a subfield of $E$.


This can be expressed:
:$E$ is a field extension over a field $F$
or:
:$E$ over $F$ is a field extension 
as:
:$E / F$ is a field extension.  


$E / F$ can be voiced as $E$ over $F$.
Let $F$ be a field.


A field extension over $F$ is a field $E$ where $F \subseteq E$.

That is, such that $F$ is a subfield of $E$.


This can be expressed:
:$E$ is a field extension over a field $F$
or:
:$E$ over $F$ is a field extension 
as:
:$E / F$ is a field extension.  


$E / F$ can be voiced as $E$ over $F$.
Let $F$ be a field.


A field extension over $F$ is a field $E$ where $F \subseteq E$.

That is, such that $F$ is a subfield of $E$.


This can be expressed:
:$E$ is a field extension over a field $F$
or:
:$E$ over $F$ is a field extension 
as:
:$E / F$ is a field extension.  


$E / F$ can be voiced as $E$ over $F$.
Let $F$ be a field.


A field extension over $F$ is a field $E$ where $F \subseteq E$.

That is, such that $F$ is a subfield of $E$.


This can be expressed:
:$E$ is a field extension over a field $F$
or:
:$E$ over $F$ is a field extension 
as:
:$E / F$ is a field extension.  


$E / F$ can be voiced as $E$ over $F$.
",Definition:Field Extension,['Definitions/Field Extensions'],"Let F be a field.


A field extension over F is a field E where F ⊆ E.

That is, such that F is a subfield of E.


This can be expressed:
:E is a field extension over a field F
or:
:E over F is a field extension 
as:
:E / F is a field extension.  


E / F can be voiced as E over F.
Let F be a field.


A field extension over F is a field E where F ⊆ E.

That is, such that F is a subfield of E.


This can be expressed:
:E is a field extension over a field F
or:
:E over F is a field extension 
as:
:E / F is a field extension.  


E / F can be voiced as E over F.
Let F be a field.


A field extension over F is a field E where F ⊆ E.

That is, such that F is a subfield of E.


This can be expressed:
:E is a field extension over a field F
or:
:E over F is a field extension 
as:
:E / F is a field extension.  


E / F can be voiced as E over F.
Let F be a field.


A field extension over F is a field E where F ⊆ E.

That is, such that F is a subfield of E.


This can be expressed:
:E is a field extension over a field F
or:
:E over F is a field extension 
as:
:E / F is a field extension.  


E / F can be voiced as E over F.
"
Definition:Extension,Extension,"Let $R$ and $S$ be commutative rings with unity.

Let $\phi : R \to S$ be a ring monomorphism.

Then $\phi : R \to S$ is a ring extension of $R$.


Alternatively, we can define $S$ to be a ring extension of $R$ if $R$ is a subring of $S$ (provided we insist that a subring inherits the multiplicative identity from its parent ring).


These definitions are equivalent up to isomorphism, for if $R \subseteq S$ is a subring, then the identity $\operatorname{id} : R \to S$ is a monomorphism.

Conversely if $\phi : R \to S$ is a monomorphism, then $\operatorname{im}\phi \subseteq S$ is a subring of $S$.

Moreover by Surgery for Rings, we can find a ring $T$, isomorphic to $S$, that contains $R$ as a subring.

",Definition:Ring Extension,['Definitions/Ring Theory'],"Let R and S be commutative rings with unity.

Let ϕ : R → S be a ring monomorphism.

Then ϕ : R → S is a ring extension of R.


Alternatively, we can define S to be a ring extension of R if R is a subring of S (provided we insist that a subring inherits the multiplicative identity from its parent ring).


These definitions are equivalent up to isomorphism, for if R ⊆ S is a subring, then the identity id : R → S is a monomorphism.

Conversely if ϕ : R → S is a monomorphism, then imϕ⊆ S is a subring of S.

Moreover by Surgery for Rings, we can find a ring T, isomorphic to S, that contains R as a subring.

"
Definition:Extension,Extension,"Let $A$ and $B$ be commutative ring with unity.

Let $f : A \to B$ be a ring homomorphism.

Let $\mathfrak a$ be an ideal of $A$.


The extension of $\mathfrak a$ by $f$ is the ideal generated by its image under $f$:
:$\mathfrak a^e = \left\langle f \sqbrk {\mathfrak a} \right\rangle$",Definition:Extension of Ideal,['Definitions/Ideal Theory'],"Let A and B be commutative ring with unity.

Let f : A → B be a ring homomorphism.

Let 𝔞 be an ideal of A.


The extension of 𝔞 by f is the ideal generated by its image under f:
:𝔞^e = ⟨ f 𝔞⟩"
Definition:Extension,Extension,"Let $\AA, \BB$ be structures for a signature $\LL$.


Then $\BB$ is an extension of $\AA$  $\AA$ is a substructure of $\BB$.",Definition:Extension of Structure,['Definitions/Model Theory for Predicate Logic'],"Let Å, be structures for a signature .


Then  is an extension of Å  Å is a substructure of ."
Definition:Exterior,Exterior,"Let an exact $n$-form $\omega$ be given on an $m$-manifold, with local coordinates $x_1, x_2, \dots, x_m$.

Let a local coordinate expression for $\omega$ be given:

:$\omega = \map f {x_1, \ldots, x_m} \rd x_{\map \phi 1} \wedge \d x_{\map \phi 2} \wedge \cdots \wedge \d x_{\map \phi n}$

where:
:$\phi: \set {1, \ldots, n} \to \set {1, \ldots, m}$ is an injection which determines which coordinate vectors $\omega$ acts on.
:$\wedge$ denotes the wedge product.


The exterior derivative $\d \omega$ is the $\paren {n + 1}$-form defined as:

:$\ds \d \omega = \paren {\sum_{k \mathop = 1}^m \frac {\partial f} {\partial x_k} \rd x_k} \wedge \d x_{\map \phi 1} \wedge \d x_{\map \phi 2} \wedge \dots \wedge \d x_{\map \phi n}$


For inexact forms:
:$\map \d {a + b} = \d a + \d b$",Definition:Exterior Derivative,['Definitions/Differential Forms'],"Let an exact n-form ω be given on an m-manifold, with local coordinates x_1, x_2, …, x_m.

Let a local coordinate expression for ω be given:

:ω =  f x_1, …, x_m x_ϕ 1∧x̣_ϕ 2∧⋯∧x̣_ϕ n

where:
:ϕ: 1, …, n→1, …, m is an injection which determines which coordinate vectors ω acts on.
:∧ denotes the wedge product.


The exterior derivative ω̣ is the n + 1-form defined as:

:ω̣= ∑_k  = 1^m ∂ f/∂ x_k x_k∧x̣_ϕ 1∧x̣_ϕ 2∧…∧x̣_ϕ n


For inexact forms:
:ạ ̣+̣ ̣ḅ = ạ + ḅ"
Definition:Exterior,Exterior,"Contrary to intuition, the external angle of a vertex of a polygon is not the size of the angle between the sides forming that vertex, as measured outside the polygon.

An external angle is in fact an angle formed by one side of a polygon and a line produced from an adjacent side.

:

While $\angle AFE$ is the internal angle of vertex $F$, the external angle of this vertex is $\angle EFG$.",Definition:Polygon/External Angle,"['Definitions/External Angles', 'Definitions/Polygons']","Contrary to intuition, the external angle of a vertex of a polygon is not the size of the angle between the sides forming that vertex, as measured outside the polygon.

An external angle is in fact an angle formed by one side of a polygon and a line produced from an adjacent side.

:

While ∠ AFE is the internal angle of vertex F, the external angle of this vertex is ∠ EFG."
Definition:Exterior,Exterior,":


An exterior angle of a transversal is an angle which is not between the two lines cut by a transversal.

In the above figure, the exterior angles with respect to the transversal $EF$ are:
:$\angle AHE$
:$\angle CJF$
:$\angle BHE$
:$\angle DJF$",Definition:Transversal (Geometry)/Exterior Angle,['Definitions/Transversals (Geometry)'],":


An exterior angle of a transversal is an angle which is not between the two lines cut by a transversal.

In the above figure, the exterior angles with respect to the transversal EF are:
:∠ AHE
:∠ CJF
:∠ BHE
:∠ DJF"
Definition:Exterior,Exterior,"Let $f: \closedint 0 1 \to \R^2$ be a Jordan curve.


It follows from the Jordan Curve Theorem that $\R^2 \setminus \Img f$ is a union of two disjoint connected components, one of which is unbounded.

This unbounded component is called the exterior of $f$, and is denoted as $\Ext f$.
",Definition:Jordan Curve/Exterior,['Definitions/Jordan Curves'],"Let f:  0 1 →^2 be a Jordan curve.


It follows from the Jordan Curve Theorem that ^2 ∖ f is a union of two disjoint connected components, one of which is unbounded.

This unbounded component is called the exterior of f, and is denoted as f.
"
Definition:Exterior Angle,Exterior Angle,"Contrary to intuition, the external angle of a vertex of a polygon is not the size of the angle between the sides forming that vertex, as measured outside the polygon.

An external angle is in fact an angle formed by one side of a polygon and a line produced from an adjacent side.

:

While $\angle AFE$ is the internal angle of vertex $F$, the external angle of this vertex is $\angle EFG$.",Definition:Polygon/External Angle,"['Definitions/External Angles', 'Definitions/Polygons']","Contrary to intuition, the external angle of a vertex of a polygon is not the size of the angle between the sides forming that vertex, as measured outside the polygon.

An external angle is in fact an angle formed by one side of a polygon and a line produced from an adjacent side.

:

While ∠ AFE is the internal angle of vertex F, the external angle of this vertex is ∠ EFG."
Definition:Exterior Angle,Exterior Angle,":


An exterior angle of a transversal is an angle which is not between the two lines cut by a transversal.

In the above figure, the exterior angles with respect to the transversal $EF$ are:
:$\angle AHE$
:$\angle CJF$
:$\angle BHE$
:$\angle DJF$",Definition:Transversal (Geometry)/Exterior Angle,['Definitions/Transversals (Geometry)'],":


An exterior angle of a transversal is an angle which is not between the two lines cut by a transversal.

In the above figure, the exterior angles with respect to the transversal EF are:
:∠ AHE
:∠ CJF
:∠ BHE
:∠ DJF"
Definition:Extraction of Root,Extraction of Root,The process of evaluating roots of a given real number is referred to as extraction.,Definition:Root of Number/Extraction,"['Definitions/Extraction of Roots', 'Definitions/Roots of Numbers']",The process of evaluating roots of a given real number is referred to as extraction.
Definition:Extraction of Root,Extraction of Root,The process of finding roots of a given equation is referred to as extraction.,Definition:Root of Equation/Extraction,"['Definitions/Extraction of Roots', 'Definitions/Roots of Equations']",The process of finding roots of a given equation is referred to as extraction.
Definition:Face,Face,"The faces of a polyhedron are the polygons which contain it.



",Definition:Polyhedron/Face,"['Definitions/Faces of Polyhedra', 'Definitions/Polyhedra', 'Definitions/Faces']","The faces of a polyhedron are the polygons which contain it.



"
Definition:Face,Face,"Each of the planes bounded by the defining half-lines is known as a face of the polyhedral angle.
",Definition:Polyhedral Angle/Face,"['Definitions/Faces of Polyhedral Angles', 'Definitions/Polyhedral Angles', 'Definitions/Faces']","Each of the planes bounded by the defining half-lines is known as a face of the polyhedral angle.
"
Definition:Factor,Factor,"Let $S$ and $T$ be sets.

Let $S \times T$ be the cartesian product of $S$ and $T$.


Then the sets $S$ and $T$ are called the factors of $S \times T$.",Definition:Cartesian Product/Factors,['Definitions/Cartesian Product'],"Let S and T be sets.

Let S × T be the cartesian product of S and T.


Then the sets S and T are called the factors of S × T."
Definition:Factor,Factor,"Let $\family {\struct {X_i, \tau_i} }_{i \mathop \in I}$ be an indexed family of topological spaces where $I$ is an arbitrary index set.

Let $\struct {\XX, \tau}$ be the product space of $\family {\struct {x_i, \tau_i} }_{i \mathop \in I}$.


Each of the topological spaces $\struct {X_i, \tau_i}$ are called the factors of $\struct {\XX, \tau}$, and can be referred to as factor spaces.",Definition:Product Topology/Factor Space,['Definitions/Product Topology'],"Let X_i, τ_i_i ∈ I be an indexed family of topological spaces where I is an arbitrary index set.

Let , τ be the product space of x_i, τ_i_i ∈ I.


Each of the topological spaces X_i, τ_i are called the factors of , τ, and can be referred to as factor spaces."
Definition:Family,Family,,Definition:Family,"['Definitions/Indexed Families', 'Definitions/Set Theory']",
Definition:Family,Family,"Let $I$ and $S$ be sets.

Let $x: I \to S$ be an indexing function for $S$.

Let $x_i$ denote the image of an element $i \in I$ of the domain $I$ of $x$.

Let $\family {x_i}_{i \mathop \in I}$ denote the set of the images of all the elements $i \in I$ under $x$.


The image $\Img x$, consisting of the terms $\family {x_i}_{i \mathop \in I}$, along with the indexing function $x$ itself, is called a family of elements of $S$ indexed by $I$.
",Definition:Indexing Set/Family,['Definitions/Indexed Families'],"Let I and S be sets.

Let x: I → S be an indexing function for S.

Let x_i denote the image of an element i ∈ I of the domain I of x.

Let x_i_i ∈ I denote the set of the images of all the elements i ∈ I under x.


The image x, consisting of the terms x_i_i ∈ I, along with the indexing function x itself, is called a family of elements of S indexed by I.
"
Definition:Family,Family,"Let $I$ and $S$ be sets.

Let $x: I \to S$ be an indexing function for $S$.

Let $x_i$ denote the image of an element $i \in I$ of the domain $I$ of $x$.

Let $\family {x_i}_{i \mathop \in I}$ denote the set of the images of all the elements $i \in I$ under $x$.


The image $\Img x$, consisting of the terms $\family {x_i}_{i \mathop \in I}$, along with the indexing function $x$ itself, is called a family of elements of $S$ indexed by $I$.
Let $I$ and $S$ be sets.

Let $x: I \to S$ be an indexing function for $S$.

Let $x_i$ denote the image of an element $i \in I$ of the domain $I$ of $x$.

Let $\family {x_i}_{i \mathop \in I}$ denote the set of the images of all the elements $i \in I$ under $x$.


The image $\Img x$, consisting of the terms $\family {x_i}_{i \mathop \in I}$, along with the indexing function $x$ itself, is called a family of elements of $S$ indexed by $I$.
",Definition:Indexing Set/Family of Sets,['Definitions/Indexed Families'],"Let I and S be sets.

Let x: I → S be an indexing function for S.

Let x_i denote the image of an element i ∈ I of the domain I of x.

Let x_i_i ∈ I denote the set of the images of all the elements i ∈ I under x.


The image x, consisting of the terms x_i_i ∈ I, along with the indexing function x itself, is called a family of elements of S indexed by I.
Let I and S be sets.

Let x: I → S be an indexing function for S.

Let x_i denote the image of an element i ∈ I of the domain I of x.

Let x_i_i ∈ I denote the set of the images of all the elements i ∈ I under x.


The image x, consisting of the terms x_i_i ∈ I, along with the indexing function x itself, is called a family of elements of S indexed by I.
"
Definition:Family,Family,"Let $I$ and $S$ be sets.

Let $x: I \to S$ be an indexing function for $S$.

Let $x_i$ denote the image of an element $i \in I$ of the domain $I$ of $x$.

Let $\family {x_i}_{i \mathop \in I}$ denote the set of the images of all the elements $i \in I$ under $x$.


The image $\Img x$, consisting of the terms $\family {x_i}_{i \mathop \in I}$, along with the indexing function $x$ itself, is called a family of elements of $S$ indexed by $I$.
Let $S$ be a set.

Let $I$ be an indexing set.

For each $i \in I$, let $S_i$ be a corresponding subset of $S$.

Let $\family {S_i}_{i \mathop \in I}$ be a family of subsets of $S$ indexed by $I$.


Then $\family {S_i}_{i \mathop \in I}$ is referred to as an indexed family of subsets (of $S$ by $I$).
",Definition:Indexing Set/Family of Subsets,['Definitions/Indexed Families'],"Let I and S be sets.

Let x: I → S be an indexing function for S.

Let x_i denote the image of an element i ∈ I of the domain I of x.

Let x_i_i ∈ I denote the set of the images of all the elements i ∈ I under x.


The image x, consisting of the terms x_i_i ∈ I, along with the indexing function x itself, is called a family of elements of S indexed by I.
Let S be a set.

Let I be an indexing set.

For each i ∈ I, let S_i be a corresponding subset of S.

Let S_i_i ∈ I be a family of subsets of S indexed by I.


Then S_i_i ∈ I is referred to as an indexed family of subsets (of S by I).
"
Definition:Finitary,Finitary,"A finitary operation is an operation which takes a finite number of operands.


Category:Definitions/Operations",Definition:Operation/Arity/Finitary,['Definitions/Operations'],"A finitary operation is an operation which takes a finite number of operands.


Category:Definitions/Operations"
Definition:Finitary,Finitary,"A finitary argument is a logical argument which starts with a finite number of axioms, and can be translated into a finite number of statements.
",Definition:Logical Argument/Finitary,"['Definitions/Finitary Arguments', 'Definitions/Logical Arguments']","A finitary argument is a logical argument which starts with a finite number of axioms, and can be translated into a finite number of statements.
"
Definition:Finite Type,Finite Type,"Let $R$ be a ring.

Let $M$ be a module over $R$.


Then $M$ is finitely generated  there is a generator for $M$ which is finite.",Definition:Finitely Generated Module,['Definitions/Generators of Modules'],"Let R be a ring.

Let M be a module over R.


Then M is finitely generated  there is a generator for M which is finite."
Definition:Finite Type,Finite Type,"Let $\struct {X, \OO_X}$ and $\struct {Y, \OO_Y}$ be schemes.

Let $f : \struct {X, \OO_X} \to \struct {Y, \OO_Y}$ be a morphism of schemes.


$f$ is locally of finite type  for all affine open subsets $U \subset Y$, there is an open cover $\family {V_i}_{i \mathop \in I}$ of $\map {f^{-1} } U$ by affine open subsets $V_i \subset \map {f^{-1} } U$, such that the ring homomorphism $\map {\OO_Y} U \to \map {\OO_X} {V_i}$ makes $\map {\OO_X} {V_i}$ a finitely generated commutative $\map {\OO_Y} U$-algebra.
",Definition:Morphism of Schemes of Finite Type,"['Definitions/Algebraic Geometry', 'Definitions/Schemes']","Let X, _X and Y, _Y be schemes.

Let f : X, _X→Y, _Y be a morphism of schemes.


f is locally of finite type  for all affine open subsets U ⊂ Y, there is an open cover V_i_i ∈ I of f^-1 U by affine open subsets V_i ⊂f^-1 U, such that the ring homomorphism _Y U →_XV_i makes _XV_i a finitely generated commutative _Y U-algebra.
"
Definition:Finitely Generated,Finitely Generated,"Let $\struct {A, \circ}$ be an algebraic structure.

Let $\struct {A, \circ}$ have a generator which is finite.


Then $\struct {A, \circ}$ is finitely generated.",Definition:Finitely Generated Algebraic Structure,['Definitions/Abstract Algebra'],"Let A, ∘ be an algebraic structure.

Let A, ∘ have a generator which is finite.


Then A, ∘ is finitely generated."
Definition:Finitely Generated,Finitely Generated,"Let $\struct {M, \circ}$ be a monoid.

Let $S \subseteq M$.

Let $H$ be the smallest submonoid of $M$ such that $S \subseteq H$.


Then:
:$S$ is a generator of $\struct {H, \circ}$
:$S$ generates $\struct {H, \circ}$
:$\struct {H, \circ}$ is the submonoid of $\struct {M, \circ}$ generated by $S$.


This is written $H = \gen S$.


If $S$ is a singleton, for example $S = \set x$, then we can (and usually do) write $H = \gen x$ for $H = \gen {\set x}$.",Definition:Generator of Monoid,['Definitions/Monoids'],"Let M, ∘ be a monoid.

Let S ⊆ M.

Let H be the smallest submonoid of M such that S ⊆ H.


Then:
:S is a generator of H, ∘
:S generates H, ∘
:H, ∘ is the submonoid of M, ∘ generated by S.


This is written H =  S.


If S is a singleton, for example S =  x, then we can (and usually do) write H =  x for H =  x."
Definition:Finitely Generated,Finitely Generated,"Let $G$ be a group.


$G$ is finitely generated  $G$ has a generator which is finite.",Definition:Finitely Generated Group,"['Definitions/Generators of Groups', 'Definitions/Group Theory']","Let G be a group.


G is finitely generated  G has a generator which is finite."
Definition:Finitely Generated,Finitely Generated,"Let $R$ be a ring.

Let $M$ be a module over $R$.


Then $M$ is finitely generated  there is a generator for $M$ which is finite.
",Definition:Finitely Generated Module,['Definitions/Generators of Modules'],"Let R be a ring.

Let M be a module over R.


Then M is finitely generated  there is a generator for M which is finite.
"
Definition:Finitely Generated,Finitely Generated,"Let $E / F$ be a field extension.


Then $E$ is said to be finitely generated over $F$ , for some $\alpha_1, \ldots, \alpha_n \in E$:

:$E = F \left({\alpha_1, \ldots, \alpha_n}\right)$ 

where $F \left({\alpha_1, \ldots, \alpha_n}\right)$ is the field in $E$ generated by $F \cup \left\{{\alpha_1, \ldots, \alpha_n}\right\}$.",Definition:Finitely Generated Field Extension,['Definitions/Field Extensions'],"Let E / F be a field extension.


Then E is said to be finitely generated over F , for some α_1, …, α_n ∈ E:

:E = F (α_1, …, α_n) 

where F (α_1, …, α_n) is the field in E generated by F ∪{α_1, …, α_n}."
Definition:Finitely Generated,Finitely Generated,"Let $A$ be a commutative ring.

Let $B$ be an $A$-algebra.


Then $B$ is finitely generated  $B$ has a generator which is finite.",Definition:Finitely Generated Algebra,['Definitions/Algebras'],"Let A be a commutative ring.

Let B be an A-algebra.


Then B is finitely generated  B has a generator which is finite."
Definition:Focus,Focus,"Let $\KK$ be a conic section specified in terms of:
:a given straight line $D$
:a given point $F$
:a given constant $\epsilon$

where $K$ is the locus of points $P$ such that the distance $p$ from $P$ to $D$ and the distance $q$ from $P$ to $F$ are related by the condition:
:$q = \epsilon \, p$


The point $F$ is known as the focus of $\KK$.
",Definition:Conic Section/Focus,"['Definitions/Foci of Conic Sections', 'Definitions/Conic Sections', 'Definitions/Foci']","Let  be a conic section specified in terms of:
:a given straight line D
:a given point F
:a given constant ϵ

where K is the locus of points P such that the distance p from P to D and the distance q from P to F are related by the condition:
:q = ϵ  p


The point F is known as the focus of .
"
Definition:Focus,Focus,"Let $K$ be an ellipse specified in terms of:
:a given straight line $D$
:a given point $F$
:a given constant $\epsilon$ such that $0 < \epsilon < 1$

where $K$ is the locus of points $P$ such that the distance $p$ from $P$ to $D$ and the distance $q$ from $P$ to $F$ are related by the condition:
:$q = \epsilon \, p$


The point $F$ is known as the focus of the ellipse.
",Definition:Ellipse/Focus,"['Definitions/Foci of Ellipses', 'Definitions/Foci of Conic Sections', 'Definitions/Ellipses']","Let K be an ellipse specified in terms of:
:a given straight line D
:a given point F
:a given constant ϵ such that 0 < ϵ < 1

where K is the locus of points P such that the distance p from P to D and the distance q from P to F are related by the condition:
:q = ϵ  p


The point F is known as the focus of the ellipse.
"
Definition:Focus,Focus,":


Let $K$ be a parabola specified in terms of:
:a given straight line $D$
:a given point $F$

where $K$ is the locus of points $P$ such that the distance $p$ from $P$ to $D$ equals the distance $q$ from $P$ to $F$:
:$p = q$


The point $F$ is known as the focus of the parabola.
",Definition:Parabola/Focus,"['Definitions/Parabolas', 'Definitions/Foci of Conic Sections']",":


Let K be a parabola specified in terms of:
:a given straight line D
:a given point F

where K is the locus of points P such that the distance p from P to D equals the distance q from P to F:
:p = q


The point F is known as the focus of the parabola.
"
Definition:Focus,Focus,":


Let $K$ be a hyperbola specified in terms of:
:a given straight line $D$
:a given point $F$
:a given constant $\epsilon$ such that $\epsilon > 1$

where $K$ is the locus of points $P$ such that the distance $p$ from $P$ to $D$ and the distance $q$ from $P$ to $F$ are related by the condition:
:$q = \epsilon \, p$


The point $F_1$ is known as a focus of the hyperbola.

The symmetrically-positioned point $F_2$ is also a focus of the hyperbola.
:


Let $K$ be a hyperbola specified in terms of:
:a given straight line $D$
:a given point $F$
:a given constant $\epsilon$ such that $\epsilon > 1$

where $K$ is the locus of points $P$ such that the distance $p$ from $P$ to $D$ and the distance $q$ from $P$ to $F$ are related by the condition:
:$q = \epsilon \, p$


The point $F_1$ is known as a focus of the hyperbola.

The symmetrically-positioned point $F_2$ is also a focus of the hyperbola.
",Definition:Hyperbola/Focus,"['Definitions/Hyperbolas', 'Definitions/Foci of Conic Sections']",":


Let K be a hyperbola specified in terms of:
:a given straight line D
:a given point F
:a given constant ϵ such that ϵ > 1

where K is the locus of points P such that the distance p from P to D and the distance q from P to F are related by the condition:
:q = ϵ  p


The point F_1 is known as a focus of the hyperbola.

The symmetrically-positioned point F_2 is also a focus of the hyperbola.
:


Let K be a hyperbola specified in terms of:
:a given straight line D
:a given point F
:a given constant ϵ such that ϵ > 1

where K is the locus of points P such that the distance p from P to D and the distance q from P to F are related by the condition:
:q = ϵ  p


The point F_1 is known as a focus of the hyperbola.

The symmetrically-positioned point F_2 is also a focus of the hyperbola.
"
Definition:Foot,Foot,"Let $\triangle ABC$ be a triangle.

Let $h_a$ be the altitude of $A$:

:


The point at which $h_a$ meets $BC$ (or its production) is the foot of the altitude $h_a$.


Category:Definitions/Triangles",Definition:Altitude of Triangle/Foot,['Definitions/Triangles'],"Let ABC be a triangle.

Let h_a be the altitude of A:

:


The point at which h_a meets BC (or its production) is the foot of the altitude h_a.


Category:Definitions/Triangles"
Definition:Foot,Foot,":


The foot of a perpendicular is the point where it intersects the line to which it is at right angles.

In the above diagram, the point $C$ is the foot of the perpendicular $CD$.
:


The foot of a perpendicular is the point where it intersects the line to which it is at right angles.

In the above diagram, the point $C$ is the foot of the perpendicular $CD$.
",Definition:Right Angle/Perpendicular/Foot,['Definitions/Perpendiculars'],":


The foot of a perpendicular is the point where it intersects the line to which it is at right angles.

In the above diagram, the point C is the foot of the perpendicular CD.
:


The foot of a perpendicular is the point where it intersects the line to which it is at right angles.

In the above diagram, the point C is the foot of the perpendicular CD.
"
Definition:Formula,Formula,"Let $\FF$ be a formal language whose alphabet is $\AA$.

A well-formed formula is a collation in $\AA$ which can be built by using the rules of formation of the formal grammar of $\FF$.


That is, a collation in $\AA$ is a well-formed formula in $\FF$  it has a parsing sequence in $\FF$.",Definition:Well-Formed Formula,['Definitions/Formal Languages'],"Let  be a formal language whose alphabet is Å.

A well-formed formula is a collation in Å which can be built by using the rules of formation of the formal grammar of .


That is, a collation in Å is a well-formed formula in   it has a parsing sequence in ."
Definition:Formula,Formula,"Let $\FF$ be a formal language whose alphabet is $\AA$.

A well-formed formula is a collation in $\AA$ which can be built by using the rules of formation of the formal grammar of $\FF$.


That is, a collation in $\AA$ is a well-formed formula in $\FF$  it has a parsing sequence in $\FF$.
",Definition:Logical Formula,['Definitions/Symbolic Logic'],"Let  be a formal language whose alphabet is Å.

A well-formed formula is a collation in Å which can be built by using the rules of formation of the formal grammar of .


That is, a collation in Å is a well-formed formula in   it has a parsing sequence in .
"
Definition:Free,Free,"Let $S$ be a set.


A free monoid over $S$ is a monoid $M$ together with a mapping $i: S \to M$, subject to:

:For all monoids $N$, for all mappings $f: S \to N$, there is a unique monoid homomorphism $\bar f: M \to N$, such that:

::$\bar f \circ i = f$

This condition is called the universal (mapping) property or UMP of the free monoid over $S$.


=== Category-Theoretic Formulation ===

Let $\mathbf{Mon}$ be the category of monoids, and let $\mathbf{Set}$ be the category of sets.

Let $\left\vert{\cdot}\right\vert$ be the underlying set functor on $\mathbf{Mon}$.


Let $M \in \mathbf{Mon}_0$ be a monoid, and let $i: S \to \left\vert{M}\right\vert$ be a mapping.

Then $\left({M, i}\right)$ is said to be a free monoid over $S$ :

:For all $N \in \mathbf{Mon}_0$ and $f: S \to \left\vert{N}\right\vert \in \mathbf{Set}_1$, a unique $\bar f \in \mathbf{Mon}_1$ makes the following diagram commute:

::$\begin{xy}
<0em,4em>*{\mathbf{Mon} :},
<0em,1em>*{\mathbf{Set} :},

<4em,4em>*+{M} = ""M"",
<8em,4em>*+{N} = ""N"",
""M"";""N"" **@{.} ?>*@{>} ?*!/_1em/{\bar f},

<4em,1em>*+{\left\vert{M}\right\vert} = ""MM"",
<8em,1em>*+{\left\vert{N}\right\vert} = ""NN"",
<4em,-3em>*+{S} = ""S"",

""MM"";""NN"" **@{-} ?>*@{>} ?*!/_1em/{\left\vert{\bar f}\right\vert},
""S"";""MM"" **@{-} ?>*@{>} ?*!/_1em/{i},
""S"";""NN"" **@{-} ?>*@{>} ?*!/^1em/{f}
\end{xy}$

This condition is called the universal (mapping) property or UMP of the free monoid over $S$.",Definition:Free Monoid,"['Definitions/Monoids', 'Definitions/Category of Monoids', 'Definitions/Category of Sets']","Let S be a set.


A free monoid over S is a monoid M together with a mapping i: S → M, subject to:

:For all monoids N, for all mappings f: S → N, there is a unique monoid homomorphism f̅: M → N, such that:

::f̅∘ i = f

This condition is called the universal (mapping) property or UMP of the free monoid over S.


=== Category-Theoretic Formulation ===

Let 𝐌𝐨𝐧 be the category of monoids, and let 𝐒𝐞𝐭 be the category of sets.

Let |·| be the underlying set functor on 𝐌𝐨𝐧.


Let M ∈𝐌𝐨𝐧_0 be a monoid, and let i: S →|M| be a mapping.

Then (M, i) is said to be a free monoid over S :

:For all N ∈𝐌𝐨𝐧_0 and f: S →|N|∈𝐒𝐞𝐭_1, a unique f̅∈𝐌𝐨𝐧_1 makes the following diagram commute:

::<0em,4em>*𝐌𝐨𝐧 :,
<0em,1em>*𝐒𝐞𝐭 :,

<4em,4em>*+M = ""M"",
<8em,4em>*+N = ""N"",
""M"";""N"" **@. ?>*@> ?*!/_1em/f̅,

<4em,1em>*+|M| = ""MM"",
<8em,1em>*+|N| = ""NN"",
<4em,-3em>*+S = ""S"",

""MM"";""NN"" **@- ?>*@> ?*!/_1em/|f̅|,
""S"";""MM"" **@- ?>*@> ?*!/_1em/i,
""S"";""NN"" **@- ?>*@> ?*!/^1em/f

This condition is called the universal (mapping) property or UMP of the free monoid over S."
Definition:Free,Free,"The free commutative monoid on an indexed set $X = \family {X_j: j \in J}$ is the set $M$ of all monomials under the standard multiplication.


That is, it is the set $M$ of all finite sequences of $X$.",Definition:Free Commutative Monoid,"['Definitions/Monoids', 'Definitions/Examples of Monoids', 'Definitions/Polynomial Theory']","The free commutative monoid on an indexed set X = X_j: j ∈ J is the set M of all monomials under the standard multiplication.


That is, it is the set M of all finite sequences of X."
Definition:Free,Free,"Let $R$ be a ring with unity.

Let $G$ be a unitary $R$-module.


Then $G$ is described as free  there exists a basis of $G$.
",Definition:Free Abelian Group,['Definitions/Abelian Groups'],"Let R be a ring with unity.

Let G be a unitary R-module.


Then G is described as free  there exists a basis of G.
"
Definition:Free,Free,"Let $\Z$ be the additive group of integers.

Let $S$ be a set.


The free abelian group on $S$ is the pair $\struct {\Z^{\paren S}, \iota}$ where:
* $\Z^{\paren S}$ is the direct sum of $S$ copies of $\Z$. That is, of the indexed family $S \to \set {\Z}$
* $\iota : S \to \Z^{\paren S}$ is the canonical mapping, which sends $s$ to the mapping $\delta_{st} \in \Z^{\paren S}$, where $\delta$ denotes Kronecker delta.",Definition:Free Abelian Group on Set,['Definitions/Abelian Groups'],"Let  be the additive group of integers.

Let S be a set.


The free abelian group on S is the pair ^ S, ι where:
* ^ S is the direct sum of S copies of . That is, of the indexed family S →
* ι : S →^ S is the canonical mapping, which sends s to the mapping δ_st∈^ S, where δ denotes Kronecker delta."
Definition:Free,Free,"Let $R$ be a ring with unity.

Let $G$ be a unitary $R$-module.


Then $G$ is described as free  there exists a basis of $G$.",Definition:Free Module,['Definitions/Module Theory'],"Let R be a ring with unity.

Let G be a unitary R-module.


Then G is described as free  there exists a basis of G."
Definition:Free,Free,"Let $R$ be a ring.



Let $I$ be an indexing set.


The free $R$-module on $I$ is the direct sum of $R$ as a module over itself:
:$\ds R^{\paren I} := \bigoplus_{i \mathop \in I} R$
of the family $I \to \set R$ to the singleton $\set R$.",Definition:Free Module on Set,"['Definitions/Module Theory', 'Definitions/Abstract Algebra']","Let R be a ring.



Let I be an indexing set.


The free R-module on I is the direct sum of R as a module over itself:
:R^ I := ⊕_i ∈ I R
of the family I → R to the singleton R."
Definition:Free,Free,"Let $G$ be a group with identity $e$ acting on a set $X$.


The group action is free :
:$\forall g \in G: \forall x \in X : g * x = x \implies g = e$",Definition:Free Group Action,['Definitions/Group Actions'],"Let G be a group with identity e acting on a set X.


The group action is free :
:∀ g ∈ G: ∀ x ∈ X : g * x = x  g = e"
Definition:Frequency,Frequency,"Let $f: \R \to \R$ be a periodic real function.

The frequency $\nu$ of $f$ is the reciprocal of the period $L$ of $f$:
:$\nu = \dfrac 1 L$

where:
:$\forall x \in X: \map f x = \map f {x + L}$
",Definition:Periodic Real Function/Frequency,"['Definitions/Frequency of Periodic Real Function', 'Definitions/Periodic Functions']","Let f: → be a periodic real function.

The frequency ν of f is the reciprocal of the period L of f:
:ν =  1 L

where:
:∀ x ∈ X:  f x =  f x + L
"
Definition:Frequency,Frequency,"Let $S$ be a sample or a population.

Let $\omega$ be a qualitative variable, or a class interval of a quantitative variable.


The frequency of $\omega$ is the number of individuals in $S$ satisfying $\omega$.
",Definition:Frequency (Descriptive Statistics),"['Definitions/Frequency (Descriptive Statistics)', 'Definitions/Class Intervals', 'Definitions/Qualitative Variables', 'Definitions/Descriptive Statistics']","Let S be a sample or a population.

Let ω be a qualitative variable, or a class interval of a quantitative variable.


The frequency of ω is the number of individuals in S satisfying ω.
"
Definition:Frequency,Frequency,"Let $S$ be a sample or a finite population.

Let $\omega$ be a qualitative variable, or a class interval of a quantitative variable.


The relative frequency of $\omega$ is defined as:

:$\map {\operatorname {RF} } \omega := \dfrac {f_\omega} n$

where:

:$f_\omega$ is the (absolute) frequency of $\omega$

:$n$ is the number of individuals in $S$.
Let $S$ be a sample or a population.

Let $\omega$ be a qualitative variable, or a class interval of a quantitative variable.


The frequency of $\omega$ is the number of individuals in $S$ satisfying $\omega$.
",Definition:Relative Frequency,"['Definitions/Relative Frequency', 'Definitions/Frequency (Descriptive Statistics)']","Let S be a sample or a finite population.

Let ω be a qualitative variable, or a class interval of a quantitative variable.


The relative frequency of ω is defined as:

:RFω := f_ω n

where:

:f_ω is the (absolute) frequency of ω

:n is the number of individuals in S.
Let S be a sample or a population.

Let ω be a qualitative variable, or a class interval of a quantitative variable.


The frequency of ω is the number of individuals in S satisfying ω.
"
Definition:Frequency Curve,Frequency Curve,"A frequency curve is a smooth curve approximating a frequency polygon for a large data set.
",Definition:Frequency Curve (Statistics),"['Definitions/Frequency Curves (Statistics)', 'Definitions/Frequency Polygons', 'Definitions/Frequency Curves']","A frequency curve is a smooth curve approximating a frequency polygon for a large data set.
"
Definition:Frequency Curve,Frequency Curve,"A frequency curve is a curve representing a frequency function.
",Definition:Frequency Curve (Probability Theory),"['Definitions/Frequency Curves (Probability Theory)', 'Definitions/Frequency Functions', 'Definitions/Frequency Curves']","A frequency curve is a curve representing a frequency function.
"
Definition:Fréchet Space,Fréchet Space,"Let $T = \struct {S, \tau}$ be a topological space.


=== Definition 1 ===



=== Definition 2 ===
",Definition:Fréchet Space (Topology),"['Definitions/T1 Spaces', 'Definitions/Separation Axioms']","Let T = S, τ be a topological space.


=== Definition 1 ===



=== Definition 2 ===
"
Definition:Fréchet Space,Fréchet Space,"Let $\R^\omega$ denote the countable-dimensional real Cartesian space.

Let:
:$x := \family {x_i}_{i \mathop \in \N} = \tuple {x_0, x_1, x_2, \ldots}$
and:
:$y := \family {y_i}_{i \mathop \in \N} = \tuple {y_0, y_1, y_2, \ldots}$
denote arbitrary elements of $\R^\omega$.


Let the distance function $d: \R^\omega \times \R^\omega \to \R$ be applied to $\R^\omega$ as:
:$\forall x, y \in \R^\omega: \map d {x, y} = \ds \sum_{i \mathop \in \N} \dfrac {2^{-i} \size {x_i - y_i} } {1 + \size {x_i - y_i} }$


The distance function $d$ is referred to as the Fréchet (product) metric.


The resulting metric space $\struct {\R^\omega, d}$ is then referred to as the Fréchet (metric) space.",Definition:Fréchet Space (Functional Analysis),"['Definitions/Fréchet Product Metric', 'Definitions/Examples of Metric Spaces']","Let ^ω denote the countable-dimensional real Cartesian space.

Let:
:x := x_i_i ∈ = x_0, x_1, x_2, …
and:
:y := y_i_i ∈ = y_0, y_1, y_2, …
denote arbitrary elements of ^ω.


Let the distance function d: ^ω×^ω→ be applied to ^ω as:
:∀ x, y ∈^ω:  d x, y = ∑_i ∈2^-ix_i - y_i1 + x_i - y_i


The distance function d is referred to as the Fréchet (product) metric.


The resulting metric space ^ω, d is then referred to as the Fréchet (metric) space."
Definition:Galois Group,Galois Group,"Let $L / K$ be a field extension.


The Galois group of $L / K$ is the subgroup of the automorphism group of $L$ consisting of field automorphisms that fix $K$ point-wise:
:$\Gal {L / K} = \set {\sigma \in \Aut L: \forall k \in K: \map \sigma k = k}$


=== Topological Group ===

",Definition:Galois Group of Field Extension,"['Definitions/Galois Groups of Field Extensions', 'Definitions/Field Extensions', 'Definitions/Galois Theory', 'Definitions/Examples of Groups', 'Definitions/Galois Groups']","Let L / K be a field extension.


The Galois group of L / K is the subgroup of the automorphism group of L consisting of field automorphisms that fix K point-wise:
:L / K = σ∈ L: ∀ k ∈ K: σ k = k


=== Topological Group ===

"
Definition:Galois Group,Galois Group,"Let $P$ be a polynomial.

The Galois group of $P$ is the symmetry group of the roots of $P$.
",Definition:Galois Group of Polynomial,"['Definitions/Galois Groups of Polynomials', 'Definitions/Galois Groups', 'Definitions/Polynomial Theory']","Let P be a polynomial.

The Galois group of P is the symmetry group of the roots of P.
"
Definition:Generated,Generated,"Let $\struct {A, \circ}$ be an algebraic structure.

Let $G \subseteq A$ be any subset of $A$.


The algebraic substructure generated by $G$ is the smallest substructure of $\struct {A, \circ}$ which contains $G$.


It is written $\gen G$.",Definition:Generated Algebraic Substructure,"['Definitions/Abstract Algebra', 'Definitions/Algebraic Structures']","Let A, ∘ be an algebraic structure.

Let G ⊆ A be any subset of A.


The algebraic substructure generated by G is the smallest substructure of A, ∘ which contains G.


It is written G."
Definition:Generated,Generated,"Let $\struct {M, \circ}$ be a monoid whose identity is $e_M$.

Let $S \subseteq M$

Let $H$ be the smallest (with respect to set inclusion) submonoid of $M$ such that $\paren {S \cup \set {e_M} } \subseteq H$.


Then $\struct {H, \circ}$ is the submonoid of $\struct {M, \circ}$ generated by $S$.


This is written $H = \gen S$.


If $S$ is a singleton, for example $S = \set x$, then we can (and usually do) write $H = \gen x$ for $H = \gen {\set x}$.


=== Generator ===
",Definition:Generated Submonoid,['Definitions/Monoids'],"Let M, ∘ be a monoid whose identity is e_M.

Let S ⊆ M

Let H be the smallest (with respect to set inclusion) submonoid of M such that S ∪e_M⊆ H.


Then H, ∘ is the submonoid of M, ∘ generated by S.


This is written H =  S.


If S is a singleton, for example S =  x, then we can (and usually do) write H =  x for H =  x.


=== Generator ===
"
Definition:Generated,Generated,"Let $\struct {R, +, \circ}$ be a ring.

Let $S \subseteq R$ be a subset.


The subring generated by $S$ is the smallest subring of $R$ containing $S$; that is, it is the intersection of all subrings of $R$ containing $S$.
",Definition:Generated Subring,['Definitions/Ring Theory'],"Let R, +, ∘ be a ring.

Let S ⊆ R be a subset.


The subring generated by S is the smallest subring of R containing S; that is, it is the intersection of all subrings of R containing S.
"
Definition:Generated,Generated,"Let $X$ be a set.

Let $\GG \subseteq \powerset X$ be a collection of subsets of $X$.


Then the Dynkin system generated by $\GG$, denoted $\map \delta \GG$, is the smallest Dynkin system on $X$ that contains $\GG$.

That is, $\map \delta \GG$ is subject to:

:$(1):\quad \GG \subseteq \map \delta \GG$
:$(2):\quad \GG \subseteq \DD \implies \map \delta \GG \subseteq \DD$ for any Dynkin system $\DD$ on $X$


In fact, $\map \delta \GG$ always exists, and is unique, as proved on Existence and Uniqueness of Dynkin System Generated by Collection of Subsets.


=== Generator ===

One says that $\GG$ is a generator for $\map \delta \GG$.",Definition:Dynkin System Generated by Collection of Subsets,['Definitions/Dynkin Systems'],"Let X be a set.

Let ⊆ X be a collection of subsets of X.


Then the Dynkin system generated by , denoted δ, is the smallest Dynkin system on X that contains .

That is, δ is subject to:

:(1):  ⊆δ
:(2):  ⊆δ⊆ for any Dynkin system  on X


In fact, δ always exists, and is unique, as proved on Existence and Uniqueness of Dynkin System Generated by Collection of Subsets.


=== Generator ===

One says that  is a generator for δ."
Definition:Generated,Generated,,Definition:Generated Sigma-Algebra,['Definitions/Sigma-Algebras'],
Definition:Generated,Generated,"Let $I$ be an indexing set.

Let $\family {\struct {X_i, \Sigma_i} }_{i \mathop \in I}$ be a family of measurable spaces.

Let $X$ be a set.

Let $\family {f_i: X \to X_i}_{i \mathop \in I}$ be a family of mappings.


Then the $\sigma$-algebra generated by $\family {f_i}_{i \mathop \in I}$, $\map \sigma {f_i: i \in I}$, is the smallest $\sigma$-algebra on $X$ such that every $f_i$ is $\map \sigma {f_i: i \in I} \, / \, \Sigma_i$-measurable.

That is, $\map \sigma {f_i: i \in I}$ is subject to:

:$(1):\quad \forall i \in I: \forall E_i \in \Sigma_i: \map {f_i^{-1} } {E_i} \in \map \sigma {f_i: i \in I}$
:$(2):\quad \map \sigma {f_i: i \in I} \subseteq \Sigma$ for all $\sigma$-algebras $\Sigma$ on $X$ satisfying $(1)$


In fact, $\map \sigma {f_i: i \in I}$ always exists, and is unique, as proved on Existence and Uniqueness of Sigma-Algebra Generated by Collection of Mappings.",Definition:Sigma-Algebra Generated by Collection of Mappings,['Definitions/Sigma-Algebras'],"Let I be an indexing set.

Let X_i, Σ_i_i ∈ I be a family of measurable spaces.

Let X be a set.

Let f_i: X → X_i_i ∈ I be a family of mappings.


Then the σ-algebra generated by f_i_i ∈ I, σf_i: i ∈ I, is the smallest σ-algebra on X such that every f_i is σf_i: i ∈ I  /  Σ_i-measurable.

That is, σf_i: i ∈ I is subject to:

:(1):  ∀ i ∈ I: ∀ E_i ∈Σ_i: f_i^-1E_i∈σf_i: i ∈ I
:(2):  σf_i: i ∈ I⊆Σ for all σ-algebras Σ on X satisfying (1)


In fact, σf_i: i ∈ I always exists, and is unique, as proved on Existence and Uniqueness of Sigma-Algebra Generated by Collection of Mappings."
Definition:Generated,Generated,"Let $X$ be a set.

Let $\GG \subseteq \powerset X$ be a collection of subsets of $X$.


Then the monotone class generated by $\GG$, $\map {\mathfrak m} \GG$, is the smallest monotone class on $X$ that contains $\GG$.

That is, $\map {\mathfrak m} \GG$ is subject to:

:$(1): \quad \GG \subseteq \map {\mathfrak m} \GG$
:$(2): \quad \GG \subseteq \MM \implies \map {\mathfrak m} \GG \subseteq \MM$ for any monotone class $\MM$ on $X$


=== Generator ===

One says that $\GG$ is a generator for $\map {\mathfrak m} \GG$.",Definition:Monotone Class Generated by Collection of Subsets,['Definitions/Set Systems'],"Let X be a set.

Let ⊆ X be a collection of subsets of X.


Then the monotone class generated by , 𝔪, is the smallest monotone class on X that contains .

That is, 𝔪 is subject to:

:(1):   ⊆𝔪
:(2):   ⊆𝔪⊆ for any monotone class  on X


=== Generator ===

One says that  is a generator for 𝔪."
Definition:Generated,Generated,"Let $S$ be a set.

Let $\powerset S$ be the power set of $S$.

Let $\BB \subset \powerset S$ be a filter basis of a filter $\FF$ on $S$.


$\FF$ is said to be generated by $\BB$.",Definition:Filter Basis/Generated Filter,['Definitions/Filter Bases'],"Let S be a set.

Let S be the power set of S.

Let ⊂ S be a filter basis of a filter  on S.


 is said to be generated by ."
Definition:Generator,Generator,"Let $\struct {A, \circ}$ be an algebraic structure.

Let $G \subset A$ be a subset.


=== Definition 1 ===

The subset $G$ is a generator of $A$  $A$ is the algebraic substructure generated by $G$.


=== Definition 2 ===

The subset $G$ is a generator of $A$ :

:$\forall x, y \in G: x \circ y \in A$;
:$\forall z \in A: \exists x, y \in \map W G: z = x \circ y$
where $\map W G$ is the set of words of $G$.

That is, every element in $A$ can be formed as the product of a finite number of elements of $G$.


If $G$ is such a set, then we can write $A = \gen G$.",Definition:Generator of Algebraic Structure,['Definitions/Algebraic Structures'],"Let A, ∘ be an algebraic structure.

Let G ⊂ A be a subset.


=== Definition 1 ===

The subset G is a generator of A  A is the algebraic substructure generated by G.


=== Definition 2 ===

The subset G is a generator of A :

:∀ x, y ∈ G: x ∘ y ∈ A;
:∀ z ∈ A: ∃ x, y ∈ W G: z = x ∘ y
where W G is the set of words of G.

That is, every element in A can be formed as the product of a finite number of elements of G.


If G is such a set, then we can write A =  G."
Definition:Generator,Generator,"Let $\struct {S, \circ}$ be a semigroup.

Let $\O \subset X \subseteq S$.

Let $\struct {T, \circ}$ be the smallest subsemigroup of $\struct {S, \circ}$ such that $X \subseteq T$.


Then:
:$X$ is a generator  of $\struct {T, \circ}$
:$X$ generates $\struct {T, \circ}$
:$\struct {T, \circ}$ is the subsemigroup of $\struct {S, \circ}$ generated by $X$.


This is written:
:$T = \gen X$",Definition:Generator of Subsemigroup,"['Definitions/Semigroups', 'Definitions/Subsemigroups']","Let S, ∘ be a semigroup.

Let Ø⊂ X ⊆ S.

Let T, ∘ be the smallest subsemigroup of S, ∘ such that X ⊆ T.


Then:
:X is a generator  of T, ∘
:X generates T, ∘
:T, ∘ is the subsemigroup of S, ∘ generated by X.


This is written:
:T =  X"
Definition:Generator,Generator,"Let $\struct {M, \circ}$ be a monoid.

Let $S \subseteq M$.

Let $H$ be the smallest submonoid of $M$ such that $S \subseteq H$.


Then:
:$S$ is a generator of $\struct {H, \circ}$
:$S$ generates $\struct {H, \circ}$
:$\struct {H, \circ}$ is the submonoid of $\struct {M, \circ}$ generated by $S$.


This is written $H = \gen S$.


If $S$ is a singleton, for example $S = \set x$, then we can (and usually do) write $H = \gen x$ for $H = \gen {\set x}$.",Definition:Generator of Monoid,['Definitions/Monoids'],"Let M, ∘ be a monoid.

Let S ⊆ M.

Let H be the smallest submonoid of M such that S ⊆ H.


Then:
:S is a generator of H, ∘
:S generates H, ∘
:H, ∘ is the submonoid of M, ∘ generated by S.


This is written H =  S.


If S is a singleton, for example S =  x, then we can (and usually do) write H =  x for H =  x."
Definition:Generator,Generator,"Let $\struct {G, \circ}$ be a group.

Let $S \subseteq G$.


Then $S$ is a generator of $G$, denoted $G = \gen S$,  $G$ is the subgroup generated by $S$.
",Definition:Generator of Group,"['Definitions/Generators of Groups', 'Definitions/Group Theory']","Let G, ∘ be a group.

Let S ⊆ G.


Then S is a generator of G, denoted G =  S,  G is the subgroup generated by S.
"
Definition:Generator,Generator,"Let $\struct {M, \circ}$ be a monoid

Let $S \subseteq M$.

Let $\struct {H, \circ}$ be the submonoid of $\struct {M, \circ}$ generated by $S$.

Then $S$ is known as a generator of $\struct {H, \circ}$.
",Definition:Generated Submonoid/Generator,['Definitions/Monoids'],"Let M, ∘ be a monoid

Let S ⊆ M.

Let H, ∘ be the submonoid of M, ∘ generated by S.

Then S is known as a generator of H, ∘.
"
Definition:Generator,Generator,"Let $\struct {G, \circ}$ be a group.

Let $S \subseteq G$.

Let $H$ be the subgroup generated by $S$.


Then $S$ is a generator of $H$, denoted $H = \gen S$,  $H$ is the subgroup generated by $S$.


=== Definition by Predicate ===

",Definition:Generator of Subgroup,"['Definitions/Group Theory', 'Definitions/Generators of Groups']","Let G, ∘ be a group.

Let S ⊆ G.

Let H be the subgroup generated by S.


Then S is a generator of H, denoted H =  S,  H is the subgroup generated by S.


=== Definition by Predicate ===

"
Definition:Generator,Generator,"Let $G$ be a cyclic group generated by the element $g$.

Let $a \in G$ be an element of $G$ such that $\gen a = G$.

Then $a$ is a generator of $G$.
",Definition:Cyclic Group/Generator,"['Definitions/Cyclic Groups', 'Definitions/Generators of Groups']","Let G be a cyclic group generated by the element g.

Let a ∈ G be an element of G such that a = G.

Then a is a generator of G.
"
Definition:Generator,Generator,"Let $R$ be a commutative ring. 

Let $I \subset R$ be an ideal.

Let $S \subset I$ be a subset.


Then:
:$S$ is a generator of $I$
:
:$I$ is the ideal generated by $S$.
",Definition:Generator of Ideal of Ring,"['Definitions/Generators of Ideals', 'Definitions/Ideal Theory', 'Definitions/Ring Theory']","Let R be a commutative ring. 

Let I ⊂ R be an ideal.

Let S ⊂ I be a subset.


Then:
:S is a generator of I
:
:I is the ideal generated by S.
"
Definition:Generator,Generator,"Let $R$ be a ring.

Let $S \subset R$ be a subset.


Then $S$ is a generator of $R$  $R$ is the subring generated by $S$.",Definition:Generator of Ring,['Definitions/Ring Theory'],"Let R be a ring.

Let S ⊂ R be a subset.


Then S is a generator of R  R is the subring generated by S."
Definition:Generator,Generator,"Let $\struct {D, +, \circ}$ be a division ring.

Let $S \subseteq D$.


The division subring generated by $S$ is the smallest division subring of $D$ containing $S$.",Definition:Generated Division Subring,['Definitions/Division Rings'],"Let D, +, ∘ be a division ring.

Let S ⊆ D.


The division subring generated by S is the smallest division subring of D containing S."
Definition:Generator,Generator,"Let $F$ be a field.

Let $S \subseteq F$ be a subset and $K \le F$ a subfield.


The field generated by $S$ is the smallest subfield of $F$ containing $S$.

The subring of $F$ generated by $K \cup S$, written $K \sqbrk S$, is the smallest subring of $F$ containing $K \cup S$.

The subfield of $F$ generated by $K \cup S$, written $\map K S$, is the smallest subfield of $F$ containing $K \cup S$.",Definition:Generator of Field,['Definitions/Field Theory'],"Let F be a field.

Let S ⊆ F be a subset and K ≤ F a subfield.


The field generated by S is the smallest subfield of F containing S.

The subring of F generated by K ∪ S, written K  S, is the smallest subring of F containing K ∪ S.

The subfield of F generated by K ∪ S, written K S, is the smallest subfield of F containing K ∪ S."
Definition:Generator,Generator,"Let $\struct {A_R, \oplus}$ be an algebra over a ring $R$.

Let $S \subseteq A_R$ be a subset of $A_R$.


The subalgebra generated by $S$ is the smallest subalgebra $B_R$ of $A_R$ which contains $S$.",Definition:Generator of Algebra,['Definitions/Algebras'],"Let A_R, ⊕ be an algebra over a ring R.

Let S ⊆ A_R be a subset of A_R.


The subalgebra generated by S is the smallest subalgebra B_R of A_R which contains S."
Definition:Generator,Generator,"Let $K$ be a division ring.

Let $\mathbf V$ be a vector space over $K$.

Let $S \subseteq \mathbf V$ be a subset of $\mathbf V$.


$S$ is a generator of $\mathbf V$  every element of $\mathbf V$ is a linear combination of elements of $S$.
",Definition:Generator of Vector Space,"['Definitions/Generators of Vector Spaces', 'Definitions/Vector Spaces', 'Definitions/Linear Algebra']","Let K be a division ring.

Let 𝐕 be a vector space over K.

Let S ⊆𝐕 be a subset of 𝐕.


S is a generator of 𝐕  every element of 𝐕 is a linear combination of elements of S.
"
Definition:Genus,Genus,"Let $S$ be a surface.

Let $G = \struct {V, E}$ be a graph which is embedded in $S$.

Let $G$ be such that each of its faces is a simple closed curve.

Let $\map \chi G = v - e + f = 2 - 2 p$ be the Euler characteristic of $G$ where:
:$v = \size V$ is the number of vertices
:$e = \size E$ is the number of edges
:$f$ is the number of faces.

Then $p$ is known as the genus of $S$.
",Definition:Genus of Surface,"['Definitions/Genera of Surfaces', 'Definitions/Graph Theory', 'Definitions/Genera']","Let S be a surface.

Let G = V, E be a graph which is embedded in S.

Let G be such that each of its faces is a simple closed curve.

Let χ G = v - e + f = 2 - 2 p be the Euler characteristic of G where:
:v =  V is the number of vertices
:e =  E is the number of edges
:f is the number of faces.

Then p is known as the genus of S.
"
Definition:Genus,Genus,"The genus of a compact topological manifold is the number of handles it has.
",Definition:Genus of Manifold,"['Definitions/Genera of Manifolds', 'Definitions/Topological Manifolds', 'Definitions/Genera']","The genus of a compact topological manifold is the number of handles it has.
"
Definition:Genus,Genus,"The genus of a Riemann surface $R$ is the number of linearly independent holomorphic $1$-forms that are defined on $R$.
",Definition:Genus of Riemann Surface,"['Definitions/Genera of Riemann Surfaces', 'Definitions/Riemann Surfaces', 'Definitions/Genera']","The genus of a Riemann surface R is the number of linearly independent holomorphic 1-forms that are defined on R.
"
Definition:Genus,Genus,"Let $\CC$ be a plane algebraic curve with no singular points.

The genus of $\CC$ is defined as:
:$\dbinom {d - 1} 2$
where $d$ denotes the degree of $\CC$.


=== Singular Points ===

",Definition:Genus of Plane Algebraic Curve,"['Definitions/Genera of Plane Algebraic Curves', 'Definitions/Algebraic Curves', 'Definitions/Genera']","Let  be a plane algebraic curve with no singular points.

The genus of  is defined as:
:d - 1 2
where d denotes the degree of .


=== Singular Points ===

"
Definition:Graph,Graph,"A metagraph $\GG$ consists of:

* objects $X, Y, Z, \ldots$
* morphisms $f, g, h, \ldots$ between its objects

These are subjected to the following two axioms:







A metagraph is purely axiomatic, and does not use set theory.

For example, the objects are not ""elements of the set of objects"", because these axioms are (without further interpretation) unfounded in set theory.
A metagraph $\GG$ consists of:

* objects $X, Y, Z, \ldots$
* morphisms $f, g, h, \ldots$ between its objects

These are subjected to the following two axioms:







A metagraph is purely axiomatic, and does not use set theory.

For example, the objects are not ""elements of the set of objects"", because these axioms are (without further interpretation) unfounded in set theory.
",Definition:Graph (Category Theory),['Definitions/Category Theory'],"A metagraph  consists of:

* objects X, Y, Z, …
* morphisms f, g, h, … between its objects

These are subjected to the following two axioms:







A metagraph is purely axiomatic, and does not use set theory.

For example, the objects are not ""elements of the set of objects"", because these axioms are (without further interpretation) unfounded in set theory.
A metagraph  consists of:

* objects X, Y, Z, …
* morphisms f, g, h, … between its objects

These are subjected to the following two axioms:







A metagraph is purely axiomatic, and does not use set theory.

For example, the objects are not ""elements of the set of objects"", because these axioms are (without further interpretation) unfounded in set theory.
"
Definition:Graph,Graph,"Let $U \subseteq \R^n$ be an open subset of $n$-dimensional Euclidean space.

Let $f : U \to \R^k$ be a real function.


The graph $\map \Gamma f$ of the function $f$ is the subset of $\R^n \times \R^k$ such that:

:$\map \Gamma f = \set {\tuple {x, y} \in \R^n \times \R^k: x \in U \subseteq \R^n : \map f x = y}$

where $\times$ denotes the Cartesian product.",Definition:Graph of Real Function,"['Definitions/Real Functions', 'Definitions/Mapping Theory']","Let U ⊆^n be an open subset of n-dimensional Euclidean space.

Let f : U →^k be a real function.


The graph Γ f of the function f is the subset of ^n ×^k such that:

:Γ f = x, y∈^n ×^k: x ∈ U ⊆^n :  f x = y

where × denotes the Cartesian product."
Definition:Graph,Graph,"Let $S \times T$ be the cartesian product of two sets $S$ and $T$.

Let $\RR$ be a relation on $S \times T$.


The graph of $\RR$ is the set of all ordered pairs $\tuple {s, t}$ of $S \times T$ such that $s \mathrel \RR t$:
:$\map \TT \RR = \set {\tuple {s, t}: s \mathrel \RR t}$
",Definition:Relation/Graph,"['Definitions/Graphs of Relations', 'Definitions/Relations']","Let S × T be the cartesian product of two sets S and T.

Let  be a relation on S × T.


The graph of  is the set of all ordered pairs s, t of S × T such that s  t:
:= s, t: s  t
"
Definition:Graph,Graph,"A pictograph is a variant of a bar chart whose  bars consist of a number of icons or other pictorial symbols arranged in a (usually) line.


:$\begin {array} {r|l} \text {Happy} & 😊😊😊😊😊😊 \\ \text {Sad} & 😢😢😢 \\ \text {Unimpressed} & 😒😒😒😒😒😒😒😒 \\ \text {Disgusted} & 🤢🤢🤢🤢 \\ \end {array}$
",Definition:Graph (Statistics),"['Definitions/Statistics', 'Definitions/Diagrams', 'Definitions/Graphs (Statistics)']","A pictograph is a variant of a bar chart whose  bars consist of a number of icons or other pictorial symbols arranged in a (usually) line.


:[       Happy      😊😊😊😊😊😊;         Sad         😢😢😢; Unimpressed    😒😒😒😒😒😒😒😒;   Disgusted        🤢🤢🤢🤢;             ]
"
Definition:Harmonic,Harmonic,"Let $x_1, x_2, \ldots, x_n \in \R$ be real numbers which are all strictly positive.

The harmonic mean $H_n$ of $x_1, x_2, \ldots, x_n$ is defined as:

:$\ds \dfrac 1 {H_n} := \frac 1 n \paren {\sum_{k \mathop = 1}^n \frac 1 {x_k} }$

That is, to find the harmonic mean of a set of $n$ numbers, take the reciprocal of the arithmetic mean of their reciprocals.
Let $x_1, x_2, \ldots, x_n \in \R$ be real numbers which are all strictly positive.

The harmonic mean $H_n$ of $x_1, x_2, \ldots, x_n$ is defined as:

:$\ds \dfrac 1 {H_n} := \frac 1 n \paren {\sum_{k \mathop = 1}^n \frac 1 {x_k} }$

That is, to find the harmonic mean of a set of $n$ numbers, take the reciprocal of the arithmetic mean of their reciprocals.
",Definition:Harmonic Mean,"['Definitions/Harmonic Mean', 'Definitions/Pythagorean Means', 'Definitions/Measures of Central Tendency', 'Definitions/Algebra', 'Definitions/Number Theory', 'Definitions/Analysis']","Let x_1, x_2, …, x_n ∈ be real numbers which are all strictly positive.

The harmonic mean H_n of x_1, x_2, …, x_n is defined as:

:1 H_n := 1/n∑_k  = 1^n 1/x_k

That is, to find the harmonic mean of a set of n numbers, take the reciprocal of the arithmetic mean of their reciprocals.
Let x_1, x_2, …, x_n ∈ be real numbers which are all strictly positive.

The harmonic mean H_n of x_1, x_2, …, x_n is defined as:

:1 H_n := 1/n∑_k  = 1^n 1/x_k

That is, to find the harmonic mean of a set of n numbers, take the reciprocal of the arithmetic mean of their reciprocals.
"
Definition:Harmonic,Harmonic,"The harmonic numbers are denoted $H_n$ and are defined for positive integers $n$:
:$\ds \forall n \in \Z, n \ge 0: H_n = \sum_{k \mathop = 1}^n \frac 1 k$

From the definition of vacuous summation it is clear that $H_0 = 0$.

=== General Harmonic Numbers ===

Let $r \in \R_{>0}$.

For $n \in \N_{> 0}$ the harmonic numbers order $r$ are defined as follows:
:$\ds \map {H^{\paren r} } n = \sum_{k \mathop = 1}^n \frac 1 {k^r}$


=== Complex Extension ===



",Definition:Harmonic Numbers,"['Definitions/Harmonic Numbers', 'Definitions/Discrete Mathematics', 'Definitions/Number Theory', 'Definitions/Real Analysis']","The harmonic numbers are denoted H_n and are defined for positive integers n:
:∀ n ∈, n ≥ 0: H_n = ∑_k  = 1^n 1/k

From the definition of vacuous summation it is clear that H_0 = 0.

=== General Harmonic Numbers ===

Let r ∈_>0.

For n ∈_> 0 the harmonic numbers order r are defined as follows:
:H^ r n = ∑_k  = 1^n 1/k^r


=== Complex Extension ===



"
Definition:Harmonic,Harmonic,"Let $r \in \R_{>0}$.

For $n \in \N_{> 0}$ the harmonic numbers order $r$ are defined as follows:
:$\ds \map {H^{\paren r} } n = \sum_{k \mathop = 1}^n \frac 1 {k^r}$


=== Complex Extension ===



The harmonic numbers are denoted $H_n$ and are defined for positive integers $n$:
:$\ds \forall n \in \Z, n \ge 0: H_n = \sum_{k \mathop = 1}^n \frac 1 k$

From the definition of vacuous summation it is clear that $H_0 = 0$.

=== General Harmonic Numbers ===

",Definition:Harmonic Numbers/General Definition,"['Definitions/General Harmonic Numbers', 'Definitions/Harmonic Numbers', 'Definitions/P-Series', 'Definitions/Riemann Zeta Function']","Let r ∈_>0.

For n ∈_> 0 the harmonic numbers order r are defined as follows:
:H^ r n = ∑_k  = 1^n 1/k^r


=== Complex Extension ===



The harmonic numbers are denoted H_n and are defined for positive integers n:
:∀ n ∈, n ≥ 0: H_n = ∑_k  = 1^n 1/k

From the definition of vacuous summation it is clear that H_0 = 0.

=== General Harmonic Numbers ===

"
Definition:Harmonic,Harmonic,"Let $x_1, x_2, \ldots, x_n \in \R$ be real numbers which are all strictly positive.

The harmonic mean $H_n$ of $x_1, x_2, \ldots, x_n$ is defined as:

:$\ds \dfrac 1 {H_n} := \frac 1 n \paren {\sum_{k \mathop = 1}^n \frac 1 {x_k} }$

That is, to find the harmonic mean of a set of $n$ numbers, take the reciprocal of the arithmetic mean of their reciprocals.
",Definition:Ore Number,"['Definitions/Ore Numbers', 'Definitions/Number Theory']","Let x_1, x_2, …, x_n ∈ be real numbers which are all strictly positive.

The harmonic mean H_n of x_1, x_2, …, x_n is defined as:

:1 H_n := 1/n∑_k  = 1^n 1/x_k

That is, to find the harmonic mean of a set of n numbers, take the reciprocal of the arithmetic mean of their reciprocals.
"
Definition:Harmonic,Harmonic,"A harmonic function is a is a twice continuously differentiable function $f: U \to \R$ (where $U$ is an open set of $\R^n$) which satisfies Laplace's equation:

:$\dfrac {\partial^2 f} {\partial {x_1}^2} + \dfrac {\partial^2 f} {\partial {x_2}^2} + \cdots + \dfrac {\partial^2 f} {\partial {x_n}^2} = 0$

everywhere on $U$.


This is usually written using the $\nabla^2$ symbol to denote the Laplacian, as:

:$\nabla^2 f = 0$


=== Riemannian Manifold ===
",Definition:Harmonic Function,"['Definitions/Harmonic Functions', 'Definitions/Potential Theory']","A harmonic function is a is a twice continuously differentiable function f: U → (where U is an open set of ^n) which satisfies Laplace's equation:

:∂^2 f∂x_1^2 + ∂^2 f∂x_2^2 + ⋯ + ∂^2 f∂x_n^2 = 0

everywhere on U.


This is usually written using the ∇^2 symbol to denote the Laplacian, as:

:∇^2 f = 0


=== Riemannian Manifold ===
"
Definition:Harmonic,Harmonic,"Harmonic analysis is the study of functions by expressing them as the sum of series of a family of functions such as sines and cosines.
",Definition:Harmonic Analysis,"['Definitions/Harmonic Analysis', 'Definitions/Analysis']","Harmonic analysis is the study of functions by expressing them as the sum of series of a family of functions such as sines and cosines.
"
Definition:Harmonic,Harmonic,"The series defined as:
:$\ds \sum_{n \mathop = 1}^\infty \frac 1 n = 1 + \frac 1 2 + \frac 1 3 + \frac 1 4 + \cdots$

is known as the harmonic series.


=== General Harmonic Series ===

Let $\sequence {x_n}$ be a sequence of numbers such that $\sequence {\size {x_n} }$ is a harmonic sequence.


Then the series defined as:
:$\ds \sum_{n \mathop = 1}^\infty x_n$

is a harmonic series.
",Definition:Harmonic Series,"['Definitions/Harmonic Series', 'Definitions/Series', 'Definitions/Harmonic Numbers', 'Definitions/Real Analysis']","The series defined as:
:∑_n  = 1^∞1/n = 1 + 1/2 + 1/3 + 1/4 + ⋯

is known as the harmonic series.


=== General Harmonic Series ===

Let x_n be a sequence of numbers such that x_n is a harmonic sequence.


Then the series defined as:
:∑_n  = 1^∞ x_n

is a harmonic series.
"
Definition:Harmonic,Harmonic,"Mercator's constant is the real number:







",Definition:Mercator's Constant,"[""Definitions/Mercator's Constant"", 'Definitions/Harmonic Series', 'Definitions/Specific Numbers']","Mercator's constant is the real number:







"
Definition:Harmonic,Harmonic,"A harmonic is a solution $\phi$ to Laplace's equation in $2$ dimensions:
:$\nabla^2 \phi = 0$
that is:
:$\dfrac {\partial^2 \phi} {\partial x^2} + \dfrac {\partial^2 \phi} {\partial y^2} = 0$
",Definition:Harmonic (Analysis),"['Definitions/Harmonics (Analysis)', ""Definitions/Laplace's Equation""]","A harmonic is a solution ϕ to Laplace's equation in 2 dimensions:
:∇^2 ϕ = 0
that is:
:∂^2 ϕ∂ x^2 + ∂^2 ϕ∂ y^2 = 0
"
Definition:Harmonic,Harmonic,"A spherical harmonic is a solution $\phi$ to Laplace's equation in $3$ dimensions when expressed in spherical coordinates.
",Definition:Spherical Harmonic,"['Definitions/Spherical Harmonics', ""Definitions/Laplace's Equation""]","A spherical harmonic is a solution ϕ to Laplace's equation in 3 dimensions when expressed in spherical coordinates.
"
Definition:Harmonic,Harmonic,"A surface harmonic is a spherical harmonic:
:$r^n \paren {a_n \map {P_n} {\cos \theta} + \ds \sum_{m \mathop = 1}^n \paren { {a_n}^m \cos m \phi + {b_n}^m \sin m \phi} \map { {P_n}^m} {\cos \theta} }$

such that $r = 1$.

That is:
:$a_n \map {P_n} {\cos \theta} + \ds \sum_{m \mathop = 1}^n \paren { {a_n}^m \cos m \phi + {b_n}^m \sin m \phi} \map { {P_n}^m} {\cos \theta}$
A spherical harmonic is a solution $\phi$ to Laplace's equation in $3$ dimensions when expressed in spherical coordinates.
",Definition:Surface Harmonic,"['Definitions/Surface Harmonics', 'Definitions/Spherical Harmonics']","A surface harmonic is a spherical harmonic:
:r^n a_n P_ncosθ + ∑_m  = 1^n a_n^m cos m ϕ + b_n^m sin m ϕP_n^mcosθ

such that r = 1.

That is:
:a_n P_ncosθ + ∑_m  = 1^n a_n^m cos m ϕ + b_n^m sin m ϕP_n^mcosθ
A spherical harmonic is a solution ϕ to Laplace's equation in 3 dimensions when expressed in spherical coordinates.
"
Definition:Harmonic,Harmonic,"A tesseral harmonic is a surface harmonic in the form:
:$\cos m \phi \, \map { {P_n}^m} {\cos \theta}$
or:
:$\sin m \phi \, \map { {P_n}^m} {\cos \theta}$

such that $m < n$.
A surface harmonic is a spherical harmonic:
:$r^n \paren {a_n \map {P_n} {\cos \theta} + \ds \sum_{m \mathop = 1}^n \paren { {a_n}^m \cos m \phi + {b_n}^m \sin m \phi} \map { {P_n}^m} {\cos \theta} }$

such that $r = 1$.

That is:
:$a_n \map {P_n} {\cos \theta} + \ds \sum_{m \mathop = 1}^n \paren { {a_n}^m \cos m \phi + {b_n}^m \sin m \phi} \map { {P_n}^m} {\cos \theta}$
",Definition:Tesseral Harmonic,"['Definitions/Tesseral Harmonics', 'Definitions/Surface Harmonics', 'Definitions/Spherical Harmonics']","A tesseral harmonic is a surface harmonic in the form:
:cos m ϕ P_n^mcosθ
or:
:sin m ϕ P_n^mcosθ

such that m < n.
A surface harmonic is a spherical harmonic:
:r^n a_n P_ncosθ + ∑_m  = 1^n a_n^m cos m ϕ + b_n^m sin m ϕP_n^mcosθ

such that r = 1.

That is:
:a_n P_ncosθ + ∑_m  = 1^n a_n^m cos m ϕ + b_n^m sin m ϕP_n^mcosθ
"
Definition:Harmonic,Harmonic,"A sectoral harmonic is a surface harmonic in the form:
:$\cos m \phi \, \map { {P_n}^m} {\cos \theta}$
or:
:$\sin m \phi \, \map { {P_n}^m} {\cos \theta}$

such that $m = n$.
A surface harmonic is a spherical harmonic:
:$r^n \paren {a_n \map {P_n} {\cos \theta} + \ds \sum_{m \mathop = 1}^n \paren { {a_n}^m \cos m \phi + {b_n}^m \sin m \phi} \map { {P_n}^m} {\cos \theta} }$

such that $r = 1$.

That is:
:$a_n \map {P_n} {\cos \theta} + \ds \sum_{m \mathop = 1}^n \paren { {a_n}^m \cos m \phi + {b_n}^m \sin m \phi} \map { {P_n}^m} {\cos \theta}$
",Definition:Sectoral Harmonic,"['Definitions/Sectoral Harmonics', 'Definitions/Surface Harmonics', 'Definitions/Spherical Harmonics']","A sectoral harmonic is a surface harmonic in the form:
:cos m ϕ P_n^mcosθ
or:
:sin m ϕ P_n^mcosθ

such that m = n.
A surface harmonic is a spherical harmonic:
:r^n a_n P_ncosθ + ∑_m  = 1^n a_n^m cos m ϕ + b_n^m sin m ϕP_n^mcosθ

such that r = 1.

That is:
:a_n P_ncosθ + ∑_m  = 1^n a_n^m cos m ϕ + b_n^m sin m ϕP_n^mcosθ
"
Definition:Harmonic,Harmonic,"A spherical harmonic is a solution $\phi$ to Laplace's equation in $3$ dimensions when expressed in spherical coordinates.
Let $H$ be a spherical harmonic in the form:
:$r^n \paren {a_n \map {P_n} {\cos \theta} + \ds \sum_{m \mathop = 1}^n \paren { {a_n}^m \cos m \phi + {b_n}^m \sin m \phi} \map { {P_n}^m} {\cos \theta} }$

The function $\map {P_n} {\cos \theta}$ is known as a zonal harmonic.
",Definition:Zonal Harmonic,"['Definitions/Zonal Harmonics', 'Definitions/Spherical Harmonics']","A spherical harmonic is a solution ϕ to Laplace's equation in 3 dimensions when expressed in spherical coordinates.
Let H be a spherical harmonic in the form:
:r^n a_n P_ncosθ + ∑_m  = 1^n a_n^m cos m ϕ + b_n^m sin m ϕP_n^mcosθ

The function P_ncosθ is known as a zonal harmonic.
"
Definition:Harmonic,Harmonic,"Let $P$ be a physical particle.

Let its position $\map x t$ be a real function, where $t$ is time.

Let $k > 0$.


Then the potential energy of the form:

:$\map U x = \dfrac 1 2 k x^2$

is called the harmonic potential energy.",Definition:Harmonic Potential Energy,"['Definitions/Lagrangian Mechanics', 'Definitions/Physics']","Let P be a physical particle.

Let its position x t be a real function, where t is time.

Let k > 0.


Then the potential energy of the form:

:U x =  1 2 k x^2

is called the harmonic potential energy."
Definition:Harmonic,Harmonic,"Let $P$ be a physical particle.

Let its position $\map x t$ be a real function, where $t$ is time.

Let $k > 0$.


Then the potential energy of the form:

:$\map U x = \dfrac 1 2 k x^2$

is called the harmonic potential energy.
",Definition:Harmonic Oscillator,['Definitions/Physics'],"Let P be a physical particle.

Let its position x t be a real function, where t is time.

Let k > 0.


Then the potential energy of the form:

:U x =  1 2 k x^2

is called the harmonic potential energy.
"
Definition:Harmonic,Harmonic,"Let $A$, $B$, $C$ and $D$ be points on a straight line.

Let the cross-ratio $\set {A, B; C, D}$ of $A$, $B$, $C$ and $D$ be equal to $-1$:
:$\dfrac {AC / CB} {AD / DB} = -1$
that is:
:$\dfrac {AC \cdot DB} {AD \cdot CB} = -1$


Then $\set {A, B; C, D}$ is known as a harmonic ratio.
",Definition:Harmonic Ratio,"['Definitions/Harmonic Ratios', 'Definitions/Cross-Ratios']","Let A, B, C and D be points on a straight line.

Let the cross-ratio A, B; C, D of A, B, C and D be equal to -1:
:AC / CBAD / DB = -1
that is:
:AC · DBAD · CB = -1


Then A, B; C, D is known as a harmonic ratio.
"
Definition:Harmonic,Harmonic,"Let $A$, $B$, $C$ and $D$ be points on a straight line.

Let the cross-ratio $\set {A, B; C, D}$ of $A$, $B$, $C$ and $D$ be equal to $-1$:
:$\dfrac {AC / CB} {AD / DB} = -1$
that is:
:$\dfrac {AC \cdot DB} {AD \cdot CB} = -1$


Then $\set {A, B; C, D}$ is known as a harmonic ratio.
Let $A$ and $B$ be points on a straight line.

Let $P$ and $Q$ lie on $AB$ such that $P$ is on the line segment $AB$ while $Q$ is outside the line segment $AB$.


:


Let $P$ and $Q$ be positioned such that the cross-ratio $\set {A, B; P, Q}$ forms a harmonic ratio:
:$\dfrac {AP} {PB} = -\dfrac {AQ} {QB}$


Then $\tuple {AB, PQ}$ are said to be a harmonic range.
",Definition:Harmonic Range,"['Definitions/Harmonic Ranges', 'Definitions/Straight Lines', 'Definitions/Analytic Geometry']","Let A, B, C and D be points on a straight line.

Let the cross-ratio A, B; C, D of A, B, C and D be equal to -1:
:AC / CBAD / DB = -1
that is:
:AC · DBAD · CB = -1


Then A, B; C, D is known as a harmonic ratio.
Let A and B be points on a straight line.

Let P and Q lie on AB such that P is on the line segment AB while Q is outside the line segment AB.


:


Let P and Q be positioned such that the cross-ratio A, B; P, Q forms a harmonic ratio:
:APPB = -AQQB


Then AB, PQ are said to be a harmonic range.
"
Definition:Harmonic,Harmonic,"Let $AB$ and $PQ$ be line segments on a straight line such that $\tuple {AB, PQ}$ is a harmonic range.

Then $P$ and $Q$ are said to be harmonic conjugates with respect to $A$ and $B$.
Let $AB$ and $PQ$ be line segments on a straight line such that $\tuple {AB, PQ}$ is a harmonic range.

Let $O$ be a point which is not on the straight line $AB$.

Let $\map O {AB, PQ}$ be the harmonic pencil formed from $O$ and $\tuple {AB, PQ}$.


:


The rays $OP$ and $OQ$ are said to be harmonic conjugates with respect to $OA$ and $OB$.
",Definition:Harmonic Conjugates,"['Definitions/Harmonic Conjugates', 'Definitions/Harmonic Ranges']","Let AB and PQ be line segments on a straight line such that AB, PQ is a harmonic range.

Then P and Q are said to be harmonic conjugates with respect to A and B.
Let AB and PQ be line segments on a straight line such that AB, PQ is a harmonic range.

Let O be a point which is not on the straight line AB.

Let O AB, PQ be the harmonic pencil formed from O and AB, PQ.


:


The rays OP and OQ are said to be harmonic conjugates with respect to OA and OB.
"
Definition:Harmonic,Harmonic,"Let $A$ and $B$ be points on a straight line.

Let $P$ and $Q$ lie on $AB$ such that $P$ is on the line segment $AB$ while $Q$ is outside the line segment $AB$.


:


Let $P$ and $Q$ be positioned such that the cross-ratio $\set {A, B; P, Q}$ forms a harmonic ratio:
:$\dfrac {AP} {PB} = -\dfrac {AQ} {QB}$


Then $\tuple {AB, PQ}$ are said to be a harmonic range.
Let $A$ and $B$ be points on a straight line.

Let $P$ and $Q$ lie on $AB$ such that $\tuple {AB, PQ}$ is a harmonic range.


Let $O$ be a point which is not on the straight line $AB$.


:


Then the pencil $\map O {AB, PQ}$ formed by joining $O$ to the four points $A$, $B$, $P$ and $Q$ is said to be a harmonic pencil.
",Definition:Harmonic Pencil,"['Definitions/Harmonic Pencils', 'Definitions/Harmonic Ranges']","Let A and B be points on a straight line.

Let P and Q lie on AB such that P is on the line segment AB while Q is outside the line segment AB.


:


Let P and Q be positioned such that the cross-ratio A, B; P, Q forms a harmonic ratio:
:APPB = -AQQB


Then AB, PQ are said to be a harmonic range.
Let A and B be points on a straight line.

Let P and Q lie on AB such that AB, PQ is a harmonic range.


Let O be a point which is not on the straight line AB.


:


Then the pencil O AB, PQ formed by joining O to the four points A, B, P and Q is said to be a harmonic pencil.
"
Definition:Height,Height,"Height, like depth, is used as a term for linear measure in a dimension perpendicular to both length and breadth.


However, whereas depth has connotations of down, height is used for distances up from the plane.


=== Euclidean Definition ===

When discussing the size and shape of a general polygon, the words height and width are often seen.



In contrast, the width is the linear measure going across the page.",Definition:Linear Measure/Height,['Definitions/Length'],"Height, like depth, is used as a term for linear measure in a dimension perpendicular to both length and breadth.


However, whereas depth has connotations of down, height is used for distances up from the plane.


=== Euclidean Definition ===

When discussing the size and shape of a general polygon, the words height and width are often seen.



In contrast, the width is the linear measure going across the page."
Definition:Height,Height,"The height of a triangle is the length of a perpendicular from the apex to whichever side has been chosen as its base.


That is, the length of the altitude so defined.


:

Thus the length of the altitude $h_a$ so constructed is called the height of $\triangle ABC$.
",Definition:Triangle (Geometry)/Height,['Definitions/Triangles'],"The height of a triangle is the length of a perpendicular from the apex to whichever side has been chosen as its base.


That is, the length of the altitude so defined.


:

Thus the length of the altitude h_a so constructed is called the height of ABC.
"
Definition:Height,Height,"The height of a polygon is the length of a perpendicular from the base to the vertex most distant from the base.



:
",Definition:Polygon/Height,['Definitions/Polygons'],"The height of a polygon is the length of a perpendicular from the base to the vertex most distant from the base.



:
"
Definition:Height,Height,":

Let a perpendicular $AE$ be dropped from the apex of a cone to the plane containing its base.

The length $h$ of the line $AE$ is the height of the cone.",Definition:Cone (Geometry)/Height,['Definitions/Cones'],":

Let a perpendicular AE be dropped from the apex of a cone to the plane containing its base.

The length h of the line AE is the height of the cone."
Definition:Height,Height,"Let $A$ be a commutative ring with unity.

Let $\mathfrak p$ be a prime ideal in $A$.


The height of $\mathfrak p$ is the supremum over all $n$ such that there exists a chain of prime ideals:

:$\mathfrak p_0 \subsetneqq \mathfrak p_1 \subsetneqq \cdots \subsetneqq \mathfrak p_n = \mathfrak p$


It is denoted by:
:$\map {\operatorname {ht} } {\mathfrak p}$

",Definition:Height of Prime Ideal,"['Definitions/Ring Theory', 'Definitions/Commutative Algebra']","Let A be a commutative ring with unity.

Let 𝔭 be a prime ideal in A.


The height of 𝔭 is the supremum over all n such that there exists a chain of prime ideals:

:𝔭_0 ⫋𝔭_1 ⫋⋯⫋𝔭_n = 𝔭


It is denoted by:
:ht𝔭

"
Definition:Height,Height,"Let $A$ be a commutative ring with unity.

Let $\mathfrak p$ be a prime ideal in $A$.


The height of $\mathfrak p$ is the supremum over all $n$ such that there exists a chain of prime ideals:

:$\mathfrak p_0 \subsetneqq \mathfrak p_1 \subsetneqq \cdots \subsetneqq \mathfrak p_n = \mathfrak p$


It is denoted by:
:$\map {\operatorname {ht} } {\mathfrak p}$


",Definition:Height of Proper Ideal,"['Definitions/Ring Theory', 'Definitions/Commutative Algebra']","Let A be a commutative ring with unity.

Let 𝔭 be a prime ideal in A.


The height of 𝔭 is the supremum over all n such that there exists a chain of prime ideals:

:𝔭_0 ⫋𝔭_1 ⫋⋯⫋𝔭_n = 𝔭


It is denoted by:
:ht𝔭


"
Definition:Homogeneous,Homogeneous,"A homogeneous expression is an algebraic expression in which the variables can be replaced throughout by the product of that variable with a given non-zero constant, and the constant can be extracted as a factor of the resulting expression.
",Definition:Homogeneous Expression,"['Definitions/Homogeneous Expressions', 'Definitions/Expressions', 'Definitions/Algebra', 'Definitions/Homogeneity']","A homogeneous expression is an algebraic expression in which the variables can be replaced throughout by the product of that variable with a given non-zero constant, and the constant can be extracted as a factor of the resulting expression.
"
Definition:Homogeneous,Homogeneous,"A homogeneous equation is formed when a homogeneous expression is equated to zero.
A homogeneous expression is an algebraic expression in which the variables can be replaced throughout by the product of that variable with a given non-zero constant, and the constant can be extracted as a factor of the resulting expression.
",Definition:Homogeneous Equation,"['Definitions/Homogeneous Equations', 'Definitions/Homogeneous Expressions', 'Definitions/Homogeneity']","A homogeneous equation is formed when a homogeneous expression is equated to zero.
A homogeneous expression is an algebraic expression in which the variables can be replaced throughout by the product of that variable with a given non-zero constant, and the constant can be extracted as a factor of the resulting expression.
"
Definition:Homogeneous,Homogeneous,"A homogeneous quadratic equation is a quadratic equation in two variables such that each term is of degree $2$:

:$a x^2 + h x y + b y^2 = 0$",Definition:Homogeneous Quadratic Equation,['Definitions/Quadratic Equations'],"A homogeneous quadratic equation is a quadratic equation in two variables such that each term is of degree 2:

:a x^2 + h x y + b y^2 = 0"
Definition:Homogeneous,Homogeneous,A straight line or plane is homogeneous  it contains the origin.,Definition:Homogeneous (Analytic Geometry),['Definitions/Analytic Geometry'],A straight line or plane is homogeneous  it contains the origin.
Definition:Homogeneous,Homogeneous,"Let $\CC$ denote the Cartesian plane.

Let $P = \tuple {x, y}$ be an arbitrary point in $\CC$.


Let $x$ and $y$ be expressed in the forms:






where $Z$ is an arbitrary real number.


$P$ is then determined by the ordered triple $\tuple {X, Y, Z}$, the terms of which are called its homogeneous Cartesian coordinates.
",Definition:Homogeneous Cartesian Coordinates,"['Definitions/Homogeneous Cartesian Coordinates', 'Definitions/Cartesian Coordinate Systems', 'Definitions/Projective Geometry', 'Definitions/Homogeneity']","Let  denote the Cartesian plane.

Let P = x, y be an arbitrary point in .


Let x and y be expressed in the forms:






where Z is an arbitrary real number.


P is then determined by the ordered triple X, Y, Z, the terms of which are called its homogeneous Cartesian coordinates.
"
Definition:Homogeneous,Homogeneous,"A homogeneous polynomial is a polynomial whose monomials with nonzero coefficients all have the same total degree.
",Definition:Homogeneous Polynomial,"['Definitions/Homogeneous Polynomials', 'Definitions/Homogeneous Expressions', 'Definitions/Polynomial Theory', 'Definitions/Homogeneity']","A homogeneous polynomial is a polynomial whose monomials with nonzero coefficients all have the same total degree.
"
Definition:Homogeneous,Homogeneous,"A system of homogeneous linear equations is a set of simultaneous linear equations:

:$\ds \forall i \in \closedint 1 m: \sum_{j \mathop = 1}^n \alpha_{i j} x_j = \beta_i$

such that all the $\beta_i$ are equal to zero:

:$\ds \forall i \in \closedint 1 m : \sum_{j \mathop = 1}^n \alpha_{i j} x_j = 0$

That is:









=== Matrix Representation ===
",Definition:Homogeneous Linear Equations,"['Definitions/Algebra', 'Definitions/Linear Algebra']","A system of homogeneous linear equations is a set of simultaneous linear equations:

:∀ i ∈ 1 m: ∑_j  = 1^n α_i j x_j = β_i

such that all the β_i are equal to zero:

:∀ i ∈ 1 m : ∑_j  = 1^n α_i j x_j = 0

That is:









=== Matrix Representation ===
"
Definition:Homogeneous,Homogeneous,"Let $T$ be an $\LL$-theory.

Let $\kappa$ be an infinite cardinal.


A model $\MM$ of $T$ is $\kappa$-homogeneous  for every subset $A$ and element $b$ in the universe of $\MM$ with the cardinality of $A$ strictly less than $\kappa$, if $f: A \to \MM$ is partial elementary, then $f$ extends to an elementary map $f^*: A \cup \set b \to \MM$.

That is, $\MM$ is $\kappa$-homogeneous  for all $A \subseteq \MM$ with  $\card A < \kappa$ and all $b \in \MM$, every elementary $f: A \to \MM$ extends to an elementary $f^*: A \cup \set b \to \MM$.


We say $\MM$ is homogeneous  it is $\kappa$-homogeneous where $\kappa$ is the cardinality of the universe of $\MM$.",Definition:Homogeneous (Model Theory),['Definitions/Model Theory for Predicate Logic'],"Let T be an -theory.

Let κ be an infinite cardinal.


A model  of T is κ-homogeneous  for every subset A and element b in the universe of  with the cardinality of A strictly less than κ, if f: A → is partial elementary, then f extends to an elementary map f^*: A ∪ b →.

That is,  is κ-homogeneous  for all A ⊆ with  A < κ and all b ∈, every elementary f: A → extends to an elementary f^*: A ∪ b →.


We say  is homogeneous  it is κ-homogeneous where κ is the cardinality of the universe of ."
Definition:Homogeneous,Homogeneous,"A body is said to be homogeneous  the substance of any part of it is indistinguishable from any other part.


=== Warning ===

",Definition:Homogeneous (Physics),"['Definitions/Physics', 'Definitions/Homogeneity']","A body is said to be homogeneous  the substance of any part of it is indistinguishable from any other part.


=== Warning ===

"
Definition:Homomorphism,Homomorphism,"Let $\struct {G, \circ}$ and $\struct {H, *}$ be groups.

Let $\phi: G \to H$ be a mapping such that $\circ$ has the morphism property under $\phi$.


That is, $\forall a, b \in G$:
:$\map \phi {a \circ b} = \map \phi a * \map \phi b$


Then $\phi: \struct {G, \circ} \to \struct {H, *}$ is a group homomorphism.
",Definition:Group Homomorphism,"['Definitions/Group Homomorphisms', 'Definitions/Homomorphisms (Abstract Algebra)', 'Definitions/Group Theory']","Let G, ∘ and H, * be groups.

Let ϕ: G → H be a mapping such that ∘ has the morphism property under ϕ.


That is, ∀ a, b ∈ G:
:ϕa ∘ b = ϕ a * ϕ b


Then ϕ: G, ∘→H, * is a group homomorphism.
"
Definition:Homomorphism,Homomorphism,"Let $\struct {R, +, \circ}$ and $\struct {S, \oplus, *}$ be rings.

Let $\phi: R \to S$ be a mapping such that both $+$ and $\circ$ have the morphism property under $\phi$.


That is, $\forall a, b \in R$:







Then $\phi: \struct {R, +, \circ} \to \struct {S, \oplus, *}$ is a ring homomorphism.
",Definition:Ring Homomorphism,"['Definitions/Ring Homomorphisms', 'Definitions/Homomorphisms (Abstract Algebra)', 'Definitions/Ring Theory']","Let R, +, ∘ and S, ⊕, * be rings.

Let ϕ: R → S be a mapping such that both + and ∘ have the morphism property under ϕ.


That is, ∀ a, b ∈ R:







Then ϕ: R, +, ∘→S, ⊕, * is a ring homomorphism.
"
Definition:Homomorphism,Homomorphism,"Let $\struct {F, +, \times}$ and $\struct {K, \oplus, \otimes}$ be fields.

Let $\phi: F \to K$ be a mapping such that both $+$ and $\times$ have the morphism property under $\phi$.


That is, $\forall a, b \in F$:







Then $\phi: \struct {F, +, \times} \to \struct {K, \oplus, \otimes}$ is a field homomorphism.
",Definition:Field Homomorphism,"['Definitions/Field Homomorphisms', 'Definitions/Ring Homomorphisms', 'Definitions/Homomorphisms (Abstract Algebra)', 'Definitions/Field Theory']","Let F, +, × and K, ⊕, ⊗ be fields.

Let ϕ: F → K be a mapping such that both + and × have the morphism property under ϕ.


That is, ∀ a, b ∈ F:







Then ϕ: F, +, ×→K, ⊕, ⊗ is a field homomorphism.
"
Definition:Homomorphism,Homomorphism,"Let $R$ be a ring.

Let $\struct {S, \ast_1, \ast_2, \ldots, \ast_n, \circ}_R$ and $\struct {T, \odot_1, \odot_2, \ldots, \odot_n, \otimes}_R$ be $R$-algebraic structures.

Let $\phi: S \to T$ be a mapping.


Then $\phi$ is an $R$-algebraic structure homomorphism :

:$(1): \quad \forall k \in \closedint 1 n: \forall x, y \in S: \map \phi {x \ast_k y} = \map \phi x \odot_k \map \phi y$
:$(2): \quad \forall x \in S: \forall \lambda \in R: \map \phi {\lambda \circ x} = \lambda \otimes \map \phi x$
where $\closedint 1 n = \set {1, 2, \ldots, n}$ denotes an integer interval.


Note that this definition also applies to modules and vector spaces.
",Definition:R-Algebraic Structure Homomorphism,"['Definitions/R-Algebraic Structure Homomorphisms', 'Definitions/Homomorphisms (Abstract Algebra)', 'Definitions/Linear Algebra']","Let R be a ring.

Let S, ∗_1, ∗_2, …, ∗_n, ∘_R and T, ⊙_1, ⊙_2, …, ⊙_n, ⊗_R be R-algebraic structures.

Let ϕ: S → T be a mapping.


Then ϕ is an R-algebraic structure homomorphism :

:(1):   ∀ k ∈ 1 n: ∀ x, y ∈ S: ϕx ∗_k y = ϕ x ⊙_k ϕ y
:(2):   ∀ x ∈ S: ∀λ∈ R: ϕλ∘ x = λ⊗ϕ x
where 1 n = 1, 2, …, n denotes an integer interval.


Note that this definition also applies to modules and vector spaces.
"
Definition:Homomorphism,Homomorphism,"Let $G = \struct {\map V G, \map E G}$ and $H = \struct {\map V H, \map E H}$ be graphs.


Let there exist a mapping $F: \map V G \to \map V H$ such that:
:for each edge $\set {u, v} \in \map E G$
:there exists an edge $\set {\map F u, \map F v} \in \map E H$.


Then $G$ and $H$ are homomorphic.

The mapping $F$ is called a homomorphism from $G$ to $H$.
",Definition:Homomorphism (Graph Theory),"['Definitions/Graph Theory', 'Definitions/Homomorphisms']","Let G =  V G,  E G and H =  V H,  E H be graphs.


Let there exist a mapping F:  V G → V H such that:
:for each edge u, v∈ E G
:there exists an edge F u,  F v∈ E H.


Then G and H are homomorphic.

The mapping F is called a homomorphism from G to H.
"
Definition:Homomorphism,Homomorphism,,Definition:Bundle Homomorphism,"['Definitions/Riemannian Geometry', 'Definitions/Homomorphisms']",
Definition:Ideal,Ideal,"Let $\struct {R, +, \circ}$ be a ring.

Let $\struct {J, +}$ be a subgroup of $\struct {R, +}$.


$J$ is a left ideal of $R$ :
:$\forall j \in J: \forall r \in R: r \circ j \in J$

that is, :
:$\forall r \in R: r \circ J \subseteq J$
",Definition:Ideal of Ring/Left Ideal,['Definitions/Ideal Theory'],"Let R, +, ∘ be a ring.

Let J, + be a subgroup of R, +.


J is a left ideal of R :
:∀ j ∈ J: ∀ r ∈ R: r ∘ j ∈ J

that is, :
:∀ r ∈ R: r ∘ J ⊆ J
"
Definition:Ideal,Ideal,"Let $\struct {R, +, \circ}$ be a ring.

Let $\struct {J, +}$ be a subgroup of $\struct {R, +}$.


$J$ is a right ideal of $R$ :
:$\forall j \in J: \forall r \in R: j \circ r \in J$

that is, :
:$\forall r \in R: J \circ r \subseteq J$
",Definition:Ideal of Ring/Right Ideal,['Definitions/Ideal Theory'],"Let R, +, ∘ be a ring.

Let J, + be a subgroup of R, +.


J is a right ideal of R :
:∀ j ∈ J: ∀ r ∈ R: j ∘ r ∈ J

that is, :
:∀ r ∈ R: J ∘ r ⊆ J
"
Definition:Ideal,Ideal,"Let $R$ be a ring.

Let $\struct {A, \ast}$ be an $R$-algebra.

Let $\struct {J, \ast}$ be a subalgebra of $\struct {A, \ast}$. 


We say that $J$ is a left ideal of $A$ :
:for each $a \in A$ and $x \in J$ we have $a \ast x \in J$.
Let $R$ be a ring.

Let $\struct {A, \ast}$ be an $R$-algebra.

Let $\struct {J, \ast}$ be a subalgebra of $\struct {A, \ast}$. 


We say that $J$ is a left ideal of $A$ :
:for each $a \in A$ and $x \in J$ we have $x \ast a \in J$.
Let $R$ be a ring.

Let $\struct {A, \ast}$ be an $R$-algebra.

Let $\struct {J, \ast}$ be a subalgebra of $\struct {A, \ast}$. 


We say that $J$ is a proper ideal of $A$ :
:$J \ne A$
",Definition:Ideal of Algebra,['Definitions/Algebras'],"Let R be a ring.

Let A, ∗ be an R-algebra.

Let J, ∗ be a subalgebra of A, ∗. 


We say that J is a left ideal of A :
:for each a ∈ A and x ∈ J we have a ∗ x ∈ J.
Let R be a ring.

Let A, ∗ be an R-algebra.

Let J, ∗ be a subalgebra of A, ∗. 


We say that J is a left ideal of A :
:for each a ∈ A and x ∈ J we have x ∗ a ∈ J.
Let R be a ring.

Let A, ∗ be an R-algebra.

Let J, ∗ be a subalgebra of A, ∗. 


We say that J is a proper ideal of A :
:J  A
"
Definition:Ideal,Ideal,"Let $R$ be a ring.

Let $\struct {A, \ast}$ be an $R$-algebra.

Let $\struct {J, \ast}$ be a subalgebra of $\struct {A, \ast}$. 


We say that $J$ is a left ideal of $A$ :
:for each $a \in A$ and $x \in J$ we have $a \ast x \in J$.",Definition:Ideal of Algebra/Left Ideal,['Definitions/Ideals of Algebras'],"Let R be a ring.

Let A, ∗ be an R-algebra.

Let J, ∗ be a subalgebra of A, ∗. 


We say that J is a left ideal of A :
:for each a ∈ A and x ∈ J we have a ∗ x ∈ J."
Definition:Ideal,Ideal,"Let $R$ be a ring.

Let $\struct {A, \ast}$ be an $R$-algebra.

Let $\struct {J, \ast}$ be a subalgebra of $\struct {A, \ast}$. 


We say that $J$ is a left ideal of $A$ :
:for each $a \in A$ and $x \in J$ we have $x \ast a \in J$.",Definition:Ideal of Algebra/Right Ideal,['Definitions/Ideals of Algebras'],"Let R be a ring.

Let A, ∗ be an R-algebra.

Let J, ∗ be a subalgebra of A, ∗. 


We say that J is a left ideal of A :
:for each a ∈ A and x ∈ J we have x ∗ a ∈ J."
Definition:Ideal,Ideal,"Let $\struct {S, \preccurlyeq}$ be an ordered set.

Let $\II$ be an ideal on $\struct {S, \preccurlyeq}$.


Then:
:$\II$ is a proper ideal on $S$
:
:$\II \ne S$

That is,  $\II$ is a proper subset of $S$.
",Definition:Ideal (Order Theory),['Definitions/Order Theory'],"Let S, ≼ be an ordered set.

Let  be an ideal on S, ≼.


Then:
: is a proper ideal on S
:
:S

That is,   is a proper subset of S.
"
Definition:Ideal,Ideal,"Let $\struct {S, \vee, \preceq}$ be a join semilattice.

Let $I \subseteq S$ be a non-empty subset of $S$.


Then $I$ is a join semilattice ideal of $S$  $I$ satisifies the join semilattice ideal axioms:
",Definition:Join Semilattice Ideal,['Definitions/Lattice Theory'],"Let S, ∨, ≼ be a join semilattice.

Let I ⊆ S be a non-empty subset of S.


Then I is a join semilattice ideal of S  I satisifies the join semilattice ideal axioms:
"
Definition:Ideal,Ideal,An ideal (or idealized) object is one in which certain attributes are approximated to zero or infinity.,Definition:Ideal (Physics),"['Definitions/Physics', 'Definitions/Applied Mathematics', 'Definitions/Ideals in Physics']",An ideal (or idealized) object is one in which certain attributes are approximated to zero or infinity.
Definition:Identity,Identity,"An identity is an equation which is true for all values attained by the variables it contains.


=== Symbol ===

",Definition:Identity (Equation),"['Definitions/Identities (Equations)', 'Definitions/Equations', 'Definitions/Identities']","An identity is an equation which is true for all values attained by the variables it contains.


=== Symbol ===

"
Definition:Identity,Identity,"Let $\struct {S, \circ}$ be an algebraic structure.

An element $e_L \in S$ is called a left identity (element) :
:$\forall x \in S: e_L \circ x = x$
Let $\struct {S, \circ}$ be an algebraic structure.

An element $e_R \in S$ is called a right identity (element) :
:$\forall x \in S: x \circ e_R = x$
Let $\struct {S, \circ}$ be an algebraic structure.

An element $e \in S$ is called an identity (element)  it is both a left identity and a right identity:

:$\forall x \in S: x \circ e = x = e \circ x$


In Identity is Unique it is established that an identity element, if it exists, is unique within $\struct {S, \circ}$.

Thus it is justified to refer to it as the identity (of a given algebraic structure).


This identity is often denoted $e_S$, or $e$ if it is clearly understood what structure is being discussed.
",Definition:Identity (Abstract Algebra),"['Definitions/Identity Elements', 'Definitions/Abstract Algebra']","Let S, ∘ be an algebraic structure.

An element e_L ∈ S is called a left identity (element) :
:∀ x ∈ S: e_L ∘ x = x
Let S, ∘ be an algebraic structure.

An element e_R ∈ S is called a right identity (element) :
:∀ x ∈ S: x ∘ e_R = x
Let S, ∘ be an algebraic structure.

An element e ∈ S is called an identity (element)  it is both a left identity and a right identity:

:∀ x ∈ S: x ∘ e = x = e ∘ x


In Identity is Unique it is established that an identity element, if it exists, is unique within S, ∘.

Thus it is justified to refer to it as the identity (of a given algebraic structure).


This identity is often denoted e_S, or e if it is clearly understood what structure is being discussed.
"
Definition:Improper,Improper,"An improper fraction is a fraction representing a rational number whose absolute value is greater than $1$.

Specifically, when expressed in the form $r = \dfrac p q$, where $p$ and $q$ are integers such that (the absolute value of) the numerator is greater than (the absolute value of) the denominator: $\size p > \size q$.
",Definition:Fraction/Improper,"['Definitions/Improper Fractions', 'Definitions/Fractions']","An improper fraction is a fraction representing a rational number whose absolute value is greater than 1.

Specifically, when expressed in the form r =  p q, where p and q are integers such that (the absolute value of) the numerator is greater than (the absolute value of) the denominator: p >  q.
"
Definition:Improper,Improper,"An improper integral is a definite integral over an interval which is not closed, that is, open or half open, and whose limits of integration are the end points of that interval.

When the end point is not actually in the interval, the conventional definition of the definite integral is not valid.

Therefore we use the technique of limits to specify the integral.


Note: In the below, in all cases the necessary limits must exist in order for the definition to hold.
",Definition:Improper Integral,"['Definitions/Improper Integrals', 'Definitions/Definite Integrals', 'Definitions/Integral Calculus']","An improper integral is a definite integral over an interval which is not closed, that is, open or half open, and whose limits of integration are the end points of that interval.

When the end point is not actually in the interval, the conventional definition of the definite integral is not valid.

Therefore we use the technique of limits to specify the integral.


Note: In the below, in all cases the necessary limits must exist in order for the definition to hold.
"
Definition:Independent,Independent,,Definition:Independent Random Variables,"['Definitions/Probability Theory', 'Definitions/Random Variables', 'Definitions/Independent Random Variables']",
Definition:Independent,Independent,"Let $p$ and $q$ be statements.

Let it be the case that:
:$(1): \quad p$ and $q$ are not contrary
:$(2): \quad p$ and $q$ are not subcontrary
:$(3): \quad p$ is not superimplicant to $q$
:$(4): \quad p$ is not subimplicant to $q$
:$(5): \quad p$ and $q$ are not equivalent
:$(6): \quad p$ and $q$ are not contradictory.


Then $p$ and $q$ are independent statements.",Definition:Independent Statements,['Definitions/Logic'],"Let p and q be statements.

Let it be the case that:
:(1):    p and q are not contrary
:(2):    p and q are not subcontrary
:(3):    p is not superimplicant to q
:(4):    p is not subimplicant to q
:(5):    p and q are not equivalent
:(6):    p and q are not contradictory.


Then p and q are independent statements."
Definition:Index,Index,"Let $I$ and $S$ be sets.

Let $x: I \to S$ be a mapping.

Let $x_i$ denote the image of an element $i \in I$ of the domain $I$ of $x$.

Let $\family {x_i}_{i \mathop \in I}$ denote the set of the images of all the element $i \in I$ under $x$.


An element of the domain $I$ of $x$ is called an index.
",Definition:Indexing Set/Index,"['Definitions/Indexed Families', 'Definitions/Indices']","Let I and S be sets.

Let x: I → S be a mapping.

Let x_i denote the image of an element i ∈ I of the domain I of x.

Let x_i_i ∈ I denote the set of the images of all the element i ∈ I under x.


An element of the domain I of x is called an index.
"
Definition:Index,Index,"Let $I$ and $S$ be sets.

Let $x: I \to S$ be a mapping.

Let $x_i$ denote the image of an element $i \in I$ of the domain $I$ of $x$.

Let $\family {x_i}_{i \mathop \in I}$ denote the set of the images of all the element $i \in I$ under $x$.


An element of the domain $I$ of $x$ is called an index.
Let $I$ and $S$ be sets.

Let $x: I \to S$ be a mapping.

Let $x_i$ denote the image of an element $i \in I$ of the domain $I$ of $x$.

Let $\family {x_i}_{i \mathop \in I}$ denote the set of the images of all the elements $i \in I$ under $x$.


The image of $x$, that is, $x \sqbrk I$ or $\Img x$, is called an indexed set.

That is, it is the set indexed by $I$.
Let $I$ and $S$ be sets.

Let $x: I \to S$ be a mapping.

Let $x_i$ denote the image of an element $i \in I$ of the domain $I$ of $x$.

Let $\family {x_i}_{i \mathop \in I}$ denote the set of the images of all the element $i \in I$ under $x$.


When used in this context, the mapping $x$ is referred to as an indexing function for $S$.


=== Notation ===

",Definition:Indexing Set,"['Definitions/Indexed Families', 'Definitions/Set Theory', 'Definitions/Mapping Theory']","Let I and S be sets.

Let x: I → S be a mapping.

Let x_i denote the image of an element i ∈ I of the domain I of x.

Let x_i_i ∈ I denote the set of the images of all the element i ∈ I under x.


An element of the domain I of x is called an index.
Let I and S be sets.

Let x: I → S be a mapping.

Let x_i denote the image of an element i ∈ I of the domain I of x.

Let x_i_i ∈ I denote the set of the images of all the elements i ∈ I under x.


The image of x, that is, x  I or x, is called an indexed set.

That is, it is the set indexed by I.
Let I and S be sets.

Let x: I → S be a mapping.

Let x_i denote the image of an element i ∈ I of the domain I of x.

Let x_i_i ∈ I denote the set of the images of all the element i ∈ I under x.


When used in this context, the mapping x is referred to as an indexing function for S.


=== Notation ===

"
Definition:Index,Index,"Let $\sequence {x_n}$ be a sequence.

Let $x_k$ be the $k$th term of $\sequence {x_n}$.

Then the integer $k$ is known as the index of $x_k$.",Definition:Term of Sequence/Index,"['Definitions/Sequences', 'Definitions/Indices']","Let x_n be a sequence.

Let x_k be the kth term of x_n.

Then the integer k is known as the index of x_k."
Definition:Index,Index,"Let $\mathbf A$ be an $m \times n$ matrix.

Let $a_{i j}$ be the element in row $i$ and column $j$ of $\mathbf A$.


Then the subscripts $i$ and $j$ are referred to as the indices (singular: index) of $a_{i j}$.
",Definition:Matrix/Indices,"['Definitions/Matrices', 'Definitions/Indices']","Let 𝐀 be an m × n matrix.

Let a_i j be the element in row i and column j of 𝐀.


Then the subscripts i and j are referred to as the indices (singular: index) of a_i j.
"
Definition:Index,Index,"In the power operation $x^r$, the number $r$ is known as the exponent of $x$, particularly for $r \in \R$.",Definition:Power (Algebra)/Exponent,"['Definitions/Exponents', 'Definitions/Powers', 'Definitions/Indices']","In the power operation x^r, the number r is known as the exponent of x, particularly for r ∈."
Definition:Index,Index,"Let $\sqrt [n] x$ denote the $n$th root of $x$.

The number $n$ is known as the index of the root.


If $n$ is not specified, that is $\sqrt x$ is presented, this means the square root.
",Definition:Root of Number/Index,"['Definitions/Roots of Numbers', 'Definitions/Indices']","Let √(x) denote the nth root of x.

The number n is known as the index of the root.


If n is not specified, that is √(x) is presented, this means the square root.
"
Definition:Index,Index,"Consider the summation, in either of the three forms:

:$\ds \sum_{j \mathop = 1}^n a_j \qquad \sum_{1 \mathop \le j \mathop \le n} a_j \qquad \sum_{\map R j} a_j$


The variable $j$, an example of a bound variable, is known as the index variable of the summation.
",Definition:Summation/Index Variable,"['Definitions/Summations', 'Definitions/Indices']","Consider the summation, in either of the three forms:

:∑_j  = 1^n a_j     ∑_1 ≤ j ≤ n a_j     ∑_ R j a_j


The variable j, an example of a bound variable, is known as the index variable of the summation.
"
Definition:Indicator,Indicator,"Let $G$ be a finite group.

Let $a \in G$.

Let $H$ be a subgroup of $G$.


The indicator of $a$ in $H$ is the least strictly positive integer $n$ such that $a^n \in H$.",Definition:Indicator of Group Element,['Definitions/Group Theory'],"Let G be a finite group.

Let a ∈ G.

Let H be a subgroup of G.


The indicator of a in H is the least strictly positive integer n such that a^n ∈ H."
Definition:Indicator,Indicator,,Definition:Indicator Function,[],
Definition:Inertia,Inertia,"Inertia is the tendency of a body to maintain the same velocity in the absence of an external force, in accordance with Newton's First Law of Motion.

Equivalently put, inertia is the resistance of a body to a change in its motion.

Inertia is equivalent to mass.
Inertia is the tendency of a body to maintain the same velocity in the absence of an external force, in accordance with Newton's First Law of Motion.

Equivalently put, inertia is the resistance of a body to a change in its motion.

Inertia is equivalent to mass.
Inertia is the tendency of a body to maintain the same velocity in the absence of an external force, in accordance with Newton's First Law of Motion.

Equivalently put, inertia is the resistance of a body to a change in its motion.

Inertia is equivalent to mass.
",Definition:Inertia (Physics),"['Definitions/Inertia (Physics)', 'Definitions/Physics', 'Definitions/Inertia']","Inertia is the tendency of a body to maintain the same velocity in the absence of an external force, in accordance with Newton's First Law of Motion.

Equivalently put, inertia is the resistance of a body to a change in its motion.

Inertia is equivalent to mass.
Inertia is the tendency of a body to maintain the same velocity in the absence of an external force, in accordance with Newton's First Law of Motion.

Equivalently put, inertia is the resistance of a body to a change in its motion.

Inertia is equivalent to mass.
Inertia is the tendency of a body to maintain the same velocity in the absence of an external force, in accordance with Newton's First Law of Motion.

Equivalently put, inertia is the resistance of a body to a change in its motion.

Inertia is equivalent to mass.
"
Definition:Inertia,Inertia,"Let $\mathbf H$ be a Hermitian matrix.

The inertia of $\mathbf H$ is an ordered triple of integers comprising:
:the number of positive eigenvalues of $\mathbf H$
:the number of negative eigenvalues of $\mathbf H$
:the number of zero eigenvalues of $\mathbf H$
in that order.
",Definition:Inertia of Hermitian Matrix,"['Definitions/Inertia of Hermitian Matrices', 'Definitions/Hermitian Matrices', 'Definitions/Inertia']","Let 𝐇 be a Hermitian matrix.

The inertia of 𝐇 is an ordered triple of integers comprising:
:the number of positive eigenvalues of 𝐇
:the number of negative eigenvalues of 𝐇
:the number of zero eigenvalues of 𝐇
in that order.
"
Definition:Infimum,Infimum,"Let $\struct {S, \preccurlyeq}$ be an ordered set.

Let $T \subseteq S$.


An element $c \in S$ is the infimum of $T$ in $S$ :

:$(1): \quad c$ is a lower bound of $T$ in $S$
:$(2): \quad d \preccurlyeq c$ for all lower bounds $d$ of $T$ in $S$.


If there exists an infimum of $T$ (in $S$), we say that:
:$T$ admits an infimum (in $S$) or
:$T$ has an infimum (in $S$).


=== Subset of Real Numbers ===

The concept is often encountered where $\struct {S, \preccurlyeq}$ is the set of real numbers under the usual ordering $\struct {\R, \le}$:



The infimum of $T$ is denoted $\inf T$ or $\map \inf T$.


=== Finite Infimum ===

Let $\struct {S, \preccurlyeq}$ be an ordered set.

Let $T \subseteq S$.


An element $c \in S$ is the infimum of $T$ in $S$ :

:$(1): \quad c$ is a lower bound of $T$ in $S$
:$(2): \quad d \preccurlyeq c$ for all lower bounds $d$ of $T$ in $S$.


If there exists an infimum of $T$ (in $S$), we say that:
:$T$ admits an infimum (in $S$) or
:$T$ has an infimum (in $S$).


=== Subset of Real Numbers ===

The concept is often encountered where $\struct {S, \preccurlyeq}$ is the set of real numbers under the usual ordering $\struct {\R, \le}$:



The infimum of $T$ is denoted $\inf T$ or $\map \inf T$.


=== Finite Infimum ===

Let $\struct {S, \preccurlyeq}$ be an ordered set.

Let $T \subseteq S$.


An element $c \in S$ is the infimum of $T$ in $S$ :

:$(1): \quad c$ is a lower bound of $T$ in $S$
:$(2): \quad d \preccurlyeq c$ for all lower bounds $d$ of $T$ in $S$.


If there exists an infimum of $T$ (in $S$), we say that:
:$T$ admits an infimum (in $S$) or
:$T$ has an infimum (in $S$).


=== Subset of Real Numbers ===

The concept is often encountered where $\struct {S, \preccurlyeq}$ is the set of real numbers under the usual ordering $\struct {\R, \le}$:



The infimum of $T$ is denoted $\inf T$ or $\map \inf T$.


=== Finite Infimum ===

Let $\struct {S, \preccurlyeq}$ be an ordered set.

Let $T \subseteq S$ admit a infimum $\map \inf T$.


If $T$ is finite, $\map \inf T$ is called a finite infimum.
",Definition:Infimum of Set,['Definitions/Infima'],"Let S, ≼ be an ordered set.

Let T ⊆ S.


An element c ∈ S is the infimum of T in S :

:(1):    c is a lower bound of T in S
:(2):    d ≼ c for all lower bounds d of T in S.


If there exists an infimum of T (in S), we say that:
:T admits an infimum (in S) or
:T has an infimum (in S).


=== Subset of Real Numbers ===

The concept is often encountered where S, ≼ is the set of real numbers under the usual ordering , ≤:



The infimum of T is denoted inf T or inf T.


=== Finite Infimum ===

Let S, ≼ be an ordered set.

Let T ⊆ S.


An element c ∈ S is the infimum of T in S :

:(1):    c is a lower bound of T in S
:(2):    d ≼ c for all lower bounds d of T in S.


If there exists an infimum of T (in S), we say that:
:T admits an infimum (in S) or
:T has an infimum (in S).


=== Subset of Real Numbers ===

The concept is often encountered where S, ≼ is the set of real numbers under the usual ordering , ≤:



The infimum of T is denoted inf T or inf T.


=== Finite Infimum ===

Let S, ≼ be an ordered set.

Let T ⊆ S.


An element c ∈ S is the infimum of T in S :

:(1):    c is a lower bound of T in S
:(2):    d ≼ c for all lower bounds d of T in S.


If there exists an infimum of T (in S), we say that:
:T admits an infimum (in S) or
:T has an infimum (in S).


=== Subset of Real Numbers ===

The concept is often encountered where S, ≼ is the set of real numbers under the usual ordering , ≤:



The infimum of T is denoted inf T or inf T.


=== Finite Infimum ===

Let S, ≼ be an ordered set.

Let T ⊆ S admit a infimum inf T.


If T is finite, inf T is called a finite infimum.
"
Definition:Infimum,Infimum,"Let $T \subseteq \R$.


A real number $c \in \R$ is the infimum of $T$ in $\R$ :

:$(1): \quad c$ is a lower bound of $T$ in $\R$
:$(2): \quad d \le c$ for all lower bounds $d$ of $T$ in $\R$.


If there exists an infimum of $T$ (in $\R$), we say that $T$ admits an infimum (in $\R$).


The infimum of $T$ is denoted $\inf T$ or $\map \inf T$.",Definition:Infimum of Set/Real Numbers,['Definitions/Infima'],"Let T ⊆.


A real number c ∈ is the infimum of T in  :

:(1):    c is a lower bound of T in 
:(2):    d ≤ c for all lower bounds d of T in .


If there exists an infimum of T (in ), we say that T admits an infimum (in ).


The infimum of T is denoted inf T or inf T."
Definition:Infimum,Infimum,"Let $\struct {S, \preccurlyeq}$ be an ordered set.

Let $T \subseteq S$.


An element $c \in S$ is the infimum of $T$ in $S$ :

:$(1): \quad c$ is a lower bound of $T$ in $S$
:$(2): \quad d \preccurlyeq c$ for all lower bounds $d$ of $T$ in $S$.


If there exists an infimum of $T$ (in $S$), we say that:
:$T$ admits an infimum (in $S$) or
:$T$ has an infimum (in $S$).


=== Subset of Real Numbers ===

The concept is often encountered where $\struct {S, \preccurlyeq}$ is the set of real numbers under the usual ordering $\struct {\R, \le}$:



The infimum of $T$ is denoted $\inf T$ or $\map \inf T$.


=== Finite Infimum ===

",Definition:Infimum of Mapping,['Definitions/Infima'],"Let S, ≼ be an ordered set.

Let T ⊆ S.


An element c ∈ S is the infimum of T in S :

:(1):    c is a lower bound of T in S
:(2):    d ≼ c for all lower bounds d of T in S.


If there exists an infimum of T (in S), we say that:
:T admits an infimum (in S) or
:T has an infimum (in S).


=== Subset of Real Numbers ===

The concept is often encountered where S, ≼ is the set of real numbers under the usual ordering , ≤:



The infimum of T is denoted inf T or inf T.


=== Finite Infimum ===

"
Definition:Infimum,Infimum,"Let $S$ be a set.

Let $\struct {T, \preceq}$ be an ordered set.

Let $f: S \to T$ be a mapping from $S$ to $T$.

Let $f \sqbrk S$, the image of $f$, admit an infimum.


Then the infimum of $f$ (on $S$) is defined by:
:$\ds \inf_{x \mathop \in S} \map f x = \inf f \sqbrk S$


=== Real-Valued Function ===

Let $\struct {S, \preccurlyeq}$ be an ordered set.

Let $T \subseteq S$.


An element $c \in S$ is the infimum of $T$ in $S$ :

:$(1): \quad c$ is a lower bound of $T$ in $S$
:$(2): \quad d \preccurlyeq c$ for all lower bounds $d$ of $T$ in $S$.


If there exists an infimum of $T$ (in $S$), we say that:
:$T$ admits an infimum (in $S$) or
:$T$ has an infimum (in $S$).


=== Subset of Real Numbers ===

The concept is often encountered where $\struct {S, \preccurlyeq}$ is the set of real numbers under the usual ordering $\struct {\R, \le}$:



The infimum of $T$ is denoted $\inf T$ or $\map \inf T$.


=== Finite Infimum ===

",Definition:Infimum of Sequence,"['Definitions/Sequences', 'Definitions/Infima']","Let S be a set.

Let T, ≼ be an ordered set.

Let f: S → T be a mapping from S to T.

Let f  S, the image of f, admit an infimum.


Then the infimum of f (on S) is defined by:
:inf_x ∈ S f x = inf f  S


=== Real-Valued Function ===

Let S, ≼ be an ordered set.

Let T ⊆ S.


An element c ∈ S is the infimum of T in S :

:(1):    c is a lower bound of T in S
:(2):    d ≼ c for all lower bounds d of T in S.


If there exists an infimum of T (in S), we say that:
:T admits an infimum (in S) or
:T has an infimum (in S).


=== Subset of Real Numbers ===

The concept is often encountered where S, ≼ is the set of real numbers under the usual ordering , ≤:



The infimum of T is denoted inf T or inf T.


=== Finite Infimum ===

"
Definition:Integral,Integral,"Let $A$ be a commutative ring with unity.

Let $R \subseteq A$ be a subring.


Then $a \in A$ is said to be integral over $R$  is is a root of a monic nonzero polynomial over $R$.
",Definition:Integral Closure,"['Definitions/Algebraic Number Theory', 'Definitions/Commutative Algebra']","Let A be a commutative ring with unity.

Let R ⊆ A be a subring.


Then a ∈ A is said to be integral over R  is is a root of a monic nonzero polynomial over R.
"
Definition:Integral,Integral,"Let $R$ be an integral domain.


Then $R$ is integrally closed  it is integrally closed in its field of fractions.
",Definition:Integrally Closed,"['Definitions/Algebraic Number Theory', 'Definitions/Commutative Algebra']","Let R be an integral domain.


Then R is integrally closed  it is integrally closed in its field of fractions.
"
Definition:Integral,Integral,"Let $A$ be a commutative ring with unity.

Let $R \subseteq A$ be a subring.


Then $a \in A$ is said to be integral over $R$  is is a root of a monic nonzero polynomial over $R$.",Definition:Integral Element of Ring Extension,"['Definitions/Algebraic Number Theory', 'Definitions/Commutative Algebra']","Let A be a commutative ring with unity.

Let R ⊆ A be a subring.


Then a ∈ A is said to be integral over R  is is a root of a monic nonzero polynomial over R."
Definition:Integral,Integral,"An integral polynomial is a polynomial over the ring of integers $\Z$.
",Definition:Integral Polynomial,"['Definitions/Integral Polynomials', 'Definitions/Polynomial Theory', 'Definitions/Integers']","An integral polynomial is a polynomial over the ring of integers .
"
Definition:Integral,Integral,"Integral calculus is a subfield of calculus which is concerned with the study of the rates at which quantities accumulate.

Equivalently, given the rate of change of a quantity integral calculus provides techniques of providing the quantity itself.

The equivalence of the two uses are demonstrated in the Fundamental Theorem of Calculus.

The technique is also frequently used for the purpose of calculating areas and volumes of curved geometric figures.
Integral calculus is a subfield of calculus which is concerned with the study of the rates at which quantities accumulate.

Equivalently, given the rate of change of a quantity integral calculus provides techniques of providing the quantity itself.

The equivalence of the two uses are demonstrated in the Fundamental Theorem of Calculus.

The technique is also frequently used for the purpose of calculating areas and volumes of curved geometric figures.
",Definition:Calculus/Integral,"['Definitions/Integral Calculus', 'Definitions/Calculus', 'Definitions/Branches of Mathematics']","Integral calculus is a subfield of calculus which is concerned with the study of the rates at which quantities accumulate.

Equivalently, given the rate of change of a quantity integral calculus provides techniques of providing the quantity itself.

The equivalence of the two uses are demonstrated in the Fundamental Theorem of Calculus.

The technique is also frequently used for the purpose of calculating areas and volumes of curved geometric figures.
Integral calculus is a subfield of calculus which is concerned with the study of the rates at which quantities accumulate.

Equivalently, given the rate of change of a quantity integral calculus provides techniques of providing the quantity itself.

The equivalence of the two uses are demonstrated in the Fundamental Theorem of Calculus.

The technique is also frequently used for the purpose of calculating areas and volumes of curved geometric figures.
"
Definition:Integral,Integral,"Let $\closedint a b$ be a closed real interval.

Let $f: \closedint a b \to \R$ be a real function.

Let $\Delta$ be a finite subdivision of $\closedint a b$, $\Delta = \set {x_0, \ldots, x_n}$, $x_0 = a$ and $x_n = b$.

Let there for $\Delta$ be a corresponding sequence $C$ of sample points $c_i$, $C = \tuple {c_1, \ldots, c_n}$, where $c_i \in \closedint {x_{i - 1} } {x_i}$ for every $i \in \set {1, \ldots, n}$.

Let $\map S {f; \Delta, C}$ denote the Riemann sum of $f$ for the subdivision $\Delta$ and the sample point sequence $C$.


Then $f$ is said to be (properly) Riemann integrable on $\closedint a b$ :
:$\exists L \in \R: \forall \epsilon \in \R_{>0}: \exists \delta \in \R_{>0}: \forall$ finite subdivisions $\Delta$ of $\closedint a b: \forall$ sample point sequences $C$ of $\Delta: \norm \Delta < \delta \implies \size {\map S {f; \Delta, C} - L} < \epsilon$
where $\norm \Delta$ denotes the norm of $\Delta$.


The real number $L$ is called the Riemann integral of $f$ over $\closedint a b$ and is denoted:
:$\ds \int_a^b \map f x \rd x$


More usually (and informally), we say:
:$f$ is (Riemann) integrable over $\closedint a b$.


=== Riemann Integral as Integral Operator ===

Let $\closedint a b$ be a closed real interval.

Let $f: \closedint a b \to \R$ be a real function.

Let $f$ be bounded on $\closedint a b$.


Suppose that:
:$\ds \underline {\int_a^b} \map f x \rd x = \overline {\int_a^b} \map f x \rd x$
where $\ds \underline {\int_a^b}$ and $\ds \overline {\int_a^b}$ denote the lower Darboux integral and upper Darboux integral, respectively.


Then the definite (Darboux) integral of $f$ over $\closedint a b$ is defined as:
:$\ds \int_a^b \map f x \rd x = \underline {\int_a^b} \map f x \rd x = \overline {\int_a^b} \map f x \rd x$


$f$ is formally defined as (properly) integrable over $\closedint a b$ in the sense of Darboux, or (properly) Darboux integrable over $\closedint a b$.


More usually (and informally), we say:
:$f$ is (Darboux) integrable over $\closedint a b$.
",Definition:Definite Integral,"['Definitions/Definite Integrals', 'Definitions/Integral Calculus']","Let a b be a closed real interval.

Let f:  a b → be a real function.

Let Δ be a finite subdivision of a b, Δ = x_0, …, x_n, x_0 = a and x_n = b.

Let there for Δ be a corresponding sequence C of sample points c_i, C = c_1, …, c_n, where c_i ∈x_i - 1x_i for every i ∈1, …, n.

Let S f; Δ, C denote the Riemann sum of f for the subdivision Δ and the sample point sequence C.


Then f is said to be (properly) Riemann integrable on a b :
:∃ L ∈: ∀ϵ∈_>0: ∃δ∈_>0: ∀ finite subdivisions Δ of a b: ∀ sample point sequences C of Δ: Δ < δ S f; Δ, C - L < ϵ
where Δ denotes the norm of Δ.


The real number L is called the Riemann integral of f over a b and is denoted:
:∫_a^b  f x  x


More usually (and informally), we say:
:f is (Riemann) integrable over a b.


=== Riemann Integral as Integral Operator ===

Let a b be a closed real interval.

Let f:  a b → be a real function.

Let f be bounded on a b.


Suppose that:
:∫_a^b f x  x = ∫_a^b f x  x
where ∫_a^b and ∫_a^b denote the lower Darboux integral and upper Darboux integral, respectively.


Then the definite (Darboux) integral of f over a b is defined as:
:∫_a^b  f x  x = ∫_a^b f x  x = ∫_a^b f x  x


f is formally defined as (properly) integrable over a b in the sense of Darboux, or (properly) Darboux integrable over a b.


More usually (and informally), we say:
:f is (Darboux) integrable over a b.
"
Definition:Interior,Interior,"Let $S \subseteq \C$ be a subset of the complex plane.

Let $z \in S$.


$z$ is an interior point of $S$  $z$ has an $\epsilon$-neighborhood $\map {N_\epsilon} z$ such that $\map {N_\epsilon} z \subseteq S$.
",Definition:Interior (Complex Analysis),['Definitions/Complex Analysis'],"Let S ⊆ be a subset of the complex plane.

Let z ∈ S.


z is an interior point of S  z has an ϵ-neighborhood N_ϵ z such that N_ϵ z ⊆ S.
"
Definition:Interior,Interior,"Let $S \subseteq \C$ be a subset of the complex plane.

Let $z \in S$.


$z$ is an interior point of $S$  $z$ has an $\epsilon$-neighborhood $\map {N_\epsilon} z$ such that $\map {N_\epsilon} z \subseteq S$.",Definition:Interior Point (Complex Analysis),['Definitions/Complex Analysis'],"Let S ⊆ be a subset of the complex plane.

Let z ∈ S.


z is an interior point of S  z has an ϵ-neighborhood N_ϵ z such that N_ϵ z ⊆ S."
Definition:Interior,Interior,":


An interior angle of a transversal is an angle which is between the two lines cut by that transversal.

In the above figure, the interior angles with respect to the transversal $EF$ are:
:$\angle AHJ$
:$\angle CJH$
:$\angle BHJ$
:$\angle DJH$
:


An interior angle of a transversal is an angle which is between the two lines cut by that transversal.

In the above figure, the interior angles with respect to the transversal $EF$ are:
:$\angle AHJ$
:$\angle CJH$
:$\angle BHJ$
:$\angle DJH$
",Definition:Transversal (Geometry)/Interior Angle,['Definitions/Transversals (Geometry)'],":


An interior angle of a transversal is an angle which is between the two lines cut by that transversal.

In the above figure, the interior angles with respect to the transversal EF are:
:∠ AHJ
:∠ CJH
:∠ BHJ
:∠ DJH
:


An interior angle of a transversal is an angle which is between the two lines cut by that transversal.

In the above figure, the interior angles with respect to the transversal EF are:
:∠ AHJ
:∠ CJH
:∠ BHJ
:∠ DJH
"
Definition:Interior,Interior,"The internal angle of a vertex of a polygon is the size of the angle between the sides adjacent to that vertex, as measured inside the polygon.",Definition:Polygon/Internal Angle,"['Definitions/Internal Angles', 'Definitions/Polygons']","The internal angle of a vertex of a polygon is the size of the angle between the sides adjacent to that vertex, as measured inside the polygon."
Definition:Interior,Interior,"Let $f: \closedint 0 1 \to \R^2$ be a Jordan curve.


It follows from the Jordan Curve Theorem that $\R^2 \setminus \Img f$ is a union of two disjoint connected components, one of which is bounded.

This bounded component is called the interior of $f$, and is denoted as $\Int f$.
",Definition:Jordan Curve/Interior,['Definitions/Jordan Curves'],"Let f:  0 1 →^2 be a Jordan curve.


It follows from the Jordan Curve Theorem that ^2 ∖ f is a union of two disjoint connected components, one of which is bounded.

This bounded component is called the interior of f, and is denoted as f.
"
Definition:Interior Angle,Interior Angle,"The internal angle of a vertex of a polygon is the size of the angle between the sides adjacent to that vertex, as measured inside the polygon.",Definition:Polygon/Internal Angle,"['Definitions/Internal Angles', 'Definitions/Polygons']","The internal angle of a vertex of a polygon is the size of the angle between the sides adjacent to that vertex, as measured inside the polygon."
Definition:Interior Angle,Interior Angle,":


An interior angle of a transversal is an angle which is between the two lines cut by that transversal.

In the above figure, the interior angles with respect to the transversal $EF$ are:
:$\angle AHJ$
:$\angle CJH$
:$\angle BHJ$
:$\angle DJH$
:


An interior angle of a transversal is an angle which is between the two lines cut by that transversal.

In the above figure, the interior angles with respect to the transversal $EF$ are:
:$\angle AHJ$
:$\angle CJH$
:$\angle BHJ$
:$\angle DJH$
",Definition:Transversal (Geometry)/Interior Angle,['Definitions/Transversals (Geometry)'],":


An interior angle of a transversal is an angle which is between the two lines cut by that transversal.

In the above figure, the interior angles with respect to the transversal EF are:
:∠ AHJ
:∠ CJH
:∠ BHJ
:∠ DJH
:


An interior angle of a transversal is an angle which is between the two lines cut by that transversal.

In the above figure, the interior angles with respect to the transversal EF are:
:∠ AHJ
:∠ CJH
:∠ BHJ
:∠ DJH
"
Definition:Inverse,Inverse,"The inverse of the conditional:
: $p \implies q$
is the statement:
:$\neg p \implies \neg q$",Definition:Inverse Statement,['Definitions/Conditional'],"The inverse of the conditional:
: p  q
is the statement:
:p  q"
Definition:Inverse,Inverse,"Let $\struct {S, \circ}$ be an algebraic structure with an identity element $e_S$.

Let $x, y \in S$ be elements.


The element $y$ is an inverse of $x$ :
:$y \circ x = e_S = x \circ y$
that is,  $y$ is both:
:a left inverse of $x$
and:
:a right inverse of $x$.
Let $\struct {S, \circ}$ be a monoid whose identity is $e_S$.

An element $x_L \in S$ is called a left inverse of $x$ :
:$x_L \circ x = e_S$
Let $\struct {S, \circ}$ be a monoid whose identity is $e_S$.

An element $x_R \in S$ is called a right inverse of $x$ :
:$x \circ x_R = e_S$
",Definition:Inverse (Abstract Algebra)/Inverse,['Definitions/Inverse Elements'],"Let S, ∘ be an algebraic structure with an identity element e_S.

Let x, y ∈ S be elements.


The element y is an inverse of x :
:y ∘ x = e_S = x ∘ y
that is,  y is both:
:a left inverse of x
and:
:a right inverse of x.
Let S, ∘ be a monoid whose identity is e_S.

An element x_L ∈ S is called a left inverse of x :
:x_L ∘ x = e_S
Let S, ∘ be a monoid whose identity is e_S.

An element x_R ∈ S is called a right inverse of x :
:x ∘ x_R = e_S
"
Definition:Inverse,Inverse,"Let $\struct {S, \circ}$ be an algebraic structure with an identity element $e_S$.

Let $x, y \in S$ be elements.


The element $y$ is an inverse of $x$ :
:$y \circ x = e_S = x \circ y$
that is,  $y$ is both:
:a left inverse of $x$
and:
:a right inverse of $x$.
",Definition:Product Inverse,['Definitions/Ring Theory'],"Let S, ∘ be an algebraic structure with an identity element e_S.

Let x, y ∈ S be elements.


The element y is an inverse of x :
:y ∘ x = e_S = x ∘ y
that is,  y is both:
:a left inverse of x
and:
:a right inverse of x.
"
Definition:Inverse,Inverse,"Let $(S, \circ)$ be an inverse semigroup.

Let $a\in S$.


The inverse of $a$ is the unique element $b\in S$ such that:
:$a = a \circ b \circ a$ and $b = b \circ a \circ b$


Category:Definitions/Inverse Semigroups
",Definition:Inverse Semigroup,"['Definitions/Algebraic Structures', 'Definitions/Semigroups', 'Definitions/Inverse Semigroups']","Let (S, ∘) be an inverse semigroup.

Let a∈ S.


The inverse of a is the unique element b∈ S such that:
:a = a ∘ b ∘ a and b = b ∘ a ∘ b


Category:Definitions/Inverse Semigroups
"
Definition:Inverse,Inverse,"An inverse semigroup is a semigroup $\struct {S, \circ}$ such that:

:$\forall a \in S: \exists! b \in S: a = a \circ b \circ a, b = b \circ a \circ b$


=== Inverse ===

",Definition:Inverse in Inverse Semigroup,['Definitions/Inverse Semigroups'],"An inverse semigroup is a semigroup S, ∘ such that:

:∀ a ∈ S: ∃! b ∈ S: a = a ∘ b ∘ a, b = b ∘ a ∘ b


=== Inverse ===

"
Definition:Inverse,Inverse,"Let $\RR \subseteq S \times T$ be a relation.


The inverse relation to (or of) $\RR$ is defined as:

:$\RR^{-1} := \set {\tuple {t, s}: \tuple {s, t} \in \RR}$


That is, $\RR^{-1} \subseteq T \times S$ is the relation which satisfies:

:$\forall s \in S: \forall t \in T: \tuple {t, s} \in \RR^{-1} \iff \tuple {s, t} \in \RR$


=== Class Theoretical Definition ===


",Definition:Inverse Relation,"['Definitions/Inverse Relations', 'Definitions/Relation Theory']","Let ⊆ S × T be a relation.


The inverse relation to (or of)  is defined as:

:^-1 := t, s: s, t∈


That is, ^-1⊆ T × S is the relation which satisfies:

:∀ s ∈ S: ∀ t ∈ T: t, s∈^-1s, t∈


=== Class Theoretical Definition ===


"
Definition:Inverse,Inverse,,Definition:Inverse Image,[],
Definition:Inverse Image,Inverse Image,"Let $S$ and $T$ be sets.

Let $\powerset S$ and $\powerset T$ be their power sets.


=== Relation ===

Let $\RR \subseteq S \times T$ be a relation on $S \times T$.



=== Mapping ===

Let $f: S \to T$ be a mapping.

",Definition:Inverse Image Mapping,"['Definitions/Induced Mappings', 'Definitions/Preimages', 'Definitions/Inverse Image Mappings']","Let S and T be sets.

Let S and T be their power sets.


=== Relation ===

Let ⊆ S × T be a relation on S × T.



=== Mapping ===

Let f: S → T be a mapping.

"
Definition:Inverse Image,Inverse Image,"Let $T_1 = \struct {S_1, \tau_1}$ and $T_2 = \struct {S_2, \tau_2}$ be topological spaces.

Let $f: T_1 \to T_2$ be continuous.

Let $\mathbf C$ be a category which has all small inductive limits.

Let $\FF$ be a $\mathbf C$-valued presheaf on $T_2$.


The inverse image presheaf of $\FF$ via $f$ is the presheaf $f^{-1}_{\operatorname {Psh} } \FF$ on $T_1$ with:
:$\map {\paren {f^{-1}_{\operatorname{Psh} } \FF} } U = \ds \varinjlim_{V \mathop \supseteq \map f U} \map \FF V$ where the inductive limit goes over open $V \subseteq Y$
:$\operatorname {res}^U_W$ is the induced map on the inductive limit of the subset $\set {V: V \supseteq \map f U} \subseteq \set {V : V \supseteq \map f W}$",Definition:Inverse Image Presheaf,['Definitions/Sheaf Theory'],"Let T_1 = S_1, τ_1 and T_2 = S_2, τ_2 be topological spaces.

Let f: T_1 → T_2 be continuous.

Let 𝐂 be a category which has all small inductive limits.

Let  be a 𝐂-valued presheaf on T_2.


The inverse image presheaf of  via f is the presheaf f^-1_Psh on T_1 with:
:f^-1_Psh U = _V ⊇ f U V where the inductive limit goes over open V ⊆ Y
:res^U_W is the induced map on the inductive limit of the subset V: V ⊇ f U⊆V : V ⊇ f W"
Definition:Invertible,Invertible,"Let $\left({S, \circ}\right)$ be an algebraic structure.


The operation $\circ$ is invertible :
:$\forall a, b \in S: \exists r, s \in S: a \circ r = b = s \circ a$
",Definition:Invertible Element,['Definitions/Abstract Algebra'],"Let (S, ∘) be an algebraic structure.


The operation ∘ is invertible :
:∀ a, b ∈ S: ∃ r, s ∈ S: a ∘ r = b = s ∘ a
"
Definition:Invertible,Invertible,"Let $\struct {R, +, \circ}$ be a ring with unity.

Let $n \in \Z_{>0}$ be a (strictly) positive integer.

Let $\mathbf A$ be an element of the ring of square matrices $\struct {\map {\MM_R} n, +, \times}$.


Then $\mathbf A$ is invertible :
:$\exists \mathbf B \in \struct {\map {\MM_R} n, +, \times}: \mathbf A \mathbf B = \mathbf I_n = \mathbf B \mathbf A$
where $\mathbf I_n$ denotes the unit matrix of order $n$.


Such a $\mathbf B$ is the inverse of $\mathbf A$.

It is usually denoted $\mathbf A^{-1}$.
",Definition:Invertible Matrix,"['Definitions/Inverse Matrices', 'Definitions/Matrices']","Let R, +, ∘ be a ring with unity.

Let n ∈_>0 be a (strictly) positive integer.

Let 𝐀 be an element of the ring of square matrices _R n, +, ×.


Then 𝐀 is invertible :
:∃𝐁∈_R n, +, ×: 𝐀𝐁 = 𝐈_n = 𝐁𝐀
where 𝐈_n denotes the unit matrix of order n.


Such a 𝐁 is the inverse of 𝐀.

It is usually denoted 𝐀^-1.
"
Definition:Invertible,Invertible,"=== Normed Vector Space ===
 

=== Inner Product Space ===
",Definition:Invertible Bounded Linear Operator,"['Definitions/Invertible Bounded Linear Operator', 'Definitions/Invertible Bounded Linear Operators', 'Definitions/Bounded Linear Operators', 'Definitions/Invertible Bounded Linear Operators']","=== Normed Vector Space ===
 

=== Inner Product Space ===
"
Definition:Irreducible,Irreducible,"Let $T = \struct {S, \tau}$ be a topological space.

Let $A$ be a subset of $S$.


Then $A$ is irreducible (subset) 
:$A$ is non-empty and closed and for all closed subsets $B, C$ of $S$: $A = B \cup C \implies A = B$ or $A = C$",Definition:Irreducible Subset (Topology),['Definitions/Topology'],"Let T = S, τ be a topological space.

Let A be a subset of S.


Then A is irreducible (subset) 
:A is non-empty and closed and for all closed subsets B, C of S: A = B ∪ C  A = B or A = C"
Definition:Irreducible,Irreducible,"Let $\rho: G \to \GL V$ be a linear representation.

Then $\rho$ is irreducible  it is not reducible.


That is,  there exists no non-trivial proper vector subspace $W$ of $V$ such that:
: $\forall g \in G: \map {\map \rho g} W \subseteq W$


Category:Definitions/Representation Theory",Definition:Irreducible (Representation Theory)/Linear Representation,['Definitions/Representation Theory'],"Let ρ: G → V be a linear representation.

Then ρ is irreducible  it is not reducible.


That is,  there exists no non-trivial proper vector subspace W of V such that:
: ∀ g ∈ G: ρ g W ⊆ W


Category:Definitions/Representation Theory"
Definition:Irreducible,Irreducible,"Let $\left({S, \wedge, \preceq}\right)$ be a meet semilattice.

Let $g \in S$.


Then $g$ is meet irreducible :
:$\forall x, y \in S: g = x \wedge y \implies g = x$ or $g = y$",Definition:Meet Irreducible,['Definitions/Order Theory'],"Let (S, ∧, ≼) be a meet semilattice.

Let g ∈ S.


Then g is meet irreducible :
:∀ x, y ∈ S: g = x ∧ y  g = x or g = y"
Definition:Irreducible,Irreducible,"Let $\struct {S, \preceq}$ be an ordered set.

Let $p \in S$.

An element $p$ is completely irreducible 
:$p^\succeq \setminus \set p$ admits a minimum element
where $p^\succeq$ denotes the upper closure of $p$.",Definition:Completely Irreducible,['Definitions/Order Theory'],"Let S, ≼ be an ordered set.

Let p ∈ S.

An element p is completely irreducible 
:p^≽∖ p admits a minimum element
where p^≽ denotes the upper closure of p."
Definition:Isometry,Isometry,"Let $M_1 = \struct {A_1, d_1}$ and $M_2 = \struct {A_2, d_2}$ be metric spaces or pseudometric spaces.


Let $\phi: A_1 \to A_2$ be an injection such that:
:$\forall a, b \in A_1: \map {d_1} {a, b} = \map {d_2} {\map \phi a, \map \phi b}$


Then $\phi$ is called an isometry (from $M_1$) into $M_2$.


That is, an isometry (from $M_1$) into $M_2$ is an isometry which is not actually a surjection, but satisfies the other conditions for being an isometry.
",Definition:Isometry (Metric Spaces),"['Definitions/Isometries (Metric Spaces)', 'Definitions/Metric Spaces', 'Definitions/Pseudometric Spaces', 'Definitions/Isometries']","Let M_1 = A_1, d_1 and M_2 = A_2, d_2 be metric spaces or pseudometric spaces.


Let ϕ: A_1 → A_2 be an injection such that:
:∀ a, b ∈ A_1: d_1a, b = d_2ϕ a, ϕ b


Then ϕ is called an isometry (from M_1) into M_2.


That is, an isometry (from M_1) into M_2 is an isometry which is not actually a surjection, but satisfies the other conditions for being an isometry.
"
Definition:Isometry,Isometry,"Let $\EE$ be a real Euclidean space.


Let $\phi: \EE \to \EE$ be a bijection such that:
:$\forall P, Q \in \EE: PQ = P'Q'$
where:
:$P$ and $Q$ are arbitrary points in $\EE$
:$P'$ and $Q'$ are the images of $P$ and $Q$ respectively
:$PQ$ and $P'Q'$ denote the lengths of the straight line segments $PQ$ and $P'Q'$ respectively.


Then $\phi$ is an isometry.


That is, an isometry is a bijection which preserves distance between points.


=== Context ===

Let $\EE$ be a real Euclidean space.


Let $\phi: \EE \to \EE$ be a bijection such that:
:$\forall P, Q \in \EE: PQ = P'Q'$
where:
:$P$ and $Q$ are arbitrary points in $\EE$
:$P'$ and $Q'$ are the images of $P$ and $Q$ respectively
:$PQ$ and $P'Q'$ denote the lengths of the straight line segments $PQ$ and $P'Q'$ respectively.


Then $\phi$ is an isometry.


That is, an isometry is a bijection which preserves distance between points.


=== Context ===

",Definition:Isometry (Euclidean Geometry),"['Definitions/Isometries (Euclidean Geometry)', 'Definitions/Euclidean Geometry', 'Definitions/Isometries']","Let  be a real Euclidean space.


Let ϕ: → be a bijection such that:
:∀ P, Q ∈: PQ = P'Q'
where:
:P and Q are arbitrary points in 
:P' and Q' are the images of P and Q respectively
:PQ and P'Q' denote the lengths of the straight line segments PQ and P'Q' respectively.


Then ϕ is an isometry.


That is, an isometry is a bijection which preserves distance between points.


=== Context ===

Let  be a real Euclidean space.


Let ϕ: → be a bijection such that:
:∀ P, Q ∈: PQ = P'Q'
where:
:P and Q are arbitrary points in 
:P' and Q' are the images of P and Q respectively
:PQ and P'Q' denote the lengths of the straight line segments PQ and P'Q' respectively.


Then ϕ is an isometry.


That is, an isometry is a bijection which preserves distance between points.


=== Context ===

"
Definition:Isometry,Isometry,"Let $V$ and $W$ be inner product spaces.

Let their inner products be $\innerprod \cdot \cdot_V$ and $\innerprod \cdot \cdot_W$ respectively.

Let the mapping $F : V \to W$ be a vector space isomorphism that preserves inner products:

:$\forall v_1, v_2 \in V : \innerprod {v_1} {v_2}_V = \innerprod {\map F {v_1}} {\map F {v_2}}_W$


Then $F$ is called a (linear) isometry.


=== Hilbert Spaces ===

",Definition:Isometry (Inner Product Spaces),"['Definitions/Isometries (Inner Product Spaces)', 'Definitions/Inner Product Spaces', 'Definitions/Isometries']","Let V and W be inner product spaces.

Let their inner products be ··_V and ··_W respectively.

Let the mapping F : V → W be a vector space isomorphism that preserves inner products:

:∀ v_1, v_2 ∈ V : v_1v_2_V =  F v_1 F v_2_W


Then F is called a (linear) isometry.


=== Hilbert Spaces ===

"
Definition:Isometry,Isometry,"Let $\struct {M, g}$ and $\struct {\tilde M, \tilde g}$ be Riemannian manifolds with Riemannian metrics $g$ and $\tilde g$ respectively.

Let the mapping $\phi : M \to \tilde M$ be a diffeomorphism such that:

:$\phi^* \tilde g = g$


Then $\phi$ is called an isometry from $\struct {M, g}$ to $\struct {\tilde M, \tilde g}$.
",Definition:Isometry (Riemannian Manifolds),"['Definitions/Isometries (Riemannian Manifolds)', 'Definitions/Riemannian Manifolds', 'Definitions/Isometries']","Let M, g and M̃, g̃ be Riemannian manifolds with Riemannian metrics g and g̃ respectively.

Let the mapping ϕ : M →M̃ be a diffeomorphism such that:

:ϕ^* g̃ = g


Then ϕ is called an isometry from M, g to M̃, g̃.
"
Definition:Isomorphism,Isomorphism,"Let $\struct {G, \circ}$ and $\struct {H, *}$ be groups.

Let $\phi: G \to H$ be a (group) homomorphism.


Then $\phi$ is a group isomorphism  $\phi$ is a bijection.


That is, $\phi$ is a group isomorphism  $\phi$ is both a monomorphism and an epimorphism.


If $G$ is isomorphic to $H$, then the notation $G \cong H$ can be used (although notation varies).
Let $\struct {G, \circ}$ and $\struct {H, *}$ be groups.

Let $\phi: G \to H$ be a (group) homomorphism.


Then $\phi$ is a group isomorphism  $\phi$ is a bijection.


That is, $\phi$ is a group isomorphism  $\phi$ is both a monomorphism and an epimorphism.


If $G$ is isomorphic to $H$, then the notation $G \cong H$ can be used (although notation varies).
",Definition:Isomorphism (Abstract Algebra)/Group Isomorphism,['Definitions/Group Homomorphisms'],"Let G, ∘ and H, * be groups.

Let ϕ: G → H be a (group) homomorphism.


Then ϕ is a group isomorphism  ϕ is a bijection.


That is, ϕ is a group isomorphism  ϕ is both a monomorphism and an epimorphism.


If G is isomorphic to H, then the notation G ≅ H can be used (although notation varies).
Let G, ∘ and H, * be groups.

Let ϕ: G → H be a (group) homomorphism.


Then ϕ is a group isomorphism  ϕ is a bijection.


That is, ϕ is a group isomorphism  ϕ is both a monomorphism and an epimorphism.


If G is isomorphic to H, then the notation G ≅ H can be used (although notation varies).
"
Definition:Isomorphism,Isomorphism,"Let $\struct {R, +, \circ}$ and $\struct {S, \oplus, *}$ be rings.

Let $\phi: R \to S$ be a (ring) homomorphism.


Then $\phi$ is a ring isomorphism  $\phi$ is a bijection.

That is, $\phi$ is a ring isomorphism  $\phi$ is both a monomorphism and an epimorphism.
Let $\struct {R, +, \circ}$ and $\struct {S, \oplus, *}$ be rings.

Let $\phi: R \to S$ be a (ring) homomorphism.


Then $\phi$ is a ring isomorphism  $\phi$ is a bijection.

That is, $\phi$ is a ring isomorphism  $\phi$ is both a monomorphism and an epimorphism.
",Definition:Isomorphism (Abstract Algebra)/Ring Isomorphism,"['Definitions/Isomorphisms (Abstract Algebra)', 'Definitions/Ring Homomorphisms']","Let R, +, ∘ and S, ⊕, * be rings.

Let ϕ: R → S be a (ring) homomorphism.


Then ϕ is a ring isomorphism  ϕ is a bijection.

That is, ϕ is a ring isomorphism  ϕ is both a monomorphism and an epimorphism.
Let R, +, ∘ and S, ⊕, * be rings.

Let ϕ: R → S be a (ring) homomorphism.


Then ϕ is a ring isomorphism  ϕ is a bijection.

That is, ϕ is a ring isomorphism  ϕ is both a monomorphism and an epimorphism.
"
Definition:Isomorphism,Isomorphism,"Let $\struct {S_1, \RR_1}$ and $\struct {S_2, \RR_2}$ be relational structures.

Let there exist a bijection $\phi: S_1 \to S_2$ such that:
:$(1): \quad \forall \tuple {s_1, t_1} \in \RR_1: \tuple {\map \phi {s_1}, \map \phi {t_1} } \in \RR_2$
:$(2): \quad \forall \tuple {s_2, t_2} \in \RR_2: \tuple {\map {\phi^{-1} } {s_2}, \map {\phi^{-1} } {t_2} } \in \RR_1$


Then $\struct {S_1, \RR_1}$ and $\struct {S_2, \RR_2}$ are isomorphic, and this is denoted $S_1 \cong S_2$.


The function $\phi$ is called a relation isomorphism, or just an isomorphism, from $\struct {S_1, \RR_1}$ to $\struct {S_2, \RR_2}$.",Definition:Relation Isomorphism,"['Definitions/Relational Structures', 'Definitions/Mapping Theory']","Let S_1, _1 and S_2, _2 be relational structures.

Let there exist a bijection ϕ: S_1 → S_2 such that:
:(1):   ∀s_1, t_1∈_1: ϕs_1, ϕt_1∈_2
:(2):   ∀s_2, t_2∈_2: ϕ^-1s_2, ϕ^-1t_2∈_1


Then S_1, _1 and S_2, _2 are isomorphic, and this is denoted S_1 ≅ S_2.


The function ϕ is called a relation isomorphism, or just an isomorphism, from S_1, _1 to S_2, _2."
Definition:Isomorphism,Isomorphism,"Let $\mathbf C$ be a category, and let $X, Y$ be objects of $\mathbf C$.


A morphism $f: X \to Y$ is an isomorphism  there exists a morphism $g: Y \to X$ such that:

:$g \circ f = I_X$
:$f \circ g = I_Y$

where $I_X$ denotes the identity morphism on $X$.

It can be seen that this is equivalent to $g$ being both a retraction and a section of $f$.


=== Inverse Morphism ===
",Definition:Isomorphism (Category Theory),['Definitions/Category Theory'],"Let 𝐂 be a category, and let X, Y be objects of 𝐂.


A morphism f: X → Y is an isomorphism  there exists a morphism g: Y → X such that:

:g ∘ f = I_X
:f ∘ g = I_Y

where I_X denotes the identity morphism on X.

It can be seen that this is equivalent to g being both a retraction and a section of f.


=== Inverse Morphism ===
"
Definition:Isomorphism,Isomorphism,"Let $\mathbf C$ and $\mathbf D$ be metacategories.

Let $F: \mathbf C \to \mathbf D$ be a functor.


Then $F$ is an isomorphism (of categories)  there exists a functor $G: \mathbf C \to \mathbf D$ such that:

:$G F: \mathbf C \to \mathbf C$ is the identity functor $I_{\mathbf C}$
:$F G: \mathbf D \to \mathbf D$ is the identity functor $I_{\mathbf D}$


=== Isomorphic Categories ===
",Definition:Isomorphism of Categories,"['Definitions/Category Theory', 'Definitions/Isomorphisms']","Let 𝐂 and 𝐃 be metacategories.

Let F: 𝐂→𝐃 be a functor.


Then F is an isomorphism (of categories)  there exists a functor G: 𝐂→𝐃 such that:

:G F: 𝐂→𝐂 is the identity functor I_𝐂
:F G: 𝐃→𝐃 is the identity functor I_𝐃


=== Isomorphic Categories ===
"
Definition:Isomorphism,Isomorphism,"Let $G = \struct {\map V G, \map E G}$ and $H = \struct {\map V H, \map E H}$ be graphs.

Let there exist a bijection $F: \map V G \to \map V H$ such that:
:for each edge $\set {u, v} \in \map E G$, there exists an edge $\set {\map F u, \map F v} \in \map E H$.


That is, that:
:$F: \map V G \to \map V H$ is a homomorphism, and

:$F^{-1}: \map V H \to \map V G$ is a homomorphism.


Then $G$ and $H$ are isomorphic, and this is denoted $G \cong H$.

The function $F$ is called an isomorphism from $G$ to $H$.",Definition:Isomorphism (Graph Theory),['Definitions/Graph Theory'],"Let G =  V G,  E G and H =  V H,  E H be graphs.

Let there exist a bijection F:  V G → V H such that:
:for each edge u, v∈ E G, there exists an edge F u,  F v∈ E H.


That is, that:
:F:  V G → V H is a homomorphism, and

:F^-1:  V H → V G is a homomorphism.


Then G and H are isomorphic, and this is denoted G ≅ H.

The function F is called an isomorphism from G to H."
Definition:Isomorphism,Isomorphism,"Let $H, K$ be Hilbert spaces.

Denote by $\innerprod \cdot \cdot_H$ and $\innerprod \cdot \cdot_K$ their respective inner products.

An isomorphism between $H$ and $K$ is a map $U: H \to K$, such that:

:$(1): \quad U$ is a linear map
:$(2): \quad U$ is surjective
:$(3): \quad \forall g, h \in H: \innerprod g h_H = \innerprod {U g} {U h}_K$

These three requirements may be summarized by stating that $U$ be a surjective isometry.

Furthermore, Surjection that Preserves Inner Product is Linear shows that requirement $(1)$ is superfluous.


If such an isomorphism $U$ exists, $H$ and $K$ are said to be isomorphic.


As the name isomorphism suggests, Hilbert Space Isomorphism is Equivalence Relation.",Definition:Isomorphism (Hilbert Spaces),['Definitions/Hilbert Spaces'],"Let H, K be Hilbert spaces.

Denote by ··_H and ··_K their respective inner products.

An isomorphism between H and K is a map U: H → K, such that:

:(1):    U is a linear map
:(2):    U is surjective
:(3):   ∀ g, h ∈ H:  g h_H = U gU h_K

These three requirements may be summarized by stating that U be a surjective isometry.

Furthermore, Surjection that Preserves Inner Product is Linear shows that requirement (1) is superfluous.


If such an isomorphism U exists, H and K are said to be isomorphic.


As the name isomorphism suggests, Hilbert Space Isomorphism is Equivalence Relation."
Definition:Isotropy,Isotropy,"Isotropy is a property of a physical system such that measurements of that system are independent of the direction in which those measurements are taken.

",Definition:Isotropy (Physics),['Definitions/Physics'],"Isotropy is a property of a physical system such that measurements of that system are independent of the direction in which those measurements are taken.

"
Definition:Isotropy,Isotropy,,Definition:Isotropy Representation,['Definitions/Riemannian Geometry'],
Definition:Join,Join,"Let $\struct {G, \circ}$ be a group.

Let $A$ and $B$ be subgroups of $G$.


The join of $A$ and $B$ is written and defined as:
:$A \vee B := \gen {A \cup B}$
where $\gen {A \cup B}$ is the subgroup generated by $A \cup B$.


By the definition of subgroup generator, this can alternatively be written:

:$\ds A \vee B := \bigcap \set {T: T \text { is a subgroup of } G: A \cup B \subseteq T}$


=== General Definition ===
",Definition:Join of Subgroups,['Definitions/Group Theory'],"Let G, ∘ be a group.

Let A and B be subgroups of G.


The join of A and B is written and defined as:
:A ∨ B := A ∪ B
where A ∪ B is the subgroup generated by A ∪ B.


By the definition of subgroup generator, this can alternatively be written:

:A ∨ B := ⋂T: T  is a subgroup of  G: A ∪ B ⊆ T


=== General Definition ===
"
Definition:Join,Join,"Consider the Boolean algebra $\struct {S, \vee, \wedge, \neg}$


The operation $\vee$ is called join.
",Definition:Boolean Algebra/Join,['Definitions/Boolean Algebras'],"Consider the Boolean algebra S, ∨, ∧,


The operation ∨ is called join.
"
Definition:Join,Join,"Let $\struct {S, \preceq}$ be an ordered set.

Let $a, b \in S$.

Let their supremum $\sup \set {a, b}$ exist in $S$.


Then the join of $a$ and $b$ is defined as:

:$a \vee b = \sup \set {a, b}$


Expanding the definition of supremum, one sees that $c = a \vee b$ :

:$(1): \quad a \preceq c$ and $b \preceq c$
:$(2): \quad \forall s \in S: a \preceq s$ and $b \preceq s \implies c \preceq s$",Definition:Join (Order Theory),"['Definitions/Order Theory', 'Definitions/Lattice Theory']","Let S, ≼ be an ordered set.

Let a, b ∈ S.

Let their supremum supa, b exist in S.


Then the join of a and b is defined as:

:a ∨ b = supa, b


Expanding the definition of supremum, one sees that c = a ∨ b :

:(1):    a ≼ c and b ≼ c
:(2):   ∀ s ∈ S: a ≼ s and b ≼ s  c ≼ s"
Definition:Join,Join,"Let $G = \struct {V, E}$ be a graph.

Let $u$ and $v$ be vertices of $G$.

Let $e = u v$ be an edge of $G$.


Then $e$ joins the vertices $u$ and $v$.",Definition:Graph (Graph Theory)/Edge/Join,"['Definitions/Edges of Graphs', 'Definitions/Vertices of Graphs']","Let G = V, E be a graph.

Let u and v be vertices of G.

Let e = u v be an edge of G.


Then e joins the vertices u and v."
Definition:Kernel,Kernel,"Let $\struct {S, \circ}$ be a magma.

Let $\struct {T, *}$ be an algebraic structure with an identity element $e$.

Let $\phi: \struct {S, \circ} \to \struct {T, *}$ be a homomorphism.


The kernel of $\phi$ is the subset of the domain of $\phi$ defined as:
:$\map \ker \phi = \set {x \in S: \map \phi x = e}$


That is, $\map \ker \phi$ is the subset of $S$ that maps to the identity of $T$.
Let $\struct {G, \circ}$ and $\struct {H, *}$ be groups.

Let $\phi: \struct {G, \circ} \to \struct {H, *}$ be a group homomorphism.


The kernel of $\phi$ is the subset of the domain of $\phi$ defined as:
:$\map \ker \phi := \phi^{-1} \sqbrk {e_H} = \set {x \in G: \map \phi x = e_H}$
where $e_H$ is the identity of $H$.


That is, $\map \ker \phi$ is the subset of $G$ that maps to the identity of $H$.
Let $\struct {R_1, +_1, \circ_1}$ and $\struct {R_2, +_2, \circ_2}$ be rings.

Let $\phi: \struct {R_1, +_1, \circ_1} \to \struct {R_2, +_2, \circ_2}$ be a ring homomorphism.


The kernel of $\phi$ is the subset of the domain of $\phi$ defined as:
:$\map \ker \phi = \set {x \in R_1: \map \phi x = 0_{R_2} }$
where $0_{R_2}$ is the zero of $R_2$.


That is, $\map \ker \phi$ is the subset of $R_1$ that maps to the zero of $R_2$.


From Ring Homomorphism Preserves Zero it follows that $0_{R_1} \in \map \ker \phi$ where $0_{R_1}$ is the zero of $R_1$.
Let $\phi: G \to H$ be a linear transformation where $G$ and $H$ are $R$-modules.

Let $e_H$ be the identity of $H$.


The kernel of $\phi$ is defined as:

:$\map \ker \phi := \phi^{-1} \sqbrk {\set {e_H} }$

where $\phi^{-1} \sqbrk S$ denotes the preimage of $S$ under $\phi$.


=== In Vector Space ===

",Definition:Kernel (Abstract Algebra),"['Definitions/Kernels (Abstract Algebra)', 'Definitions/Abstract Algebra']","Let S, ∘ be a magma.

Let T, * be an algebraic structure with an identity element e.

Let ϕ: S, ∘→T, * be a homomorphism.


The kernel of ϕ is the subset of the domain of ϕ defined as:
:ϕ = x ∈ S: ϕ x = e


That is, ϕ is the subset of S that maps to the identity of T.
Let G, ∘ and H, * be groups.

Let ϕ: G, ∘→H, * be a group homomorphism.


The kernel of ϕ is the subset of the domain of ϕ defined as:
:ϕ := ϕ^-1e_H = x ∈ G: ϕ x = e_H
where e_H is the identity of H.


That is, ϕ is the subset of G that maps to the identity of H.
Let R_1, +_1, ∘_1 and R_2, +_2, ∘_2 be rings.

Let ϕ: R_1, +_1, ∘_1→R_2, +_2, ∘_2 be a ring homomorphism.


The kernel of ϕ is the subset of the domain of ϕ defined as:
:ϕ = x ∈ R_1: ϕ x = 0_R_2
where 0_R_2 is the zero of R_2.


That is, ϕ is the subset of R_1 that maps to the zero of R_2.


From Ring Homomorphism Preserves Zero it follows that 0_R_1∈ϕ where 0_R_1 is the zero of R_1.
Let ϕ: G → H be a linear transformation where G and H are R-modules.

Let e_H be the identity of H.


The kernel of ϕ is defined as:

:ϕ := ϕ^-1e_H

where ϕ^-1 S denotes the preimage of S under ϕ.


=== In Vector Space ===

"
Definition:Kernel,Kernel,"Let $\phi: G \to H$ be a linear transformation where $G$ and $H$ are $R$-modules.

Let $e_H$ be the identity of $H$.


The kernel of $\phi$ is defined as:

:$\map \ker \phi := \phi^{-1} \sqbrk {\set {e_H} }$

where $\phi^{-1} \sqbrk S$ denotes the preimage of $S$ under $\phi$.


=== In Vector Space ===

",Definition:Kernel of Linear Transformation,"['Definitions/Kernels of Linear Transformations', 'Definitions/Linear Algebra', 'Definitions/Kernels (Abstract Algebra)']","Let ϕ: G → H be a linear transformation where G and H are R-modules.

Let e_H be the identity of H.


The kernel of ϕ is defined as:

:ϕ := ϕ^-1e_H

where ϕ^-1 S denotes the preimage of S under ϕ.


=== In Vector Space ===

"
Definition:Kernel,Kernel,"Let $\phi: G \to H$ be a linear transformation where $G$ and $H$ are $R$-modules.

Let $e_H$ be the identity of $H$.


The kernel of $\phi$ is defined as:

:$\map \ker \phi := \phi^{-1} \sqbrk {\set {e_H} }$

where $\phi^{-1} \sqbrk S$ denotes the preimage of $S$ under $\phi$.


=== In Vector Space ===

Let $\struct {R, +, \cdot}$ be a ring.

Let:
:$M: \quad \cdots \longrightarrow M_i \stackrel {d_i} \longrightarrow M_{i + 1} \stackrel {d_{i + 1} } \longrightarrow M_{i + 2} \stackrel {d_{i + 2} } \longrightarrow \cdots$
and
:$N: \quad \cdots \longrightarrow N_i \stackrel {d'_i} \longrightarrow N_{i + 1} \stackrel {d'_{i + 1} } \longrightarrow N_{i + 2} \stackrel {d'_{i + 2} } \longrightarrow \cdots$
be two differential complexes of $R$-modules.

Let $\phi = \set {\phi_i : i \in \Z}$ be a homomorphism $M \to N$.

For each $i \in \Z$ let $K_i$ be the kernel of $\phi_i$.

For each $i \in \Z$ let $f_i$ be the restriction of $d_i$ to $K_i$.


Then the kernel of $\phi$ is:

:$\ker \phi : \quad \cdots \longrightarrow K_i \stackrel {f_i} \longrightarrow K_{i + 1} \stackrel {f_{i + 1} } \longrightarrow K_{i + 2} \stackrel {f_{i + 2} } \longrightarrow \cdots$
",Definition:Kernel of Homomorphism of Differential Complexes,['Definitions/Homological Algebra'],"Let ϕ: G → H be a linear transformation where G and H are R-modules.

Let e_H be the identity of H.


The kernel of ϕ is defined as:

:ϕ := ϕ^-1e_H

where ϕ^-1 S denotes the preimage of S under ϕ.


=== In Vector Space ===

Let R, +, · be a ring.

Let:
:M:   ⋯⟶ M_i d_i⟶ M_i + 1d_i + 1⟶ M_i + 2d_i + 2⟶⋯
and
:N:   ⋯⟶ N_i d'_i⟶ N_i + 1d'_i + 1⟶ N_i + 2d'_i + 2⟶⋯
be two differential complexes of R-modules.

Let ϕ = ϕ_i : i ∈ be a homomorphism M → N.

For each i ∈ let K_i be the kernel of ϕ_i.

For each i ∈ let f_i be the restriction of d_i to K_i.


Then the kernel of ϕ is:

:ϕ :   ⋯⟶ K_i f_i⟶ K_i + 1f_i + 1⟶ K_i + 2f_i + 2⟶⋯
"
Definition:Kernel,Kernel,"Let $\map F p$ be an integral transform:

:$\map F p = \ds \int_a^b \map f x \map K {p, x} \rd x$


The function $\map K {p, x}$ is the kernel of $\map F p$.
",Definition:Integral Transform/Kernel,"['Definitions/Integral Transforms', 'Definitions/Kernels']","Let F p be an integral transform:

:F p = ∫_a^b  f x  K p, x x


The function K p, x is the kernel of F p.
"
Definition:Kernel,Kernel,"Consider the integral equation:

:of the first kind:
::$\map f x = \lambda \ds \int_{\map a x}^{\map b x} \map K {x, y} \map g y \rd x$

:of the second kind:
::$\map g x = \map f x + \lambda \ds \int_{\map a x}^{\map b x} \map K {x, y} \map g y \rd x$

:of the third kind:
::$\map u x \map g x = \map f x + \lambda \ds \int_{\map a x}^{\map b x} \map K {x, y} \map g y \rd x$


The function $\map K {x, y}$ is known as the kernel of the integral equation.
",Definition:Integral Equation/Kernel,"['Definitions/Integral Equations', 'Definitions/Kernels']","Consider the integral equation:

:of the first kind:
::f x = λ∫_ a x^ b x K x, y g y  x

:of the second kind:
::g x =  f x + λ∫_ a x^ b x K x, y g y  x

:of the third kind:
::u x  g x =  f x + λ∫_ a x^ b x K x, y g y  x


The function K x, y is known as the kernel of the integral equation.
"
Definition:Kernel,Kernel,"Let $\struct {X, \Sigma}$ be a measurable space.

Let $\overline \R_{\ge 0}$ be the set of positive extended real numbers.


A kernel is a mapping $N: X \times \Sigma \to \overline{\R}_{\ge0}$ such that:

:$(1): \quad \forall x \in X: N_x: \Sigma \to \overline \R_{\ge 0}, E \mapsto \map N {x, E}$ is a measure
:$(2): \quad \forall E \in \Sigma: N_E: X \to \overline \R_{\ge 0}, x \mapsto \map N {x, E}$ is a positive $\Sigma$-measurable function

",Definition:Kernel (Measure Theory),['Definitions/Measure Theory'],"Let X, Σ be a measurable space.

Let _≥ 0 be the set of positive extended real numbers.


A kernel is a mapping N: X ×Σ→_≥0 such that:

:(1):   ∀ x ∈ X: N_x: Σ→_≥ 0, E ↦ N x, E is a measure
:(2):   ∀ E ∈Σ: N_E: X →_≥ 0, x ↦ N x, E is a positive Σ-measurable function

"
Definition:Kernel,Kernel,"Let $\struct {X, \Sigma}$ be a measurable space.

Let $\overline \R_{\ge 0}$ be the set of positive extended real numbers.


A kernel is a mapping $N: X \times \Sigma \to \overline{\R}_{\ge0}$ such that:

:$(1): \quad \forall x \in X: N_x: \Sigma \to \overline \R_{\ge 0}, E \mapsto \map N {x, E}$ is a measure
:$(2): \quad \forall E \in \Sigma: N_E: X \to \overline \R_{\ge 0}, x \mapsto \map N {x, E}$ is a positive $\Sigma$-measurable function


",Definition:Kernel Transformation of Measure,['Definitions/Measure Theory'],"Let X, Σ be a measurable space.

Let _≥ 0 be the set of positive extended real numbers.


A kernel is a mapping N: X ×Σ→_≥0 such that:

:(1):   ∀ x ∈ X: N_x: Σ→_≥ 0, E ↦ N x, E is a measure
:(2):   ∀ E ∈Σ: N_E: X →_≥ 0, x ↦ N x, E is a positive Σ-measurable function


"
Definition:Knot,Knot,"The knot is a unit of speed which is used for air and sea navigation.

It is defined as $1$ nautical mile per hour.

It is now defined as exactly $1 \, 852$ metres per hour.


=== Conversion Factors ===

",Definition:Knot (Unit of Measurement),"['Definitions/Knot (Unit of Measurement)', 'Definitions/Velocity', 'Definitions/Units of Measurement']","The knot is a unit of speed which is used for air and sea navigation.

It is defined as 1 nautical mile per hour.

It is now defined as exactly 1   852 metres per hour.


=== Conversion Factors ===

"
Definition:Knot,Knot,"Let $\closedint a b$ be a closed real interval.

Let $T := \set {a = t_0, t_1, t_2, \ldots, t_{n - 1}, t_n = b}$ form a subdivision of $\closedint a b$.

Let $S: \closedint a b \to \R$ be a spline function on $\closedint a b$ on $T$.


The ordered $n + 1$-tuple $\mathbf t := \tuple {t_0, t_1, t_2, \ldots, t_{n - 1}, t_n}$ of $S$ is known as the knot vector.


Category:Definitions/Knots of Splines
",Definition:Spline Function/Knot,"['Definitions/Knots of Splines', 'Definitions/Splines']","Let a b be a closed real interval.

Let T := a = t_0, t_1, t_2, …, t_n - 1, t_n = b form a subdivision of a b.

Let S:  a b → be a spline function on a b on T.


The ordered n + 1-tuple 𝐭 := t_0, t_1, t_2, …, t_n - 1, t_n of S is known as the knot vector.


Category:Definitions/Knots of Splines
"
Definition:Leading Coefficient,Leading Coefficient,"Let $R$ be a commutative ring with unity.

Let $P \in R \sqbrk X$ be a nonzero polynomial over $R$.

Let $n$ be the degree of $P$.


The leading coefficient of $P$ is the coefficient of $x^n$ in $P$.



Let $\struct {R, +, \circ}$ be a ring.

Let $\struct {S, +, \circ}$ be a subring of $R$.

Let $\ds f = \sum_{k \mathop = 0}^n a_k \circ x^k$ be a polynomial in $x$ over $S$.


The coefficient $a_n \ne 0_R$ is called the leading coefficient of $f$.


=== Polynomial Form ===

",Definition:Leading Coefficient of Polynomial,['Definitions/Polynomial Theory'],"Let R be a commutative ring with unity.

Let P ∈ R  X be a nonzero polynomial over R.

Let n be the degree of P.


The leading coefficient of P is the coefficient of x^n in P.



Let R, +, ∘ be a ring.

Let S, +, ∘ be a subring of R.

Let f = ∑_k  = 0^n a_k ∘ x^k be a polynomial in x over S.


The coefficient a_n  0_R is called the leading coefficient of f.


=== Polynomial Form ===

"
Definition:Leading Coefficient,Leading Coefficient,"Let $\mathbf A = \sqbrk a_{m n}$ be an $m \times n$ matrix.

The leading coefficient of each row of $\mathbf A$ is the leftmost non-zero element of that row.


A zero row has no leading coefficient.",Definition:Leading Coefficient of Matrix,['Definitions/Matrix Theory'],"Let 𝐀 =  a_m n be an m × n matrix.

The leading coefficient of each row of 𝐀 is the leftmost non-zero element of that row.


A zero row has no leading coefficient."
Definition:Left,Left,"The direction left is that way:
:$\gets$",Definition:Left (Direction),['Definitions/Language Definitions'],"The direction left is that way:
:"
Definition:Left,Left,"In an equation:
:$\text {Expression $1$} = \text {Expression $2$}$
the term $\text {Expression $1$}$ is the left hand side.",Definition:Left Hand Side,['Definitions/Language Definitions'],"In an equation:
:Expression 1 = Expression 2
the term Expression 1 is the left hand side."
Definition:Left,Left,"Let $\RR \subseteq S \times T$ be a relation.


Then $\RR$ is left-total :
:$\forall s \in S: \exists t \in T: \tuple {s, t} \in \RR$


That is,  every element of $S$ relates to some element of $T$.",Definition:Left-Total Relation,"['Definitions/Relation Theory', 'Definitions/Left-Total Relations']","Let ⊆ S × T be a relation.


Then  is left-total :
:∀ s ∈ S: ∃ t ∈ T: s, t∈


That is,  every element of S relates to some element of T."
Definition:Left,Left,"Let $\RR \subseteq S \times S$ be a relation in $S$.


$\RR$ is left-Euclidean :

:$\tuple {x, z} \in \RR \land \tuple {y, z} \in \RR \implies \tuple {x, y} \in \RR$",Definition:Euclidean Relation/Left-Euclidean,['Definitions/Euclidean Relations'],"Let ⊆ S × S be a relation in S.


 is left-Euclidean :

:x, z∈y, z∈x, y∈"
Definition:Left,Left,"Let $A$ be a class.

Let $\RR$ be a relation on $A$.


An element $x$ of $A$ is left normal with respect to $\RR$ :
:$\forall y \in A: \map \RR {x, y}$ holds.",Definition:Left Normal Element of Relation,['Definitions/Relations'],"Let A be a class.

Let  be a relation on A.


An element x of A is left normal with respect to  :
:∀ y ∈ A: x, y holds."
Definition:Left,Left,"A mapping $f: Y \to Z$ is left cancellable (or left-cancellable) :

:$\forall X: \forall \struct {g_1, g_2: X \to Y}: f \circ g_1 = f \circ g_2 \implies g_1 = g_2$

That is, for any set $X$, if $g_1$ and $g_2$ are mappings from $X$ to $Y$:
:If $f \circ g_1 = f \circ g_2$
:then $g_1 = g_2$.",Definition:Left Cancellable Mapping,"['Definitions/Mapping Theory', 'Definitions/Cancellability']","A mapping f: Y → Z is left cancellable (or left-cancellable) :

:∀ X: ∀g_1, g_2: X → Y: f ∘ g_1 = f ∘ g_2  g_1 = g_2

That is, for any set X, if g_1 and g_2 are mappings from X to Y:
:If f ∘ g_1 = f ∘ g_2
:then g_1 = g_2."
Definition:Left,Left,"Let $S, T$ be sets where $S \ne \O$, that is, $S$ is not empty.

Let $f: S \to T$ be a mapping.


Let $g: T \to S$ be a mapping such that:
:$g \circ f = I_S$
where:
:$g \circ f$ denotes the composite mapping $f$ followed by $g$;
:$I_S$ is the identity mapping on $S$.


Then $g: T \to S$ is called a left inverse (mapping).",Definition:Left Inverse Mapping,['Definitions/Inverse Mappings'],"Let S, T be sets where S Ø, that is, S is not empty.

Let f: S → T be a mapping.


Let g: T → S be a mapping such that:
:g ∘ f = I_S
where:
:g ∘ f denotes the composite mapping f followed by g;
:I_S is the identity mapping on S.


Then g: T → S is called a left inverse (mapping)."
Definition:Left,Left,"Let $\struct {S, \preccurlyeq}$ be an ordered set.

Let $a, b \in S$.


The left half-open interval between $a$ and $b$ is the set:

:$\hointl a b := a^\succ \cap b^\preccurlyeq = \set {s \in S: \paren {a \prec s} \land \paren {s \preccurlyeq b} }$

where:
:$a^\succ$ denotes the strict upper closure of $a$
:$b^\preccurlyeq$ denotes the lower closure of $b$.
",Definition:Interval/Ordered Set/Left Half-Open,['Definitions/Intervals'],"Let S, ≼ be an ordered set.

Let a, b ∈ S.


The left half-open interval between a and b is the set:

:a b := a^≻∩ b^≼ = s ∈ S: a ≺ ss ≼ b

where:
:a^≻ denotes the strict upper closure of a
:b^≼ denotes the lower closure of b.
"
Definition:Left,Left,"Let $\openint a b$ be an open real interval.

Let $f: \openint a b \to \R$ be a real function.

Let $L \in \R$.


Suppose that:
:$\forall \epsilon \in \R_{>0}: \exists \delta \in \R_{>0}: \forall x \in \R: b - \delta < x < b \implies \size {\map f x - L} < \epsilon$

where $\R_{>0}$ denotes the set of strictly positive real numbers.

That is, for every real strictly positive $\epsilon$ there exists a real strictly positive $\delta$ such that every real number in the domain of $f$, less than $b$ but within $\delta$ of $b$, has an image within $\epsilon$ of $L$.


:

Then $\map f x$ is said to tend to the limit $L$ as $x$ tends to $b$ from the left, and we write:
:$\map f x \to L$ as $x \to b^-$
or
:$\ds \lim_{x \mathop \to b^-} \map f x = L$


This is voiced:
:the limit of $\map f x$ as $x$ tends to $b$ from the left
and such an $L$ is called:
:a limit from the left.
",Definition:Left-Hand Derivative,['Definitions/Derivatives'],"Let a b be an open real interval.

Let f:  a b → be a real function.

Let L ∈.


Suppose that:
:∀ϵ∈_>0: ∃δ∈_>0: ∀ x ∈: b - δ < x < b  f x - L < ϵ

where _>0 denotes the set of strictly positive real numbers.

That is, for every real strictly positive ϵ there exists a real strictly positive δ such that every real number in the domain of f, less than b but within δ of b, has an image within ϵ of L.


:

Then f x is said to tend to the limit L as x tends to b from the left, and we write:
:f x → L as x → b^-
or
:lim_x → b^- f x = L


This is voiced:
:the limit of f x as x tends to b from the left
and such an L is called:
:a limit from the left.
"
Definition:Left,Left,"Let $V$ be a vector space over the real numbers $\R$.

Let $f: \R \to V$ be a function.


A left difference quotient is an expression of the form:
:$\dfrac {\map f {x + h} - \map f x} h$
where $h < 0$ is a strictly negative real number.",Definition:Difference Quotient/Left,['Definitions/Difference Quotients'],"Let V be a vector space over the real numbers .

Let f: → V be a function.


A left difference quotient is an expression of the form:
:f x + h -  f x h
where h < 0 is a strictly negative real number."
Definition:Left,Left,"Let $\openint a b$ be an open real interval.

Let $f: \openint a b \to \R$ be a real function.

Let $L \in \R$.


Suppose that:
:$\forall \epsilon \in \R_{>0}: \exists \delta \in \R_{>0}: \forall x \in \R: b - \delta < x < b \implies \size {\map f x - L} < \epsilon$

where $\R_{>0}$ denotes the set of strictly positive real numbers.

That is, for every real strictly positive $\epsilon$ there exists a real strictly positive $\delta$ such that every real number in the domain of $f$, less than $b$ but within $\delta$ of $b$, has an image within $\epsilon$ of $L$.


:

Then $\map f x$ is said to tend to the limit $L$ as $x$ tends to $b$ from the left, and we write:
:$\map f x \to L$ as $x \to b^-$
or
:$\ds \lim_{x \mathop \to b^-} \map f x = L$


This is voiced:
:the limit of $\map f x$ as $x$ tends to $b$ from the left
and such an $L$ is called:
:a limit from the left.
Let $\openint a b$ be an open real interval.

Let $f: \openint a b \to \R$ be a real function.

Let $L \in \R$.


Suppose that:
:$\forall \epsilon \in \R_{>0}: \exists \delta \in \R_{>0}: \forall x \in \R: b - \delta < x < b \implies \size {\map f x - L} < \epsilon$

where $\R_{>0}$ denotes the set of strictly positive real numbers.

That is, for every real strictly positive $\epsilon$ there exists a real strictly positive $\delta$ such that every real number in the domain of $f$, less than $b$ but within $\delta$ of $b$, has an image within $\epsilon$ of $L$.


:

Then $\map f x$ is said to tend to the limit $L$ as $x$ tends to $b$ from the left, and we write:
:$\map f x \to L$ as $x \to b^-$
or
:$\ds \lim_{x \mathop \to b^-} \map f x = L$


This is voiced:
:the limit of $\map f x$ as $x$ tends to $b$ from the left
and such an $L$ is called:
:a limit from the left.
",Definition:Continuous Real Function/Left-Continuous,['Definitions/Continuous Real Functions'],"Let a b be an open real interval.

Let f:  a b → be a real function.

Let L ∈.


Suppose that:
:∀ϵ∈_>0: ∃δ∈_>0: ∀ x ∈: b - δ < x < b  f x - L < ϵ

where _>0 denotes the set of strictly positive real numbers.

That is, for every real strictly positive ϵ there exists a real strictly positive δ such that every real number in the domain of f, less than b but within δ of b, has an image within ϵ of L.


:

Then f x is said to tend to the limit L as x tends to b from the left, and we write:
:f x → L as x → b^-
or
:lim_x → b^- f x = L


This is voiced:
:the limit of f x as x tends to b from the left
and such an L is called:
:a limit from the left.
Let a b be an open real interval.

Let f:  a b → be a real function.

Let L ∈.


Suppose that:
:∀ϵ∈_>0: ∃δ∈_>0: ∀ x ∈: b - δ < x < b  f x - L < ϵ

where _>0 denotes the set of strictly positive real numbers.

That is, for every real strictly positive ϵ there exists a real strictly positive δ such that every real number in the domain of f, less than b but within δ of b, has an image within ϵ of L.


:

Then f x is said to tend to the limit L as x tends to b from the left, and we write:
:f x → L as x → b^-
or
:lim_x → b^- f x = L


This is voiced:
:the limit of f x as x tends to b from the left
and such an L is called:
:a limit from the left.
"
Definition:Left,Left,"Let $a, b \in \R$ be real numbers.


The left half-open (real) interval from $a$ to $b$ is the subset:
:$\hointl a b := \set {x \in \R: a < x \le b}$
",Definition:Real Interval/Half-Open/Left,[],"Let a, b ∈ be real numbers.


The left half-open (real) interval from a to b is the subset:
:a b := x ∈: a < x ≤ b
"
Definition:Left,Left,"There are two unbounded closed intervals involving a real number $a \in \R$, defined as:




",Definition:Real Interval/Unbounded Closed,['Definitions/Real Intervals'],"There are two unbounded closed intervals involving a real number a ∈, defined as:




"
Definition:Left,Left,"There are two unbounded open intervals involving a real number $a \in \R$, defined as:




",Definition:Real Interval/Unbounded Open,['Definitions/Real Intervals'],"There are two unbounded open intervals involving a real number a ∈, defined as:




"
Definition:Left,Left,"Let $\struct {S, \circ}$ be an algebraic structure.

An element $z_L \in S$ is called a left zero element (or just left zero) :
:$\forall x \in S: z_L \circ x = z_L$",Definition:Left Zero,['Definitions/Zero Elements'],"Let S, ∘ be an algebraic structure.

An element z_L ∈ S is called a left zero element (or just left zero) :
:∀ x ∈ S: z_L ∘ x = z_L"
Definition:Left,Left,"Let $\struct {S, \circ}$ be an algebraic structure.

An element $e_L \in S$ is called a left identity (element) :
:$\forall x \in S: e_L \circ x = x$
",Definition:Identity (Abstract Algebra)/Left Identity,['Definitions/Identity Elements'],"Let S, ∘ be an algebraic structure.

An element e_L ∈ S is called a left identity (element) :
:∀ x ∈ S: e_L ∘ x = x
"
Definition:Left,Left,"Let $S$ be a set.

For any $x, y \in S$, the left operation on $S$ is the binary operation defined as:
:$\forall x, y \in S: x \gets y = x$",Definition:Left Operation,"['Definitions/Abstract Algebra', 'Definitions/Left Operation']","Let S be a set.

For any x, y ∈ S, the left operation on S is the binary operation defined as:
:∀ x, y ∈ S: x  y = x"
Definition:Left,Left,"Let $\struct {S, \circ}$ be an algebraic structure.


An element $x \in \struct {S, \circ}$ is left cancellable :

:$\forall a, b \in S: x \circ a = x \circ b \implies a = b$",Definition:Cancellable Element/Left Cancellable,['Definitions/Cancellability'],"Let S, ∘ be an algebraic structure.


An element x ∈S, ∘ is left cancellable :

:∀ a, b ∈ S: x ∘ a = x ∘ b  a = b"
Definition:Left,Left,"Let $\struct {S, \circ}$ be an algebraic structure.


An element $x \in \struct {S, \circ}$ is left cancellable :

:$\forall a, b \in S: x \circ a = x \circ b \implies a = b$
",Definition:Left Cancellable Operation,['Definitions/Cancellability'],"Let S, ∘ be an algebraic structure.


An element x ∈S, ∘ is left cancellable :

:∀ a, b ∈ S: x ∘ a = x ∘ b  a = b
"
Definition:Left,Left,"Let $S$ be a set on which is defined two binary operations, defined on all the elements of $S \times S$, denoted here as $\circ$ and $*$.

The operation $\circ$ is left distributive over the operation $*$ :

:$\forall a, b, c \in S: a \circ \paren {b * c} = \paren {a \circ b} * \paren {a \circ c}$",Definition:Distributive Operation/Left,['Definitions/Distributive Operations'],"Let S be a set on which is defined two binary operations, defined on all the elements of S × S, denoted here as ∘ and *.

The operation ∘ is left distributive over the operation * :

:∀ a, b, c ∈ S: a ∘b * c = a ∘ b * a ∘ c"
Definition:Left,Left,"Let $\struct {S, \circ}$ be a monoid whose identity is $e_S$.

An element $x_L \in S$ is called a left inverse of $x$ :
:$x_L \circ x = e_S$
",Definition:Inverse (Abstract Algebra)/Left Inverse,['Definitions/Inverse Elements'],"Let S, ∘ be a monoid whose identity is e_S.

An element x_L ∈ S is called a left inverse of x :
:x_L ∘ x = e_S
"
Definition:Left,Left,"Let $\struct {S, \circ}$ be a magma.

The mapping $\lambda_a: S \to S$ is defined as:

:$\forall x \in S: \map {\lambda_a} x = a \circ x$


This is known as the left regular representation of $\struct {S, \circ}$ with respect to $a$.
",Definition:Quasigroup/Left Quasigroup,['Definitions/Quasigroups'],"Let S, ∘ be a magma.

The mapping λ_a: S → S is defined as:

:∀ x ∈ S: λ_a x = a ∘ x


This is known as the left regular representation of S, ∘ with respect to a.
"
Definition:Left,Left,"Let $x$ and $y$ be elements which are operated on by a given operation $\circ$.

The left-hand product of $x$ by $y$ is the product $y \circ x$.",Definition:Operation/Binary Operation/Product/Left,['Definitions/Operations'],"Let x and y be elements which are operated on by a given operation ∘.

The left-hand product of x by y is the product y ∘ x."
Definition:Left,Left,"Let $\struct {S, \circ, \preceq}$ be a positively totally ordered semigroup.


Then $\struct {S, \circ, \preceq}$ is a left naturally totally ordered semigroup :

:$a \prec b \implies \exists x \in S: b = x \circ a$",Definition:Left Naturally Totally Ordered Semigroup,['Definitions/Naturally Ordered Semigroup'],"Let S, ∘, ≼ be a positively totally ordered semigroup.


Then S, ∘, ≼ is a left naturally totally ordered semigroup :

:a ≺ b ∃ x ∈ S: b = x ∘ a"
Definition:Left,Left,"Let $\struct {S, \circ}$ be an algebraic structure.

Let $\struct {H, \circ}$ be a subgroup of $\struct {S, \circ}$.


The left coset of $x$ modulo $H$, or left coset of $H$ by $x$, is:

:$x \circ H = \set {y \in S: \exists h \in H: y = x \circ h}$


That is, it is the subset product with singleton:

:$x \circ H = \set x \circ H$",Definition:Coset/Left Coset,['Definitions/Cosets'],"Let S, ∘ be an algebraic structure.

Let H, ∘ be a subgroup of S, ∘.


The left coset of x modulo H, or left coset of H by x, is:

:x ∘ H = y ∈ S: ∃ h ∈ H: y = x ∘ h


That is, it is the subset product with singleton:

:x ∘ H =  x ∘ H"
Definition:Left,Left,"Let $G$ be a group.

Let $H$ be a subgroup of $G$.


We can use $H$ to define a relation on $G$ as follows:

:$\RR^l_H := \set {\tuple {x, y} \in G \times G: x^{-1} y \in H}$

This is called left congruence modulo $H$.
Let $\struct {S, \circ}$ be an algebraic structure.

Let $\struct {H, \circ}$ be a subgroup of $\struct {S, \circ}$.


The left coset of $x$ modulo $H$, or left coset of $H$ by $x$, is:

:$x \circ H = \set {y \in S: \exists h \in H: y = x \circ h}$


That is, it is the subset product with singleton:

:$x \circ H = \set x \circ H$
",Definition:Coset Space/Left Coset Space,['Definitions/Cosets'],"Let G be a group.

Let H be a subgroup of G.


We can use H to define a relation on G as follows:

:^l_H := x, y∈ G × G: x^-1 y ∈ H

This is called left congruence modulo H.
Let S, ∘ be an algebraic structure.

Let H, ∘ be a subgroup of S, ∘.


The left coset of x modulo H, or left coset of H by x, is:

:x ∘ H = y ∈ S: ∃ h ∈ H: y = x ∘ h


That is, it is the subset product with singleton:

:x ∘ H =  x ∘ H
"
Definition:Left,Left,"Let $\struct {S, \circ}$ be an algebraic structure.

Let $\struct {H, \circ}$ be a subgroup of $\struct {S, \circ}$.


The left coset of $x$ modulo $H$, or left coset of $H$ by $x$, is:

:$x \circ H = \set {y \in S: \exists h \in H: y = x \circ h}$


That is, it is the subset product with singleton:

:$x \circ H = \set x \circ H$
",Definition:Transversal (Group Theory)/Left Transversal,['Definitions/Transversals (Group Theory)'],"Let S, ∘ be an algebraic structure.

Let H, ∘ be a subgroup of S, ∘.


The left coset of x modulo H, or left coset of H by x, is:

:x ∘ H = y ∈ S: ∃ h ∈ H: y = x ∘ h


That is, it is the subset product with singleton:

:x ∘ H =  x ∘ H
"
Definition:Left,Left,"Let $X$ be a set.

Let $\struct {G, \circ}$ be a group whose identity is $e$.


A (left) group action is an operation $\phi: G \times X \to X$ such that:

:$\forall \tuple {g, x} \in G \times X: g * x := \map \phi {g, x} \in X$

in such a way that the group action axioms are satisfied:
",Definition:Group Action/Left Group Action,['Definitions/Group Actions'],"Let X be a set.

Let G, ∘ be a group whose identity is e.


A (left) group action is an operation ϕ: G × X → X such that:

:∀g, x∈ G × X: g * x := ϕg, x∈ X

in such a way that the group action axioms are satisfied:
"
Definition:Left,Left,"Let $\struct {G, \circ}$ be a group.

Let $\powerset G$ be the power set of $G$.


The (left) subset product action of $G$ is the group action $*: G \times \powerset G \to \powerset G$:
:$\forall g \in G, S \in \powerset G: g * S = g \circ S$",Definition:Subset Product Action/Left,['Definitions/Subset Product Action'],"Let G, ∘ be a group.

Let G be the power set of G.


The (left) subset product action of G is the group action *: G × G → G:
:∀ g ∈ G, S ∈ G: g * S = g ∘ S"
Definition:Left,Left,"Let $G$ be a group.

Let $H$ be a subgroup of $G$.


We can use $H$ to define a relation on $G$ as follows:

:$\RR^l_H := \set {\tuple {x, y} \in G \times G: x^{-1} y \in H}$

This is called left congruence modulo $H$.",Definition:Congruence Modulo Subgroup/Left Congruence,['Definitions/Congruence Modulo Subgroup'],"Let G be a group.

Let H be a subgroup of G.


We can use H to define a relation on G as follows:

:^l_H := x, y∈ G × G: x^-1 y ∈ H

This is called left congruence modulo H."
Definition:Left,Left,"Let $\struct {R, +, \circ}$ be a ring.


A left zero divisor (in $R$) is an element $x \in R$ such that:
:$\exists y \in R^*: x \circ y = 0_R$

where $R^*$ is defined as $R \setminus \set {0_R}$.
",Definition:Left Zero Divisor,['Definitions/Zero Divisors'],"Let R, +, ∘ be a ring.


A left zero divisor (in R) is an element x ∈ R such that:
:∃ y ∈ R^*: x ∘ y = 0_R

where R^* is defined as R ∖0_R.
"
Definition:Left,Left,"Let $R$ be a ring.

Let $M$ be an abelian group.

Let $\circ : R \times M \to M$ be a mapping from the cartesian product $R \times M$.


$\circ$ is a left linear ring action of $R$ on $M$  $\circ$ satisfies the left ring action axioms:
",Definition:Linear Ring Action/Left,['Definitions/Linear Ring Actions'],"Let R be a ring.

Let M be an abelian group.

Let ∘ : R × M → M be a mapping from the cartesian product R × M.


∘ is a left linear ring action of R on M  ∘ satisfies the left ring action axioms:
"
Definition:Left,Left,"Let $\struct {R, +, \circ}$ be a ring.

Let $\struct {J, +}$ be a subgroup of $\struct {R, +}$.


$J$ is a left ideal of $R$ :
:$\forall j \in J: \forall r \in R: r \circ j \in J$

that is, :
:$\forall r \in R: r \circ J \subseteq J$
",Definition:Ideal of Ring/Left Ideal,['Definitions/Ideal Theory'],"Let R, +, ∘ be a ring.

Let J, + be a subgroup of R, +.


J is a left ideal of R :
:∀ j ∈ J: ∀ r ∈ R: r ∘ j ∈ J

that is, :
:∀ r ∈ R: r ∘ J ⊆ J
"
Definition:Left,Left,"Let $\struct {R, +, \circ}$ be a ring.

Let $\struct {J, +}$ be a subgroup of $\struct {R, +}$.


$J$ is a left ideal of $R$ :
:$\forall j \in J: \forall r \in R: r \circ j \in J$

that is, :
:$\forall r \in R: r \circ J \subseteq J$
Let $\struct {R, +, \circ}$ be a ring.

Let $\struct {J, +}$ be a subgroup of $\struct {R, +}$.


$J$ is a left ideal of $R$ :
:$\forall j \in J: \forall r \in R: r \circ j \in J$

that is, :
:$\forall r \in R: r \circ J \subseteq J$
",Definition:Maximal Ideal of Ring/Left,['Definitions/Maximal Ideals of Rings'],"Let R, +, ∘ be a ring.

Let J, + be a subgroup of R, +.


J is a left ideal of R :
:∀ j ∈ J: ∀ r ∈ R: r ∘ j ∈ J

that is, :
:∀ r ∈ R: r ∘ J ⊆ J
Let R, +, ∘ be a ring.

Let J, + be a subgroup of R, +.


J is a left ideal of R :
:∀ j ∈ J: ∀ r ∈ R: r ∘ j ∈ J

that is, :
:∀ r ∈ R: r ∘ J ⊆ J
"
Definition:Left,Left,"Let $\struct {R, +_R, \times_R}$ be a ring.

Let $\struct {G, +_G}$ be an abelian group.


A left module over $R$ is an $R$-algebraic structure $\struct {G, +_G, \circ}_R$ with one operation $\circ$, the (left) ring action, which satisfies the left module axioms:

",Definition:Left Module,"['Definitions/Left Modules', 'Definitions/Module Theory']","Let R, +_R, ×_R be a ring.

Let G, +_G be an abelian group.


A left module over R is an R-algebraic structure G, +_G, ∘_R with one operation ∘, the (left) ring action, which satisfies the left module axioms:

"
Definition:Left,Left,"Let $R$ be a ring.

Let $\mathbf A$ be a matrix in the matrix space $\map {\MM_{m, n} } R$.

Let $\mathbf A^\intercal$ be the transpose of $\mathbf A$.

The left null space $\mathbf A$ is defined as the null space of $\mathbf A^\intercal$.",Definition:Left Null Space,"['Definitions/Linear Algebra', 'Definitions/Null Spaces']","Let R be a ring.

Let 𝐀 be a matrix in the matrix space _m, n R.

Let 𝐀^⊺ be the transpose of 𝐀.

The left null space 𝐀 is defined as the null space of 𝐀^⊺."
Definition:Left,Left,"A left-truncatable prime is a prime number which remains prime when any number of digits are removed from the left hand end.

Zeroes are excluded, in order to eliminate, for example, prime numbers of the form $10^n + 3$ for arbitrarily large $n$.


=== Sequence ===
",Definition:Left-Truncatable Prime,"['Definitions/Prime Numbers', 'Definitions/Recreational Mathematics', 'Definitions/Left-Truncatable Primes']","A left-truncatable prime is a prime number which remains prime when any number of digits are removed from the left hand end.

Zeroes are excluded, in order to eliminate, for example, prime numbers of the form 10^n + 3 for arbitrarily large n.


=== Sequence ===
"
Definition:Left,Left,"A Cartesian plane is defined as being left-handed if it has the following property:

Let a left hand be placed, with palm uppermost, such that the thumb points along the $x$-axis in the positive direction, such that the thumb and index finger are at right-angles to each other.

Then the index finger is pointed along the $y$-axis in the positive direction.


:",Definition:Orientation of Coordinate Axes/Cartesian Plane/Left-Handed,['Definitions/Orientation (Coordinate Axes)'],"A Cartesian plane is defined as being left-handed if it has the following property:

Let a left hand be placed, with palm uppermost, such that the thumb points along the x-axis in the positive direction, such that the thumb and index finger are at right-angles to each other.

Then the index finger is pointed along the y-axis in the positive direction.


:"
Definition:Left,Left,"A Cartesian $3$-Space is defined as being left-handed if it has the following property:

Let a left hand be placed such that:
:the thumb and index finger are at right-angles to each other
:the $3$rd finger is at right-angles to the thumb and index finger, upwards from the palm
:the thumb points along the $x$-axis in the positive direction
:the index finger points along the $y$-axis in the positive direction.

Then the $3$rd finger is pointed along the $z$-axis in the positive direction.


:",Definition:Orientation of Coordinate Axes/Cartesian 3-Space/Left-Handed,['Definitions/Orientation (Coordinate Axes)'],"A Cartesian 3-Space is defined as being left-handed if it has the following property:

Let a left hand be placed such that:
:the thumb and index finger are at right-angles to each other
:the 3rd finger is at right-angles to the thumb and index finger, upwards from the palm
:the thumb points along the x-axis in the positive direction
:the index finger points along the y-axis in the positive direction.

Then the 3rd finger is pointed along the z-axis in the positive direction.


:"
Definition:Left,Left,,Definition:Left Cancellable,['Definitions/Cancellability'],
Definition:Left Cancellable,Left Cancellable,"Let $\struct {S, \circ}$ be an algebraic structure.


An element $x \in \struct {S, \circ}$ is left cancellable :

:$\forall a, b \in S: x \circ a = x \circ b \implies a = b$",Definition:Cancellable Element/Left Cancellable,['Definitions/Cancellability'],"Let S, ∘ be an algebraic structure.


An element x ∈S, ∘ is left cancellable :

:∀ a, b ∈ S: x ∘ a = x ∘ b  a = b"
Definition:Left Cancellable,Left Cancellable,"Let $\struct {S, \circ}$ be an algebraic structure.


An element $x \in \struct {S, \circ}$ is left cancellable :

:$\forall a, b \in S: x \circ a = x \circ b \implies a = b$
",Definition:Left Cancellable Operation,['Definitions/Cancellability'],"Let S, ∘ be an algebraic structure.


An element x ∈S, ∘ is left cancellable :

:∀ a, b ∈ S: x ∘ a = x ∘ b  a = b
"
Definition:Left Cancellable,Left Cancellable,"A mapping $f: Y \to Z$ is left cancellable (or left-cancellable) :

:$\forall X: \forall \struct {g_1, g_2: X \to Y}: f \circ g_1 = f \circ g_2 \implies g_1 = g_2$

That is, for any set $X$, if $g_1$ and $g_2$ are mappings from $X$ to $Y$:
:If $f \circ g_1 = f \circ g_2$
:then $g_1 = g_2$.",Definition:Left Cancellable Mapping,"['Definitions/Mapping Theory', 'Definitions/Cancellability']","A mapping f: Y → Z is left cancellable (or left-cancellable) :

:∀ X: ∀g_1, g_2: X → Y: f ∘ g_1 = f ∘ g_2  g_1 = g_2

That is, for any set X, if g_1 and g_2 are mappings from X to Y:
:If f ∘ g_1 = f ∘ g_2
:then g_1 = g_2."
Definition:Left Inverse,Left Inverse,"Let $S, T$ be sets where $S \ne \O$, that is, $S$ is not empty.

Let $f: S \to T$ be a mapping.


Let $g: T \to S$ be a mapping such that:
:$g \circ f = I_S$
where:
:$g \circ f$ denotes the composite mapping $f$ followed by $g$;
:$I_S$ is the identity mapping on $S$.


Then $g: T \to S$ is called a left inverse (mapping).",Definition:Left Inverse Mapping,['Definitions/Inverse Mappings'],"Let S, T be sets where S Ø, that is, S is not empty.

Let f: S → T be a mapping.


Let g: T → S be a mapping such that:
:g ∘ f = I_S
where:
:g ∘ f denotes the composite mapping f followed by g;
:I_S is the identity mapping on S.


Then g: T → S is called a left inverse (mapping)."
Definition:Left Inverse,Left Inverse,"Let $\struct {S, \circ}$ be a monoid whose identity is $e_S$.

An element $x_L \in S$ is called a left inverse of $x$ :
:$x_L \circ x = e_S$
",Definition:Inverse (Abstract Algebra)/Left Inverse,['Definitions/Inverse Elements'],"Let S, ∘ be a monoid whose identity is e_S.

An element x_L ∈ S is called a left inverse of x :
:x_L ∘ x = e_S
"
Definition:Left Inverse,Left Inverse,"
",Definition:Inverse Matrix/Left,['Definitions/Inverse Matrices'],"
"
Definition:Length,Length,"Length is linear measure taken in a particular direction.

Usually, in multi-dimensional figures, the dimension in which the linear measure is greatest is referred to as length.

It is the most widely used term for linear measure, as it is the standard term used when only one dimension is under consideration.


Length is the fundamental notion of Euclidean geometry, never defined but regarded as an intuitive concept at the basis of every geometrical theorem.


=== Vector Definition ===

Length is linear measure taken in a particular direction.

Usually, in multi-dimensional figures, the dimension in which the linear measure is greatest is referred to as length.

It is the most widely used term for linear measure, as it is the standard term used when only one dimension is under consideration.


Length is the fundamental notion of Euclidean geometry, never defined but regarded as an intuitive concept at the basis of every geometrical theorem.


=== Vector Definition ===

Length is linear measure taken in a particular direction.

Usually, in multi-dimensional figures, the dimension in which the linear measure is greatest is referred to as length.

It is the most widely used term for linear measure, as it is the standard term used when only one dimension is under consideration.


Length is the fundamental notion of Euclidean geometry, never defined but regarded as an intuitive concept at the basis of every geometrical theorem.


=== Vector Definition ===

",Definition:Linear Measure/Length,"['Definitions/Length', 'Definitions/Geometry', 'Definitions/Fundamental Dimensions']","Length is linear measure taken in a particular direction.

Usually, in multi-dimensional figures, the dimension in which the linear measure is greatest is referred to as length.

It is the most widely used term for linear measure, as it is the standard term used when only one dimension is under consideration.


Length is the fundamental notion of Euclidean geometry, never defined but regarded as an intuitive concept at the basis of every geometrical theorem.


=== Vector Definition ===

Length is linear measure taken in a particular direction.

Usually, in multi-dimensional figures, the dimension in which the linear measure is greatest is referred to as length.

It is the most widely used term for linear measure, as it is the standard term used when only one dimension is under consideration.


Length is the fundamental notion of Euclidean geometry, never defined but regarded as an intuitive concept at the basis of every geometrical theorem.


=== Vector Definition ===

Length is linear measure taken in a particular direction.

Usually, in multi-dimensional figures, the dimension in which the linear measure is greatest is referred to as length.

It is the most widely used term for linear measure, as it is the standard term used when only one dimension is under consideration.


Length is the fundamental notion of Euclidean geometry, never defined but regarded as an intuitive concept at the basis of every geometrical theorem.


=== Vector Definition ===

"
Definition:Length,Length,"Let:
:$\closedint a b$
or
:$\hointr a b$
or
:$\hointl a b$
or
:$\openint a b$
be a real interval.


The difference $b - a$ between the endpoints is called the length of the interval.",Definition:Real Interval/Length,['Definitions/Real Intervals'],"Let:
:a b
or
:a b
or
:a b
or
:a b
be a real interval.


The difference b - a between the endpoints is called the length of the interval."
Definition:Length,Length,"Length is linear measure taken in a particular direction.

Usually, in multi-dimensional figures, the dimension in which the linear measure is greatest is referred to as length.

It is the most widely used term for linear measure, as it is the standard term used when only one dimension is under consideration.


Length is the fundamental notion of Euclidean geometry, never defined but regarded as an intuitive concept at the basis of every geometrical theorem.


=== Vector Definition ===

Length is linear measure taken in a particular direction.

Usually, in multi-dimensional figures, the dimension in which the linear measure is greatest is referred to as length.

It is the most widely used term for linear measure, as it is the standard term used when only one dimension is under consideration.


Length is the fundamental notion of Euclidean geometry, never defined but regarded as an intuitive concept at the basis of every geometrical theorem.


=== Vector Definition ===

",Definition:Length of Curve,['Definitions/Analytic Geometry'],"Length is linear measure taken in a particular direction.

Usually, in multi-dimensional figures, the dimension in which the linear measure is greatest is referred to as length.

It is the most widely used term for linear measure, as it is the standard term used when only one dimension is under consideration.


Length is the fundamental notion of Euclidean geometry, never defined but regarded as an intuitive concept at the basis of every geometrical theorem.


=== Vector Definition ===

Length is linear measure taken in a particular direction.

Usually, in multi-dimensional figures, the dimension in which the linear measure is greatest is referred to as length.

It is the most widely used term for linear measure, as it is the standard term used when only one dimension is under consideration.


Length is the fundamental notion of Euclidean geometry, never defined but regarded as an intuitive concept at the basis of every geometrical theorem.


=== Vector Definition ===

"
Definition:Length,Length,"The length of a finite sequence is the number of terms it contains, or equivalently, the cardinality of its domain.


=== Sequence of $n$ Terms ===

",Definition:Length of Sequence,['Definitions/Sequences'],"The length of a finite sequence is the number of terms it contains, or equivalently, the cardinality of its domain.


=== Sequence of n Terms ===

"
Definition:Length,Length,,Definition:Vector Length/Complex Plane,"['Definitions/Vector Length', 'Definitions/Complex Analysis']",
Definition:Length,Length,"Let $C$ be a contour in $\C$ defined by the (finite) sequence $\sequence {C_1, \ldots, C_n}$ of directed smooth curves in $\C$.

Let $C_k$ be parameterized by the smooth path $\gamma_k: \closedint {a_k} {b_k} \to \C$ for all $k \in \set {1, \ldots, n}$.


The length of $C$ is defined as:

:$\ds \map L C := \sum_{k \mathop = 1}^n \int_{a_k}^{b_k} \size {\map {\gamma_k'} t} \rd t$


It follows from Length of Contour is Well-Defined that $\map L C$ is defined and independent of the parameterizations of $C_1, \ldots, C_n$.",Definition:Contour/Length/Complex Plane,['Definitions/Complex Analysis'],"Let C be a contour in  defined by the (finite) sequence C_1, …, C_n of directed smooth curves in .

Let C_k be parameterized by the smooth path γ_k: a_kb_k→ for all k ∈1, …, n.


The length of C is defined as:

:L C := ∑_k  = 1^n ∫_a_k^b_kγ_k' t t


It follows from Length of Contour is Well-Defined that L C is defined and independent of the parameterizations of C_1, …, C_n."
Definition:Length,Length,"The length of a vector $\mathbf v$ in a normed vector space $\struct {V, \norm {\, \cdot \,} }$ is defined as $\norm {\mathbf v}$, the norm of $\mathbf v$.


=== Arrow Representation ===
",Definition:Vector Length,"['Definitions/Vectors', 'Definitions/Vector Length']","The length of a vector 𝐯 in a normed vector space V,  · is defined as 𝐯, the norm of 𝐯.


=== Arrow Representation ===
"
Definition:Length,Length,"Let $\struct {S, \preceq}$ be an ordered set.

Let $T$ be a chain in $S$.

Let $T$ be finite and non-empty.


The length of the chain $T$ is its cardinality minus $1$.


Category:Definitions/Chains (Order Theory)",Definition:Length of Chain,['Definitions/Chains (Order Theory)'],"Let S, ≼ be an ordered set.

Let T be a chain in S.

Let T be finite and non-empty.


The length of the chain T is its cardinality minus 1.


Category:Definitions/Chains (Order Theory)"
Definition:Length,Length,"The length of a finite sequence is the number of terms it contains, or equivalently, the cardinality of its domain.


=== Sequence of $n$ Terms ===

The length of a finite sequence is the number of terms it contains, or equivalently, the cardinality of its domain.


=== Sequence of $n$ Terms ===

",Definition:Ordinal Sequence/Length,['Definitions/Ordinal Sequences'],"The length of a finite sequence is the number of terms it contains, or equivalently, the cardinality of its domain.


=== Sequence of n Terms ===

The length of a finite sequence is the number of terms it contains, or equivalently, the cardinality of its domain.


=== Sequence of n Terms ===

"
Definition:Length,Length,"The number of partial denominators in a cycle of a periodic continued fraction is called the cycle length.
",Definition:Periodic Continued Fraction/Cycle/Length,['Definitions/Continued Fractions'],"The number of partial denominators in a cycle of a periodic continued fraction is called the cycle length.
"
Definition:Length,Length,"Let $k$ be a field.

Let $C$ be a continued fraction in $k$, either finite or infinite.


The length of $C$ is an extended natural number equal to:
*$\infty$ if $C$ is an infinite continued fraction.
*$n$ if $C$ is a finite continued fraction with domain the integer interval $\left[0 \,.\,.\, n\right]$.",Definition:Length of Continued Fraction,['Definitions/Continued Fractions'],"Let k be a field.

Let C be a continued fraction in k, either finite or infinite.


The length of C is an extended natural number equal to:
*∞ if C is an infinite continued fraction.
*n if C is a finite continued fraction with domain the integer interval [0  . .  n]."
Definition:Length,Length,"A zero length walk is a walk which consists of one vertex.


Category:Definitions/Walks
",Definition:Walk (Graph Theory)/Length,['Definitions/Walks'],"A zero length walk is a walk which consists of one vertex.


Category:Definitions/Walks
"
Definition:Length,Length,"Let $T$ be a rooted tree with root node $r_T$.

Let $\Gamma$ be a finite branch of $T$.


The length of $\Gamma$ is defined as the number of ancestors of the leaf at the end of that branch.",Definition:Rooted Tree/Branch/Length,['Definitions/Rooted Trees'],"Let T be a rooted tree with root node r_T.

Let Γ be a finite branch of T.


The length of Γ is defined as the number of ancestors of the leaf at the end of that branch."
Definition:Length,Length,"The length of a finite string in a formal language is the number of symbols it contains.


The length of a string $S$ can be denoted $\map \len S$ or $\size S$.",Definition:Length of String,['Definitions/Collations'],"The length of a finite string in a formal language is the number of symbols it contains.


The length of a string S can be denoted S or S."
Definition:Length,Length,"The length of a tableau proof is the number of lines it has.


Category:Definitions/Proof Systems",Definition:Tableau Proof (Formal Systems)/Length,['Definitions/Proof Systems'],"The length of a tableau proof is the number of lines it has.


Category:Definitions/Proof Systems"
Definition:Length,Length,"Let $G$ be a group acting on a set $X$.

Let $x \in X$.


Let $\Orb x$ be the orbit of $x$.

The length of the orbit $\Orb x$ of $x$ is the number of elements of $X$ it contains:
:$\size {\Orb x}$


Category:Definitions/Group Actions",Definition:Orbit (Group Theory)/Length,['Definitions/Group Actions'],"Let G be a group acting on a set X.

Let x ∈ X.


Let x be the orbit of x.

The length of the orbit x of x is the number of elements of X it contains:
:x


Category:Definitions/Group Actions"
Definition:Length,Length,"Let $G$ be a group whose identity is $e$.

Let $\sequence {G_i}_{i \mathop \in \closedint 0 n}$ be a normal series for $G$:
:$\sequence {G_i}_{i \mathop \in \closedint 0 n} = \tuple {\set e = G_0 \lhd G_1 \lhd \cdots \lhd G_{n-1} \lhd G_n = G}$


The length of $\sequence {G_i}_{i \mathop \in \closedint 0 n}$ is the number of (normal) subgroups which make it.

In this context, the length of $\sequence {G_i}_{i \mathop \in \closedint 0 n}$ is $n$.


If such a normal series is infinite, then its length is not defined.


Category:Definitions/Normal Series",Definition:Normal Series/Length,['Definitions/Normal Series'],"Let G be a group whose identity is e.

Let G_i_i ∈ 0 n be a normal series for G:
:G_i_i ∈ 0 n =  e = G_0  G_1 ⋯ G_n-1 G_n = G


The length of G_i_i ∈ 0 n is the number of (normal) subgroups which make it.

In this context, the length of G_i_i ∈ 0 n is n.


If such a normal series is infinite, then its length is not defined.


Category:Definitions/Normal Series"
Definition:Length,Length,"Let $G$ be a group whose identity is $e$.

Let $\sequence {G_i}_{i \mathop \in \closedint 0 n}$ be a normal series for $G$:
:$\sequence {G_i}_{i \mathop \in \closedint 0 n} = \tuple {\set e = G_0 \lhd G_1 \lhd \cdots \lhd G_{n-1} \lhd G_n = G}$


The length of $\sequence {G_i}_{i \mathop \in \closedint 0 n}$ is the number of (normal) subgroups which make it.

In this context, the length of $\sequence {G_i}_{i \mathop \in \closedint 0 n}$ is $n$.


If such a normal series is infinite, then its length is not defined.


Category:Definitions/Normal Series
Let $G$ be a group whose identity is $e$.

Let $\sequence {G_i}_{i \mathop \in \closedint 0 n}$ be a normal series for $G$:
:$\sequence {G_i}_{i \mathop \in \closedint 0 n} = \tuple {\set e = G_0 \lhd G_1 \lhd \cdots \lhd G_{n-1} \lhd G_n = G}$


The length of $\sequence {G_i}_{i \mathop \in \closedint 0 n}$ is the number of (normal) subgroups which make it.

In this context, the length of $\sequence {G_i}_{i \mathop \in \closedint 0 n}$ is $n$.


If such a normal series is infinite, then its length is not defined.


Category:Definitions/Normal Series
",Definition:Length of Group,"['Definitions/Composition Series', 'Definitions/Group Theory']","Let G be a group whose identity is e.

Let G_i_i ∈ 0 n be a normal series for G:
:G_i_i ∈ 0 n =  e = G_0  G_1 ⋯ G_n-1 G_n = G


The length of G_i_i ∈ 0 n is the number of (normal) subgroups which make it.

In this context, the length of G_i_i ∈ 0 n is n.


If such a normal series is infinite, then its length is not defined.


Category:Definitions/Normal Series
Let G be a group whose identity is e.

Let G_i_i ∈ 0 n be a normal series for G:
:G_i_i ∈ 0 n =  e = G_0  G_1 ⋯ G_n-1 G_n = G


The length of G_i_i ∈ 0 n is the number of (normal) subgroups which make it.

In this context, the length of G_i_i ∈ 0 n is n.


If such a normal series is infinite, then its length is not defined.


Category:Definitions/Normal Series
"
Definition:Length,Length,"Let $S_n$ denote the symmetric group on $n$ letters.

Let $\rho \in S_n$ be a permutation on $S$.


Then $\rho$ is a cyclic permutation of length $k$  there exists $k \in \Z: k > 0$ and $i \in \Z$ such that:
:$(1): \quad k$ is the smallest such that $\map {\rho^k} i = i$

:$(2): \quad \rho$ fixes each $j$ not in $\set {i, \map \rho i, \ldots, \map {\rho^{k - 1} } i}$.


$\rho$ is usually denoted using cycle notation as:
:$\begin{pmatrix} i & \map \rho i & \ldots & \map {\rho^{k - 1} } i \end{pmatrix}$

but some sources introduce it using two-row notation:

:$\begin{pmatrix} a_1 & a_2 & \cdots & a_k & \cdots & i & \cdots \\ a_2 & a_3 & \cdots & a_1 & \cdots & i & \cdots \end{pmatrix}$
",Definition:Cyclic Permutation,"['Definitions/Cyclic Permutations', 'Definitions/Permutation Theory']","Let S_n denote the symmetric group on n letters.

Let ρ∈ S_n be a permutation on S.


Then ρ is a cyclic permutation of length k  there exists k ∈: k > 0 and i ∈ such that:
:(1):    k is the smallest such that ρ^k i = i

:(2):   ρ fixes each j not in i, ρ i, …, ρ^k - 1 i.


ρ is usually denoted using cycle notation as:
:[         i       ρ i         … ρ^k - 1 i ]

but some sources introduce it using two-row notation:

:[ a_1 a_2   ⋯ a_k   ⋯   i   ⋯; a_2 a_3   ⋯ a_1   ⋯   i   ⋯ ]
"
Definition:Letter,Letter,,Definition:Formal Language/Alphabet/Letter,['Definitions/Alphabets (Formal Language)'],
Definition:Letter,Letter,"A letter is one of the symbols of the alphabet of a natural language.

The letters of the English alphabet, for example, are:
:$\texttt {A B C D E F G H I J K L M N O P Q R S T U V W X Y Z}$
:$\texttt {a b c d e f g h i j k l m n o p q r s t u v w x y z}$


Category:Definitions/Language Definitions
A letter is one of the symbols of the alphabet of a natural language.

The letters of the English alphabet, for example, are:
:$\texttt {A B C D E F G H I J K L M N O P Q R S T U V W X Y Z}$
:$\texttt {a b c d e f g h i j k l m n o p q r s t u v w x y z}$


Category:Definitions/Language Definitions
",Definition:Letter of Alphabet,['Definitions/Language Definitions'],"A letter is one of the symbols of the alphabet of a natural language.

The letters of the English alphabet, for example, are:
:
:


Category:Definitions/Language Definitions
A letter is one of the symbols of the alphabet of a natural language.

The letters of the English alphabet, for example, are:
:
:


Category:Definitions/Language Definitions
"
Definition:Like,Like,"Let $\mathbf a$ and $\mathbf b$ be vector quantities.

Then $\mathbf a$ and $\mathbf b$ are known as like vector quantities  they have the same direction.",Definition:Like Vector Quantities,['Definitions/Vectors'],"Let 𝐚 and 𝐛 be vector quantities.

Then 𝐚 and 𝐛 are known as like vector quantities  they have the same direction."
Definition:Like,Like,"If $2$ electric charges are of the same polarity, they are referred to as being like (electric) charges.


The usage is archaic; the word one would expect is alike.",Definition:Electric Charge/Polarity/Like,['Definitions/Electric Charge'],"If 2 electric charges are of the same polarity, they are referred to as being like (electric) charges.


The usage is archaic; the word one would expect is alike."
Definition:Limit,Limit,,Definition:Limit of Mapping,"['Definitions/Limits of Mappings', 'Definitions/Metric Spaces', 'Definitions/Analysis']",
Definition:Limit,Limit,"Let $S \subseteq \C$ be a subset of the set of complex numbers.

Let $z_0 \in \C$. 

Let $\map {N_\epsilon} {z_0}$ be the $\epsilon$-neighborhood of $z_0$ for a given $\epsilon \in \R$ such that $\epsilon > 0$.


Then $z_0$ is a limit point of $S$  every deleted $\epsilon$-neighborhood $\map {N_\epsilon} {z_0} \setminus \set {z_0}$ of $z_0$ contains a point in $S$:
:$\forall \epsilon \in \R_{>0}: \paren {\map {N_\epsilon} {z_0} \setminus \set {z_0} } \cap S \ne \O$
that is:
:$\forall \epsilon \in \R_{>0}: \set {z \in S: 0 < \cmod {z - z_0} < \epsilon} \ne \O$


Note that $z_0$ does not itself have to be an element of $S$ to be a limit point, although it may well be.

Informally, there are points in $S$ which are arbitrarily close to it.
",Definition:Limit of Complex Function,"['Definitions/Limits of Complex Functions', 'Definitions/Limits of Mappings', 'Definitions/Limits', 'Definitions/Complex Analysis']","Let S ⊆ be a subset of the set of complex numbers.

Let z_0 ∈. 

Let N_ϵz_0 be the ϵ-neighborhood of z_0 for a given ϵ∈ such that ϵ > 0.


Then z_0 is a limit point of S  every deleted ϵ-neighborhood N_ϵz_0∖z_0 of z_0 contains a point in S:
:∀ϵ∈_>0: N_ϵz_0∖z_0∩ S Ø
that is:
:∀ϵ∈_>0: z ∈ S: 0 < z - z_0 < ϵØ


Note that z_0 does not itself have to be an element of S to be a limit point, although it may well be.

Informally, there are points in S which are arbitrarily close to it.
"
Definition:Limit,Limit,"Let $\xi \in \R$ be a real number.

Let $\ds \map S x = \sum_{n \mathop = 0}^\infty a_n \paren {x - \xi}^n$ be a power series about $\xi$.

Let $I$ be the interval of convergence of $\map S x$.

Let the endpoints of $I$ be $\xi - R$ and $\xi + R$.

(This follows from the fact that $\xi$ is the midpoint of $I$.)


Then $R$ is called the radius of convergence of $\map S x$.


If $\map S x$ is convergent over the whole of $\R$, then $I = \R$ and thus the radius of convergence is infinite.",Definition:Radius of Convergence/Real Domain,"['Definitions/Convergence', 'Definitions/Power Series', 'Definitions/Real Analysis']","Let ξ∈ be a real number.

Let S x = ∑_n  = 0^∞ a_n x - ξ^n be a power series about ξ.

Let I be the interval of convergence of S x.

Let the endpoints of I be ξ - R and ξ + R.

(This follows from the fact that ξ is the midpoint of I.)


Then R is called the radius of convergence of S x.


If S x is convergent over the whole of , then I = and thus the radius of convergence is infinite."
Definition:Limit,Limit,"Let $\mathbf C$ be a metacategory.

Let $D: \mathbf J \to \mathbf C$ be a $\mathbf J$-diagram in $\mathbf C$.

Let $\varprojlim_j D_j$ be a limit for $D$.


Then $\varprojlim_j D_j$ is called a finite limit  $\mathbf J$ is a finite category.
",Definition:Limit (Category Theory),"['Definitions/Category Theory', 'Definitions/Limits and Colimits']","Let 𝐂 be a metacategory.

Let D: 𝐉→𝐂 be a 𝐉-diagram in 𝐂.

Let _j D_j be a limit for D.


Then _j D_j is called a finite limit  𝐉 is a finite category.
"
Definition:Limit,Limit,"Let $\struct {F, \norm {\,\cdot\,} }$ be a valued field.

Let $C = \sequence {a_n}_{n \mathop \ge 0}$ be a infinite continued fraction in $F$.


Let $C$ converge to $x \in F$:

Then $x$ is the value of $C$.",Definition:Value of Continued Fraction/Infinite,['Definitions/Continued Fractions'],"Let F,  · be a valued field.

Let C = a_n_n ≥ 0 be a infinite continued fraction in F.


Let C converge to x ∈ F:

Then x is the value of C."
Definition:Limit,Limit,"Let $A$ be a class.

Let $\preccurlyeq$ be a well-ordering on $A$.

Let $x$ be neither the smallest element of $A$ nor an immediate successor of any element of $A$.


Then $x$ is a limit element of $A$ (under $\preccurlyeq$).",Definition:Limit Element under Well-Ordering,['Definitions/Well-Orderings'],"Let A be a class.

Let ≼ be a well-ordering on A.

Let x be neither the smallest element of A nor an immediate successor of any element of A.


Then x is a limit element of A (under ≼)."
Definition:Linear Algebra,Linear Algebra,"Linear algebra is the branch of algebra which studies vector spaces and linear transformations between them.
",Definition:Linear Algebra (Mathematical Branch),"['Definitions/Linear Algebra', 'Definitions/Algebra', 'Definitions/Linearity', 'Definitions/Branches of Mathematics']","Linear algebra is the branch of algebra which studies vector spaces and linear transformations between them.
"
Definition:Linear Algebra,Linear Algebra,"Let $F$ be a field.


An algebra over $F$ is an ordered pair $\struct {A, *}$ where:
:$A$ is a vector space over $F$
:$* : A^2 \to A$ is a bilinear mapping


That is, it is an algebra $\struct {A, *}$ over the ring $F$ where:
:$F$ is a field
:the $F$-module $A$ is a vector space.


The symbol $A$ is often used for such an algebra, more so as the level of abstraction increases.


=== Multiplication ===
",Definition:Algebra over Field,"['Definitions/Algebras over Fields', 'Definitions/Algebras', 'Definitions/Field Theory']","Let F be a field.


An algebra over F is an ordered pair A, * where:
:A is a vector space over F
:* : A^2 → A is a bilinear mapping


That is, it is an algebra A, * over the ring F where:
:F is a field
:the F-module A is a vector space.


The symbol A is often used for such an algebra, more so as the level of abstraction increases.


=== Multiplication ===
"
Definition:Linear Form,Linear Form,"Let $\struct {R, +, \times}$ be a commutative ring.

Let $\struct {R, +_R, \circ}_R$ denote the $R$-module $R$.

Let $\struct {G, +_G, \circ}_R$ be a module over $R$.


Let $\phi: \struct {G, +_G, \circ}_R \to \struct {R, +_R, \circ}_R$ be a linear transformation from $G$ to the $R$-module $R$.


$\phi$ is called a linear form on $G$.
",Definition:Linear Form (Linear Algebra),"['Definitions/Linear Forms (Linear Algebra)', 'Definitions/Linear Algebra', 'Definitions/Linear Transformations', 'Definitions/Linear Forms', 'Definitions/Linearity']","Let R, +, × be a commutative ring.

Let R, +_R, ∘_R denote the R-module R.

Let G, +_G, ∘_R be a module over R.


Let ϕ: G, +_G, ∘_R →R, +_R, ∘_R be a linear transformation from G to the R-module R.


ϕ is called a linear form on G.
"
Definition:Linear Form,Linear Form,"A linear form is a form whose variables are of degree $1$.
",Definition:Linear Form (Polynomial Theory),"['Definitions/Linear Forms (Polynomial Theory)', 'Definitions/Forms', 'Definitions/Polynomial Theory', 'Definitions/Linear Forms', 'Definitions/Linearity']","A linear form is a form whose variables are of degree 1.
"
Definition:Locally Finite,Locally Finite,,Definition:Locally Finite Set of Subsets,['Definitions/Topology'],
Definition:Locally Finite,Locally Finite,"Let $T = \struct {S, \tau}$ be a topological space.

Let $\CC$ be a cover of $S$.


Then $\CC$ is locally finite  each element of $S$ has a neighborhood which intersects a finite number of sets in $\CC$.",Definition:Locally Finite Cover,['Definitions/Covers'],"Let T = S, τ be a topological space.

Let  be a cover of S.


Then  is locally finite  each element of S has a neighborhood which intersects a finite number of sets in ."
Definition:Locally Finite,Locally Finite,A locally finite graph $G$ is an infinite graph where every vertex of $G$ has finite degree.,Definition:Locally Finite Graph,['Definitions/Graph Theory'],A locally finite graph G is an infinite graph where every vertex of G has finite degree.
Definition:Loop,Loop,"A loop is a part of a plane curve that intersects itself.

Hence it encloses a bounded set of points.
",Definition:Loop (Plane Geometry),"['Definitions/Loops (Plane Geometry)', 'Definitions/Plane Curves']","A loop is a part of a plane curve that intersects itself.

Hence it encloses a bounded set of points.
"
Definition:Loop,Loop,"An algebra loop $\struct {S, \circ}$ is a quasigroup with an identity element.
:$\exists e \in S: \forall x \in S: x \circ e = x = e \circ x$",Definition:Algebra Loop,['Definitions/Abstract Algebra'],"An algebra loop S, ∘ is a quasigroup with an identity element.
:∃ e ∈ S: ∀ x ∈ S: x ∘ e = x = e ∘ x"
Definition:Loop,Loop,"Let $\RR$ be a relation on a set $S$.

Let $a_1, a_2, \ldots a_n$ be elements of $S$.


A relational loop on $S$ takes the form:

:$\tuple {a_1 \mathrel \RR a_2 \land a_2 \mathrel \RR a_3 \dots \land a_{n - 1} \mathrel \RR a_n \land a_n \mathrel \RR a_1}$

That is, it is a subset of $\RR$ of the form:
:$\set {\tuple {a_1, a_2}, \tuple {a_2, a_3}, \ldots, \tuple {a_{n - 1}, a_n}, \tuple {a_n, a_1} }$",Definition:Relational Loop,['Definitions/Relation Theory'],"Let  be a relation on a set S.

Let a_1, a_2, … a_n be elements of S.


A relational loop on S takes the form:

:a_1  a_2  a_2  a_3 … a_n - 1 a_n  a_n  a_1

That is, it is a subset of  of the form:
:a_1, a_2, a_2, a_3, …, a_n - 1, a_n, a_n, a_1"
Definition:Loop,Loop,"Let $T = \struct {S, \tau}$ be a topological space.

Let $\gamma: \closedint 0 1 \to S$ be a path in $T$.


$\gamma$ is a simple loop (in $T$) :
:$\map \gamma {t_1} \ne \map \gamma {t_2}$ for all $t_1 ,t_2 \in \hointr 0 1$ with $t_1 \ne t_2$
:$\map \gamma 0 = \map \gamma 1$
Let $T$ be a topological space.


The set of all loops based at $p \in T$ is denoted by $\map \Omega {T, p}$.
Let $T$ be a topological space.

Let $p \in T$.

Let $\map \Omega {T, p}$ denote the set of all loops based at $p$.


A constant loop $c_p$ is the loop $c_p \in \map \Omega {T, p}$ such that:

:$\forall t \in \closedint 0 1 : \map {c_p} t = p$
Let $T = \struct {S, \tau}$ be a topological space.

Let $\gamma$ be a loop in $T$.

Suppose $\gamma$ is path-homotopic to a constant loop.


Then $\gamma$ is said to be null-homotopic.
Let $T = \struct {S, \tau}$ be a topological space.

Let $\gamma: \closedint 0 1 \to S$ be a loop in $T$.

Let $\Bbb S^1 \subseteq \C$ be the unit circle in $\C$:

:$\Bbb S^1 = \set {z \in \C : \size z = 1}$

Suppose $\omega : \closedint 0 1 \to \Bbb S^1$ such that $\map \omega s = \map \exp {2 \pi i s}$.


Then the unique map $\tilde f : \Bbb S^1 \to T$ such that $\tilde f \circ \omega = f$ is called the circle representative of $f$.
Let $M$ be a topological manifold.

Let $\sigma : \closedint 0 1 \to M$ be a continuous path.

Let $\map \sigma 0 = \map \sigma 1$.


Then $\sigma$ is called a loop.
",Definition:Loop (Topology),"['Definitions/Loops (Topology)', 'Definitions/Topology']","Let T = S, τ be a topological space.

Let γ:  0 1 → S be a path in T.


γ is a simple loop (in T) :
:γt_1γt_2 for all t_1 ,t_2 ∈ 0 1 with t_1  t_2
:γ 0 = γ 1
Let T be a topological space.


The set of all loops based at p ∈ T is denoted by ΩT, p.
Let T be a topological space.

Let p ∈ T.

Let ΩT, p denote the set of all loops based at p.


A constant loop c_p is the loop c_p ∈ΩT, p such that:

:∀ t ∈ 0 1 : c_p t = p
Let T = S, τ be a topological space.

Let γ be a loop in T.

Suppose γ is path-homotopic to a constant loop.


Then γ is said to be null-homotopic.
Let T = S, τ be a topological space.

Let γ:  0 1 → S be a loop in T.

Let S^1 ⊆ be the unit circle in :

:S^1 = z ∈ :  z = 1

Suppose ω :  0 1 → S^1 such that ω s = exp2 π i s.


Then the unique map f̃ :  S^1 → T such that f̃∘ω = f is called the circle representative of f.
Let M be a topological manifold.

Let σ :  0 1 → M be a continuous path.

Let σ 0 = σ 1.


Then σ is called a loop.
"
Definition:Loop,Loop,"Let $M = \struct {S, \mathscr I}$ be a matroid.


A loop of $M$ is an element $x$ of $S$ such that $\set x$ is a dependent subset of $S$.

That is, $x \in S$ is a loop  $\set x \not \in \mathscr I$.
Let $M = \struct {S, \mathscr I}$ be a matroid.


A loop of $M$ is an element $x$ of $S$ such that $\set x$ is a dependent subset of $S$.

That is, $x \in S$ is a loop  $\set x \not \in \mathscr I$.
",Definition:Loop (Matroid),['Definitions/Matroid Theory'],"Let M = S, ℐ be a matroid.


A loop of M is an element x of S such that x is a dependent subset of S.

That is, x ∈ S is a loop  x ∉ℐ.
Let M = S, ℐ be a matroid.


A loop of M is an element x of S such that x is a dependent subset of S.

That is, x ∈ S is a loop  x ∉ℐ.
"
Definition:Lower Bound,Lower Bound,"Let $\struct {S, \preceq}$ be an ordered set.

Let $T$ be a subset of $S$.


A lower bound for $T$ (in $S$) is an element $m \in S$ such that:
:$\forall t \in T: m \preceq t$

That is, $m$ precedes every element of $T$.


=== Subset of Real Numbers ===

The concept is usually encountered where $\struct {S, \preceq}$ is the set of real numbers under the usual ordering $\struct {\R, \le}$:

",Definition:Lower Bound of Set,['Definitions/Boundedness'],"Let S, ≼ be an ordered set.

Let T be a subset of S.


A lower bound for T (in S) is an element m ∈ S such that:
:∀ t ∈ T: m ≼ t

That is, m precedes every element of T.


=== Subset of Real Numbers ===

The concept is usually encountered where S, ≼ is the set of real numbers under the usual ordering , ≤:

"
Definition:Lower Bound,Lower Bound,"Let $\R$ be the set of real numbers.

Let $T$ be a subset of $S$.


A lower bound for $T$ (in $\R$) is an element $m \in \R$ such that:
:$\forall t \in T: m \le t$",Definition:Lower Bound of Set/Real Numbers,['Definitions/Boundedness'],"Let  be the set of real numbers.

Let T be a subset of S.


A lower bound for T (in ) is an element m ∈ such that:
:∀ t ∈ T: m ≤ t"
Definition:Lower Bound,Lower Bound,"Let $f: S \to T$ be a mapping whose codomain is an ordered set $\struct {T, \preceq}$.


Let $f$ be bounded below in $T$ by $H \in T$.


Then $H$ is a lower bound of $f$.


=== Real-Valued Function ===

The concept is usually encountered where $\struct {T, \preceq}$ is the set of real numbers under the usual ordering $\struct {\R, \le}$:

",Definition:Lower Bound of Mapping,['Definitions/Boundedness'],"Let f: S → T be a mapping whose codomain is an ordered set T, ≼.


Let f be bounded below in T by H ∈ T.


Then H is a lower bound of f.


=== Real-Valued Function ===

The concept is usually encountered where T, ≼ is the set of real numbers under the usual ordering , ≤:

"
Definition:Lower Bound,Lower Bound,"When considering the lower bound of a set of numbers, it is commonplace to ignore the set and instead refer just to the number itself.

Thus the construction:

:The set of numbers which fulfil the propositional function $P \left({n}\right)$ is bounded below with the lower bound $N$

would be reported as:

:The number $n$ such that $P \left({n}\right)$ has the lower bound $N$.


This construct obscures the details of what is actually being stated. Its use on  is considered an abuse of notation and so discouraged.


This also applies in the case where it is the lower bound of a mapping which is under discussion.


Category:Definitions/Numbers
",Definition:Lower Bound of Mapping/Real-Valued,['Definitions/Boundedness'],"When considering the lower bound of a set of numbers, it is commonplace to ignore the set and instead refer just to the number itself.

Thus the construction:

:The set of numbers which fulfil the propositional function P (n) is bounded below with the lower bound N

would be reported as:

:The number n such that P (n) has the lower bound N.


This construct obscures the details of what is actually being stated. Its use on  is considered an abuse of notation and so discouraged.


This also applies in the case where it is the lower bound of a mapping which is under discussion.


Category:Definitions/Numbers
"
Definition:Lower Bound,Lower Bound,"Let $f: S \to T$ be a mapping whose codomain is an ordered set $\struct {T, \preceq}$.


Let $f$ be bounded below in $T$ by $H \in T$.


Then $H$ is a lower bound of $f$.


=== Real-Valued Function ===

The concept is usually encountered where $\struct {T, \preceq}$ is the set of real numbers under the usual ordering $\struct {\R, \le}$:


",Definition:Lower Bound of Sequence,['Definitions/Boundedness'],"Let f: S → T be a mapping whose codomain is an ordered set T, ≼.


Let f be bounded below in T by H ∈ T.


Then H is a lower bound of f.


=== Real-Valued Function ===

The concept is usually encountered where T, ≼ is the set of real numbers under the usual ordering , ≤:


"
Definition:Machine,Machine,"A machine, in the context of mechanics, is a physical artefact which takes energy in a particular form and converts it into energy in another form for a specific purpose.
",Definition:Machine (Mechanics),['Definitions/Machines'],"A machine, in the context of mechanics, is a physical artefact which takes energy in a particular form and converts it into energy in another form for a specific purpose.
"
Definition:Machine,Machine,"A finite state machine is an ordered tuple:

: $F = \left({ S, A, I, \Sigma, T }\right)$

where:

: $S$ is the (finite) set of states
: $A \subseteq S$ is the set of accepting states
: $I \in S$ is the initial state
: $\Sigma$ is the alphabet of symbols that can be fed into the machine
: $T : \left({ S \times \Sigma }\right) \rightarrow S$ is the transition function.


A finite state machine operates as follows:

:$(1): \quad$ At the beginning, the current state $s$ of the finite state machine is $I$.
:$(2): \quad$ One by one, the input (a sequence of symbols from $\Sigma$) is fed into the machine.
:$(3): \quad$ After each input symbol $\sigma$, the current state $s$ is set to the result of $T\left({s, \sigma}\right)$.


If, at the end of processing an input word $w$, $s \in A$, the finite state machine is said to accept $w$, otherwise to reject it.


The set of words $w$ accepted by the machine $F$ is called the accepted language $L\left({F}\right)$.

Category:Definitions/Abstract Machines",Definition:Finite State Machine,['Definitions/Abstract Machines'],"A finite state machine is an ordered tuple:

: F = ( S, A, I, Σ, T )

where:

: S is the (finite) set of states
: A ⊆ S is the set of accepting states
: I ∈ S is the initial state
: Σ is the alphabet of symbols that can be fed into the machine
: T : ( S ×Σ) → S is the transition function.


A finite state machine operates as follows:

:(1): At the beginning, the current state s of the finite state machine is I.
:(2): One by one, the input (a sequence of symbols from Σ) is fed into the machine.
:(3): After each input symbol σ, the current state s is set to the result of T(s, σ).


If, at the end of processing an input word w, s ∈ A, the finite state machine is said to accept w, otherwise to reject it.


The set of words w accepted by the machine F is called the accepted language L(F).

Category:Definitions/Abstract Machines"
Definition:Machine,Machine,"An abstract machine is a hypothetical computing machine defined in terms of the operations it performs rather than its internal physical structure.
",Definition:Unlimited Register Machine,"['Definitions/Examples of Abstract Machines', 'Definitions/Unlimited Register Machines']","An abstract machine is a hypothetical computing machine defined in terms of the operations it performs rather than its internal physical structure.
"
Definition:Machine,Machine,"A Turing machine is an abstract machine which works by manipulating symbols on an imaginary piece of paper by means of a specific set of algorithmic rules.

To simplify things, the piece of paper being worked on is in the form of a series of boxes on a one-dimensional ""tape"" divided into squares.

Each square can be either blank or can contain a symbol taken from a finite set, e.g. $s_1, s_2, \ldots, s_\alpha$.


The machine examines one square at a time, and carries out an action determined by both:
:$(1): \quad$ the symbol in the square
:$(2): \quad$ the current internal state of the machine.

The internal state of the machine is a way of providing a device that can keep track of the symbols in other squares.

There can be only a finite set of these states, say $q_1, q_2, \ldots, q_\beta$.


The actions that the machine can take are as follows:
:$(1): \quad$ Replace the symbol in the square with another symbol
:$(2): \quad$ Move to examine the square in the immediate left of the current square being looked at
:$(3): \quad$ Move to examine the square in the immediate right of the current square being looked at.

After carrying out an action, the machine may change to a different internal state.


The program for the machine is a set of instructions which specify:
:$(1): \quad$ what action to take in some possible combinations of the internal state and symbol in the square currently being read
:$(2): \quad$ which internal state the machine moves into after carrying out that action.

Thus the instructions have the following form:
:$q_i \quad s_j \quad A \quad q_t$
which is interpreted as:

""If:
:* the machine is in internal state $q_i$
: the symbol in the square currently being examined is $s_j$
then:
: Carry out action $A$
: Move into internal state $q_t$.


The actions can be abbreviated to:
: $L$: Move one square to the left
: $R$: Move one square to the right
: $s_k$: Replace the symbol in the square currently being read with symbol $s_k$.


The computation stops when there is no instruction which specifies what should be done in the current combination of internal state and symbol being read.

=== Formal Definition ===


A Turing machine is a 7-tuple $\paren {Q, \Sigma, \Gamma, \delta, q_0, B, F}$ that satisfies the following:
* $Q$ is a finite set, the states of the machine.
* $\Sigma$ is a finite set, the input symbols.
* $\Gamma \supsetneq \Sigma$ is a finite superset of the input symbols, called the tape symbols.
** For convenience, we also require that $\Gamma$ and $Q$ are disjoint.
* $\delta : Q \times \Gamma \to Q \times \Gamma \times \set {L, R}$ is a partial mapping, the transition function.
** $L$ and $R$ are arbitrary constants called directions.
* $q_0 \in Q$ is a distinguished state called the start state.
* $B \in \Gamma$ is a distinguished tape symbol called the blank symbol. $B$ must not be an element of $\Sigma$.
* $F \subset Q$ be a designated subset of the states called accepting states.


An instantaneous description of a Turing machine is a finite sequence of elements of $\Gamma \cup Q$, subject to the following conditions:
* There is exactly one element of $Q$ in the sequence.
* The first entry in the sequence is not $B$.
* The last entry in the sequence is not in $Q$.
* If the last entry in the sequence is $B$, then the second-to-last is in $Q$.

By this definition, an instantaneous description can always be written as:
:$X_m X_{m-1} \dotsm X_2 X_1 q Y Z_1 Z_2 \dotsm Z_{n-1} Z_n$
where $m$ or $n$ may be $0$; $X_i$, $Y$, and $Z_j$ are all elements of $\Gamma$; and $q$ is an element of $Q$.

Additionally, $X_m$ and $Z_n$ are not $B$ if they exist; that is, if $m$ and $n$ are not $0$, respectively.


A move reduces one instantaneous description into another by applying the transition function.

We write $A \vdash B$ if a machine with instantaneous description $A$ has, after a single move, instantaneous description $B$.

Let $\map \delta {q, Y} = \paren{q', Y', d}$.

Then there are seven cases to consider:

* If $d = L$ and $m > 0$, and either $n > 0$ or $Y' \neq B$ then:
:$X_m \dotsm X_2 X_1 q Y Z_1 \dotsm Z_n \vdash X_m \dotsm X_2 q' X_1 Y' Z_1 \dotsm Z_n$

* If $d = L$ and $m > 0$, but $n = 0$ and $Y' = B$ then:
:$X_m \dotsm X_2 X_1 q Y \vdash X_m \dotsm X_2 q' X_1$

* If $d = L$ but $m = 0$, and either $n > 0$ or $Y' \neq B$ then:
:$q Y Z_1 \dotsm Z_n \vdash q' B Y' Z_1 \dotsm Z_n$

* If $d = R$ and $n > 0$, and either $m > 0$ or $Y' \neq B$ then:
:$X_m \dotsm X_1 q Y Z_1 Z_2 \dotsm Z_n \vdash X_m \dotsm X_1 Y' q' Z_1 Z_2 \dotsm Z_n$

* If $d = R$ and $n > 0$, but $m = 0$ and $Y' = B$ then:
:$q Y Z_1 Z_2 \dotsm Z_n \vdash q' Z_1 Z_2 \dotsm Z_n$

* If $d = R$ but $n = 0$, and either $m > 0$ or $Y' \neq B$ then:
:$X_m \dotsm X_1 q Y \vdash X_m \dotsm X_1 Y' q' B$

* If $m = 0$, $n = 0$, and $Y' = B$ then regardless of $d$:
:$q Y \vdash q' B$


Let $A \vdash^* B$ indicate that there exists a finite sequence $\sequence {A_i}_{0 \leq i \leq n}$ such that:
:$A = A_0 \vdash A_1 \vdash A_2 \vdash \dotso \vdash A_n = B$


The language $\map L M$ accepted by the machine $M$ is the set of strings $w \in \Sigma^*$ for which, for some $\alpha, \beta \in \Gamma^*$ and $p \in F$:
:$q_0 w \vdash^* \alpha p \beta$

As a special case, the null string is in the language  the above holds for $w = B$.


A machine $M$ halts on an input $w$, using the same special case for the null string as above,  for some $\alpha, \beta \in \Gamma^*$, $X \in \Gamma$, and $q \in Q$:
:$q_0 w \vdash^* \alpha q X \beta$
where $\map \delta {q, X}$ is undefined.
",Definition:Nondeterministic Turing Machine,['Definitions/Turing Machines'],"A Turing machine is an abstract machine which works by manipulating symbols on an imaginary piece of paper by means of a specific set of algorithmic rules.

To simplify things, the piece of paper being worked on is in the form of a series of boxes on a one-dimensional ""tape"" divided into squares.

Each square can be either blank or can contain a symbol taken from a finite set, e.g. s_1, s_2, …, s_α.


The machine examines one square at a time, and carries out an action determined by both:
:(1): the symbol in the square
:(2): the current internal state of the machine.

The internal state of the machine is a way of providing a device that can keep track of the symbols in other squares.

There can be only a finite set of these states, say q_1, q_2, …, q_β.


The actions that the machine can take are as follows:
:(1): Replace the symbol in the square with another symbol
:(2): Move to examine the square in the immediate left of the current square being looked at
:(3): Move to examine the square in the immediate right of the current square being looked at.

After carrying out an action, the machine may change to a different internal state.


The program for the machine is a set of instructions which specify:
:(1): what action to take in some possible combinations of the internal state and symbol in the square currently being read
:(2): which internal state the machine moves into after carrying out that action.

Thus the instructions have the following form:
:q_i    s_j    A    q_t
which is interpreted as:

""If:
:* the machine is in internal state q_i
: the symbol in the square currently being examined is s_j
then:
: Carry out action A
: Move into internal state q_t.


The actions can be abbreviated to:
: L: Move one square to the left
: R: Move one square to the right
: s_k: Replace the symbol in the square currently being read with symbol s_k.


The computation stops when there is no instruction which specifies what should be done in the current combination of internal state and symbol being read.

=== Formal Definition ===


A Turing machine is a 7-tuple Q, Σ, Γ, δ, q_0, B, F that satisfies the following:
* Q is a finite set, the states of the machine.
* Σ is a finite set, the input symbols.
* Γ⊋Σ is a finite superset of the input symbols, called the tape symbols.
** For convenience, we also require that Γ and Q are disjoint.
* δ : Q ×Γ→ Q ×Γ×L, R is a partial mapping, the transition function.
** L and R are arbitrary constants called directions.
* q_0 ∈ Q is a distinguished state called the start state.
* B ∈Γ is a distinguished tape symbol called the blank symbol. B must not be an element of Σ.
* F ⊂ Q be a designated subset of the states called accepting states.


An instantaneous description of a Turing machine is a finite sequence of elements of Γ∪ Q, subject to the following conditions:
* There is exactly one element of Q in the sequence.
* The first entry in the sequence is not B.
* The last entry in the sequence is not in Q.
* If the last entry in the sequence is B, then the second-to-last is in Q.

By this definition, an instantaneous description can always be written as:
:X_m X_m-1… X_2 X_1 q Y Z_1 Z_2 … Z_n-1 Z_n
where m or n may be 0; X_i, Y, and Z_j are all elements of Γ; and q is an element of Q.

Additionally, X_m and Z_n are not B if they exist; that is, if m and n are not 0, respectively.


A move reduces one instantaneous description into another by applying the transition function.

We write A ⊢ B if a machine with instantaneous description A has, after a single move, instantaneous description B.

Let δq, Y = q', Y', d.

Then there are seven cases to consider:

* If d = L and m > 0, and either n > 0 or Y' ≠ B then:
:X_m … X_2 X_1 q Y Z_1 … Z_n ⊢ X_m … X_2 q' X_1 Y' Z_1 … Z_n

* If d = L and m > 0, but n = 0 and Y' = B then:
:X_m … X_2 X_1 q Y ⊢ X_m … X_2 q' X_1

* If d = L but m = 0, and either n > 0 or Y' ≠ B then:
:q Y Z_1 … Z_n ⊢ q' B Y' Z_1 … Z_n

* If d = R and n > 0, and either m > 0 or Y' ≠ B then:
:X_m … X_1 q Y Z_1 Z_2 … Z_n ⊢ X_m … X_1 Y' q' Z_1 Z_2 … Z_n

* If d = R and n > 0, but m = 0 and Y' = B then:
:q Y Z_1 Z_2 … Z_n ⊢ q' Z_1 Z_2 … Z_n

* If d = R but n = 0, and either m > 0 or Y' ≠ B then:
:X_m … X_1 q Y ⊢ X_m … X_1 Y' q' B

* If m = 0, n = 0, and Y' = B then regardless of d:
:q Y ⊢ q' B


Let A ⊢^* B indicate that there exists a finite sequence A_i_0 ≤ i ≤ n such that:
:A = A_0 ⊢ A_1 ⊢ A_2 ⊢…⊢ A_n = B


The language L M accepted by the machine M is the set of strings w ∈Σ^* for which, for some α, β∈Γ^* and p ∈ F:
:q_0 w ⊢^* α p β

As a special case, the null string is in the language  the above holds for w = B.


A machine M halts on an input w, using the same special case for the null string as above,  for some α, β∈Γ^*, X ∈Γ, and q ∈ Q:
:q_0 w ⊢^* α q X β
where δq, X is undefined.
"
Definition:Major,Major,"Let $a, b \in \R_{>0}$ in the forms:
:$a = \dfrac \rho {\sqrt 2} \sqrt {1 + \dfrac k {\sqrt {1 + k^2} } }$
:$b = \dfrac \rho {\sqrt 2} \sqrt {1 - \dfrac k {\sqrt {1 + k^2} } }$

where:
: $\rho$ is a rational number
: $k$ is a rational number whose square root is irrational.


Then $a + b$ is a major.


",Definition:Major (Euclidean),['Definitions/Euclidean Number Theory'],"Let a, b ∈_>0 in the forms:
:a = ρ√(2)√(1 +  k √(1 + k^2))
:b = ρ√(2)√(1 -  k √(1 + k^2))

where:
: ρ is a rational number
: k is a rational number whose square root is irrational.


Then a + b is a major.


"
Definition:Major,Major,,Definition:Hyperbola/Major Axis,"['Definitions/Major Axis', 'Definitions/Hyperbolas']",
Definition:Major,Major,"Consider the lemniscate of Bernoulli defined as the locus $M$ described by the equation:
:$P_1 M \times P_2 M = \paren {\dfrac {P_1 P_2} 2}^2$


:


The line $P_1 P_2$ is the major axis of the lemniscate.


Category:Definitions/Lemniscate of Bernoulli
",Definition:Lemniscate of Bernoulli/Major Axis,['Definitions/Lemniscate of Bernoulli'],"Consider the lemniscate of Bernoulli defined as the locus M described by the equation:
:P_1 M × P_2 M = P_1 P_2 2^2


:


The line P_1 P_2 is the major axis of the lemniscate.


Category:Definitions/Lemniscate of Bernoulli
"
Definition:Major,Major,"Consider the lemniscate of Bernoulli defined as the locus $M$ described by the equation:
:$P_1 M \times P_2 M = \paren {\dfrac {P_1 P_2} 2}^2$
where $O$ is the point at the center where the branches cross.


:


Each of the lines $O P_1$ and $O P_2$ is a major semiaxis of the lemniscate.


Category:Definitions/Lemniscate of Bernoulli
",Definition:Lemniscate of Bernoulli/Major Semiaxis,['Definitions/Lemniscate of Bernoulli'],"Consider the lemniscate of Bernoulli defined as the locus M described by the equation:
:P_1 M × P_2 M = P_1 P_2 2^2
where O is the point at the center where the branches cross.


:


Each of the lines O P_1 and O P_2 is a major semiaxis of the lemniscate.


Category:Definitions/Lemniscate of Bernoulli
"
Definition:Major,Major,"The major premise of a categorical syllogism is conventionally stated first.

It is a categorical statement which expresses the logical relationship between the primary term and the middle term of the syllogism.",Definition:Categorical Syllogism/Premises/Major Premise,['Definitions/Categorical Syllogisms'],"The major premise of a categorical syllogism is conventionally stated first.

It is a categorical statement which expresses the logical relationship between the primary term and the middle term of the syllogism."
Definition:Manifold,Manifold,"Let $M$ be a second-countable locally Euclidean space of dimension $d$. 

Let $\mathscr F$ be a $d$-dimensional differentiable structure on $M$ of class $\CC^k$, where $k \ge 1$.


Then $\struct {M, \mathscr F}$ is a differentiable manifold of class $\CC^k$ and dimension $d$.",Definition:Topological Manifold/Differentiable Manifold,"['Definitions/Topological Manifolds', 'Definitions/Differentiable Manifolds']","Let M be a second-countable locally Euclidean space of dimension d. 

Let ℱ be a d-dimensional differentiable structure on M of class ^k, where k ≥ 1.


Then M, ℱ is a differentiable manifold of class ^k and dimension d."
Definition:Manifold,Manifold,"Let $M$ be a second-countable locally Euclidean space of dimension $d$. 

Let $\mathscr F$ be a smooth differentiable structure on $M$.


Then $\struct {M, \mathscr F}$ is called a smooth manifold of dimension $d$.
",Definition:Topological Manifold/Smooth Manifold,"['Definitions/Smooth Manifolds', 'Definitions/Topological Manifolds', 'Definitions/Differentiable Manifolds']","Let M be a second-countable locally Euclidean space of dimension d. 

Let ℱ be a smooth differentiable structure on M.


Then M, ℱ is called a smooth manifold of dimension d.
"
Definition:Manifold,Manifold,"Let $M$ be a second-countable, complex locally Euclidean space of dimension $d$. 

Let $\mathscr F$ be a complex analytic differentiable structure on $M$.


Then $\struct {M, \mathscr F}$ is called a complex manifold of dimension $d$.",Definition:Topological Manifold/Complex Manifold,['Definitions/Topological Manifolds'],"Let M be a second-countable, complex locally Euclidean space of dimension d. 

Let ℱ be a complex analytic differentiable structure on M.


Then M, ℱ is called a complex manifold of dimension d."
Definition:Manifold,Manifold,"Let $K$ be a division ring.

Let $\left({S, +, \circ}\right)_K$ be a $K$-algebraic structure with one operation.


Let $T$ be a closed subset of $S$.

Let $\left({T, +_T, \circ_T}\right)_K$ be a $K$-vector space where:
: $+_T$ is the restriction of $+$ to $T \times T$ and
: $\circ_T$ is the restriction of $\circ$ to $K \times T$.


Then $\left({T, +_T, \circ_T}\right)_K$ is a (vector) subspace of $\left({S, +, \circ}\right)_K$.


When considering Hilbert spaces, one wants to deal with projections onto subspaces.

These projections however require the linear subspace to be closed in topological sense in order to be well-defined.

Therefore, in treatises of Hilbert spaces, one encounters the terminology linear manifold for the concept of vector subspace defined above.

The adapted definition of linear subspace is then that it is a topologically closed linear manifold.",Definition:Vector Subspace/Hilbert Spaces,"['Definitions/Linear Algebra', 'Definitions/Hilbert Spaces']","Let K be a division ring.

Let (S, +, ∘)_K be a K-algebraic structure with one operation.


Let T be a closed subset of S.

Let (T, +_T, ∘_T)_K be a K-vector space where:
: +_T is the restriction of + to T × T and
: ∘_T is the restriction of ∘ to K × T.


Then (T, +_T, ∘_T)_K is a (vector) subspace of (S, +, ∘)_K.


When considering Hilbert spaces, one wants to deal with projections onto subspaces.

These projections however require the linear subspace to be closed in topological sense in order to be well-defined.

Therefore, in treatises of Hilbert spaces, one encounters the terminology linear manifold for the concept of vector subspace defined above.

The adapted definition of linear subspace is then that it is a topologically closed linear manifold."
Definition:Mean,Mean,"Let $x_1, x_2, \ldots, x_n \in \R$ be real numbers.

The arithmetic mean of $x_1, x_2, \ldots, x_n$ is defined as:

:$\ds A_n := \dfrac 1 n \sum_{k \mathop = 1}^n x_k$

That is, to find out the arithmetic mean of a set of numbers, add them all up and divide by how many there are.
Let $x_1, x_2, \ldots, x_n \in \R$ be real numbers.

The arithmetic mean of $x_1, x_2, \ldots, x_n$ is defined as:

:$\ds A_n := \dfrac 1 n \sum_{k \mathop = 1}^n x_k$

That is, to find out the arithmetic mean of a set of numbers, add them all up and divide by how many there are.
",Definition:Arithmetic Mean,"['Definitions/Arithmetic Mean', 'Definitions/Pythagorean Means', 'Definitions/Algebra', 'Definitions/Measures of Central Tendency']","Let x_1, x_2, …, x_n ∈ be real numbers.

The arithmetic mean of x_1, x_2, …, x_n is defined as:

:A_n :=  1 n ∑_k  = 1^n x_k

That is, to find out the arithmetic mean of a set of numbers, add them all up and divide by how many there are.
Let x_1, x_2, …, x_n ∈ be real numbers.

The arithmetic mean of x_1, x_2, …, x_n is defined as:

:A_n :=  1 n ∑_k  = 1^n x_k

That is, to find out the arithmetic mean of a set of numbers, add them all up and divide by how many there are.
"
Definition:Mean,Mean,"Let $x_1, x_2, \ldots, x_n \in \R_{>0}$ be (strictly) positive real numbers.

The geometric mean of $x_1, x_2, \ldots, x_n$ is defined as:

:$\ds G_n := \paren {\prod_{k \mathop = 1}^n x_k}^{1/n}$


That is, to find out the geometric mean of a set of $n$ numbers, multiply them together and take the $n$th root.


=== Mean Proportional ===

Let $x_1, x_2, \ldots, x_n \in \R_{>0}$ be (strictly) positive real numbers.

The geometric mean of $x_1, x_2, \ldots, x_n$ is defined as:

:$\ds G_n := \paren {\prod_{k \mathop = 1}^n x_k}^{1/n}$


That is, to find out the geometric mean of a set of $n$ numbers, multiply them together and take the $n$th root.


=== Mean Proportional ===

In the language of , the geometric mean of two magnitudes is called the mean proportional.

Thus the mean proportional of $a$ and $b$ is defined as that magnitude $c$ such that:
:$a : c = c : b$
where $a : c$ denotes the ratio between $a$ and $c$.


From the definition of ratio it is seen that $\dfrac a c = \dfrac c b$ from which it follows that $c = \sqrt {a b}$ demonstrating that the definitions are logically equivalent.


=== General Definition ===

",Definition:Geometric Mean,"['Definitions/Geometric Mean', 'Definitions/Pythagorean Means', 'Definitions/Algebra', 'Definitions/Measures of Central Tendency']","Let x_1, x_2, …, x_n ∈_>0 be (strictly) positive real numbers.

The geometric mean of x_1, x_2, …, x_n is defined as:

:G_n := ∏_k  = 1^n x_k^1/n


That is, to find out the geometric mean of a set of n numbers, multiply them together and take the nth root.


=== Mean Proportional ===

Let x_1, x_2, …, x_n ∈_>0 be (strictly) positive real numbers.

The geometric mean of x_1, x_2, …, x_n is defined as:

:G_n := ∏_k  = 1^n x_k^1/n


That is, to find out the geometric mean of a set of n numbers, multiply them together and take the nth root.


=== Mean Proportional ===

In the language of , the geometric mean of two magnitudes is called the mean proportional.

Thus the mean proportional of a and b is defined as that magnitude c such that:
:a : c = c : b
where a : c denotes the ratio between a and c.


From the definition of ratio it is seen that a c =  c b from which it follows that c = √(a b) demonstrating that the definitions are logically equivalent.


=== General Definition ===

"
Definition:Mean,Mean,"The arithmetic-geometric mean of two numbers $a$ and $b$ is the limit of the sequences obtained by the arithmetic-geometric mean iteration.

This is denoted $\map M {a, b}$.


=== Arithmetic-Geometric Mean Iteration ===

Let $a$ and $b$ be numbers.

Let $\sequence {a_n}$ and $\sequence {b_n}$ be defined as the recursive sequences:







where:






The above process is known as the arithmetic-geometric mean iteration.
Let $a$ and $b$ be numbers.

Let $\sequence {a_n}$ and $\sequence {b_n}$ be defined as the recursive sequences:







where:






The above process is known as the arithmetic-geometric mean iteration.
",Definition:Arithmetic-Geometric Mean,"['Definitions/Arithmetic-Geometric Mean', 'Definitions/Measures of Central Tendency']","The arithmetic-geometric mean of two numbers a and b is the limit of the sequences obtained by the arithmetic-geometric mean iteration.

This is denoted M a, b.


=== Arithmetic-Geometric Mean Iteration ===

Let a and b be numbers.

Let a_n and b_n be defined as the recursive sequences:







where:






The above process is known as the arithmetic-geometric mean iteration.
Let a and b be numbers.

Let a_n and b_n be defined as the recursive sequences:







where:






The above process is known as the arithmetic-geometric mean iteration.
"
Definition:Mean,Mean,"Let $x_1, x_2, \ldots, x_n \in \R$ be real numbers which are all strictly positive.

The harmonic mean $H_n$ of $x_1, x_2, \ldots, x_n$ is defined as:

:$\ds \dfrac 1 {H_n} := \frac 1 n \paren {\sum_{k \mathop = 1}^n \frac 1 {x_k} }$

That is, to find the harmonic mean of a set of $n$ numbers, take the reciprocal of the arithmetic mean of their reciprocals.
Let $x_1, x_2, \ldots, x_n \in \R$ be real numbers which are all strictly positive.

The harmonic mean $H_n$ of $x_1, x_2, \ldots, x_n$ is defined as:

:$\ds \dfrac 1 {H_n} := \frac 1 n \paren {\sum_{k \mathop = 1}^n \frac 1 {x_k} }$

That is, to find the harmonic mean of a set of $n$ numbers, take the reciprocal of the arithmetic mean of their reciprocals.
Let $x_1, x_2, \ldots, x_n \in \R$ be real numbers.

The arithmetic mean of $x_1, x_2, \ldots, x_n$ is defined as:

:$\ds A_n := \dfrac 1 n \sum_{k \mathop = 1}^n x_k$

That is, to find out the arithmetic mean of a set of numbers, add them all up and divide by how many there are.
",Definition:Harmonic Mean,"['Definitions/Harmonic Mean', 'Definitions/Pythagorean Means', 'Definitions/Measures of Central Tendency', 'Definitions/Algebra', 'Definitions/Number Theory', 'Definitions/Analysis']","Let x_1, x_2, …, x_n ∈ be real numbers which are all strictly positive.

The harmonic mean H_n of x_1, x_2, …, x_n is defined as:

:1 H_n := 1/n∑_k  = 1^n 1/x_k

That is, to find the harmonic mean of a set of n numbers, take the reciprocal of the arithmetic mean of their reciprocals.
Let x_1, x_2, …, x_n ∈ be real numbers which are all strictly positive.

The harmonic mean H_n of x_1, x_2, …, x_n is defined as:

:1 H_n := 1/n∑_k  = 1^n 1/x_k

That is, to find the harmonic mean of a set of n numbers, take the reciprocal of the arithmetic mean of their reciprocals.
Let x_1, x_2, …, x_n ∈ be real numbers.

The arithmetic mean of x_1, x_2, …, x_n is defined as:

:A_n :=  1 n ∑_k  = 1^n x_k

That is, to find out the arithmetic mean of a set of numbers, add them all up and divide by how many there are.
"
Definition:Mean,Mean,"Let $x_1, x_2, \ldots, x_n \in \R$ be real numbers.

The quadratic mean of $x_1, x_2, \ldots, x_n$ is defined as:

:$Q_n := \ds \sqrt {\frac 1 n \sum_{k \mathop = 1}^n x_k^2}$",Definition:Quadratic Mean,"['Definitions/Algebra', 'Definitions/Measures of Central Tendency']","Let x_1, x_2, …, x_n ∈ be real numbers.

The quadratic mean of x_1, x_2, …, x_n is defined as:

:Q_n := √(1/n∑_k  = 1^n x_k^2)"
Definition:Mean,Mean,"Let $S = \sequence {x_1, x_2, \ldots, x_n}$ be a sequence of real numbers.

Let $W$ be a weight function to be applied to the terms of $S$.


The weighted mean of $S$  $W$ is defined as:
:$\bar x := \dfrac {\ds \sum_{i \mathop = 1}^n \map W {x_i} x_i} {\ds \sum_{i \mathop = 1}^n \map W {x_i} }$

This means that elements of $S$ with a larger weight contribute more to the weighted mean than those with a smaller weight.


If we write:
:$\forall i: 1 \le i \le n: w_i = \map W {x_i}$
we can write this weighted mean as:
:$\bar x := \dfrac {w_1 x_1 + w_2 x_2 + \cdots + w_n x_n} {w_1 + w_2 + \cdots + w_n}$


From the definition of the weight function, none of the weights can be negative.

While some of the weights may be zero, not all of them can, otherwise we would be dividing by zero.


=== Normalized Weighted Mean ===

Let $S = \sequence {x_1, x_2, \ldots, x_n}$ be a sequence of real numbers.

Let $W$ be a weight function to be applied to the terms of $S$.


The weighted mean of $S$  $W$ is defined as:
:$\bar x := \dfrac {\ds \sum_{i \mathop = 1}^n \map W {x_i} x_i} {\ds \sum_{i \mathop = 1}^n \map W {x_i} }$

This means that elements of $S$ with a larger weight contribute more to the weighted mean than those with a smaller weight.


If we write:
:$\forall i: 1 \le i \le n: w_i = \map W {x_i}$
we can write this weighted mean as:
:$\bar x := \dfrac {w_1 x_1 + w_2 x_2 + \cdots + w_n x_n} {w_1 + w_2 + \cdots + w_n}$


From the definition of the weight function, none of the weights can be negative.

While some of the weights may be zero, not all of them can, otherwise we would be dividing by zero.


=== Normalized Weighted Mean ===

Let $S = \sequence {x_1, x_2, \ldots, x_n}$ be a sequence of real numbers.

Let $W$ be a weight function to be applied to the terms of $S$.


The weighted mean of $S$  $W$ is defined as:
:$\bar x := \dfrac {\ds \sum_{i \mathop = 1}^n \map W {x_i} x_i} {\ds \sum_{i \mathop = 1}^n \map W {x_i} }$

This means that elements of $S$ with a larger weight contribute more to the weighted mean than those with a smaller weight.


If we write:
:$\forall i: 1 \le i \le n: w_i = \map W {x_i}$
we can write this weighted mean as:
:$\bar x := \dfrac {w_1 x_1 + w_2 x_2 + \cdots + w_n x_n} {w_1 + w_2 + \cdots + w_n}$


From the definition of the weight function, none of the weights can be negative.

While some of the weights may be zero, not all of them can, otherwise we would be dividing by zero.


=== Normalized Weighted Mean ===

Let $S = \sequence {x_1, x_2, \ldots, x_n}$ be a sequence of real numbers.

Let $\map W x$ be a weight function to be applied to the terms of $S$.

Let the weights be normalized.

Then the weighted mean of $S$  $W$ can be expressed in the form:
:$\ds \bar x := \sum_{i \mathop = 1}^n \map W {x_i} x_i$

as by definition of normalized weight function all the weights add up to $1$.

This weighted mean is known as a  normalized weighted mean.



",Definition:Weighted Mean,"['Definitions/Weighted Means', 'Definitions/Measures of Central Tendency', 'Definitions/Algebra']","Let S = x_1, x_2, …, x_n be a sequence of real numbers.

Let W be a weight function to be applied to the terms of S.


The weighted mean of S  W is defined as:
:x̅ := ∑_i  = 1^n  W x_i x_i∑_i  = 1^n  W x_i

This means that elements of S with a larger weight contribute more to the weighted mean than those with a smaller weight.


If we write:
:∀ i: 1 ≤ i ≤ n: w_i =  W x_i
we can write this weighted mean as:
:x̅ := w_1 x_1 + w_2 x_2 + ⋯ + w_n x_nw_1 + w_2 + ⋯ + w_n


From the definition of the weight function, none of the weights can be negative.

While some of the weights may be zero, not all of them can, otherwise we would be dividing by zero.


=== Normalized Weighted Mean ===

Let S = x_1, x_2, …, x_n be a sequence of real numbers.

Let W be a weight function to be applied to the terms of S.


The weighted mean of S  W is defined as:
:x̅ := ∑_i  = 1^n  W x_i x_i∑_i  = 1^n  W x_i

This means that elements of S with a larger weight contribute more to the weighted mean than those with a smaller weight.


If we write:
:∀ i: 1 ≤ i ≤ n: w_i =  W x_i
we can write this weighted mean as:
:x̅ := w_1 x_1 + w_2 x_2 + ⋯ + w_n x_nw_1 + w_2 + ⋯ + w_n


From the definition of the weight function, none of the weights can be negative.

While some of the weights may be zero, not all of them can, otherwise we would be dividing by zero.


=== Normalized Weighted Mean ===

Let S = x_1, x_2, …, x_n be a sequence of real numbers.

Let W be a weight function to be applied to the terms of S.


The weighted mean of S  W is defined as:
:x̅ := ∑_i  = 1^n  W x_i x_i∑_i  = 1^n  W x_i

This means that elements of S with a larger weight contribute more to the weighted mean than those with a smaller weight.


If we write:
:∀ i: 1 ≤ i ≤ n: w_i =  W x_i
we can write this weighted mean as:
:x̅ := w_1 x_1 + w_2 x_2 + ⋯ + w_n x_nw_1 + w_2 + ⋯ + w_n


From the definition of the weight function, none of the weights can be negative.

While some of the weights may be zero, not all of them can, otherwise we would be dividing by zero.


=== Normalized Weighted Mean ===

Let S = x_1, x_2, …, x_n be a sequence of real numbers.

Let W x be a weight function to be applied to the terms of S.

Let the weights be normalized.

Then the weighted mean of S  W can be expressed in the form:
:x̅ := ∑_i  = 1^n  W x_i x_i

as by definition of normalized weight function all the weights add up to 1.

This weighted mean is known as a  normalized weighted mean.



"
Definition:Mean,Mean,"Let $S = \sequence {x_1, x_2, \ldots, x_n}$ be a sequence of real numbers.

Let $W$ be a weight function to be applied to the terms of $S$.


The weighted mean of $S$  $W$ is defined as:
:$\bar x := \dfrac {\ds \sum_{i \mathop = 1}^n \map W {x_i} x_i} {\ds \sum_{i \mathop = 1}^n \map W {x_i} }$

This means that elements of $S$ with a larger weight contribute more to the weighted mean than those with a smaller weight.


If we write:
:$\forall i: 1 \le i \le n: w_i = \map W {x_i}$
we can write this weighted mean as:
:$\bar x := \dfrac {w_1 x_1 + w_2 x_2 + \cdots + w_n x_n} {w_1 + w_2 + \cdots + w_n}$


From the definition of the weight function, none of the weights can be negative.

While some of the weights may be zero, not all of them can, otherwise we would be dividing by zero.


=== Normalized Weighted Mean ===

Let $x_1, x_2, \ldots, x_n \in \R$ be real numbers.

The arithmetic mean of $x_1, x_2, \ldots, x_n$ is defined as:

:$\ds A_n := \dfrac 1 n \sum_{k \mathop = 1}^n x_k$

That is, to find out the arithmetic mean of a set of numbers, add them all up and divide by how many there are.
Let $x_1, x_2, \ldots, x_n \in \R_{>0}$ be (strictly) positive real numbers.

The geometric mean of $x_1, x_2, \ldots, x_n$ is defined as:

:$\ds G_n := \paren {\prod_{k \mathop = 1}^n x_k}^{1/n}$


That is, to find out the geometric mean of a set of $n$ numbers, multiply them together and take the $n$th root.


=== Mean Proportional ===

",Definition:Heronian Mean,['Definitions/Algebra'],"Let S = x_1, x_2, …, x_n be a sequence of real numbers.

Let W be a weight function to be applied to the terms of S.


The weighted mean of S  W is defined as:
:x̅ := ∑_i  = 1^n  W x_i x_i∑_i  = 1^n  W x_i

This means that elements of S with a larger weight contribute more to the weighted mean than those with a smaller weight.


If we write:
:∀ i: 1 ≤ i ≤ n: w_i =  W x_i
we can write this weighted mean as:
:x̅ := w_1 x_1 + w_2 x_2 + ⋯ + w_n x_nw_1 + w_2 + ⋯ + w_n


From the definition of the weight function, none of the weights can be negative.

While some of the weights may be zero, not all of them can, otherwise we would be dividing by zero.


=== Normalized Weighted Mean ===

Let x_1, x_2, …, x_n ∈ be real numbers.

The arithmetic mean of x_1, x_2, …, x_n is defined as:

:A_n :=  1 n ∑_k  = 1^n x_k

That is, to find out the arithmetic mean of a set of numbers, add them all up and divide by how many there are.
Let x_1, x_2, …, x_n ∈_>0 be (strictly) positive real numbers.

The geometric mean of x_1, x_2, …, x_n is defined as:

:G_n := ∏_k  = 1^n x_k^1/n


That is, to find out the geometric mean of a set of n numbers, multiply them together and take the nth root.


=== Mean Proportional ===

"
Definition:Mean,Mean,"Let $f$ be an integrable function on some closed interval $\closedint a b$.

The mean value of $f$ on $\closedint a b$ is defined as:

:$\ds \frac 1 {b - a} \int_a^b \map f x \rd x$
",Definition:Mean Value of Function,"['Definitions/Mean Value of Function', 'Definitions/Integral Calculus', 'Definitions/Measures of Central Tendency']","Let f be an integrable function on some closed interval a b.

The mean value of f on a b is defined as:

:1/b - a∫_a^b  f x  x
"
Definition:Mean,Mean,"The mean squared error is the expected value of the square of the difference between an estimator $T$ and the true parameter value $\theta$.
",Definition:Mean Squared Error,"['Definitions/Mean Squared Error', 'Definitions/Variance', 'Definitions/Statistics']","The mean squared error is the expected value of the square of the difference between an estimator T and the true parameter value θ.
"
Definition:Median,Median,"Let $\triangle ABC$ be a triangle.

:

A median is a cevian which bisects the opposite.


In the above diagram, $CD$ is a median.
Let $\triangle ABC$ be a triangle.

:

A median is a cevian which bisects the opposite.


In the above diagram, $CD$ is a median.
",Definition:Median of Triangle,"['Definitions/Medians of Triangles', 'Definitions/Cevians', 'Definitions/Triangles']","Let ABC be a triangle.

:

A median is a cevian which bisects the opposite.


In the above diagram, CD is a median.
Let ABC be a triangle.

:

A median is a cevian which bisects the opposite.


In the above diagram, CD is a median.
"
Definition:Median,Median,"Let $\Box ABCD$ be a trapezium.

:

The median of $\Box ABCD$ is the straight line through the midpoints of the legs of $\Box ABCD$.


In the above diagram, $EF$ is the median of the trapezium $\Box ABCD$.
Let $\Box ABCD$ be a trapezium.

:

The median of $\Box ABCD$ is the straight line through the midpoints of the legs of $\Box ABCD$.


In the above diagram, $EF$ is the median of the trapezium $\Box ABCD$.
",Definition:Median of Trapezium,"['Definitions/Medians of Trapezia', 'Definitions/Trapezia']","Let ABCD be a trapezium.

:

The median of ABCD is the straight line through the midpoints of the legs of ABCD.


In the above diagram, EF is the median of the trapezium ABCD.
Let ABCD be a trapezium.

:

The median of ABCD is the straight line through the midpoints of the legs of ABCD.


In the above diagram, EF is the median of the trapezium ABCD.
"
Definition:Median,Median,"Let $S$ be a set of quantitative data.

Let $S$ be arranged in order of size.

The median is the element of $S$ that is in the middle of that ordered set.


Suppose there are an odd number of elements of $S$ such that $S$ has cardinality $2 n - 1$.

The median of $S$ in that case is the $n$th element of $S$.


Suppose there are an even number of elements of $S$ such that $S$ has cardinality $2 n$.

Then the middle of $S$ is not well-defined, and so the median of $S$ in that case is the arithmetic mean of the $n$th and $n + 1$th elements of $S$.


=== Continuous Random Variable ===
",Definition:Median (Statistics),"['Definitions/Medians', 'Definitions/Measures of Central Tendency']","Let S be a set of quantitative data.

Let S be arranged in order of size.

The median is the element of S that is in the middle of that ordered set.


Suppose there are an odd number of elements of S such that S has cardinality 2 n - 1.

The median of S in that case is the nth element of S.


Suppose there are an even number of elements of S such that S has cardinality 2 n.

Then the middle of S is not well-defined, and so the median of S in that case is the arithmetic mean of the nth and n + 1th elements of S.


=== Continuous Random Variable ===
"
Definition:Median,Median,"Let $X$ be a continuous random variable on a probability space $\struct {\Omega, \Sigma, \Pr}$.

Let $X$ have probability density function $f_X$. 

A median of $X$ is defined as a real number $m_X$ such that: 

:$\ds \map \Pr {X < m_X} = \int_{-\infty}^{m_X} \map {f_X} x \rd x = \frac 1 2$

That is, $m_X$ is the first $2$-quantile of $X$.",Definition:Median of Continuous Random Variable,['Definitions/Medians'],"Let X be a continuous random variable on a probability space Ω, Σ,.

Let X have probability density function f_X. 

A median of X is defined as a real number m_X such that: 

:X < m_X = ∫_-∞^m_Xf_X x  x = 1/2

That is, m_X is the first 2-quantile of X."
Definition:Meet,Meet,"Let $\struct {S, \preceq}$ be an ordered set.

Let $a, b \in S$, and suppose that their infimum $\inf \set {a, b}$ exists in $S$.


Then $a \wedge b$, the meet of $a$ and $b$, is defined as:

:$a \wedge b = \inf \set {a, b}$


Expanding the definition of infimum, one sees that $c = a \wedge b$ :

:$(1): \quad c \preceq a$ and $c \preceq b$
:$(2): \quad \forall s \in S: s \preceq a$ and $s \preceq b \implies s \preceq c$",Definition:Meet (Order Theory),"['Definitions/Order Theory', 'Definitions/Lattice Theory']","Let S, ≼ be an ordered set.

Let a, b ∈ S, and suppose that their infimum infa, b exists in S.


Then a ∧ b, the meet of a and b, is defined as:

:a ∧ b = infa, b


Expanding the definition of infimum, one sees that c = a ∧ b :

:(1):    c ≼ a and c ≼ b
:(2):   ∀ s ∈ S: s ≼ a and s ≼ b  s ≼ c"
Definition:Meet,Meet,"Consider the Boolean algebra $\struct {S, \vee, \wedge, \neg}$


The operation $\wedge$ is called meet.
",Definition:Boolean Algebra/Meet,['Definitions/Boolean Algebras'],"Consider the Boolean algebra S, ∨, ∧,


The operation ∧ is called meet.
"
Definition:Meet,Meet,"Let $\family {S_i}_{i \mathop \in I}$ be an family of sets indexed by some  indexing set $I$.

The sets in $\family {S_i}$ are said to meet  their intersection is not empty.


That is, :
:$\ds \bigcap_{i \mathop \in I} \family {S_i} \ne \O$


That is,  $\family {S_i}_{i \mathop \in I}$ is not disjoint.",Definition:Set Meeting Set,['Definitions/Set Intersection'],"Let S_i_i ∈ I be an family of sets indexed by some  indexing set I.

The sets in S_i are said to meet  their intersection is not empty.


That is, :
:⋂_i ∈ IS_iØ


That is,  S_i_i ∈ I is not disjoint."
Definition:Mercury,Mercury,"Mercury is the innermost planet of the solar system.

Its orbit lies within that of Venus.

Category:Definitions/Solar System",Definition:Mercury (Planet),['Definitions/Solar System'],"Mercury is the innermost planet of the solar system.

Its orbit lies within that of Venus.

Category:Definitions/Solar System"
Definition:Mercury,Mercury,,Definition:Mercury (Chemical Element),"['Definitions/Mercury', 'Definitions/Examples of Metals', 'Definitions/Examples of Chemical Elements']",
Definition:Minor,Minor,"Let $a, b \in \R_{>0}$ in the forms:
:$a = \dfrac \rho {\sqrt 2} \sqrt {1 + \dfrac k {\sqrt {1 + k^2} } }$
:$b = \dfrac \rho {\sqrt 2} \sqrt {1 - \dfrac k {\sqrt {1 + k^2} } }$

where:
:$\rho$ is a rational number
:$k$ is a rational number whose square root is irrational.


Then $a - b$ is a minor.


",Definition:Minor (Euclidean),['Definitions/Euclidean Number Theory'],"Let a, b ∈_>0 in the forms:
:a = ρ√(2)√(1 +  k √(1 + k^2))
:b = ρ√(2)√(1 -  k √(1 + k^2))

where:
:ρ is a rational number
:k is a rational number whose square root is irrational.


Then a - b is a minor.


"
Definition:Minor,Minor,,Definition:Hyperbola/Minor Axis,"['Definitions/Minor Axis', 'Definitions/Hyperbolas']",
Definition:Minor,Minor,"The minor premise of a categorical syllogism is conventionally stated second.

It is a categorical statement which expresses the logical relationship between the secondary term and the middle term of the syllogism.",Definition:Categorical Syllogism/Premises/Minor Premise,['Definitions/Categorical Syllogisms'],"The minor premise of a categorical syllogism is conventionally stated second.

It is a categorical statement which expresses the logical relationship between the secondary term and the middle term of the syllogism."
Definition:Minor,Minor,"Let $\mathbf A = \sqbrk a_n$ be a square matrix of order $n$.

Consider the order $k$ square submatrix $\mathbf B$ obtained by deleting $n - k$ rows and $n - k$ columns from $\mathbf A$.


Let $\map \det {\mathbf B}$ denote the determinant of $\mathbf B$.

Then $\map \det {\mathbf B}$ is an order-$k$ minor of $\map \det {\mathbf A}$.


Thus a minor is a determinant formed from the elements (in the same relative order) of $k$ specified rows and columns.
Let $\mathbf A = \sqbrk a_n$ be a square matrix of order $n$.

Consider the order $k$ square submatrix $\mathbf B$ obtained by deleting $n - k$ rows and $n - k$ columns from $\mathbf A$.


Let $\map \det {\mathbf B}$ denote the determinant of $\mathbf B$.

Then $\map \det {\mathbf B}$ is an order-$k$ minor of $\map \det {\mathbf A}$.


Thus a minor is a determinant formed from the elements (in the same relative order) of $k$ specified rows and columns.
",Definition:Minor of Determinant,['Definitions/Determinants'],"Let 𝐀 =  a_n be a square matrix of order n.

Consider the order k square submatrix 𝐁 obtained by deleting n - k rows and n - k columns from 𝐀.


Let 𝐁 denote the determinant of 𝐁.

Then 𝐁 is an order-k minor of 𝐀.


Thus a minor is a determinant formed from the elements (in the same relative order) of k specified rows and columns.
Let 𝐀 =  a_n be a square matrix of order n.

Consider the order k square submatrix 𝐁 obtained by deleting n - k rows and n - k columns from 𝐀.


Let 𝐁 denote the determinant of 𝐁.

Then 𝐁 is an order-k minor of 𝐀.


Thus a minor is a determinant formed from the elements (in the same relative order) of k specified rows and columns.
"
Definition:Model,Model,"Let $\mathscr M$ be a formal semantics for a logical language $\LL$.

Let $\MM$ be a structure of $\mathscr M$.


Let $\phi$ be a logical formula of $\LL$.

Then $\MM$ is a model of $\phi$ :

:$\MM \models_{\mathscr M} \phi$

that is,  $\phi$ is valid in $\MM$.




Category:Definitions/Model Theory
Category:Definitions/Formal Semantics
Let $\mathscr M$ be a formal semantics for a logical language $\LL$.

Let $\MM$ be a structure of $\mathscr M$.


Let $\FF$ be a set of logical formulas of $\LL$.

Then $\MM$ is a model of $\FF$ :

:$\MM \models_{\mathscr M} \phi$ for every $\phi \in \FF$

that is,  it is a model of every logical formula $\phi \in \FF$.
",Definition:Model (Logic),"['Definitions/Model Theory', 'Definitions/Symbolic Logic', 'Definitions/Formal Semantics']","Let ℳ be a formal semantics for a logical language .

Let  be a structure of ℳ.


Let ϕ be a logical formula of .

Then  is a model of ϕ :

:_ℳϕ

that is,  ϕ is valid in .




Category:Definitions/Model Theory
Category:Definitions/Formal Semantics
Let ℳ be a formal semantics for a logical language .

Let  be a structure of ℳ.


Let  be a set of logical formulas of .

Then  is a model of  :

:_ℳϕ for every ϕ∈

that is,  it is a model of every logical formula ϕ∈.
"
Definition:Model,Model,"Let $\LL_1$ be the language of predicate logic.

Let $\AA$ be a structure for predicate logic.


Then $\AA$ models a sentence $\mathbf A$ :

:$\map {\operatorname{val}_\AA} {\mathbf A} = \T$

where $\map {\operatorname{val}_\AA} {\mathbf A}$ denotes the value of $\mathbf A$ in $\AA$.


This relationship is denoted:

:$\AA \models_{\mathrm{PL} } \mathbf A$


When pertaining to a collection of sentences $\FF$, one says $\AA$ models $\FF$ :

:$\forall \mathbf A \in \FF: \AA \models_{\mathrm{PL} } \mathbf A$

that is,  it models all elements of $\FF$.

This can be expressed symbolically as:

:$\AA \models_{\mathrm {PL} } \FF$",Definition:Model (Predicate Logic),['Definitions/Model Theory for Predicate Logic'],"Let _1 be the language of predicate logic.

Let Å be a structure for predicate logic.


Then Å models a sentence 𝐀 :

:val_Å𝐀 =

where val_Å𝐀 denotes the value of 𝐀 in Å.


This relationship is denoted:

:Å_PL𝐀


When pertaining to a collection of sentences , one says Å models  :

:∀𝐀∈: Å_PL𝐀

that is,  it models all elements of .

This can be expressed symbolically as:

:Å_PL"
Definition:Modulus,Modulus,"Let $k = \sequence {k_j}_{j \mathop \in J}$ be a multiindex.


The modulus of such a multiindex $k$  is defined by:
:$\ds \size k = \sum_{j \mathop \in J} k_j$




Note that, since by definition all but finitely many of the $k_j$ are zero, this summation is convergent.",Definition:Multiindex/Modulus,['Definitions/Polynomial Theory'],"Let k = k_j_j ∈ J be a multiindex.


The modulus of such a multiindex k  is defined by:
:k = ∑_j ∈ J k_j




Note that, since by definition all but finitely many of the k_j are zero, this summation is convergent."
Definition:Modulus,Modulus,"Let $z = a + i b$ be a complex number, where $a, b \in \R$.


The (complex) modulus of $z$ is written $\cmod z$, and is defined as the square root of the sum of the squares of the real and imaginary parts:

:$\cmod z := \sqrt {a^2 + b^2}$


The (complex) modulus is a real-valued function, and, as and when appropriate, can be referred to as the (complex) modulus function.
Let $z = a + i b$ be a complex number, where $a, b \in \R$.


The (complex) modulus of $z$ is written $\cmod z$, and is defined as the square root of the sum of the squares of the real and imaginary parts:

:$\cmod z := \sqrt {a^2 + b^2}$


The (complex) modulus is a real-valued function, and, as and when appropriate, can be referred to as the (complex) modulus function.
Let $z = a + i b$ be a complex number, where $a, b \in \R$.


The (complex) modulus of $z$ is written $\cmod z$, and is defined as the square root of the sum of the squares of the real and imaginary parts:

:$\cmod z := \sqrt {a^2 + b^2}$


The (complex) modulus is a real-valued function, and, as and when appropriate, can be referred to as the (complex) modulus function.
",Definition:Complex Modulus,"['Definitions/Complex Modulus', 'Definitions/Complex Numbers', 'Definitions/Complex Analysis', 'Definitions/Polar Form of Complex Number', 'Definitions/Examples of Norms']","Let z = a + i b be a complex number, where a, b ∈.


The (complex) modulus of z is written z, and is defined as the square root of the sum of the squares of the real and imaginary parts:

:z := √(a^2 + b^2)


The (complex) modulus is a real-valued function, and, as and when appropriate, can be referred to as the (complex) modulus function.
Let z = a + i b be a complex number, where a, b ∈.


The (complex) modulus of z is written z, and is defined as the square root of the sum of the squares of the real and imaginary parts:

:z := √(a^2 + b^2)


The (complex) modulus is a real-valued function, and, as and when appropriate, can be referred to as the (complex) modulus function.
Let z = a + i b be a complex number, where a, b ∈.


The (complex) modulus of z is written z, and is defined as the square root of the sum of the squares of the real and imaginary parts:

:z := √(a^2 + b^2)


The (complex) modulus is a real-valued function, and, as and when appropriate, can be referred to as the (complex) modulus function.
"
Definition:Modulus,Modulus,"Let $f: S \to \C$ be a complex-valued function.


Then the (complex) modulus of $f$ is written $\cmod f : S \to \R$ and is the real-valued function defined as:

:$\forall z \in S: \map {\cmod f} z = \cmod {\map f z}$",Definition:Modulus of Complex-Valued Function,['Definitions/Complex Analysis'],"Let f: S → be a complex-valued function.


Then the (complex) modulus of f is written f : S → and is the real-valued function defined as:

:∀ z ∈ S:  f z =  f z"
Definition:Modulus,Modulus,"In geometric function theory, the term modulus is used to denote certain conformal invariants of configurations or curve families.

More precisely, the modulus of a curve family $\Gamma$ is the reciprocal of its extremal length:
:$\mod \Gamma := \dfrac 1 {\map \lambda \Gamma}$


=== Modulus of a Quadrilateral ===

Consider a quadrilateral; that is, a Jordan domain $Q$ in the complex plane (or some other Riemann surface), together with two disjoint closed boundary arcs $\alpha$ and $\alpha'$.

Then the modulus of the quadrilateral $\map Q {\alpha, \alpha'}$ is the extremal length of the family of curves in $Q$ that connect $\alpha$ and $\alpha'$.

Equivalently, there exists a rectangle $R = \set {x + i y: \cmod x < a, \cmod y < b}$ and a conformal isomorphism between $Q$ and $R$ under which $\alpha$ and $\alpha'$ correspond to the vertical sides of $R$.

Then the modulus of $\map Q {\alpha, \alpha'}$ is equal to the ratio $a/b$.


See Modulus of a Quadrilateral.


=== Modulus of an Annulus ===

Consider an annulus $A$; that is, a domain whose boundary consists of two Jordan curves.

Then the modulus $\mod A$ is the extremal length of the family of curves in $A$ that connect the two boundary components of $A$.

Equivalently, there is a round annulus $\tilde A = \set {z \in \C: r < \cmod z < R}$ that is conformally equivalent to $A$.

Then:
:$\mod A := \dfrac 1 {2 \pi} \map \ln {\dfrac R r}$

The modulus of $A$ can also be denoted $\map M R$.
",Definition:Modulus (Geometric Function Theory),['Definitions/Geometric Function Theory'],"In geometric function theory, the term modulus is used to denote certain conformal invariants of configurations or curve families.

More precisely, the modulus of a curve family Γ is the reciprocal of its extremal length:
:Γ :=  1 λΓ


=== Modulus of a Quadrilateral ===

Consider a quadrilateral; that is, a Jordan domain Q in the complex plane (or some other Riemann surface), together with two disjoint closed boundary arcs α and α'.

Then the modulus of the quadrilateral Q α, α' is the extremal length of the family of curves in Q that connect α and α'.

Equivalently, there exists a rectangle R = x + i y:  x < a,  y < b and a conformal isomorphism between Q and R under which α and α' correspond to the vertical sides of R.

Then the modulus of Q α, α' is equal to the ratio a/b.


See Modulus of a Quadrilateral.


=== Modulus of an Annulus ===

Consider an annulus A; that is, a domain whose boundary consists of two Jordan curves.

Then the modulus A is the extremal length of the family of curves in A that connect the two boundary components of A.

Equivalently, there is a round annulus à = z ∈: r <  z < R that is conformally equivalent to A.

Then:
:A :=  1 2 πln R r

The modulus of A can also be denoted M R.
"
Definition:Modulus,Modulus,"Let $x$ be congruent to $y$ modulo $m$.

The number $m$ in this congruence is known as the modulus of the congruence.
",Definition:Congruence (Number Theory)/Modulus,['Definitions/Congruence (Number Theory)'],"Let x be congruent to y modulo m.

The number m in this congruence is known as the modulus of the congruence.
"
Definition:Modulus,Modulus,"Let $B$ be an elastic body.

The bulk modulus of $B$ is a physical property of $B$ which measures its resistance to change in volume without change of shape.

It is defined as the ratio of compressive stress per surface area of $B$ to its change of volume per unit volume associated with this stress, assuming uniform pressure over the surface of $B$.
",Definition:Bulk Modulus,"['Definitions/Bulk Modulus', 'Definitions/Physical Quantities']","Let B be an elastic body.

The bulk modulus of B is a physical property of B which measures its resistance to change in volume without change of shape.

It is defined as the ratio of compressive stress per surface area of B to its change of volume per unit volume associated with this stress, assuming uniform pressure over the surface of B.
"
Definition:Monotone,Monotone,"Let $\SS$ be an algebra of sets.

Let $f: \SS \to \overline \R$ be an extended real-valued function, where $\overline \R$ denotes the set of extended real numbers.


Then $f$ is defined as monotone or monotonic :
:$\forall A, B \in \SS: A \subseteq B \iff \map f A \le \map f B$

Category:Definitions/Set Systems",Definition:Monotone (Measure Theory),['Definitions/Set Systems'],"Let  be an algebra of sets.

Let f: → be an extended real-valued function, where  denotes the set of extended real numbers.


Then f is defined as monotone or monotonic :
:∀ A, B ∈: A ⊆ B  f A ≤ f B

Category:Definitions/Set Systems"
Definition:Monotone,Monotone,"Let $X$ be a set, and let $\powerset X$ be its power set.

Let $\MM \subseteq \powerset X$ be a collection of subsets of $X$.


Then $\MM$ is said to be a monotone class (on $X$)  for every countable, nonempty, index set $I$, it holds that:

:$\ds \family {A_i}_{i \mathop \in I} \in \MM \implies \bigcup_{i \mathop \in I} A_i \in \MM$
:$\ds \family {A_i}_{i \mathop \in I} \in \MM \implies \bigcap_{i \mathop \in I} A_i \in \MM$

that is,  $\MM$ is closed under countable unions and countable intersections.",Definition:Monotone Class,['Definitions/Set Systems'],"Let X be a set, and let X be its power set.

Let ⊆ X be a collection of subsets of X.


Then  is said to be a monotone class (on X)  for every countable, nonempty, index set I, it holds that:

:A_i_i ∈ I∈⋃_i ∈ I A_i ∈
:A_i_i ∈ I∈⋂_i ∈ I A_i ∈

that is,   is closed under countable unions and countable intersections."
Definition:Multiplicative Function,Multiplicative Function,"Let $K$ be a field.

Let $f: K \to K$ be a function on $K$.


Then $f$ is described as completely multiplicative :

:$\forall m, n \in K: \map f {m n} = \map f m \map f n$


That is, a completely multiplicative function is one where the value of a product of two numbers equals the product of the value of each one individually.
",Definition:Completely Multiplicative Function,"['Definitions/Multiplicative Functions', 'Definitions/Field Theory', 'Definitions/Number Theory']","Let K be a field.

Let f: K → K be a function on K.


Then f is described as completely multiplicative :

:∀ m, n ∈ K:  f m n =  f m  f n


That is, a completely multiplicative function is one where the value of a product of two numbers equals the product of the value of each one individually.
"
Definition:Multiplicative Function,Multiplicative Function,"Let $\struct {R, +, \circ}$ be a ring.

Let $f: R \to \R$ be a (real-valued) function on $R$.


$f$ is a multiplicative function on $R$ :

:$\forall x, y \in R: \map f {x \circ y} = \map f x \times \map f y$


That is, a multiplicative function on $R$ is one where the value of the product of two elements of $R$ equals the product of their values.
Let $\struct {R, +, \circ}$ be a ring.

Let $f: R \to \R$ be a (real-valued) function on $R$.


$f$ is a multiplicative function on $R$ :

:$\forall x, y \in R: \map f {x \circ y} = \map f x \times \map f y$


That is, a multiplicative function on $R$ is one where the value of the product of two elements of $R$ equals the product of their values.
",Definition:Multiplicative Function on Ring,"['Definitions/Multiplicative Functions', 'Definitions/Ring Theory', 'Definitions/Number Theory']","Let R, +, ∘ be a ring.

Let f: R → be a (real-valued) function on R.


f is a multiplicative function on R :

:∀ x, y ∈ R:  f x ∘ y =  f x × f y


That is, a multiplicative function on R is one where the value of the product of two elements of R equals the product of their values.
Let R, +, ∘ be a ring.

Let f: R → be a (real-valued) function on R.


f is a multiplicative function on R :

:∀ x, y ∈ R:  f x ∘ y =  f x × f y


That is, a multiplicative function on R is one where the value of the product of two elements of R equals the product of their values.
"
Definition:Multiplicative Function,Multiplicative Function,"Let $R$ be a unique factorization domain.

Let $f : R \to \C$ be a complex-valued function.


Then $f$ is multiplicative :
:For all coprime $x, y\in R$: $f \left({x y}\right) = f \left({x}\right) f \left({y}\right)$


=== Arithmetic Function ===
",Definition:Multiplicative Function on UFD,['Definitions/Ring Theory'],"Let R be a unique factorization domain.

Let f : R → be a complex-valued function.


Then f is multiplicative :
:For all coprime x, y∈ R: f (x y) = f (x) f (y)


=== Arithmetic Function ===
"
Definition:Multiplicity,Multiplicity,"The multiplicity of a multiple root is the number of times it appears.
",Definition:Multiple Root/Multiplicity,['Definitions/Multiple Roots'],"The multiplicity of a multiple root is the number of times it appears.
"
Definition:Multiplicity,Multiplicity,"Let $f: \C \to \C$ be a function.

Suppose there is $a \in \C$ such that $\map f a = 0$.

Then $a$ is said to be a zero of multiplicity $k$  there exists non-zero $L \in \R$ such that:

:$\ds \lim_{z \mathop \to a} \dfrac {\cmod {\map f z} } {\cmod {z - a}^k} = L$



Category:Definitions/Complex Analysis",Definition:Multiplicity (Complex Analysis),['Definitions/Complex Analysis'],"Let f: → be a function.

Suppose there is a ∈ such that f a = 0.

Then a is said to be a zero of multiplicity k  there exists non-zero L ∈ such that:

:lim_z → a f zz - a^k = L



Category:Definitions/Complex Analysis"
Definition:Multiplicity,Multiplicity,"Let $G = \struct {V, E}$ be a multigraph.

The multiplicity of an edge is the number of edges having the same pair of endvertices.


For example, simple edges have multiplicity $1$.

Thus, an edge is a multiple edge  its multiplicity exceeds $1$.

Category:Definitions/Multigraphs",Definition:Multiple Edge/Multiplicity,['Definitions/Multigraphs'],"Let G = V, E be a multigraph.

The multiplicity of an edge is the number of edges having the same pair of endvertices.


For example, simple edges have multiplicity 1.

Thus, an edge is a multiple edge  its multiplicity exceeds 1.

Category:Definitions/Multigraphs"
Definition:Multiplicity,Multiplicity,"Let $G = \struct {V, E}$ be a multigraph.

The multiplicity of an edge is the number of edges having the same pair of endvertices.


For example, simple edges have multiplicity $1$.

Thus, an edge is a multiple edge  its multiplicity exceeds $1$.

Category:Definitions/Multigraphs
",Definition:Multigraph/Multiplicity,['Definitions/Multigraphs'],"Let G = V, E be a multigraph.

The multiplicity of an edge is the number of edges having the same pair of endvertices.


For example, simple edges have multiplicity 1.

Thus, an edge is a multiple edge  its multiplicity exceeds 1.

Category:Definitions/Multigraphs
"
Definition:Multiplicity,Multiplicity,"Let $n > 1 \in \Z$.

Let:
:$n = p_1^{k_1} p_2^{k_2} \cdots p_r^{k_r}$
be the prime decomposition of $n$, where:
:$p_1 < p_2 < \cdots < p_r$ are distinct primes
:$k_1, k_2, \ldots, k_r$ are (strictly) positive integers.


For each $p_j \in \set {p_1, p_2, \ldots, p_r}$, its power $k_j$ is known as the multiplicity of $p_j$.",Definition:Prime Decomposition/Multiplicity,['Definitions/Prime Decompositions'],"Let n > 1 ∈.

Let:
:n = p_1^k_1 p_2^k_2⋯ p_r^k_r
be the prime decomposition of n, where:
:p_1 < p_2 < ⋯ < p_r are distinct primes
:k_1, k_2, …, k_r are (strictly) positive integers.


For each p_j ∈p_1, p_2, …, p_r, its power k_j is known as the multiplicity of p_j."
Definition:Net,Net,"Let $X$ be a nonempty set.

Let $\struct {\Lambda, \precsim}$ be a preordered set.

Let $F: \Lambda \to X$ be a mapping.


Then $F$ is referred to as a net.",Definition:Net (Preordered Set),['Definitions/Nets (Topology)'],"Let X be a nonempty set.

Let Λ, ≾ be a preordered set.

Let F: Λ→ X be a mapping.


Then F is referred to as a net."
Definition:Net,Net,"Let $M$ be a metric space.

Let $\epsilon \in \R_{>0}$ be a strictly positive real number.


A finite $\epsilon$-net for $M$ is an $\epsilon$-net for $M$ which is a finite set.
",Definition:Net (Metric Space),['Definitions/Metric Spaces'],"Let M be a metric space.

Let ϵ∈_>0 be a strictly positive real number.


A finite ϵ-net for M is an ϵ-net for M which is a finite set.
"
Definition:Neutral,Neutral,A body which has no electric charge on it is described as (electrically) neutral.,Definition:Electric Charge/Neutral,['Definitions/Electric Charge'],A body which has no electric charge on it is described as (electrically) neutral.
Definition:Neutral,Neutral,"Let $B$ be a particle, a system of particles, or a body which is in equilibrium.

Let $B$ be such that if a small displacement is applied, $B$ remains in its new position.


$B$ is then said to be in neutral equilibrium.
",Definition:Equilibrium (Mechanics)/Neutral,"['Definitions/Neutral Equilibrium', 'Definitions/Equilibrium (Mechanics)']","Let B be a particle, a system of particles, or a body which is in equilibrium.

Let B be such that if a small displacement is applied, B remains in its new position.


B is then said to be in neutral equilibrium.
"
Definition:Neutral,Neutral,"Let $\struct {S, \circ}$ be an algebraic structure.

An element $e \in S$ is called an identity (element)  it is both a left identity and a right identity:

:$\forall x \in S: x \circ e = x = e \circ x$


In Identity is Unique it is established that an identity element, if it exists, is unique within $\struct {S, \circ}$.

Thus it is justified to refer to it as the identity (of a given algebraic structure).


This identity is often denoted $e_S$, or $e$ if it is clearly understood what structure is being discussed.",Definition:Identity (Abstract Algebra)/Two-Sided Identity,"['Definitions/Identity Elements', 'Definitions/Identities']","Let S, ∘ be an algebraic structure.

An element e ∈ S is called an identity (element)  it is both a left identity and a right identity:

:∀ x ∈ S: x ∘ e = x = e ∘ x


In Identity is Unique it is established that an identity element, if it exists, is unique within S, ∘.

Thus it is justified to refer to it as the identity (of a given algebraic structure).


This identity is often denoted e_S, or e if it is clearly understood what structure is being discussed."
Definition:Node,Node,"The vertices of a tree are called its nodes.
",Definition:Node (Graph Theory),"['Definitions/Tree Theory', 'Definitions/Graph Theory']","The vertices of a tree are called its nodes.
"
Definition:Node,Node,"An acnode is a singular point of the locus of an equation describing a curve which is not actually on that curve.
",Definition:Acnode,"['Definitions/Acnodes', 'Definitions/Analytic Geometry']","An acnode is a singular point of the locus of an equation describing a curve which is not actually on that curve.
"
Definition:Node,Node,"A crunode is a double point $P$ of the locus of an equation describing a curve which intersects itself in such a way that there are $2$ distinct tangents at $P$.
",Definition:Crunode,"['Definitions/Crunodes', 'Definitions/Double Points', 'Definitions/Analytic Geometry']","A crunode is a double point P of the locus of an equation describing a curve which intersects itself in such a way that there are 2 distinct tangents at P.
"
Definition:Node,Node,"A metagraph $\GG$ consists of:

* objects $X, Y, Z, \ldots$
* morphisms $f, g, h, \ldots$ between its objects

These are subjected to the following two axioms:







A metagraph is purely axiomatic, and does not use set theory.

For example, the objects are not ""elements of the set of objects"", because these axioms are (without further interpretation) unfounded in set theory.",Definition:Metagraph,['Definitions/Category Theory'],"A metagraph  consists of:

* objects X, Y, Z, …
* morphisms f, g, h, … between its objects

These are subjected to the following two axioms:







A metagraph is purely axiomatic, and does not use set theory.

For example, the objects are not ""elements of the set of objects"", because these axioms are (without further interpretation) unfounded in set theory."
Definition:Node,Node,"Let $\closedint a b$ be a closed real interval.

Let $T := \set {a = t_0, t_1, t_2, \ldots, t_{n - 1}, t_n = b}$ form a subdivision of $\closedint a b$.

Let $S: \closedint a b \to \R$ be a spline function on $\closedint a b$ on $T$.


The points $T := \set {t_0, t_1, t_2, \ldots, t_{n - 1}, t_n}$ of $S$ are known as the knots.


=== Knot Vector ===
",Definition:Spline Function/Knot,"['Definitions/Knots of Splines', 'Definitions/Splines']","Let a b be a closed real interval.

Let T := a = t_0, t_1, t_2, …, t_n - 1, t_n = b form a subdivision of a b.

Let S:  a b → be a spline function on a b on T.


The points T := t_0, t_1, t_2, …, t_n - 1, t_n of S are known as the knots.


=== Knot Vector ===
"
Definition:Normal,Normal,"Let $X = \struct {M_1, d_1}$ and $Y = \struct {M_2, d_2}$ be complete metric spaces.

Let $\FF = \family {f_i}_{i \mathop \in I}$ be a family of continuous mappings $f_i: X \to Y$.


Then $\FF$ is a normal family :
:every sequence of mappings in $\FF$ contains a subsequence which converges uniformly on compact subsets of $X$ to a continuous function $f: X \to Y$.",Definition:Normal Family,['Definitions/Metric Spaces'],"Let X = M_1, d_1 and Y = M_2, d_2 be complete metric spaces.

Let = f_i_i ∈ I be a family of continuous mappings f_i: X → Y.


Then  is a normal family :
:every sequence of mappings in  contains a subsequence which converges uniformly on compact subsets of X to a continuous function f: X → Y."
Definition:Normal,Normal,"Let $\HH$ be a Hilbert space.

Let $\mathbf T: \HH \to \HH$ be a bounded linear operator.


Then $\mathbf T$ is said to be normal :

:$\mathbf T^* \mathbf T = \mathbf T \mathbf T^*$

where $\mathbf T^*$ denotes the adjoint of $\mathbf T$.",Definition:Normal Operator,"['Definitions/Adjoints', 'Definitions/Linear Operators', 'Definitions/Linear Transformations on Hilbert Spaces']","Let  be a Hilbert space.

Let 𝐓: → be a bounded linear operator.


Then 𝐓 is said to be normal :

:𝐓^* 𝐓 = 𝐓𝐓^*

where 𝐓^* denotes the adjoint of 𝐓."
Definition:Normal,Normal,"Let $C$ be a curve embedded in the plane.

The normal to $C$ at a point $P$ is defined as the straight line which lies perpendicular to the tangent at $P$ and in the same plane as $P$.",Definition:Normal to Curve,['Definitions/Analytic Geometry'],"Let C be a curve embedded in the plane.

The normal to C at a point P is defined as the straight line which lies perpendicular to the tangent at P and in the same plane as P."
Definition:Normal,Normal,"Let $S$ be a surface in ordinary $3$-space.

Let $P$ be a point of $S$.


Let $\mathbf n$ be a vector whose initial point is at $P$ such that $\mathbf n$ is perpendicular to $S$ at $P$.


Then $\mathbf n$ is a normal vector to $S$ at $P$.",Definition:Normal Vector,"['Definitions/Normal Vectors', 'Definitions/Vectors']","Let S be a surface in ordinary 3-space.

Let P be a point of S.


Let 𝐧 be a vector whose initial point is at P such that 𝐧 is perpendicular to S at P.


Then 𝐧 is a normal vector to S at P."
Definition:Normal,Normal,"Let $\struct {\tilde M, \tilde g}$ be a Riemannian manifold.

Let $M \subseteq \tilde M$ be a smooth submanifold with or without boundary in $\tilde M$.

Let $p \in M$ be a point in $M$.

Suppose $v$ is normal to $M$ at $p$.


The set of all such $v$ at $p$ is called the normal space (of $M$ at $p$) and is denoted by $N_p M = \paren {T_p M}^\perp$
Let $\struct {\tilde M, \tilde g}$ be a Riemannian manifold.

Let $M \subseteq \tilde M$ be a smooth submanifold with or without boundary in $\tilde M$.

Let $p \in M$ be a point in $M$.

Suppose $v$ is normal to $M$ at $p$.


The set of all such $v$ at $p$ is called the normal space (of $M$ at $p$) and is denoted by $N_p M = \paren {T_p M}^\perp$
",Definition:Normal Bundle,['Definitions/Topology'],"Let M̃, g̃ be a Riemannian manifold.

Let M ⊆M̃ be a smooth submanifold with or without boundary in M̃.

Let p ∈ M be a point in M.

Suppose v is normal to M at p.


The set of all such v at p is called the normal space (of M at p) and is denoted by N_p M = T_p M^⊥
Let M̃, g̃ be a Riemannian manifold.

Let M ⊆M̃ be a smooth submanifold with or without boundary in M̃.

Let p ∈ M be a point in M.

Suppose v is normal to M at p.


The set of all such v at p is called the normal space (of M at p) and is denoted by N_p M = T_p M^⊥
"
Definition:Normal,Normal,"Let $x$ be an ordinal.


The Cantor normal form of $x$ is an ordinal summation:

:$x = \omega^{a_1} n_1 + \dots + \omega^{a_k} n_k$

where:

:$k \in \N$ is a natural number
:$\omega$ is the minimally inductive set
:$\sequence {a_i}$ is a strictly decreasing finite sequence of ordinals.
:$\sequence {n_i}$ is a finite sequence of finite ordinals


In summation notation:

:$x = \ds \sum_{i \mathop = 1}^k \omega^{a_i} n_i$

",Definition:Cantor Normal Form,['Definitions/Ordinal Arithmetic'],"Let x be an ordinal.


The Cantor normal form of x is an ordinal summation:

:x = ω^a_1 n_1 + … + ω^a_k n_k

where:

:k ∈ is a natural number
:ω is the minimally inductive set
:a_i is a strictly decreasing finite sequence of ordinals.
:n_i is a finite sequence of finite ordinals


In summation notation:

:x = ∑_i  = 1^k ω^a_i n_i

"
Definition:Normal,Normal,"A real number $r$ is normal with respect to a number base $b$  its basis expansion in number base $b$ is such that:

:no finite sequence of digits of $r$ of length $n$ occurs more frequently than any other such finite sequence of length $n$.


In particular, for number base $b$, all digits of $r$ have the same natural density in the basis expansion of $r$.",Definition:Normal Number,"['Definitions/Normal Numbers', 'Definitions/Numbers']","A real number r is normal with respect to a number base b  its basis expansion in number base b is such that:

:no finite sequence of digits of r of length n occurs more frequently than any other such finite sequence of length n.


In particular, for number base b, all digits of r have the same natural density in the basis expansion of r."
Definition:Normal,Normal,"Let $X$ be a continuous random variable on a probability space $\struct {\Omega, \Sigma, \Pr}$.


Then $X$ has a Gaussian distribution  the probability density function of $X$ is:

:$\map {f_X} x = \dfrac 1 {\sigma \sqrt {2 \pi} } \map \exp {-\dfrac {\paren {x - \mu}^2} {2 \sigma^2} }$

for $\mu \in \R, \sigma \in \R_{> 0}$.


This is written: 

:$X \sim \Gaussian \mu {\sigma^2}$",Definition:Gaussian Distribution,"['Definitions/Gaussian Distribution', 'Definitions/Examples of Probability Distributions']","Let X be a continuous random variable on a probability space Ω, Σ,.


Then X has a Gaussian distribution  the probability density function of X is:

:f_X x =  1 σ√(2 π)exp-x - μ^22 σ^2

for μ∈, σ∈_> 0.


This is written: 

:X ∼μσ^2"
Definition:Normal,Normal,"A propositional formula $P$ is in conjunctive normal form  it consists of a conjunction of:
:$(1): \quad$ disjunctions of literals
and/or:
:$(2): \quad$ literals.",Definition:Conjunctive Normal Form,"['Definitions/Conjunctive Normal Form', 'Definitions/Conjunction', 'Definitions/Propositional Logic']","A propositional formula P is in conjunctive normal form  it consists of a conjunction of:
:(1): disjunctions of literals
and/or:
:(2): literals."
Definition:Normal,Normal,"A propositional formula $P$ is in disjunctive normal form  it consists of a disjunction of:
:$(1): \quad$ conjunctions of literals
and/or:
:$(2): \quad$ literals.
",Definition:Disjunctive Normal Form,"['Definitions/Disjunctive Normal Form', 'Definitions/Propositional Logic']","A propositional formula P is in disjunctive normal form  it consists of a disjunction of:
:(1): conjunctions of literals
and/or:
:(2): literals.
"
Definition:Normal,Normal,"A propositional formula $P$ is in negation normal form (NNF) :
* The only logical connectives connecting substatements of $P$ are Not, And and Or, that is, elements of the set $\left\{{\neg, \land, \lor}\right\}$;
* The Not sign $\neg$ appears only in front of atomic statements.

That is $P$ is in negation normal form iff it consists of literals, conjunctions and disjunctions.",Definition:Negation Normal Form,['Definitions/Propositional Logic'],"A propositional formula P is in negation normal form (NNF) :
* The only logical connectives connecting substatements of P are Not, And and Or, that is, elements of the set {, , };
* The Not sign  appears only in front of atomic statements.

That is P is in negation normal form iff it consists of literals, conjunctions and disjunctions."
Definition:Normal,Normal,"Let $T = \struct {S, \tau}$ be a topological space.


$\struct {S, \tau}$ is a normal space :
:$\struct {S, \tau}$ is a $T_4$ space
:$\struct {S, \tau}$ is a $T_1$ (Fréchet) space.


That is:
:$\forall A, B \in \map \complement \tau, A \cap B = \O: \exists U, V \in \tau: A \subseteq U, B \subseteq V, U \cap V = \O$ 

:$\forall x, y \in S$, both:
::$\exists U \in \tau: x \in U, y \notin U$
::$\exists V \in \tau: y \in V, x \notin V$



This space is also referred to as normal Hausdorff.
",Definition:Normal Space,"['Definitions/Normal Spaces', 'Definitions/T4 Spaces', 'Definitions/T1 Spaces', 'Definitions/Separation Axioms']","Let T = S, τ be a topological space.


S, τ is a normal space :
:S, τ is a T_4 space
:S, τ is a T_1 (Fréchet) space.


That is:
:∀ A, B ∈∁τ, A ∩ B = Ø: ∃ U, V ∈τ: A ⊆ U, B ⊆ V, U ∩ V = Ø 

:∀ x, y ∈ S, both:
::∃ U ∈τ: x ∈ U, y ∉ U
::∃ V ∈τ: y ∈ V, x ∉ V



This space is also referred to as normal Hausdorff.
"
Definition:Normal,Normal,"Let $T = \struct {S, \tau}$ be a topological space.


$\struct {S, \tau}$ is a completely normal space :
:$\struct {S, \tau}$ is a $T_5$ space
:$\struct {S, \tau}$ is a $T_1$ (Fréchet) space.


That is:

:$\forall A, B \subseteq S, A^- \cap B = A \cap B^- = \O: \exists U, V \in \tau: A \subseteq U, B \subseteq V, U \cap V = \O$ 

:$\forall x, y \in S$, both:
::$\exists U \in \tau: x \in U, y \notin U$
::$\exists V \in \tau: y \in V, x \notin V$


",Definition:Completely Normal Space,"['Definitions/Completely Normal Spaces', 'Definitions/Normal Spaces', 'Definitions/T5 Spaces', 'Definitions/T1 Spaces', 'Definitions/Separation Axioms']","Let T = S, τ be a topological space.


S, τ is a completely normal space :
:S, τ is a T_5 space
:S, τ is a T_1 (Fréchet) space.


That is:

:∀ A, B ⊆ S, A^- ∩ B = A ∩ B^- = Ø: ∃ U, V ∈τ: A ⊆ U, B ⊆ V, U ∩ V = Ø 

:∀ x, y ∈ S, both:
::∃ U ∈τ: x ∈ U, y ∉ U
::∃ V ∈τ: y ∈ V, x ∉ V


"
Definition:Normal,Normal,"Let $T = \struct {S, \tau}$ be a topological space.


$T$ is fully normal :
:Every open cover of $S$ has a star refinement
:All points of $T$ are closed.


That is, $T$ is fully normal :
:$T$ is fully $T_4$
:$T$ is a $T_1$ (Fréchet) space.


Let $T = \struct {S, \tau}$ be a topological space.


$T$ is fully normal :
:Every open cover of $S$ has a star refinement
:All points of $T$ are closed.


That is, $T$ is fully normal :
:$T$ is fully $T_4$
:$T$ is a $T_1$ (Fréchet) space.


",Definition:Fully Normal Space,"['Definitions/Fully Normal Spaces', 'Definitions/Fully T4 Spaces', 'Definitions/T1 Spaces', 'Definitions/Compact Spaces', 'Definitions/Separation Axioms']","Let T = S, τ be a topological space.


T is fully normal :
:Every open cover of S has a star refinement
:All points of T are closed.


That is, T is fully normal :
:T is fully T_4
:T is a T_1 (Fréchet) space.


Let T = S, τ be a topological space.


T is fully normal :
:Every open cover of S has a star refinement
:All points of T are closed.


That is, T is fully normal :
:T is fully T_4
:T is a T_1 (Fréchet) space.


"
Definition:Normal,Normal,"Let $T = \struct {S, \tau}$ be a topological space.


$\struct {S, \tau}$ is a perfectly normal space :
:$\struct {S, \tau}$ is a perfectly $T_4$ space
:$\struct {S, \tau}$ is a $T_1$ (Fréchet) space.


That is:

:Every closed set in $T$ is a $G_\delta$ set.

:$\forall x, y \in S$, both:
::$\exists U \in \tau: x \in U, y \notin U$
::$\exists V \in \tau: y \in V, x \notin V$
",Definition:Perfectly Normal Space,"['Definitions/Perfectly Normal Spaces', 'Definitions/Normal Spaces', 'Definitions/Perfectly T4 Spaces', 'Definitions/T1 Spaces', 'Definitions/Separation Axioms']","Let T = S, τ be a topological space.


S, τ is a perfectly normal space :
:S, τ is a perfectly T_4 space
:S, τ is a T_1 (Fréchet) space.


That is:

:Every closed set in T is a G_δ set.

:∀ x, y ∈ S, both:
::∃ U ∈τ: x ∈ U, y ∉ U
::∃ V ∈τ: y ∈ V, x ∉ V
"
Definition:Normal,Normal,"Let $S$ be a set.

Then $S$ is an ordinary set :
:$S \notin S$

That is,  $S$ is not an element of itself.",Definition:Ordinary Set,['Definitions/Set Theory'],"Let S be a set.

Then S is an ordinary set :
:S ∉ S

That is,  S is not an element of itself."
Definition:Normal,Normal,"Let $A$ be a class.

Let $\RR$ be a relation on $A$.


An element $x$ of $A$ is left normal with respect to $\RR$ :
:$\forall y \in A: \map \RR {x, y}$ holds.",Definition:Left Normal Element of Relation,['Definitions/Relations'],"Let A be a class.

Let  be a relation on A.


An element x of A is left normal with respect to  :
:∀ y ∈ A: x, y holds."
Definition:Normal,Normal,"Let $A$ be a class.

Let $\RR$ be a relation on $A$.


An element $x$ of $A$ is right normal with respect to $\RR$ :
:$\forall y \in A: \map \RR {y, x}$ holds.",Definition:Right Normal Element of Relation,['Definitions/Relations'],"Let A be a class.

Let  be a relation on A.


An element x of A is right normal with respect to  :
:∀ y ∈ A: y, x holds."
Definition:Normal,Normal,"Let $G$ be a group.

Let $S$ be a subset of $G$.


Then the normalizer of $S$ in $G$ is the set $\map {N_G} S$ defined as:
:$\map {N_G} S := \set {a \in G: S^a = S}$

where $S^a$ is the $G$-conjugate of $S$ by $a$.


If $S$ is a singleton such that $S = \set s$, we may also write $\map {N_G} s$ for $\map {N_G} S = \map {N_G} {\set s}$, as long as there is no possibility of confusion.",Definition:Normalizer,['Definitions/Normality in Groups'],"Let G be a group.

Let S be a subset of G.


Then the normalizer of S in G is the set N_G S defined as:
:N_G S := a ∈ G: S^a = S

where S^a is the G-conjugate of S by a.


If S is a singleton such that S =  s, we may also write N_G s for N_G S = N_G s, as long as there is no possibility of confusion."
Definition:Null,Null,"Let $\struct {X, \Sigma, \mu}$ be a measure space.

A set $N \in \Sigma$ is called a ($\mu$-)null set  $\map \mu N = 0$.




=== Family of Null Sets ===

The family of $\mu$-null sets, $\set {N \in \Sigma: \map \mu N = 0}$, is denoted $\NN_\mu$. 


=== Signed Measure ===

Let $\struct {X, \Sigma, \mu}$ be a measure space.

A set $N \in \Sigma$ is called a ($\mu$-)null set  $\map \mu N = 0$.




=== Family of Null Sets ===

The family of $\mu$-null sets, $\set {N \in \Sigma: \map \mu N = 0}$, is denoted $\NN_\mu$. 


=== Signed Measure ===

",Definition:Null Set,"['Definitions/Null Sets', 'Definitions/Measure Theory', 'Definitions/Topology']","Let X, Σ, μ be a measure space.

A set N ∈Σ is called a (μ-)null set  μ N = 0.




=== Family of Null Sets ===

The family of μ-null sets, N ∈Σ: μ N = 0, is denoted _μ. 


=== Signed Measure ===

Let X, Σ, μ be a measure space.

A set N ∈Σ is called a (μ-)null set  μ N = 0.




=== Family of Null Sets ===

The family of μ-null sets, N ∈Σ: μ N = 0, is denoted _μ. 


=== Signed Measure ===

"
Definition:Null,Null,"The null relation is a relation $\RR$ in $S$ to $T$ such that $\RR$ is the empty set:
:$\RR \subseteq S \times T: \RR = \O$


That is, no element of $S$ relates to any element in $T$:
:$\RR: S \times T: \forall \tuple {s, t} \in S \times T: \neg s \mathrel \RR t$
",Definition:Null Relation,"['Definitions/Null Relation', 'Definitions/Empty Set', 'Definitions/Examples of Relations']","The null relation is a relation  in S to T such that  is the empty set:
:⊆ S × T:  = Ø


That is, no element of S relates to any element in T:
:: S × T: ∀s, t∈ S × T:  s  t
"
Definition:Null,Null,"A ring with one element is called the null ring.

That is, the null ring is $\struct {\set {0_R}, +, \circ}$, where ring addition and the ring product are defined as:




",Definition:Null Ring,"['Definitions/Ring Theory', 'Definitions/Examples of Rings']","A ring with one element is called the null ring.

That is, the null ring is 0_R, +, ∘, where ring addition and the ring product are defined as:




"
Definition:Null,Null,"Let $\left({R, +_R, \circ_R}\right)$ be a ring.

Let $G$ be the trivial group.


Then the $R$-module $\left({G, +_G, \circ}\right)_R$ is known as the null module.",Definition:Null Module,['Definitions/Module Theory'],"Let (R, +_R, ∘_R) be a ring.

Let G be the trivial group.


Then the R-module (G, +_G, ∘)_R is known as the null module."
Definition:Null,Null,"Let $\struct {X, \Sigma}$ be a measurable space.


Then the null measure is the measure defined by:

:$\mu: \Sigma \to \overline \R: \map \mu E := 0$

where $\overline \R$ denotes the extended real numbers.",Definition:Null Measure,"['Definitions/Null Measure', 'Definitions/Measures']","Let X, Σ be a measurable space.


Then the null measure is the measure defined by:

:μ: Σ→: μ E := 0

where  denotes the extended real numbers."
Definition:Null,Null,"A null sequence is a sequence which converges to zero.


=== Normed Division Ring ===

",Definition:Null Sequence,"['Definitions/Null Sequences', 'Definitions/Convergence', 'Definitions/Sequences', 'Definitions/Analysis', 'Definitions/Real Analysis', 'Definitions/Complex Analysis', 'Definitions/Normed Division Rings']","A null sequence is a sequence which converges to zero.


=== Normed Division Ring ===

"
Definition:Null,Null,"The null URM program is a URM program which contains no instructions.

That is, a URM program whose length is zero.
",Definition:Unlimited Register Machine/Null Program,"['Definitions/Null URM Program', 'Definitions/URM Programs', 'Definitions/Unlimited Register Machines']","The null URM program is a URM program which contains no instructions.

That is, a URM program whose length is zero.
"
Definition:Null,Null,,Definition:Null Polynomial,['Definitions/Polynomial Theory'],
Definition:Null,Null,"The null graph is the graph which has no vertices.

That is, the null graph is the graph of order zero.


It is called the null graph because, from Empty Set is Unique, there is only one such entity.",Definition:Null Graph,"['Definitions/Null Graph', 'Definitions/Graph Theory']","The null graph is the graph which has no vertices.

That is, the null graph is the graph of order zero.


It is called the null graph because, from Empty Set is Unique, there is only one such entity."
Definition:Null,Null,"A null string is a string with no symbols in it.

In particular, the null string is a word.


The null string can be denoted $\epsilon$.
",Definition:Null String,['Definitions/Collations'],"A null string is a string with no symbols in it.

In particular, the null string is a word.


The null string can be denoted ϵ.
"
Definition:Null,Null,"Let:
$\quad \mathbf A_{m \times n} = \begin {bmatrix}
a_{11} & a_{12} & \cdots & a_{1n} \\
a_{21} & a_{22} & \cdots & a_{2n} \\
\vdots & \vdots & \ddots & \vdots \\
a_{m1} & a_{m2} & \cdots & a_{mn} \\
\end {bmatrix}$,  $\mathbf x_{n \times 1} = \begin {bmatrix} x_1 \\ x_2 \\ \vdots \\ x_n \end {bmatrix}$, $\mathbf 0_{m \times 1} = \begin {bmatrix} 0 \\ 0 \\ \vdots \\ 0 \end {bmatrix}$

be matrices where each column is a member of a real vector space.

The set of all solutions to $\mathbf A \mathbf x = \mathbf 0$:

:$\map {\mathrm N} {\mathbf A} = \set {\mathbf x \in \R^n : \mathbf {A x} = \mathbf 0}$

is called the null space of $\mathbf A$.



",Definition:Null Space,"['Definitions/Null Spaces', 'Definitions/Linear Algebra']","Let:
𝐀_m × n = [ a_11 a_12    ⋯ a_1n; a_21 a_22    ⋯ a_2n;    ⋮    ⋮    ⋱    ⋮; a_m1 a_m2    ⋯ a_mn;      ],  𝐱_n × 1 = [ x_1; x_2;   ⋮; x_n ], 0_m × 1 = [ 0; 0; ⋮; 0 ]

be matrices where each column is a member of a real vector space.

The set of all solutions to 𝐀𝐱 = 0:

:N𝐀 = 𝐱∈^n : 𝐀 𝐱 = 0

is called the null space of 𝐀.



"
Definition:Null,Null,"A ring with one element is called the null ring.

That is, the null ring is $\struct {\set {0_R}, +, \circ}$, where ring addition and the ring product are defined as:





",Definition:Null Ideal,['Definitions/Ideal Theory'],"A ring with one element is called the null ring.

That is, the null ring is 0_R, +, ∘, where ring addition and the ring product are defined as:





"
Definition:Null,Null,"Let $\mathbf C$ be a category.


A zero object of $\mathbf C$ is an object which is initial and terminal.",Definition:Zero Object,['Definitions/Category Theory'],"Let 𝐂 be a category.


A zero object of 𝐂 is an object which is initial and terminal."
Definition:Opposite,Opposite,"When a polygon has an even number of sides, each side has an opposite side, and each vertex likewise has an opposite vertex.

When a polygon has an odd number of sides, each side has an opposite vertex.


The opposite side (or opposite vertex) to a given side (or vertex) is that side (or vertex) which has the same number of sides between it and the side (or vertex) in question.",Definition:Polygon/Opposite,['Definitions/Polygons'],"When a polygon has an even number of sides, each side has an opposite side, and each vertex likewise has an opposite vertex.

When a polygon has an odd number of sides, each side has an opposite vertex.


The opposite side (or opposite vertex) to a given side (or vertex) is that side (or vertex) which has the same number of sides between it and the side (or vertex) in question."
Definition:Opposite,Opposite,"The side of a triangle which is not one of the sides adjacent to a particular vertex is referred to as its opposite.

Thus, each vertex has an opposite side, and each side has an opposite vertex.",Definition:Triangle (Geometry)/Opposite,['Definitions/Triangles'],"The side of a triangle which is not one of the sides adjacent to a particular vertex is referred to as its opposite.

Thus, each vertex has an opposite side, and each side has an opposite vertex."
Definition:Opposite,Opposite,":

The opposite face of the face $F$ of a parallelepiped $P$ is the face of $P$ which is parallel to $F$.

In the above example, the pairs of parallel planes are:

:Face $ABCD$ is opposite $HGFE$
:Face $ADEH$ is opposite $BCFG$
:Face $ABGH$ is opposite $DCFE$
:

The opposite face of the face $F$ of a parallelepiped $P$ is the face of $P$ which is parallel to $F$.

In the above example, the pairs of parallel planes are:

:Face $ABCD$ is opposite $HGFE$
:Face $ADEH$ is opposite $BCFG$
:Face $ABGH$ is opposite $DCFE$
:

The opposite face of the face $F$ of a parallelepiped $P$ is the face of $P$ which is parallel to $F$.

In the above example, the pairs of parallel planes are:

:Face $ABCD$ is opposite $HGFE$
:Face $ADEH$ is opposite $BCFG$
:Face $ABGH$ is opposite $DCFE$
",Definition:Parallelepiped/Opposite Face,['Definitions/Parallelepipeds'],":

The opposite face of the face F of a parallelepiped P is the face of P which is parallel to F.

In the above example, the pairs of parallel planes are:

:Face ABCD is opposite HGFE
:Face ADEH is opposite BCFG
:Face ABGH is opposite DCFE
:

The opposite face of the face F of a parallelepiped P is the face of P which is parallel to F.

In the above example, the pairs of parallel planes are:

:Face ABCD is opposite HGFE
:Face ADEH is opposite BCFG
:Face ABGH is opposite DCFE
:

The opposite face of the face F of a parallelepiped P is the face of P which is parallel to F.

In the above example, the pairs of parallel planes are:

:Face ABCD is opposite HGFE
:Face ADEH is opposite BCFG
:Face ABGH is opposite DCFE
"
Definition:Opposite,Opposite,"Let $\struct {R, +, \times}$ be a ring.


Let $* : R \times R \to R$ be the binary operation on $S$ defined by:
:$\forall x, y \in S: x * y = y \times x$

The opposite ring of $R$ is the algebraic structure $\struct {R, +, *}$.",Definition:Opposite Ring,['Definitions/Examples of Rings'],"Let R, +, × be a ring.


Let * : R × R → R be the binary operation on S defined by:
:∀ x, y ∈ S: x * y = y × x

The opposite ring of R is the algebraic structure R, +, *."
Definition:Opposite,Opposite,"Let $\mathbf C$ be a metacategory.


Its dual category, denoted $\mathbf C^{\text{op} }$, is defined as follows:



It can be seen that this comes down to the metacategory obtained by reversing the direction of all morphisms of $\mathbf C$.",Definition:Dual Category,"['Definitions/Category Theory', 'Definitions/Examples of Categories']","Let 𝐂 be a metacategory.


Its dual category, denoted 𝐂^op, is defined as follows:



It can be seen that this comes down to the metacategory obtained by reversing the direction of all morphisms of 𝐂."
Definition:Opposite,Opposite,"Let $\mathbf v$ and $\mathbf w$ be vectors in space.

We say that $\mathbf v$ is in the opposite direction to $\mathbf w$ :

:the lines of action of $\mathbf v$ and $\mathbf w$ are parallel
but at the same time:
:the lines of action of $\mathbf v$ and $\mathbf w$ are not the same.

Category:Definitions/Vectors",Definition:Opposite Direction,['Definitions/Vectors'],"Let 𝐯 and 𝐰 be vectors in space.

We say that 𝐯 is in the opposite direction to 𝐰 :

:the lines of action of 𝐯 and 𝐰 are parallel
but at the same time:
:the lines of action of 𝐯 and 𝐰 are not the same.

Category:Definitions/Vectors"
Definition:Order,Order,"Let $\mathbf A$ be an $n \times n$ square matrix.

That is, let $\mathbf A$ have $n$ rows (and by definition $n$ columns).


Then the order of $\mathbf A$ is defined as being $n$.
",Definition:Matrix/Square Matrix/Order,"['Definitions/Orders of Matrices', 'Definitions/Square Matrices']","Let 𝐀 be an n × n square matrix.

That is, let 𝐀 have n rows (and by definition n columns).


Then the order of 𝐀 is defined as being n.
"
Definition:Order,Order,"Let $a$ and $n$ be integers.

Let there exist a positive integer $c$ such that:

:$a^c \equiv 1 \pmod n$


Then the least such integer is called order of $a$ modulo $n$.",Definition:Multiplicative Order of Integer,['Definitions/Number Theory'],"Let a and n be integers.

Let there exist a positive integer c such that:

:a^c ≡ 1  n


Then the least such integer is called order of a modulo n."
Definition:Order,Order,"Let $m$ be a positive integer.

Let $s \left({m}\right)$ be the aliquot sum of $m$.


Let a sequence $\left\langle{a_k}\right\rangle$ be a sociable chain.

The order of $a_k$ is the smallest $r \in \Z_{>0}$ such that
:$a_r = a_0$


Category:Definitions/Sociable Numbers",Definition:Sociable Chain/Order,['Definitions/Sociable Numbers'],"Let m be a positive integer.

Let s (m) be the aliquot sum of m.


Let a sequence ⟨a_k⟩ be a sociable chain.

The order of a_k is the smallest r ∈_>0 such that
:a_r = a_0


Category:Definitions/Sociable Numbers"
Definition:Order,Order,,Definition:Degree of Polynomial,"['Definitions/Degree of Polynomial', 'Definitions/Polynomial Theory']",
Definition:Order,Order,"The order of a derivative is the number of times it has been differentiated.

For example:
:a first derivative is of first order, or order $1$
:a second derivative is of second order, or order $2$

and so on.",Definition:Derivative/Higher Derivatives/Order of Derivative,"['Definitions/Order of Derivative', 'Definitions/Higher Derivatives']","The order of a derivative is the number of times it has been differentiated.

For example:
:a first derivative is of first order, or order 1
:a second derivative is of second order, or order 2

and so on."
Definition:Order,Order,"Let $f: \R \to \mathbb F$ be a function, where $\mathbb F \in \set {\R, \C}$.

Let $f$ be continuous on the real interval $\hointr 0 \to$, except possibly for some finite number of discontinuities of the first kind in every finite subinterval of $\hointr 0 \to$.



Let  $\size {\, \cdot \,}$ be the absolute value if $f$ is real-valued, or the modulus if $f$ is complex-valued.

Let $e^{a t}$ be the exponential function, where $a \in \R$ is constant.


Then $\map f t$ is said to be of exponential order $a$, denoted $f \in \EE_a$,  there exist strictly positive real numbers $M, K$ such that:

:$\forall t \ge M: \size {\map f t} < K e^{a t}$
Let $f: \R \to \mathbb F$ be a function, where $\mathbb F \in \set {\R, \C}$.

Let $f$ be continuous on the real interval $\hointr 0 \to$, except possibly for some finite number of discontinuities of the first kind in every finite subinterval of $\hointr 0 \to$.



Let  $\size {\, \cdot \,}$ be the absolute value if $f$ is real-valued, or the modulus if $f$ is complex-valued.

Let $e^{a t}$ be the exponential function, where $a \in \R$ is constant.


Then $\map f t$ is said to be of exponential order $a$, denoted $f \in \EE_a$,  there exist strictly positive real numbers $M, K$ such that:

:$\forall t \ge M: \size {\map f t} < K e^{a t}$
",Definition:Exponential Order,"['Definitions/Real Analysis', 'Definitions/Complex Analysis']","Let f: →𝔽 be a function, where 𝔽∈,.

Let f be continuous on the real interval 0 →, except possibly for some finite number of discontinuities of the first kind in every finite subinterval of 0 →.



Let  · be the absolute value if f is real-valued, or the modulus if f is complex-valued.

Let e^a t be the exponential function, where a ∈ is constant.


Then f t is said to be of exponential order a, denoted f ∈_a,  there exist strictly positive real numbers M, K such that:

:∀ t ≥ M:  f t < K e^a t
Let f: →𝔽 be a function, where 𝔽∈,.

Let f be continuous on the real interval 0 →, except possibly for some finite number of discontinuities of the first kind in every finite subinterval of 0 →.



Let  · be the absolute value if f is real-valued, or the modulus if f is complex-valued.

Let e^a t be the exponential function, where a ∈ is constant.


Then f t is said to be of exponential order a, denoted f ∈_a,  there exist strictly positive real numbers M, K such that:

:∀ t ≥ M:  f t < K e^a t
"
Definition:Order,Order,"The order of a differential equation is defined as being the order of the highest order derivative that is present in the equation.
The order of a derivative is the number of times it has been differentiated.

For example:
:a first derivative is of first order, or order $1$
:a second derivative is of second order, or order $2$

and so on.
The order of a derivative is the number of times it has been differentiated.

For example:
:a first derivative is of first order, or order $1$
:a second derivative is of second order, or order $2$

and so on.
",Definition:Differential Equation/Order,"['Definitions/Order of Differential Equation', 'Definitions/Differential Equations']","The order of a differential equation is defined as being the order of the highest order derivative that is present in the equation.
The order of a derivative is the number of times it has been differentiated.

For example:
:a first derivative is of first order, or order 1
:a second derivative is of second order, or order 2

and so on.
The order of a derivative is the number of times it has been differentiated.

For example:
:a first derivative is of first order, or order 1
:a second derivative is of second order, or order 2

and so on.
"
Definition:Order,Order,"Let $G = \struct {V, E}$ be a graph.

The order of $G$ is the cardinality of its vertex set.


That is, the order of $G$ is $\card V$.",Definition:Graph (Graph Theory)/Order,['Definitions/Graphs (Graph Theory)'],"Let G = V, E be a graph.

The order of G is the cardinality of its vertex set.


That is, the order of G is V."
Definition:Order,Order,"Let $\mathbf L$ be an $n \times n$ Latin square.

The order of $\mathbf L$ is $n$.


Category:Definitions/Latin Squares",Definition:Latin Square/Order,['Definitions/Latin Squares'],"Let 𝐋 be an n × n Latin square.

The order of 𝐋 is n.


Category:Definitions/Latin Squares"
Definition:Order,Order,"Let $\closedint a b$ be a closed real interval.

Let $T := \set {a = t_0, t_1, t_2, \ldots, t_{n - 1}, t_n = b}$ form a subdivision of $\closedint a b$.

Let $S: \closedint a b \to \R$ be a spline function on $\closedint a b$ on $T$.


Some sources, instead of referring to the degree of a spline, use the order.

Let the maximum degree of the polynomials $P_k$ fitted between $t_k$ and $t_{k + 1}$ be $n$.

The order of $S$ is then $n + 1$.


Category:Definitions/Splines",Definition:Spline Function/Order,['Definitions/Splines'],"Let a b be a closed real interval.

Let T := a = t_0, t_1, t_2, …, t_n - 1, t_n = b form a subdivision of a b.

Let S:  a b → be a spline function on a b on T.


Some sources, instead of referring to the degree of a spline, use the order.

Let the maximum degree of the polynomials P_k fitted between t_k and t_k + 1 be n.

The order of S is then n + 1.


Category:Definitions/Splines"
Definition:Orthogonal,Orthogonal,"Two curves are orthogonal if they intersect at right angles.

The term perpendicular can also be used, but the latter term is usual when the intersecting lines are straight.


=== Orthogonal Circles ===

",Definition:Orthogonal (Analytic Geometry),['Definitions/Orthogonality (Geometry)'],"Two curves are orthogonal if they intersect at right angles.

The term perpendicular can also be used, but the latter term is usual when the intersecting lines are straight.


=== Orthogonal Circles ===

"
Definition:Orthogonal,Orthogonal,"Two circles are orthogonal if their angle of intersection is a right angle.


:
",Definition:Orthogonal Curves,"['Definitions/Orthogonal Curves', 'Definitions/Orthogonality (Geometry)', 'Definitions/Analytic Geometry']","Two circles are orthogonal if their angle of intersection is a right angle.


:
"
Definition:Orthogonal,Orthogonal,"Let $\struct {V, \innerprod \cdot \cdot}$ be an inner product space.

Let $S = \set {u_1, \ldots, u_n}$ be a subset of $V$.


Then $S$ is an orthogonal set  its elements are pairwise orthogonal:

:$\forall i \ne j: \innerprod {u_i} {u_j} = 0$
Let $\struct {V, \innerprod \cdot \cdot}$ be an inner product space.

Let $A, B \subseteq V$.

We say that $A$ and $B$ are orthogonal :

:$\forall a \in A, b \in B: a \perp b$

That is, if $a$ and $b$ are orthogonal elements of $A$ and $B$ for all $a \in A$ and $b \in B$.


We write: 

:$A \perp B$
Let $\mathbb K$ be a field.

Let $V$ be a vector space over $\mathbb K$.

Let $\struct {V, \innerprod \cdot \cdot}$ be an inner product space.

Let $S\subseteq V$ be a subset.


We define the orthogonal complement of $S$ (with respect to $\innerprod \cdot \cdot$), written $S^\perp$ as the set of all $v \in V$ which are orthogonal to all $s \in S$.

That is: 

:$S^\perp = \set {v \in V : \innerprod v s = 0 \text { for all } s \in S}$


If $S = \set v$ is a singleton, we may write $S^\perp$ as $v^\perp$.
",Definition:Orthogonal (Linear Algebra),"['Definitions/Vector Algebra', 'Definitions/Linear Algebra', 'Definitions/Inner Product Spaces', 'Definitions/Orthogonality (Linear Algebra)']","Let V, ·· be an inner product space.

Let S = u_1, …, u_n be a subset of V.


Then S is an orthogonal set  its elements are pairwise orthogonal:

:∀ i  j: u_iu_j = 0
Let V, ·· be an inner product space.

Let A, B ⊆ V.

We say that A and B are orthogonal :

:∀ a ∈ A, b ∈ B: a ⊥ b

That is, if a and b are orthogonal elements of A and B for all a ∈ A and b ∈ B.


We write: 

:A ⊥ B
Let 𝕂 be a field.

Let V be a vector space over 𝕂.

Let V, ·· be an inner product space.

Let S⊆ V be a subset.


We define the orthogonal complement of S (with respect to ··), written S^⊥ as the set of all v ∈ V which are orthogonal to all s ∈ S.

That is: 

:S^⊥ = v ∈ V :  v s = 0  for all  s ∈ S


If S =  v is a singleton, we may write S^⊥ as v^⊥.
"
Definition:Orthogonal,Orthogonal,"Let $\mathbb K$ be a field.

Let $V$ be a vector space over $\mathbb K$.

Let $b: V \times V \to \mathbb K$ be a reflexive bilinear form on $V$.

Let $S, T \subset V$ be subsets.


Then $S$ and $T$ are orthogonal  for all $s \in S$ and $t \in T$, $s$ and $t$ are orthogonal: $s \perp t$.
Let $\mathbb K$ be a field.

Let $V$ be a vector space over $\mathbb K$.

Let $b : V\times V \to \mathbb K$ be a reflexive bilinear form on $V$.

Let $S\subset V$ be a subset.


The orthogonal complement of $S$ (with respect to $b$) is the set of all $v \in V$ which are orthogonal to all $s \in S$.


This is denoted: $S^\perp$.

If $S = \set v$ is a singleton, we also write $v^\perp$.
",Definition:Orthogonal (Bilinear Form),['Definitions/Bilinear Forms (Linear Algebra)'],"Let 𝕂 be a field.

Let V be a vector space over 𝕂.

Let b: V × V →𝕂 be a reflexive bilinear form on V.

Let S, T ⊂ V be subsets.


Then S and T are orthogonal  for all s ∈ S and t ∈ T, s and t are orthogonal: s ⊥ t.
Let 𝕂 be a field.

Let V be a vector space over 𝕂.

Let b : V× V →𝕂 be a reflexive bilinear form on V.

Let S⊂ V be a subset.


The orthogonal complement of S (with respect to b) is the set of all v ∈ V which are orthogonal to all s ∈ S.


This is denoted: S^⊥.

If S =  v is a singleton, we also write v^⊥.
"
Definition:Orthogonal,Orthogonal,"Let $H$ be a Hilbert space.

Let $M, N$ be closed linear subspaces of $H$.


Then the orthogonal difference of $M$ and $N$, denoted $M \ominus N$, is the set $M \cap N^\perp$.

",Definition:Orthogonal Difference,['Definitions/Hilbert Spaces'],"Let H be a Hilbert space.

Let M, N be closed linear subspaces of H.


Then the orthogonal difference of M and N, denoted M ⊖ N, is the set M ∩ N^⊥.

"
Definition:Orthogonal,Orthogonal,,Definition:Orthogonal Complement,[],
Definition:Orthogonal Complement,Orthogonal Complement,"Let $\mathbb K$ be a field.

Let $V$ be a vector space over $\mathbb K$.

Let $\struct {V, \innerprod \cdot \cdot}$ be an inner product space.

Let $S\subseteq V$ be a subset.


We define the orthogonal complement of $S$ (with respect to $\innerprod \cdot \cdot$), written $S^\perp$ as the set of all $v \in V$ which are orthogonal to all $s \in S$.

That is: 

:$S^\perp = \set {v \in V : \innerprod v s = 0 \text { for all } s \in S}$


If $S = \set v$ is a singleton, we may write $S^\perp$ as $v^\perp$.",Definition:Orthogonal (Linear Algebra)/Orthogonal Complement,['Definitions/Orthogonality (Linear Algebra)'],"Let 𝕂 be a field.

Let V be a vector space over 𝕂.

Let V, ·· be an inner product space.

Let S⊆ V be a subset.


We define the orthogonal complement of S (with respect to ··), written S^⊥ as the set of all v ∈ V which are orthogonal to all s ∈ S.

That is: 

:S^⊥ = v ∈ V :  v s = 0  for all  s ∈ S


If S =  v is a singleton, we may write S^⊥ as v^⊥."
Definition:Orthogonal Complement,Orthogonal Complement,"Let $\mathbb K$ be a field.

Let $V$ be a vector space over $\mathbb K$.

Let $b : V\times V \to \mathbb K$ be a reflexive bilinear form on $V$.

Let $S\subset V$ be a subset.


The orthogonal complement of $S$ (with respect to $b$) is the set of all $v \in V$ which are orthogonal to all $s \in S$.


This is denoted: $S^\perp$.

If $S = \set v$ is a singleton, we also write $v^\perp$.",Definition:Orthogonal (Bilinear Form)/Orthogonal Complement,['Definitions/Bilinear Forms (Linear Algebra)'],"Let 𝕂 be a field.

Let V be a vector space over 𝕂.

Let b : V× V →𝕂 be a reflexive bilinear form on V.

Let S⊂ V be a subset.


The orthogonal complement of S (with respect to b) is the set of all v ∈ V which are orthogonal to all s ∈ S.


This is denoted: S^⊥.

If S =  v is a singleton, we also write v^⊥."
Definition:Parallel,Parallel,"Let $L$ be a straight line.

Let $P$ be a plane.

Then $L$ and $P$ are parallel , when produced indefinitely, they do not intersect at any point.


Category:Definitions/Parallel
",Definition:Parallel (Geometry),"['Definitions/Euclidean Geometry', 'Definitions/Parallel']","Let L be a straight line.

Let P be a plane.

Then L and P are parallel , when produced indefinitely, they do not intersect at any point.


Category:Definitions/Parallel
"
Definition:Parallel,Parallel,"
:


The contemporary interpretation of the concept of parallelism declares that a straight line is parallel to itself.
",Definition:Parallel (Geometry)/Lines,"['Definitions/Parallel Lines', 'Definitions/Parallel']","
:


The contemporary interpretation of the concept of parallelism declares that a straight line is parallel to itself.
"
Definition:Parallel,Parallel,"Two planes are parallel , when produced indefinitely, do not intersect at any point.





The contemporary interpretation of the concept of parallelism declares that a plane is parallel to itself.

:


The contemporary interpretation of the concept of parallelism declares that a straight line is parallel to itself.
",Definition:Parallel (Geometry)/Planes,"['Definitions/Parallel Planes', 'Definitions/Parallel']","Two planes are parallel , when produced indefinitely, do not intersect at any point.





The contemporary interpretation of the concept of parallelism declares that a plane is parallel to itself.

:


The contemporary interpretation of the concept of parallelism declares that a straight line is parallel to itself.
"
Definition:Parallel,Parallel,"Let $L$ be a straight line.

Let $P$ be a plane.

Then $L$ and $P$ are parallel , when produced indefinitely, they do not intersect at any point.


Category:Definitions/Parallel",Definition:Parallel (Geometry)/Line to Plane,['Definitions/Parallel'],"Let L be a straight line.

Let P be a plane.

Then L and P are parallel , when produced indefinitely, they do not intersect at any point.


Category:Definitions/Parallel"
Definition:Parallel,Parallel,"Let $M = \struct {S, \mathscr I}$ be a matroid.


Two elements $x, y \in S$ are said to be parallel in $M$  they are not loops but $\set {x, y}$ is a dependent subset of $S$.


That is, $x, y \in S$ are parallel :
:$\set x, \set y \in \mathscr I$ and $\set {x, y} \notin \mathscr I$.
Let $M = \struct {S, \mathscr I}$ be a matroid.


Two elements $x, y \in S$ are said to be parallel in $M$  they are not loops but $\set {x, y}$ is a dependent subset of $S$.


That is, $x, y \in S$ are parallel :
:$\set x, \set y \in \mathscr I$ and $\set {x, y} \notin \mathscr I$.
",Definition:Parallel (Matroid),['Definitions/Matroid Theory'],"Let M = S, ℐ be a matroid.


Two elements x, y ∈ S are said to be parallel in M  they are not loops but x, y is a dependent subset of S.


That is, x, y ∈ S are parallel :
:x,  y ∈ℐ and x, y∉ℐ.
Let M = S, ℐ be a matroid.


Two elements x, y ∈ S are said to be parallel in M  they are not loops but x, y is a dependent subset of S.


That is, x, y ∈ S are parallel :
:x,  y ∈ℐ and x, y∉ℐ.
"
Definition:Parameter,Parameter,"Consider the integral equation:

:of the first kind:
::$\map f x = \lambda \ds \int_{\map a x}^{\map b x} \map K {x, y} \map g y \rd x$

:of the second kind:
::$\map g x = \map f x + \lambda \ds \int_{\map a x}^{\map b x} \map K {x, y} \map g y \rd x$

:of the third kind:
::$\map u x \map g x = \map f x + \lambda \ds \int_{\map a x}^{\map b x} \map K {x, y} \map g y \rd x$


The number $\lambda$ is known as the parameter of the integral equation.
",Definition:Integral Equation/Parameter,"['Definitions/Integral Equations', 'Definitions/Parameters']","Consider the integral equation:

:of the first kind:
::f x = λ∫_ a x^ b x K x, y g y  x

:of the second kind:
::g x =  f x + λ∫_ a x^ b x K x, y g y  x

:of the third kind:
::u x  g x =  f x + λ∫_ a x^ b x K x, y g y  x


The number λ is known as the parameter of the integral equation.
"
Definition:Parameter,Parameter,"A population parameter is a numerical description of a population.
",Definition:Population Parameter,"['Definitions/Population Parameters', 'Definitions/Descriptive Statistics']","A population parameter is a numerical description of a population.
"
Definition:Parameter,Parameter,"Let $f$ be a differential equation with general solution $F$.

A parameter of $F$ is an arbitrary constant arising from the solving of a primitive during the course of obtaining the solution of $f$.",Definition:Parameter of Differential Equation,['Definitions/Differential Equations'],"Let f be a differential equation with general solution F.

A parameter of F is an arbitrary constant arising from the solving of a primitive during the course of obtaining the solution of f."
Definition:Parameter,Parameter,"Consider the implicit function $\map f {x, y, c} = 0$ in the cartesian $\tuple {x, y}$-plane where $c$ is a constant.


For each value of $c$, we have that $\map f {x, y, z, c} = 0$ defines a relation between $x$ and $y$ which can be graphed in the cartesian plane.

Thus, each value of $c$ defines a particular curve.


The complete set of all these curve for each value of $c$ is called a one-parameter family of curves.


=== Parameter ===

Let $\map f {x, y, c}$ define a one-parameter family of curves $F$.

The value $c$ is the parameter of $F$.
",Definition:Family of Curves/One-Parameter/Parameter,"['Definitions/One-Parameter Families of Curves', 'Definitions/One-Parameter Families']","Consider the implicit function f x, y, c = 0 in the cartesian x, y-plane where c is a constant.


For each value of c, we have that f x, y, z, c = 0 defines a relation between x and y which can be graphed in the cartesian plane.

Thus, each value of c defines a particular curve.


The complete set of all these curve for each value of c is called a one-parameter family of curves.


=== Parameter ===

Let f x, y, c define a one-parameter family of curves F.

The value c is the parameter of F.
"
Definition:Path,Path,"Let $T$ be a topological space.

Let $f, g: \closedint 0 1 \to T$ be paths.


$f$ and $g$ are said to be composable paths if:

:$\map f 1 = \map g 0$.
",Definition:Path (Topology),"['Definitions/Paths (Topology)', 'Definitions/Path-Connected Spaces', 'Definitions/Complex Analysis', 'Definitions/Topology']","Let T be a topological space.

Let f, g:  0 1 → T be paths.


f and g are said to be composable paths if:

:f 1 =  g 0.
"
Definition:Path,Path,"An open path is a path in which the first and last vertices are distinct.



=== Endpoint of Open Path ===

",Definition:Path (Graph Theory),"['Definitions/Paths (Graph Theory)', 'Definitions/Graph Theory']","An open path is a path in which the first and last vertices are distinct.



=== Endpoint of Open Path ===

"
Definition:Permutable,Permutable,"A permutable prime is a prime number $p$ which has the property that all anagrams of $p$ are prime.


=== Sequence ===
",Definition:Permutable Prime,"['Definitions/Permutable Primes', 'Definitions/Number Theory', 'Definitions/Recreational Mathematics', 'Definitions/Prime Numbers']","A permutable prime is a prime number p which has the property that all anagrams of p are prime.


=== Sequence ===
"
Definition:Permutable,Permutable,"Let $\struct {G, \circ}$ be a group.

Let $H$ and $K$ be subgroups of $G$.

Let $H \circ K$ denote the subset product of $H$ and $K$.


Then $H$ and $K$ are permutable :
:$H \circ K = K \circ H$",Definition:Permutable Subgroups,['Definitions/Group Theory'],"Let G, ∘ be a group.

Let H and K be subgroups of G.

Let H ∘ K denote the subset product of H and K.


Then H and K are permutable :
:H ∘ K = K ∘ H"
Definition:Permutable,Permutable,"Let $\circ$ be a binary operation.


Two elements $x, y$ are said to commute (with each other) :
:$x \circ y = y \circ x$


Thus $x$ and $y$ can be described as commutative (elements) under $\circ$.",Definition:Commutative/Elements,['Definitions/Commutativity'],"Let ∘ be a binary operation.


Two elements x, y are said to commute (with each other) :
:x ∘ y = y ∘ x


Thus x and y can be described as commutative (elements) under ∘."
Definition:Polar,Polar,"A polar equation is an equation defining the locus of a set of points in the polar coordinate plane.

Such an equation is generally presented in terms of the variables:
:$r$: the radial coordinate
:$\theta$: the angular coordinate
A plane upon which a system of polar coordinates has been applied is known as a polar coordinate plane.
",Definition:Polar Equation,"['Definitions/Polar Equations', 'Definitions/Polar Coordinates']","A polar equation is an equation defining the locus of a set of points in the polar coordinate plane.

Such an equation is generally presented in terms of the variables:
:r: the radial coordinate
:θ: the angular coordinate
A plane upon which a system of polar coordinates has been applied is known as a polar coordinate plane.
"
Definition:Polar,Polar,"The polar axis of a spherical coordinate system is the vertical straight line which passes through the origin $O$.
",Definition:Spherical Coordinate System/Polar Axis,['Definitions/Spherical Coordinates'],"The polar axis of a spherical coordinate system is the vertical straight line which passes through the origin O.
"
Definition:Polar,Polar,"Let $\triangle ABC$ be a spherical triangle on the surface of a sphere whose center is $O$.

Let the sides $a, b, c$ of $\triangle ABC$ be measured by the angles subtended at $O$, where $a, b, c$ are opposite $A, B, C$ respectively.


Let $A'$, $B'$ and $C'$ be the poles of the sides $BC$, $AC$ and $AB$ respectively which are in the same hemisphere as the points $A$, $B$ and $C$ respectively.

:

Then the spherical triangle $\triangle A'B'C'$ is the polar triangle of $\triangle ABC$.",Definition:Polar Triangle,['Definitions/Spherical Triangles'],"Let ABC be a spherical triangle on the surface of a sphere whose center is O.

Let the sides a, b, c of ABC be measured by the angles subtended at O, where a, b, c are opposite A, B, C respectively.


Let A', B' and C' be the poles of the sides BC, AC and AB respectively which are in the same hemisphere as the points A, B and C respectively.

:

Then the spherical triangle A'B'C' is the polar triangle of ABC."
Definition:Polar,Polar,A polar vector is a vector quantity whose action is along a line drawn in the direction of the vector itself.,Definition:Polar Vector,"['Definitions/Vectors', 'Definitions/Mechanics']",A polar vector is a vector quantity whose action is along a line drawn in the direction of the vector itself.
Definition:Power,Power,"Let $r \in \R$ be a real number.

(This includes the situation where $r \in \Z$ or $r \in \Q$.)

When $x=0$, $x^r$ is defined as follows:

:$0^r = \begin{cases}
1 & : r = 0 \\
0 & : r > 0 \\
\text{Undefined} & : r < 0 \\
\end{cases}$

This takes account of the awkward case $0^0$: it is ""generally accepted"" that $0^0 = 1$ as this convention agrees with certain general results which would otherwise need a special case.
",Definition:Power (Algebra),"['Definitions/Powers', 'Definitions/Algebra', 'Definitions/Numbers', 'Definitions/Real Analysis', 'Definitions/Complex Analysis', 'Definitions/Involution']","Let r ∈ be a real number.

(This includes the situation where r ∈ or r ∈.)

When x=0, x^r is defined as follows:

:0^r = 
1     : r = 0 

0     : r > 0 
Undefined    : r < 0

This takes account of the awkward case 0^0: it is ""generally accepted"" that 0^0 = 1 as this convention agrees with certain general results which would otherwise need a special case.
"
Definition:Power,Power,"Let $\struct {S, \circ}$ be a magma which has no identity element.

Let $a \in S$.


Let the mapping $\circ^n a: \N_{>0} \to S$ be recursively defined as:

:$\forall n \in \N_{>0}: \circ^n a = \begin{cases}
a & : n = 1 \\
\paren {\circ^r a} \circ a & : n = r + 1
\end{cases}$


The mapping $\circ^n a$ is known as the $n$th power of $a$ (under $\circ$).


=== Notation ===
",Definition:Power of Element/Magma,"['Definitions/Magmas', 'Definitions/Powers (Abstract Algebra)']","Let S, ∘ be a magma which has no identity element.

Let a ∈ S.


Let the mapping ∘^n a: _>0→ S be recursively defined as:

:∀ n ∈_>0: ∘^n a = 
a     : n = 1 
∘^r a∘ a     : n = r + 1


The mapping ∘^n a is known as the nth power of a (under ∘).


=== Notation ===
"
Definition:Power,Power,"Let $\struct {S, \circ}$ be a magma with an identity element $e$.

Let $a \in S$.


Let the mapping $\circ^n a: \N \to S$ be recursively defined as:

:$\forall n \in S: \circ^n a = \begin{cases}
e & : n = 0 \\
\paren {\circ^r a} \circ a & : n = r + 1
\end{cases}$


The mapping $\circ^n a$ is known as the $n$th power of $a$ (under $\circ$).


=== Notation ===


Furthermore:
:$a^0 = \circ^0 a = e$",Definition:Power of Element/Magma with Identity,"['Definitions/Magmas', 'Definitions/Powers (Abstract Algebra)']","Let S, ∘ be a magma with an identity element e.

Let a ∈ S.


Let the mapping ∘^n a: → S be recursively defined as:

:∀ n ∈ S: ∘^n a = 
e     : n = 0 
∘^r a∘ a     : n = r + 1


The mapping ∘^n a is known as the nth power of a (under ∘).


=== Notation ===


Furthermore:
:a^0 = ∘^0 a = e"
Definition:Power,Power,"Let $\struct {S, \circ}$ be a magma which has no identity element.

Let $a \in S$.


Let the mapping $\circ^n a: \N_{>0} \to S$ be recursively defined as:

:$\forall n \in \N_{>0}: \circ^n a = \begin{cases}
a & : n = 1 \\
\paren {\circ^r a} \circ a & : n = r + 1
\end{cases}$


The mapping $\circ^n a$ is known as the $n$th power of $a$ (under $\circ$).


=== Notation ===

",Definition:Power of Element/Semigroup,"['Definitions/Semigroups', 'Definitions/Powers (Abstract Algebra)']","Let S, ∘ be a magma which has no identity element.

Let a ∈ S.


Let the mapping ∘^n a: _>0→ S be recursively defined as:

:∀ n ∈_>0: ∘^n a = 
a     : n = 1 
∘^r a∘ a     : n = r + 1


The mapping ∘^n a is known as the nth power of a (under ∘).


=== Notation ===

"
Definition:Power,Power,"Let $\struct {S, \circ}$ be a semigroup which has no identity element.

Let $a \in S$.


For $n \in \N_{>0}$, the $n$th power of $a$ (under $\circ$) is defined as:

:$\circ^n a = \begin{cases} a & : n = 1 \\ \paren {\circ^m a} \circ a & : n = m + 1 \end{cases}$

That is:
:$a^n = \underbrace {a \circ a \circ \cdots \circ a}_{n \text{ copies of } a}$

which from the General Associativity Theorem is unambiguous.


=== Notation ===

",Definition:Power of Element/Monoid,['Definitions/Monoids'],"Let S, ∘ be a semigroup which has no identity element.

Let a ∈ S.


For n ∈_>0, the nth power of a (under ∘) is defined as:

:∘^n a =  a     : n = 1 
∘^m a∘ a     : n = m + 1

That is:
:a^n = a ∘ a ∘⋯∘ a_n  copies of  a

which from the General Associativity Theorem is unambiguous.


=== Notation ===

"
Definition:Power,Power,"Let $\struct {S, \circ}$ be a monoid whose identity element is $e$.

Let $a \in S$.

Let $n \in \N$.


The definition $a^n = \map {\circ^n} a$ as the $n$th power of $a$ in a semigroup can be extended to allow an exponent of $0$:

:$a^n = \begin {cases}
e & : n = 0 \\
a^{n - 1} \circ a & : n > 0
\end{cases}$

or:

:$n \cdot a = \begin {cases}
e & : n = 0 \\
\paren {\paren {n - 1} \cdot a} \circ a & : n > 0
\end{cases}$


The validity of this definition follows from the fact that a monoid has an identity element.


=== Invertible Element ===

",Definition:Power of Element/Group,['Definitions/Group Theory'],"Let S, ∘ be a monoid whose identity element is e.

Let a ∈ S.

Let n ∈.


The definition a^n = ∘^n a as the nth power of a in a semigroup can be extended to allow an exponent of 0:

:a^n = 
e     : n = 0 

a^n - 1∘ a     : n > 0

or:

:n · a = 
e     : n = 0 
n - 1· a∘ a     : n > 0


The validity of this definition follows from the fact that a monoid has an identity element.


=== Invertible Element ===

"
Definition:Power,Power,"Let $\struct {X, \circ}$ be a $B$-algebra.

For any $x \in X$ and $n \in \N$, define the $n$th power of $x$, denoted $x^n$, inductively:

:$x^n = \begin{cases}
0 & \text {if $n = 0$} \\
x^{n - 1} \circ \paren {0 \circ x} & \text {if $n \ge 1$}
\end{cases}$",Definition:Power (B-Algebra),['Definitions/B-Algebras'],"Let X, ∘ be a B-algebra.

For any x ∈ X and n ∈, define the nth power of x, denoted x^n, inductively:

:x^n = 
0    if n = 0

x^n - 1∘0 ∘ x   if n ≥ 1"
Definition:Power,Power,"The power set of a set $S$ is the set defined and denoted as:

:$\powerset S := \set {T: T \subseteq S}$

That is, the set whose elements are all of the subsets of $S$.


=== Class Theory ===
",Definition:Power Set,"['Definitions/Set Theory', 'Definitions/Power Set']","The power set of a set S is the set defined and denoted as:

:S := T: T ⊆ S

That is, the set whose elements are all of the subsets of S.


=== Class Theory ===
"
Definition:Prime Element,Prime Element,"Let $R$ be a commutative ring.

Let $p \in R \setminus \set 0$ be any non-zero element of $R$.


Then $p$ is a prime element of $R$ :
:$(1): \quad p$ is not a unit of $R$ 
:$(2): \quad$ whenever $a, b \in R$ such that $p$ divides $a b$, then either $p$ divides $a$ or $p$ divides $b$.",Definition:Prime Element of Ring,"['Definitions/Ring Theory', 'Definitions/Factorization', 'Definitions/Prime Elements of Rings']","Let R be a commutative ring.

Let p ∈ R ∖ 0 be any non-zero element of R.


Then p is a prime element of R :
:(1):    p is not a unit of R 
:(2): whenever a, b ∈ R such that p divides a b, then either p divides a or p divides b."
Definition:Prime Element,Prime Element,"Let $\struct {S, \wedge, \preceq}$ be a meet semilattice.

Let $p \in S$.


Then $p$ is a prime element (of $\struct {S, \wedge, \preceq}$) :
:$\forall x, y \in S: \paren {x \wedge y \preceq p \implies x \preceq p \text { or } y \preceq p}$",Definition:Prime Element (Order Theory),['Definitions/Order Theory'],"Let S, ∧, ≼ be a meet semilattice.

Let p ∈ S.


Then p is a prime element (of S, ∧, ≼) :
:∀ x, y ∈ S: x ∧ y ≼ p  x ≼ p  or  y ≼ p"
Definition:Prime Ideal,Prime Ideal,"Let $R$ be a ring.


A prime ideal of $R$ is a proper ideal $P$ such that:
:$I \circ J \subseteq P \implies I \subseteq P \text { or } J \subseteq P$
for any ideals $I$ and $J$ of $R$.
",Definition:Prime Ideal of Ring,"['Definitions/Ideal Theory', 'Definitions/Prime Ideals of Rings']","Let R be a ring.


A prime ideal of R is a proper ideal P such that:
:I ∘ J ⊆ P  I ⊆ P  or  J ⊆ P
for any ideals I and J of R.
"
Definition:Prime Ideal,Prime Ideal,"Let $I$ be an ideal in an ordered set $S$.


Then $I$ is a prime ideal in $S$  $S \setminus I$ is a filter.",Definition:Prime Ideal (Order Theory),['Definitions/Order Theory'],"Let I be an ideal in an ordered set S.


Then I is a prime ideal in S  S ∖ I is a filter."
Definition:Primitive,Primitive,"Let $F$ be a real function which is continuous on the closed interval $\closedint a b$ and differentiable on the open interval $\openint a b$.

Let $f$ be a real function which is continuous on the open interval $\openint a b$.


Let:
:$\forall x \in \openint a b: \map {F'} x = \map f x$
where $F'$ denotes the derivative of $F$  $x$.


Then $F$ is a primitive of $f$, and is denoted:
:$\ds F = \int \map f x \rd x$
Let $F: D \to \C$ be a complex function which is complex-differentiable on a connected domain $D$.

Let $f: D \to \C$ be a continuous complex function.


Let:
:$\forall z \in D: \map {F'} z = \map f z$
where $F'$ denotes the derivative of $F$  $z$.


Then $F$ is a primitive of $f$, and is denoted:
:$\ds F = \int \map f z \rd z$
Let $U \subset \R$ be an open set in $\R$.

Let $\mathbf f: U \to \R^n$ be a vector-valued function on $U$:

:$\forall x \in U: \map {\mathbf f} x = \ds \sum_{k \mathop = 1}^n \map {f_k} x \mathbf e_k$

where:
:$f_1, f_2, \ldots, f_n$ are real functions from $U$ to $\R$
:$\tuple {\mathbf e_1, \mathbf e_2, \ldots, \mathbf e_k}$ denotes the standard ordered basis on $\R^n$.

Let $\mathbf f$ be differentiable on $U$.


Let $\map {\mathbf g} x := \dfrac \d {\d x} \map {\mathbf f} x$ be the derivative of $\mathbf f$  $x$.


The primitive of $\mathbf g$  $x$ is defined as:

:$\ds \int \map {\mathbf g} x \rd x := \map {\mathbf f} x + \mathbf c$

where $\mathbf c$ is a arbitrary constant vector.
",Definition:Primitive (Calculus),"['Definitions/Primitives', 'Definitions/Integral Calculus']","Let F be a real function which is continuous on the closed interval a b and differentiable on the open interval a b.

Let f be a real function which is continuous on the open interval a b.


Let:
:∀ x ∈ a b: F' x =  f x
where F' denotes the derivative of F  x.


Then F is a primitive of f, and is denoted:
:F = ∫ f x  x
Let F: D → be a complex function which is complex-differentiable on a connected domain D.

Let f: D → be a continuous complex function.


Let:
:∀ z ∈ D: F' z =  f z
where F' denotes the derivative of F  z.


Then F is a primitive of f, and is denoted:
:F = ∫ f z  z
Let U ⊂ be an open set in .

Let 𝐟: U →^n be a vector-valued function on U:

:∀ x ∈ U: 𝐟 x = ∑_k  = 1^n f_k x 𝐞_k

where:
:f_1, f_2, …, f_n are real functions from U to 
:𝐞_1, 𝐞_2, …, 𝐞_k denotes the standard ordered basis on ^n.

Let 𝐟 be differentiable on U.


Let 𝐠 x := x̣̣̣𝐟 x be the derivative of 𝐟  x.


The primitive of 𝐠  x is defined as:

:∫𝐠 x  x := 𝐟 x + 𝐜

where 𝐜 is a arbitrary constant vector.
"
Definition:Primitive,Primitive,"Let $\Q \sqbrk X$ be the ring of polynomial forms over the field of rational numbers in the indeterminate $X$.

Let $f \in \Q \sqbrk X$ be such that:
:$\cont f = 1$
where $\cont f$ is the content of $f$.


That is:
:The greatest common divisor of the coefficients of $f$ is equal to $1$.



Then $f$ is described as primitive.",Definition:Primitive Polynomial (Ring Theory),['Definitions/Polynomial Theory'],"Let X be the ring of polynomial forms over the field of rational numbers in the indeterminate X.

Let f ∈ X be such that:
:f = 1
where f is the content of f.


That is:
:The greatest common divisor of the coefficients of f is equal to 1.



Then f is described as primitive."
Definition:Primitive,Primitive,"Let $\tuple {x, y, z}$ be a Pythagorean triple such that $x \perp y$ (that is, $x$ and $y$ are coprime).

Then $\tuple {x, y, z}$ is a primitive Pythagorean triple.


=== Canonical Form ===
",Definition:Pythagorean Triple/Primitive,['Definitions/Pythagorean Triples'],"Let x, y, z be a Pythagorean triple such that x ⊥ y (that is, x and y are coprime).

Then x, y, z is a primitive Pythagorean triple.


=== Canonical Form ===
"
Definition:Primitive,Primitive,"The sequence of primitive abundant numbers begins:
:$20, 70, 88, 104, 272, 304, 368, 464, 550, 572, 650, 748, 836, 945, 1184, 1312, \ldots$


",Definition:Primitive Abundant Number,"['Definitions/Abundance', 'Definitions/Abundancy', 'Definitions/Abundant Numbers', 'Definitions/Primitive Abundant Numbers']","The sequence of primitive abundant numbers begins:
:20, 70, 88, 104, 272, 304, 368, 464, 550, 572, 650, 748, 836, 945, 1184, 1312, …


"
Definition:Primitive,Primitive,"The sequence of primitive semiperfect numbers begins:
:$6, 20, 28, 88, 104, 272, 304, 350, 368, 464, 490, 496, 550, 572, \ldots$


",Definition:Primitive Semiperfect Number,"['Definitions/Semiperfect Numbers', 'Definitions/Primitive Semiperfect Numbers']","The sequence of primitive semiperfect numbers begins:
:6, 20, 28, 88, 104, 272, 304, 350, 368, 464, 490, 496, 550, 572, …


"
Definition:Primitive,Primitive,"Let $f: \N^k \to \N$ and $g: \N^{k + 2} \to \N$ be functions.

Let $\tuple {n_1, n_2, \ldots, n_k} \in \N^k$.

Then the function $h: \N^{k + 1} \to \N$ is obtained from $f$ and $g$ by primitive recursion :
:$\forall n \in \N: \map h {n_1, n_2, \ldots, n_k, n} = \begin {cases}
\map f {n_1, n_2, \ldots, n_k} & : n = 0 \\
\map g {n_1, n_2, \ldots, n_k, n - 1, \map h {n_1, n_2, \ldots, n_k, n - 1} } & : n > 0 
\end {cases}$


Category:Definitions/Recursion Theory
Let $a \in \N$ be a natural number.

Let $g: \N^2 \to \N$ be a function.

Then the function $h: \N \to \N$ is obtained from the constant $a$ and $g$ by primitive recursion :
:$\forall n \in \N: \map h n = \begin {cases}
a & : n = 0 \\
\map g {n - 1, \map h {n - 1} } & : n > 0 
\end{cases}$
Let $f: \N^k \to \N$ and $g: \N^{k+2} \to \N$ be partial functions.

Let $\tuple {n_1, n_2, \ldots, n_k} \in \N^k$.

Then the partial function $h: \N^{k + 1} \to \N$ is obtained from $f$ and $g$ by primitive recursion :
:$\forall n \in \N: \map h {n_1, n_2, \ldots, n_k, n} \approx \begin {cases}
\map f {n_1, n_2, \ldots, n_k} & : n = 0 \\
\map g {n_1, n_2, \ldots, n_k, n - 1, \map h {n_1, n_2, \ldots, n_k, n - 1} } & : n > 0 
\end{cases}$

where $\approx$ is as defined in Partial Function Equality.


Note that $\map h {n_1, n_2, \ldots, n_k, n}$ is defined only when:
:$\map h {n_1, n_2, \ldots, n_k, n - 1}$ is defined
:$\map g {n_1, n_2, \ldots, n_k, n - 1, \map h {n_1, n_2, \ldots, n_k, n - 1} }$ is defined.


Category:Definitions/Recursion Theory
",Definition:Primitive Recursion,['Definitions/Recursion Theory'],"Let f: ^k → and g: ^k + 2→ be functions.

Let n_1, n_2, …, n_k∈^k.

Then the function h: ^k + 1→ is obtained from f and g by primitive recursion :
:∀ n ∈:  h n_1, n_2, …, n_k, n =  f n_1, n_2, …, n_k    : n = 0 
 g n_1, n_2, …, n_k, n - 1,  h n_1, n_2, …, n_k, n - 1    : n > 0


Category:Definitions/Recursion Theory
Let a ∈ be a natural number.

Let g: ^2 → be a function.

Then the function h: → is obtained from the constant a and g by primitive recursion :
:∀ n ∈:  h n = 
a     : n = 0 
 g n - 1,  h n - 1    : n > 0
Let f: ^k → and g: ^k+2→ be partial functions.

Let n_1, n_2, …, n_k∈^k.

Then the partial function h: ^k + 1→ is obtained from f and g by primitive recursion :
:∀ n ∈:  h n_1, n_2, …, n_k, n≈ f n_1, n_2, …, n_k    : n = 0 
 g n_1, n_2, …, n_k, n - 1,  h n_1, n_2, …, n_k, n - 1    : n > 0

where ≈ is as defined in Partial Function Equality.


Note that h n_1, n_2, …, n_k, n is defined only when:
:h n_1, n_2, …, n_k, n - 1 is defined
:g n_1, n_2, …, n_k, n - 1,  h n_1, n_2, …, n_k, n - 1 is defined.


Category:Definitions/Recursion Theory
"
Definition:Primitive,Primitive,,Definition:Formal Language/Alphabet/Primitive Symbol,['Definitions/Alphabets (Formal Language)'],
Definition:Primitive,Primitive,"For a definition to not be circular, the definer must use already defined terms. 

However, this process cannot go on indefinitely. If we were to insist on everything being defined only using previously defined terms, we would enter an infinite regress.


Concepts that are not defined in terms of previously defined concepts are called undefined terms.

An undefined term is frequently explained by using an ostensive definition: that is, a statement that shows what something is, rather than explains.",Definition:Undefined Term,"['Definitions/Logic', 'Definitions/Definitions']","For a definition to not be circular, the definer must use already defined terms. 

However, this process cannot go on indefinitely. If we were to insist on everything being defined only using previously defined terms, we would enter an infinite regress.


Concepts that are not defined in terms of previously defined concepts are called undefined terms.

An undefined term is frequently explained by using an ostensive definition: that is, a statement that shows what something is, rather than explains."
Definition:Primitive,Primitive,"Let $F / K$ be a simple field extension such that $F = \map K \alpha$.


Then $\alpha$ is a primitive element of $F$.",Definition:Primitive Element of Field Extension,['Definitions/Field Extensions'],"Let F / K be a simple field extension such that F =  K α.


Then α is a primitive element of F."
Definition:Principal,Principal,"In a compound statement, exactly one of its logical connectives has the largest scope.

That connective is called the main connective.

The scope of the main connective comprises the entire compound statement.


",Definition:Main Connective,"['Definitions/Main Connective', 'Definitions/Propositional Logic', 'Definitions/Logic']","In a compound statement, exactly one of its logical connectives has the largest scope.

That connective is called the main connective.

The scope of the main connective comprises the entire compound statement.


"
Definition:Principal,Principal,"Let $\mathbf A = \sqbrk a_{m n}$ be a matrix.

The elements $a_{j j}: j \in \closedint 1 {\min \set {m, n} }$ constitute the main diagonal of $\mathbf A$.

That is, the main diagonal of $\mathbf A$ is the diagonal of $\mathbf A$ from the top left corner, that is, the element $a_{1 1}$, running towards the lower right corner.


=== Diagonal Elements ===
",Definition:Matrix/Diagonal/Main,"['Definitions/Main Diagonal', 'Definitions/Matrix Diagonals', 'Definitions/Matrices']","Let 𝐀 =  a_m n be a matrix.

The elements a_j j: j ∈ 1 minm, n constitute the main diagonal of 𝐀.

That is, the main diagonal of 𝐀 is the diagonal of 𝐀 from the top left corner, that is, the element a_1 1, running towards the lower right corner.


=== Diagonal Elements ===
"
Definition:Principal,Principal,"Let $A$ and $B$ be sets.

Let $f: A \to B$ be a multifunction on $A$.

Let $\sequence {S_i}_{i \mathop \in I}$ be a partitioning of the codomain of $f$ into branches.


It is usual to distinguish one such branch of $f$ from the others, and label it the principal branch of $f$.


=== Principal Value ===

Let $A$ and $B$ be sets.

Let $f: A \to B$ be a multifunction on $A$.

Let $x \in A$ be an element of the domain of $f$.

The principal value of $x$ is the element $y$ of the principal branch of $f$ such that $\map f x = y$.
",Definition:Multifunction/Principal Branch,['Definitions/Multifunctions'],"Let A and B be sets.

Let f: A → B be a multifunction on A.

Let S_i_i ∈ I be a partitioning of the codomain of f into branches.


It is usual to distinguish one such branch of f from the others, and label it the principal branch of f.


=== Principal Value ===

Let A and B be sets.

Let f: A → B be a multifunction on A.

Let x ∈ A be an element of the domain of f.

The principal value of x is the element y of the principal branch of f such that f x = y.
"
Definition:Principal,Principal,"It is understood that the argument of a complex number $z$ is unique only up to multiples of $2 k \pi$.

With this understanding, we can limit the choice of what $\theta$ can be for any given $z$ by requiring that $\theta$ lie in some half open interval of length $2 \pi$.

The most usual of these are:
:$\hointr 0 {2 \pi}$
:$\hointl {-\pi} \pi$

but in theory any such interval may be used.

This interval is known as the principal range.
",Definition:Argument of Complex Number/Principal Range,['Definitions/Argument of Complex Number'],"It is understood that the argument of a complex number z is unique only up to multiples of 2 k π.

With this understanding, we can limit the choice of what θ can be for any given z by requiring that θ lie in some half open interval of length 2 π.

The most usual of these are:
:0 2 π
:-ππ

but in theory any such interval may be used.

This interval is known as the principal range.
"
Definition:Principal,Principal,"It is understood that the argument of a complex number $z$ is unique only up to multiples of $2 k \pi$.

With this understanding, we can limit the choice of what $\theta$ can be for any given $z$ by requiring that $\theta$ lie in some half open interval of length $2 \pi$.

The most usual of these are:
:$\hointr 0 {2 \pi}$
:$\hointl {-\pi} \pi$

but in theory any such interval may be used.

This interval is known as the principal range.
Let $R$ be the principal range of the complex numbers $\C$.

The unique value of $\theta$ in $R$ is known as the principal argument, of $z$.

This is denoted $\Arg z$.

Note the capital $A$.

The standard practice is for $R$ to be $\hointl {-\pi} \pi$.

This ensures that the principal argument is continuous on the real axis for positive numbers.

Thus, if $z$ is represented in the complex plane, the principal argument $\Arg z$ is intuitively defined as the angle which $z$ yields with the real ($y = 0$) axis.



Let $R$ be the principal range of the complex numbers $\C$.

The unique value of $\theta$ in $R$ is known as the principal argument, of $z$.

This is denoted $\Arg z$.

Note the capital $A$.

The standard practice is for $R$ to be $\hointl {-\pi} \pi$.

This ensures that the principal argument is continuous on the real axis for positive numbers.

Thus, if $z$ is represented in the complex plane, the principal argument $\Arg z$ is intuitively defined as the angle which $z$ yields with the real ($y = 0$) axis.



",Definition:Argument of Complex Number/Principal Argument,['Definitions/Argument of Complex Number'],"It is understood that the argument of a complex number z is unique only up to multiples of 2 k π.

With this understanding, we can limit the choice of what θ can be for any given z by requiring that θ lie in some half open interval of length 2 π.

The most usual of these are:
:0 2 π
:-ππ

but in theory any such interval may be used.

This interval is known as the principal range.
Let R be the principal range of the complex numbers .

The unique value of θ in R is known as the principal argument, of z.

This is denoted z.

Note the capital A.

The standard practice is for R to be -ππ.

This ensures that the principal argument is continuous on the real axis for positive numbers.

Thus, if z is represented in the complex plane, the principal argument z is intuitively defined as the angle which z yields with the real (y = 0) axis.



Let R be the principal range of the complex numbers .

The unique value of θ in R is known as the principal argument, of z.

This is denoted z.

Note the capital A.

The standard practice is for R to be -ππ.

This ensures that the principal argument is continuous on the real axis for positive numbers.

Thus, if z is represented in the complex plane, the principal argument z is intuitively defined as the angle which z yields with the real (y = 0) axis.



"
Definition:Principal,Principal,"The principal branch of a complex number raised to a complex power is defined as:

:$z^k = e^{k \Ln z}$

where $\Ln z$ is the principal branch of the natural logarithm.


=== Positive Real Base ===


Category:Definitions/Complex Powers
The principal branch of the complex natural logarithm is usually defined in one of two ways:






It is important to specify which is in force during a particular exposition.
",Definition:Square Root/Complex Number/Principal Square Root,['Definitions/Complex Square Roots'],"The principal branch of a complex number raised to a complex power is defined as:

:z^k = e^k  z

where z is the principal branch of the natural logarithm.


=== Positive Real Base ===


Category:Definitions/Complex Powers
The principal branch of the complex natural logarithm is usually defined in one of two ways:






It is important to specify which is in force during a particular exposition.
"
Definition:Principal,Principal,"Let $A$ and $B$ be sets.

Let $f: A \to B$ be a multifunction on $A$.

Let $\sequence {S_i}_{i \mathop \in I}$ be a partitioning of the codomain of $f$ into branches.


It is usual to distinguish one such branch of $f$ from the others, and label it the principal branch of $f$.


=== Principal Value ===

The principal branch of the complex natural logarithm is usually defined in one of two ways:






It is important to specify which is in force during a particular exposition.
",Definition:Power (Algebra)/Complex Number/Principal Branch,['Definitions/Complex Powers'],"Let A and B be sets.

Let f: A → B be a multifunction on A.

Let S_i_i ∈ I be a partitioning of the codomain of f into branches.


It is usual to distinguish one such branch of f from the others, and label it the principal branch of f.


=== Principal Value ===

The principal branch of the complex natural logarithm is usually defined in one of two ways:






It is important to specify which is in force during a particular exposition.
"
Definition:Principal,Principal,"Let $a, b \in \Z$ be integers such that $b \ne 0$.

From the Division Theorem, we have that:

:$\forall a, b \in \Z, b \ne 0: \exists_1 q, r \in \Z: a = q b + r, 0 \le r < \size b$


The value $r$ is defined as the remainder of $a$ on division by $b$, or the remainder of $\dfrac a b$'''.


=== Real Arguments ===

When $x, y \in \R$ the remainder is still defined:

",Definition:Remainder,"['Definitions/Integer Division', 'Definitions/Integers', 'Definitions/Number Theory', 'Definitions/Discrete Mathematics']","Let a, b ∈ be integers such that b  0.

From the Division Theorem, we have that:

:∀ a, b ∈, b  0: ∃_1 q, r ∈: a = q b + r, 0 ≤ r <  b


The value r is defined as the remainder of a on division by b, or the remainder of a b”'.


=== Real Arguments ===

When x, y ∈ the remainder is still defined:

"
Definition:Principal,Principal,"Let $G$ be a finite abelian group.

The character $\chi_0: G \to \C_{\ne 0}$ defined as:

:$\forall g \in G: \map {\chi_0} g = 1$

is the trivial character on $G$.",Definition:Trivial Character,['Definitions/Analytic Number Theory'],"Let G be a finite abelian group.

The character χ_0: G →_ 0 defined as:

:∀ g ∈ G: χ_0 g = 1

is the trivial character on G."
Definition:Principal,Principal,"Let $S$ be a set.

Let $\powerset S$ denote the power set of $S$.


Let $\FF \subset \powerset S$ be an ultrafilter on $S$ which does not have a cluster point.

Then $\FF$ is a nonprincipal ultrafilter  on $S$.
",Definition:Principal Ultrafilter,['Definitions/Filter Theory'],"Let S be a set.

Let S denote the power set of S.


Let ⊂ S be an ultrafilter on S which does not have a cluster point.

Then  is a nonprincipal ultrafilter  on S.
"
Definition:Principal,Principal,"Principal is defined as a quantity of money that is either borrowed or invested.
",Definition:Principal (Economics),"['Definitions/Principal (Economics)', 'Definitions/Economics']","Principal is defined as a quantity of money that is either borrowed or invested.
"
Definition:Product,Product,"Let $a \times b$ denote the operation of multiplication on two objects $a$ and $b$.

Then the result $a \times b$ is referred to as the product of $a$ and $b$.


Note that the nature of $a$ and $b$ has deliberately been left unspecified.

They could be, for example, numbers, matrices or more complex expressions constructed from such elements.
",Definition:Multiplication/Product,['Definitions/Multiplication'],"Let a × b denote the operation of multiplication on two objects a and b.

Then the result a × b is referred to as the product of a and b.


Note that the nature of a and b has deliberately been left unspecified.

They could be, for example, numbers, matrices or more complex expressions constructed from such elements.
"
Definition:Product,Product,"Let $\sequence {S_n}$ be a sequence of sets. 

The (finite) cartesian product of $\sequence {S_n}$ is defined as:

:$\ds \prod_{k \mathop = 1}^n S_k = \set {\tuple {x_1, x_2, \ldots, x_n}: \forall k \in \N^*_n: x_k \in S_k}$


It is also denoted $S_1 \times S_2 \times \cdots \times S_n$.

Thus $S_1 \times S_2 \times \cdots \times S_n$ is the set of all ordered $n$-tuples $\tuple {x_1, x_2, \ldots, x_n}$ with $x_k \in S_k$.


In particular:
:$\ds \prod_{k \mathop = 1}^2 S_k = S_1 \times S_2$
Let $\sequence {S_n}_{n \mathop \in \N}$ be an infinite sequence of sets. 

The cartesian product of $\sequence {S_n}$ is defined as:

:$\ds \prod_{k \mathop = 1}^\infty S_k = \set {\tuple {x_1, x_2, \ldots, x_n, \ldots}: \forall k \in \N: x_k \in S_k}$


It defines the concept:
:$S_1 \times S_2 \times \cdots \times S_n \times \cdots$

Thus $\ds \prod_{k \mathop = 1}^\infty S_k$ is the set of all infinite sequences $\tuple {x_1, x_2, \ldots, x_n, \ldots}$ with $x_k \in S_k$.
",Definition:Cartesian Product,"['Definitions/Set Theory', 'Definitions/Cartesian Product']","Let S_n be a sequence of sets. 

The (finite) cartesian product of S_n is defined as:

:∏_k  = 1^n S_k = x_1, x_2, …, x_n: ∀ k ∈^*_n: x_k ∈ S_k


It is also denoted S_1 × S_2 ×⋯× S_n.

Thus S_1 × S_2 ×⋯× S_n is the set of all ordered n-tuples x_1, x_2, …, x_n with x_k ∈ S_k.


In particular:
:∏_k  = 1^2 S_k = S_1 × S_2
Let S_n_n ∈ be an infinite sequence of sets. 

The cartesian product of S_n is defined as:

:∏_k  = 1^∞ S_k = x_1, x_2, …, x_n, …: ∀ k ∈: x_k ∈ S_k


It defines the concept:
:S_1 × S_2 ×⋯× S_n ×⋯

Thus ∏_k  = 1^∞ S_k is the set of all infinite sequences x_1, x_2, …, x_n, … with x_k ∈ S_k.
"
Definition:Product,Product,"Let $\struct {G, \circ}$ be a group.


The operation $\circ$ can be referred to as the group law.
Let $\struct {R, *, \circ}$ be a ring.


The distributive operation $\circ$ in $\struct {R, *, \circ}$ is known as the (ring) product.
Let $\struct {F, +, \times}$ be a field.


The distributive operation $\times$ in $\struct {F, +, \times}$ is known as the (field) product.
",Definition:Product (Abstract Algebra),"['Definitions/Operations', 'Definitions/Abstract Algebra', 'Definitions/Multiplication']","Let G, ∘ be a group.


The operation ∘ can be referred to as the group law.
Let R, *, ∘ be a ring.


The distributive operation ∘ in R, *, ∘ is known as the (ring) product.
Let F, +, × be a field.


The distributive operation × in F, +, × is known as the (field) product.
"
Definition:Product,Product,"Let $\struct {G, \circ}$ be a group.


Let $a, b \in G$ such that $ = a \circ b$.

Then $g$ is known as the product of $a$ and $b$.
",Definition:Group Product,['Definitions/Group Theory'],"Let G, ∘ be a group.


Let a, b ∈ G such that = a ∘ b.

Then g is known as the product of a and b.
"
Definition:Product,Product,"Let $\struct {G, \circ}$ be a group.


Let $a, b \in G$ such that $ = a \circ b$.

Then $g$ is known as the product of $a$ and $b$.",Definition:Group Product/Product Element,['Definitions/Group Theory'],"Let G, ∘ be a group.


Let a, b ∈ G such that = a ∘ b.

Then g is known as the product of a and b."
Definition:Product,Product,"Let $\struct {R, *, \circ}$ be a ring.


The distributive operation $\circ$ in $\struct {R, *, \circ}$ is known as the (ring) product.",Definition:Ring (Abstract Algebra)/Product,['Definitions/Ring Theory'],"Let R, *, ∘ be a ring.


The distributive operation ∘ in R, *, ∘ is known as the (ring) product."
Definition:Product,Product,"Let $\struct {F, +, \times}$ be a field.


The distributive operation $\times$ in $\struct {F, +, \times}$ is known as the (field) product.",Definition:Field (Abstract Algebra)/Product,"['Definitions/Field Theory', 'Definitions/Multiplication']","Let F, +, × be a field.


The distributive operation × in F, +, × is known as the (field) product."
Definition:Product,Product,"Let $\mathbf C$ be a metacategory.

Let $A$ and $B$ be objects of $\mathbf C$.


A (binary) product diagram for $A$ and $B$ comprises an object $P$ and morphisms $p_1: P \to A$, $p_2: P \to B$:

::$\begin{xy}\xymatrix@+1em@L+3px{
 A
&
 P
  \ar[l]_*+{p_1}
  \ar[r]^*+{p_2}
&
 B
}\end{xy}$

subjected to the following universal mapping property:


:For any object $X$ and morphisms $x_1, x_2$ like so:

::$\begin{xy}\xymatrix@+1em@L+3px{
 A
&
 X
  \ar[l]_*+{x_1}
  \ar[r]^*+{x_2}
&
 B
}\end{xy}$

:there is a unique morphism $u: X \to P$ such that:

::$\begin{xy}\xymatrix@+1em@L+3px{
&
 X
  \ar[ld]_*+{x_1}
  \ar@{-->}[d]^*+{u}
  \ar[rd]^*+{x_2}

\\
 A
&
 P
  \ar[l]^*+{p_1}
  \ar[r]_*+{p_2}
&
 B
}\end{xy}$

:is a commutative diagram, i.e., $x_1 = p_1 \circ u$ and $x_2 = p_2 \circ u$.


In this situation, $P$ is called a (binary) product of $A$ and $B$ and may be denoted $A \times B$.

Generally, one writes $\left\langle{x_1, x_2}\right\rangle$ for the unique morphism $u$ determined by above diagram.


The morphisms $p_1$ and $p_2$ are often taken to be implicit.

They are called projections; if necessary, $p_1$ can be called the first projection and $p_2$ the second projection.

",Definition:Product (Category Theory),['Definitions/Category Theory'],"Let 𝐂 be a metacategory.

Let A and B be objects of 𝐂.


A (binary) product diagram for A and B comprises an object P and morphisms p_1: P → A, p_2: P → B:

::@+1em@L+3px
 A
   
 P
  [l]_*+p_1[r]^*+p_2   
 B

subjected to the following universal mapping property:


:For any object X and morphisms x_1, x_2 like so:

::@+1em@L+3px
 A
   
 X
  [l]_*+x_1[r]^*+x_2   
 B

:there is a unique morphism u: X → P such that:

::@+1em@L+3px   
 X
  [ld]_*+x_1@–>[d]^*+u[rd]^*+x_2

 A
   
 P
  [l]^*+p_1[r]_*+p_2   
 B

:is a commutative diagram, i.e., x_1 = p_1 ∘ u and x_2 = p_2 ∘ u.


In this situation, P is called a (binary) product of A and B and may be denoted A × B.

Generally, one writes ⟨x_1, x_2⟩ for the unique morphism u determined by above diagram.


The morphisms p_1 and p_2 are often taken to be implicit.

They are called projections; if necessary, p_1 can be called the first projection and p_2 the second projection.

"
Definition:Projection,Projection,"Let $S_1, S_2, \ldots, S_j, \ldots, S_n$ be sets.

Let $\ds \prod_{i \mathop = 1}^n S_i$ be the Cartesian product of $S_1, S_2, \ldots, S_n$.

For each $j \in \set {1, 2, \ldots, n}$, the $j$th projection on $\ds S = \prod_{i \mathop = 1}^n S_i$ is the mapping $\pr_j: S \to S_j$ defined by:
:$\map {\pr_j} {s_1, s_2, \ldots, s_j, \ldots, s_n} = s_j$

for all $\tuple {s_1, s_2, \ldots, s_n} \in S$.


=== Family of Sets ===
",Definition:Projection (Mapping Theory),"['Definitions/Mapping Theory', 'Definitions/Cartesian Product', 'Definitions/Metric Spaces', 'Definitions/Topology']","Let S_1, S_2, …, S_j, …, S_n be sets.

Let ∏_i  = 1^n S_i be the Cartesian product of S_1, S_2, …, S_n.

For each j ∈1, 2, …, n, the jth projection on S = ∏_i  = 1^n S_i is the mapping _j: S → S_j defined by:
:_js_1, s_2, …, s_j, …, s_n = s_j

for all s_1, s_2, …, s_n∈ S.


=== Family of Sets ===
"
Definition:Projection,Projection,"Let $\family {S_i}_{i \mathop \in I}$ be an indexed family of sets.

Let $\struct {P, \family {\phi_i}_{i \mathop \in I} }$ be a set product of $\family {S_i}_{i \mathop \in I}$.


The mappings $\phi_i$ are the projections of $P$.",Definition:Set Product/Projection,['Definitions/Projections'],"Let S_i_i ∈ I be an indexed family of sets.

Let P, ϕ_i_i ∈ I be a set product of S_i_i ∈ I.


The mappings ϕ_i are the projections of P."
Definition:Projection,Projection,"Let $M$ and $N$ be distinct lines in the plane.

:

The projection on $M$ along $N$ is the mapping $\pr_{M, N}$ such that:
:$\forall x \in \R^2: \map {\pr_{M, N} } x =$ the intersection of $M$ with the line through $x$ parallel to $N$.
",Definition:Projection (Geometry),"['Definitions/Geometric Projections', 'Definitions/Geometry']","Let M and N be distinct lines in the plane.

:

The projection on M along N is the mapping _M, N such that:
:∀ x ∈^2: _M, N x = the intersection of M with the line through x parallel to N.
"
Definition:Projection,Projection,"Let $H$ be a Hilbert space.

Let $P \in \map B H$ be an idempotent operator.


Then $P$ is said to be a projection :

:$\ker P = \paren {\Img P}^\perp$

where:
:$\ker P$ denotes the kernel of $P$
:$\Img P$ denotes the image of $P$
:$\perp$ denotes orthocomplementation.",Definition:Projection (Hilbert Spaces),['Definitions/Linear Transformations on Hilbert Spaces'],"Let H be a Hilbert space.

Let P ∈ B H be an idempotent operator.


Then P is said to be a projection :

:P =  P^⊥

where:
:P denotes the kernel of P
:P denotes the image of P
:⊥ denotes orthocomplementation."
Definition:Projection,Projection,"Let $H$ be a Hilbert space.

Let $K$ be a closed linear subspace of $H$.


Then the orthogonal projection on $K$ is the mapping $P_K: H \to H$ defined by

:$k = \map {P_K} h \iff k \in K$ and $\map d {h, k} = \map d {h, K}$

where the latter $d$ signifies distance to a set.


That $P_K$ is indeed a mapping is proved on Orthogonal Projection is Mapping.




The name orthogonal projection stems from the fact that $\paren {h - \map {P_K} h} \perp K$.",Definition:Orthogonal Projection,"['Definitions/Hilbert Spaces', 'Definitions/Linear Transformations on Hilbert Spaces']","Let H be a Hilbert space.

Let K be a closed linear subspace of H.


Then the orthogonal projection on K is the mapping P_K: H → H defined by

:k = P_K h  k ∈ K and d h, k =  d h, K

where the latter d signifies distance to a set.


That P_K is indeed a mapping is proved on Orthogonal Projection is Mapping.




The name orthogonal projection stems from the fact that h - P_K h⊥ K."
Definition:Proper,Proper,"Let $S$ and $T$ be sets such that $S$ is a subset of $T$.

Let $S \ne T$.

Then $S$ is referred to as a proper subset of $T$, and we write:
:$S \subsetneq T$
or:
:$S \subsetneqq T$


=== Proper Superset ===

",Definition:Subfield/Proper Subfield,['Definitions/Subfields'],"Let S and T be sets such that S is a subset of T.

Let S  T.

Then S is referred to as a proper subset of T, and we write:
:S ⊊ T
or:
:S ⫋ T


=== Proper Superset ===

"
Definition:Proper,Proper,"Let $\struct {G, \circ}$ be a group.

Let $\struct {H, \circ}$ be a subgroup of $\struct {G, \circ}$ such that $\set e \subset H \subset G$, that is:
:$H \ne \set e$
:$H \ne G$

Then $\struct {H, \circ}$ is a non-trivial proper subgroup of $\struct {G, \circ}$.
",Definition:Proper Subgroup,['Definitions/Subgroups'],"Let G, ∘ be a group.

Let H, ∘ be a subgroup of G, ∘ such that e ⊂ H ⊂ G, that is:
:H  e
:H  G

Then H, ∘ is a non-trivial proper subgroup of G, ∘.
"
Definition:Proper,Proper,"Let $\struct {R, +, \circ}$ be a ring.


A subring $S$ of $R$ is a proper subring of $R$  $S$ is neither the null ring nor $R$ itself.",Definition:Proper Subring,['Definitions/Ring Theory'],"Let R, +, ∘ be a ring.


A subring S of R is a proper subring of R  S is neither the null ring nor R itself."
Definition:Proper,Proper,"Let $K$ be a division ring.

Let $\struct {S, +, \circ}_K$ be a $K$-algebraic structure with one operation.

Let $\struct {T, +_T, \circ_T}_K$ be a vector subspace of $\struct {S, +, \circ}_K$.


If $T$ is a proper subset of $S$, then $\struct {T, +_T, \circ_T}_K$ is a proper (vector) subspace of $\struct {S, +, \circ}_K$.
",Definition:Vector Subspace,"['Definitions/Linear Algebra', 'Definitions/Vector Algebra']","Let K be a division ring.

Let S, +, ∘_K be a K-algebraic structure with one operation.

Let T, +_T, ∘_T_K be a vector subspace of S, +, ∘_K.


If T is a proper subset of S, then T, +_T, ∘_T_K is a proper (vector) subspace of S, +, ∘_K.
"
Definition:Proper,Proper,"Let $G = \struct {V, E}$ be a simple graph.


A proper (vertex) $k$-coloring of $G$ is defined as a vertex coloring from a set of $k$ colors such that no two adjacent vertices share a common color.

That is, a proper $k$-coloring of $G$ is a mapping $c: V \to \set {1, 2, \ldots k}$ such that:
:$\forall e = \set {u, v} \in E: \map c u \ne \map c v$
Let $G = \struct {V, E}$ be a simple graph.


A proper (edge) $k$-coloring of $G$ is defined as an edge coloring from a set of $k$ colors such that no two adjacent edges share a common color.

That is, a proper $k$-coloring of $G$ is a mapping $c: E \to \set {1, 2, \ldots k}$ such that:
:$\forall v \in V: \forall e = \set {u_k, v} \in E: \map c {\set {u_i, v} } \ne \map c {\set {u_j, v} }$
",Definition:Proper Coloring,['Definitions/Graph Colorings'],"Let G = V, E be a simple graph.


A proper (vertex) k-coloring of G is defined as a vertex coloring from a set of k colors such that no two adjacent vertices share a common color.

That is, a proper k-coloring of G is a mapping c: V →1, 2, … k such that:
:∀ e = u, v∈ E:  c u  c v
Let G = V, E be a simple graph.


A proper (edge) k-coloring of G is defined as an edge coloring from a set of k colors such that no two adjacent edges share a common color.

That is, a proper k-coloring of G is a mapping c: E →1, 2, … k such that:
:∀ v ∈ V: ∀ e = u_k, v∈ E:  c u_i, v c u_j, v
"
Definition:Proper,Proper,"Let $G = \struct {V, E}$ be a simple graph.


A proper (vertex) $k$-coloring of $G$ is defined as a vertex coloring from a set of $k$ colors such that no two adjacent vertices share a common color.

That is, a proper $k$-coloring of $G$ is a mapping $c: V \to \set {1, 2, \ldots k}$ such that:
:$\forall e = \set {u, v} \in E: \map c u \ne \map c v$",Definition:Proper Coloring/Vertex Coloring,"['Definitions/Graph Colorings', 'Definitions/Vertices of Graphs']","Let G = V, E be a simple graph.


A proper (vertex) k-coloring of G is defined as a vertex coloring from a set of k colors such that no two adjacent vertices share a common color.

That is, a proper k-coloring of G is a mapping c: V →1, 2, … k such that:
:∀ e = u, v∈ E:  c u  c v"
Definition:Proper,Proper,"Let $T$ be a rooted tree with root $r_T$.

Let $t$ be a node of $T$.


A proper ancestor node of $t$ is an ancestor node of $t$ that is not $t$ itself.",Definition:Rooted Tree/Ancestor Node/Proper,['Definitions/Ancestor Nodes'],"Let T be a rooted tree with root r_T.

Let t be a node of T.


A proper ancestor node of t is an ancestor node of t that is not t itself."
Definition:Proper,Proper,"A proper refinement of a normal series is a refinement which is not equal to the original normal series.


That is, it contains extra (normal) subgroups which are not present in the original normal series.",Definition:Refinement of Normal Series/Proper Refinement,['Definitions/Normal Series'],"A proper refinement of a normal series is a refinement which is not equal to the original normal series.


That is, it contains extra (normal) subgroups which are not present in the original normal series."
Definition:Proper,Proper,"Let $\mathbf Q$ be an orthogonal matrix.


Then $\mathbf Q$ is a proper orthogonal matrix :
:$\map \det {\mathbf Q} = 1$
where $\map \det {\mathbf Q}$ is the determinant of $\mathbf Q$.",Definition:Proper Orthogonal Matrix,['Definitions/Matrix Algebra'],"Let 𝐐 be an orthogonal matrix.


Then 𝐐 is a proper orthogonal matrix :
:𝐐 = 1
where 𝐐 is the determinant of 𝐐."
Definition:Proper,Proper,"Let $\FF$ be a formal language with alphabet $\AA$.

Let $\mathbf A$ be a well-formed formula of $\FF$.

Let $\mathbf B$ be a well-formed part of $\mathbf A$.


Then $\mathbf B$ is a proper well-formed part of $\mathbf A$  $\mathbf B$ is not equal to $\mathbf A$.",Definition:Well-Formed Part/Proper Well-Formed Part,['Definitions/Formal Languages'],"Let  be a formal language with alphabet Å.

Let 𝐀 be a well-formed formula of .

Let 𝐁 be a well-formed part of 𝐀.


Then 𝐁 is a proper well-formed part of 𝐀  𝐁 is not equal to 𝐀."
Definition:Proper,Proper,"Let $\struct {\Z, +, \times}$ be the ring of integers.

Let $x, y \in \Z$.


Then $x$ divides $y$ is defined as:
:$x \divides y \iff \exists t \in \Z: y = t \times x$


Then $x$ is a proper divisor of $y$ :

:$(1): \quad x \divides y$
:$(2): \quad \size x \ne \size y$
:$(3): \quad x \ne \pm 1$

That is:
:$(1): \quad x$ is a divisor of $y$
:$(2): \quad x$ and $y$ are not equal in absolute value
:$(3): \quad x$ is not equal to either $1$ or $-1$.",Definition:Proper Divisor/Integer,"['Definitions/Proper Divisors', 'Definitions/Divisors']","Let , +, × be the ring of integers.

Let x, y ∈.


Then x divides y is defined as:
:x  y ∃ t ∈: y = t × x


Then x is a proper divisor of y :

:(1):    x  y
:(2):    x  y
:(3):    x ± 1

That is:
:(1):    x is a divisor of y
:(2):    x and y are not equal in absolute value
:(3):    x is not equal to either 1 or -1."
Definition:Proper,Proper,A non-zero element of a ring which does not have a product inverse is called a proper element.,Definition:Proper Element of Ring,['Definitions/Ring Theory'],A non-zero element of a ring which does not have a product inverse is called a proper element.
Definition:Proper,Proper,"Let $\struct {R, +, \circ}$ be a ring.


A proper zero divisor of $R$ is an element $x \in R^*$ such that:

:$\exists y \in R^*: x \circ y = 0_R$

where $R^*$ is defined as $R \setminus \set {0_R}$.


That is, it is a zero divisor of $R$ which is specifically not $0_R$.


The presence of a proper zero divisor in a ring means that the product of two elements of the ring may be zero even if neither factor is zero.

That is, if $R$ has proper zero divisors, then $\struct {R^*, \circ}$ is not closed.
",Definition:Proper Zero Divisor,['Definitions/Zero Divisors'],"Let R, +, ∘ be a ring.


A proper zero divisor of R is an element x ∈ R^* such that:

:∃ y ∈ R^*: x ∘ y = 0_R

where R^* is defined as R ∖0_R.


That is, it is a zero divisor of R which is specifically not 0_R.


The presence of a proper zero divisor in a ring means that the product of two elements of the ring may be zero even if neither factor is zero.

That is, if R has proper zero divisors, then R^*, ∘ is not closed.
"
Definition:Proper,Proper,"Let $\struct {D, +, \circ}$ be an integral domain whose zero is $0_D$ and whose unity is $1_D$.

Let $U$ be the group of units of $D$.

Let $x, y \in D$.


Then $x$ is a proper divisor of $y$ :

:$(1): \quad x \divides y$
:$(2): \quad y \nmid x$
:$(3): \quad x \notin U$

That is:
:$(1): \quad x$ is a divisor of $y$
:$(2): \quad x$ is not an associate of $y$
:$(3): \quad x$ is not a unit of $D$


=== Integers ===

As the set of integers form an integral domain, the concept of a proper divisor is fully applicable to the integers.

",Definition:Proper Divisor,"['Definitions/Proper Divisors', 'Definitions/Ring Theory']","Let D, +, ∘ be an integral domain whose zero is 0_D and whose unity is 1_D.

Let U be the group of units of D.

Let x, y ∈ D.


Then x is a proper divisor of y :

:(1):    x  y
:(2):    y ∤ x
:(3):    x ∉ U

That is:
:(1):    x is a divisor of y
:(2):    x is not an associate of y
:(3):    x is not a unit of D


=== Integers ===

As the set of integers form an integral domain, the concept of a proper divisor is fully applicable to the integers.

"
Definition:Proper,Proper,"If $S$ is a proper subset of $T$, then $T$ is a proper superset of $S$.

This can be expressed by the notation $T \supsetneqq S$.
",Definition:Proper Subset,['Definitions/Subsets'],"If S is a proper subset of T, then T is a proper superset of S.

This can be expressed by the notation T ⫌ S.
"
Definition:Proper,Proper,"A proper class is a class which is not a set.

That is, $A$ is a proper class :
:$\neg \exists x: x = A$
where $x$ is a set.",Definition:Class (Class Theory)/Proper Class,['Definitions/Class Theory'],"A proper class is a class which is not a set.

That is, A is a proper class :
:∃ x: x = A
where x is a set."
Definition:Proper,Proper,"Let $A$ be a class under a total ordering $\preccurlyeq$.

Let $L$ be a lower section of $A$ such that:
:$L \ne \O$
:$L \ne A$

Then $L$ is known as a proper lower section of $A$ (by $L$).
",Definition:Proper Well-Ordering,"['Definitions/Well-Orderings', 'Definitions/Class Theory']","Let A be a class under a total ordering ≼.

Let L be a lower section of A such that:
:L Ø
:L  A

Then L is known as a proper lower section of A (by L).
"
Definition:Proper,Proper,"Let $X$ and $Y$ be topological spaces.


A mapping $f: X \to Y$ is proper  for every compact subspace $K \subset Y$, its preimage $f^{-1} \sqbrk K$ is also compact.",Definition:Proper Mapping,"['Definitions/Proper Mappings', 'Definitions/Topology', 'Definitions/Compact Spaces']","Let X and Y be topological spaces.


A mapping f: X → Y is proper  for every compact subspace K ⊂ Y, its preimage f^-1 K is also compact."
Definition:Proper,Proper,"Let $X$ and $Y$ be topological spaces.


A mapping $f: X \to Y$ is proper  for every compact subspace $K \subset Y$, its preimage $f^{-1} \sqbrk K$ is also compact.
",Definition:Proper Group Action,"['Definitions/Topology', 'Definitions/Group Theory', 'Definitions/Group Actions', 'Definitions/Topological Groups']","Let X and Y be topological spaces.


A mapping f: X → Y is proper  for every compact subspace K ⊂ Y, its preimage f^-1 K is also compact.
"
Definition:Proper,Proper,"A proper name (or just name) is a symbol or collection of symbols used to identify a particular object uniquely.


In contrast with natural language, a proper name has a wider range than being the particular identifying label attached to a particular single entity (be it a person, or a place, or whatever else).

For example:
:Sloth is a proper name for the concept of being lazy.
:Rain is a proper name for the meteorological phenomenon of water falling from the sky.
A proper name (or just name) is a symbol or collection of symbols used to identify a particular object uniquely.


In contrast with natural language, a proper name has a wider range than being the particular identifying label attached to a particular single entity (be it a person, or a place, or whatever else).

For example:
:Sloth is a proper name for the concept of being lazy.
:Rain is a proper name for the meteorological phenomenon of water falling from the sky.
A proper name (or just name) is a symbol or collection of symbols used to identify a particular object uniquely.


In contrast with natural language, a proper name has a wider range than being the particular identifying label attached to a particular single entity (be it a person, or a place, or whatever else).

For example:
:Sloth is a proper name for the concept of being lazy.
:Rain is a proper name for the meteorological phenomenon of water falling from the sky.
",Definition:Proper Name,"['Definitions/Proper Names', 'Definitions/Predicate Logic']","A proper name (or just name) is a symbol or collection of symbols used to identify a particular object uniquely.


In contrast with natural language, a proper name has a wider range than being the particular identifying label attached to a particular single entity (be it a person, or a place, or whatever else).

For example:
:Sloth is a proper name for the concept of being lazy.
:Rain is a proper name for the meteorological phenomenon of water falling from the sky.
A proper name (or just name) is a symbol or collection of symbols used to identify a particular object uniquely.


In contrast with natural language, a proper name has a wider range than being the particular identifying label attached to a particular single entity (be it a person, or a place, or whatever else).

For example:
:Sloth is a proper name for the concept of being lazy.
:Rain is a proper name for the meteorological phenomenon of water falling from the sky.
A proper name (or just name) is a symbol or collection of symbols used to identify a particular object uniquely.


In contrast with natural language, a proper name has a wider range than being the particular identifying label attached to a particular single entity (be it a person, or a place, or whatever else).

For example:
:Sloth is a proper name for the concept of being lazy.
:Rain is a proper name for the meteorological phenomenon of water falling from the sky.
"
Definition:Proper,Proper,,Definition:Improper,[],
Definition:Pullback,Pullback,"Let $\mathbf C$ be a metacategory.

Let $f: A \to C$ and $g: B \to C$ be morphisms with common codomain.


A pullback of $f$ and $g$ is a commutative diagram:

::$\begin{xy}\xymatrix{
 P
  \ar[r]^*+{p_1}
  \ar[d]_*+{p_2}
&
 A
  \ar[d]^*+{f}

\\
 B
  \ar[r]_*+{g}
&
 C
}\end{xy}$

such that $f \circ p_1 = g \circ p_2$, subject to the following UMP:


:For any commutative diagram:

:::$\begin{xy}\xymatrix{
 Q
  \ar[r]^*+{q_1}
  \ar[d]_*+{q_2}
&
 A
  \ar[d]^*+{f}

\\
 B
  \ar[r]_*+{g}
&
 C
}\end{xy}$

:there is a unique morphism $u: Q \to P$ making the following diagram commute:

:::$\begin{xy}\xymatrix@+1em{
 Q
  \ar@/^/[rrd]^*+{q_1}
  \ar@/_/[ddr]_*+{q_2}
  \ar@{-->}[rd]^*+{u}

\\
&
 P
  \ar[r]_*+{p_1}
  \ar[d]^*+{p_2}
&
 A
  \ar[d]^*+{f}

\\
&
 B
  \ar[r]_*+{g}
&
 C
}\end{xy}$


In this situation, $p_1$ is called the pullback of $f$ along $g$ and may be denoted as $g^* f$.

Similarly, $p_2$ is called the pullback of $g$ along $f$ and may be denoted $f^* g$.",Definition:Pullback (Category Theory),"['Definitions/Category Theory', 'Definitions/Pullbacks']","Let 𝐂 be a metacategory.

Let f: A → C and g: B → C be morphisms with common codomain.


A pullback of f and g is a commutative diagram:

::P
  [r]^*+p_1[d]_*+p_2   
 A
  [d]^*+f

 B
  [r]_*+g   
 C

such that f ∘ p_1 = g ∘ p_2, subject to the following UMP:


:For any commutative diagram:

:::Q
  [r]^*+q_1[d]_*+q_2   
 A
  [d]^*+f

 B
  [r]_*+g   
 C

:there is a unique morphism u: Q → P making the following diagram commute:

:::@+1em
 Q
  @/^/[rrd]^*+q_1@/_/[ddr]_*+q_2@–>[rd]^*+u
   
 P
  [r]_*+p_1[d]^*+p_2   
 A
  [d]^*+f
   
 B
  [r]_*+g   
 C


In this situation, p_1 is called the pullback of f along g and may be denoted as g^* f.

Similarly, p_2 is called the pullback of g along f and may be denoted f^* g."
Definition:Pullback,Pullback,"Let $\mathbf C$ be a metacategory.

Let $f: A \to C$ and $g: B \to C$ be morphisms with common codomain.


A pullback of $f$ and $g$ is a commutative diagram:

::$\begin{xy}\xymatrix{
 P
  \ar[r]^*+{p_1}
  \ar[d]_*+{p_2}
&
 A
  \ar[d]^*+{f}

\\
 B
  \ar[r]_*+{g}
&
 C
}\end{xy}$

such that $f \circ p_1 = g \circ p_2$, subject to the following UMP:


:For any commutative diagram:

:::$\begin{xy}\xymatrix{
 Q
  \ar[r]^*+{q_1}
  \ar[d]_*+{q_2}
&
 A
  \ar[d]^*+{f}

\\
 B
  \ar[r]_*+{g}
&
 C
}\end{xy}$

:there is a unique morphism $u: Q \to P$ making the following diagram commute:

:::$\begin{xy}\xymatrix@+1em{
 Q
  \ar@/^/[rrd]^*+{q_1}
  \ar@/_/[ddr]_*+{q_2}
  \ar@{-->}[rd]^*+{u}

\\
&
 P
  \ar[r]_*+{p_1}
  \ar[d]^*+{p_2}
&
 A
  \ar[d]^*+{f}

\\
&
 B
  \ar[r]_*+{g}
&
 C
}\end{xy}$


In this situation, $p_1$ is called the pullback of $f$ along $g$ and may be denoted as $g^* f$.

Similarly, $p_2$ is called the pullback of $g$ along $f$ and may be denoted $f^* g$.
Let $\mathbf C$ be a metacategory.

Let $f: A \to C$ and $g: B \to C$ be morphisms with common codomain.


A pullback of $f$ and $g$ is a commutative diagram:

::$\begin{xy}\xymatrix{
 P
  \ar[r]^*+{p_1}
  \ar[d]_*+{p_2}
&
 A
  \ar[d]^*+{f}

\\
 B
  \ar[r]_*+{g}
&
 C
}\end{xy}$

such that $f \circ p_1 = g \circ p_2$, subject to the following UMP:


:For any commutative diagram:

:::$\begin{xy}\xymatrix{
 Q
  \ar[r]^*+{q_1}
  \ar[d]_*+{q_2}
&
 A
  \ar[d]^*+{f}

\\
 B
  \ar[r]_*+{g}
&
 C
}\end{xy}$

:there is a unique morphism $u: Q \to P$ making the following diagram commute:

:::$\begin{xy}\xymatrix@+1em{
 Q
  \ar@/^/[rrd]^*+{q_1}
  \ar@/_/[ddr]_*+{q_2}
  \ar@{-->}[rd]^*+{u}

\\
&
 P
  \ar[r]_*+{p_1}
  \ar[d]^*+{p_2}
&
 A
  \ar[d]^*+{f}

\\
&
 B
  \ar[r]_*+{g}
&
 C
}\end{xy}$


In this situation, $p_1$ is called the pullback of $f$ along $g$ and may be denoted as $g^* f$.

Similarly, $p_2$ is called the pullback of $g$ along $f$ and may be denoted $f^* g$.
",Definition:Pullback Functor,"['Definitions/Pullbacks', 'Definitions/Slice Categories']","Let 𝐂 be a metacategory.

Let f: A → C and g: B → C be morphisms with common codomain.


A pullback of f and g is a commutative diagram:

::P
  [r]^*+p_1[d]_*+p_2   
 A
  [d]^*+f

 B
  [r]_*+g   
 C

such that f ∘ p_1 = g ∘ p_2, subject to the following UMP:


:For any commutative diagram:

:::Q
  [r]^*+q_1[d]_*+q_2   
 A
  [d]^*+f

 B
  [r]_*+g   
 C

:there is a unique morphism u: Q → P making the following diagram commute:

:::@+1em
 Q
  @/^/[rrd]^*+q_1@/_/[ddr]_*+q_2@–>[rd]^*+u
   
 P
  [r]_*+p_1[d]^*+p_2   
 A
  [d]^*+f
   
 B
  [r]_*+g   
 C


In this situation, p_1 is called the pullback of f along g and may be denoted as g^* f.

Similarly, p_2 is called the pullback of g along f and may be denoted f^* g.
Let 𝐂 be a metacategory.

Let f: A → C and g: B → C be morphisms with common codomain.


A pullback of f and g is a commutative diagram:

::P
  [r]^*+p_1[d]_*+p_2   
 A
  [d]^*+f

 B
  [r]_*+g   
 C

such that f ∘ p_1 = g ∘ p_2, subject to the following UMP:


:For any commutative diagram:

:::Q
  [r]^*+q_1[d]_*+q_2   
 A
  [d]^*+f

 B
  [r]_*+g   
 C

:there is a unique morphism u: Q → P making the following diagram commute:

:::@+1em
 Q
  @/^/[rrd]^*+q_1@/_/[ddr]_*+q_2@–>[rd]^*+u
   
 P
  [r]_*+p_1[d]^*+p_2   
 A
  [d]^*+f
   
 B
  [r]_*+g   
 C


In this situation, p_1 is called the pullback of f along g and may be denoted as g^* f.

Similarly, p_2 is called the pullback of g along f and may be denoted f^* g.
"
Definition:Pullback,Pullback,"Let $G, H$ be groups.

Let $N \lhd G, K \lhd H$ be normal subgroups of $G$ and $H$ respectively.

Let:
:$G / N \cong H / K$
where:
:$G / N$ denotes the quotient of $G$ by $N$
:$\cong$ denotes group isomorphism.

Let $\theta: G / N \to H / K$ be such a group isomorphism.


The pullback $G \times^\theta H$ of $G$ and $H$ via $\theta$ is the subset of $G \times H$ of elements of the form $\tuple {g, h}$ where $\map \theta {g N} = h K$.",Definition:Pullback of Quotient Group Isomorphism,"['Definitions/Group Theory', 'Definitions/Normality in Groups']","Let G, H be groups.

Let N  G, K  H be normal subgroups of G and H respectively.

Let:
:G / N ≅ H / K
where:
:G / N denotes the quotient of G by N
:≅ denotes group isomorphism.

Let θ: G / N → H / K be such a group isomorphism.


The pullback G ×^θ H of G and H via θ is the subset of G × H of elements of the form g, h where θg N = h K."
Definition:Quadratic Form,Quadratic Form,"Let $\mathbb K$ be a field of characteristic $\Char {\mathbb K} \ne 2$.

Let $V$ be a vector space over $\mathbb K$.


A quadratic form on $V$ is a mapping $q : V \mapsto \mathbb K$ such that:
:$\forall v \in V : \forall \kappa \in \mathbb K : \map q {\kappa v} = \kappa^2 \map q v$
:$b: V \times V \to \mathbb K: \tuple {v, w} \mapsto \map q {v + w} - \map q v - \map q w$ is a bilinear form


",Definition:Quadratic Form (Linear Algebra),"['Definitions/Quadratic Forms (Linear Algebra)', 'Definitions/Bilinear Forms (Linear Algebra)', 'Definitions/Linear Algebra', 'Definitions/Module Theory', 'Definitions/Vector Spaces', 'Definitions/Quadratic Forms']","Let 𝕂 be a field of characteristic 𝕂 2.

Let V be a vector space over 𝕂.


A quadratic form on V is a mapping q : V ↦𝕂 such that:
:∀ v ∈ V : ∀κ∈𝕂 :  q κ v = κ^2  q v
:b: V × V →𝕂: v, w↦ q v + w -  q v -  q w is a bilinear form


"
Definition:Quadratic Form,Quadratic Form,"A quadratic form is a form whose variables are of degree $2$.
",Definition:Quadratic Form (Polynomial Theory),"['Definitions/Quadratic Forms (Polynomial Theory)', 'Definitions/Forms', 'Definitions/Polynomial Theory', 'Definitions/Quadratic Forms']","A quadratic form is a form whose variables are of degree 2.
"
Definition:Quasi-Compact,Quasi-Compact,"Let $\struct {X, \OO_X}$ be a scheme.


Then $\struct {X, \OO_X}$ is quasi-compact  $X$ is compact.",Definition:Quasi-Compact Scheme,"['Definitions/Algebraic Geometry', 'Definitions/Schemes']","Let X, _X be a scheme.


Then X, _X is quasi-compact  X is compact."
Definition:Quasi-Compact,Quasi-Compact,,Definition:Quasi-Compact Morphism of Schemes,"['Definitions/Algebraic Geometry', 'Definitions/Schemes']",
Definition:Quotient Space,Quotient Space,"Let $T = \struct {S, \tau}$ be a topological space.

Let $\RR \subseteq S \times S$ be an equivalence relation on $S$.

Let $q_\RR: S \to S / \RR$ be the quotient mapping induced by $\RR$.


Let $\tau_\RR$ be the quotient topology on $S / \RR$ by $q_\RR$:
:$\tau_\RR := \set {U \subseteq S / \RR: \map {q_\RR^{-1} } U \in \tau}$


The quotient space of $S$ by $\RR$ is the topological space whose points are elements of the quotient set of $\RR$ and whose topology is $\tau_\RR$:
:$T_\RR := \struct {S / \RR, \tau_\RR}$",Definition:Quotient Topology/Quotient Space,['Definitions/Quotient Topology'],"Let T = S, τ be a topological space.

Let ⊆ S × S be an equivalence relation on S.

Let q_: S → S / be the quotient mapping induced by .


Let τ_ be the quotient topology on S / by q_:
:τ_ := U ⊆ S / : q_^-1 U ∈τ


The quotient space of S by  is the topological space whose points are elements of the quotient set of  and whose topology is τ_:
:T_ := S / , τ_"
Definition:Quotient Space,Quotient Space,,Definition:Quotient Space,[],
Definition:Radical,Radical,"Let $L / F$ be a field extension.

Then $L$ is a radical extension of $F$  there exist $\alpha_1, \ldots, \alpha_m \in F$ and $n_1, \ldots, n_2 \in \Z_{>0}$ such that:

:$(1): \quad L = K \sqbrk {\alpha_1, \ldots, \alpha_m}$

:$(2): \quad \alpha_1^{n_1} \in F$

:$(3): \quad \forall i \in \N_m: \alpha_i^{n_i} \in F \sqbrk {\alpha_1, \ldots, \alpha_{i-1} }$

where $K \sqbrk {\alpha_1, \ldots, \alpha_m}$ and $F \sqbrk {\alpha_1, \ldots, \alpha_{i-1} }$ are generated field extensions.

Category:Definitions/Field Extensions",Definition:Radical Extension,['Definitions/Field Extensions'],"Let L / F be a field extension.

Then L is a radical extension of F  there exist α_1, …, α_m ∈ F and n_1, …, n_2 ∈_>0 such that:

:(1):    L = K α_1, …, α_m

:(2):   α_1^n_1∈ F

:(3):   ∀ i ∈_m: α_i^n_i∈ F α_1, …, α_i-1

where K α_1, …, α_m and F α_1, …, α_i-1 are generated field extensions.

Category:Definitions/Field Extensions"
Definition:Radical,Radical,"Let $\mathbb K$ be a field.

Let $V$ be a vector space over $\mathbb K$.

Let $b : V\times V \to \mathbb K$ be a reflexive bilinear form on $V$.


The radical of $V$ is the orthogonal complement of $V$:
:$\map {\operatorname {rad} } V = V^\perp$",Definition:Orthogonal (Bilinear Form)/Radical,['Definitions/Bilinear Forms (Linear Algebra)'],"Let 𝕂 be a field.

Let V be a vector space over 𝕂.

Let b : V× V →𝕂 be a reflexive bilinear form on V.


The radical of V is the orthogonal complement of V:
:rad V = V^⊥"
Definition:Radical,Radical,"Let $R$ be a commutative ring with unity.

Let $\map {\operatorname{maxspec}} R$ be the set of maximal ideals of $R$.


Then the Jacobson radical of $R$ is:
:$\ds \map {\operatorname {Jac} } R = \bigcap_{m \mathop \in \map {\operatorname{maxspec}} R} m$

That is, it is the intersection of all maximal ideals of $R$.",Definition:Jacobson Radical,['Definitions/Ring Theory'],"Let R be a commutative ring with unity.

Let maxspec R be the set of maximal ideals of R.


Then the Jacobson radical of R is:
:Jac R = ⋂_m ∈maxspec R m

That is, it is the intersection of all maximal ideals of R."
Definition:Radius,Radius,":

A radius of a circle is a straight line segment whose endpoints are the center and the circumference of the circle.

In the above diagram, the line $AB$ is a radius.",Definition:Circle/Radius,['Definitions/Circles'],":

A radius of a circle is a straight line segment whose endpoints are the center and the circumference of the circle.

In the above diagram, the line AB is a radius."
Definition:Radius,Radius,"A radius of a sphere is a straight line segment whose endpoints are the center and the surface of the sphere.

The radius of a sphere is the length of one such radius.",Definition:Sphere/Geometry/Radius,['Definitions/Spheres'],"A radius of a sphere is a straight line segment whose endpoints are the center and the surface of the sphere.

The radius of a sphere is the length of one such radius."
Definition:Radius,Radius,"Let $P$ be a point in a given frame of reference whose origin is $O$.

The position vector $\mathbf p$ of $P$ is the displacement vector of $P$ from $O$.


=== Notation ===
",Definition:Position Vector,"['Definitions/Position Vectors', 'Definitions/Displacement', 'Definitions/Vectors']","Let P be a point in a given frame of reference whose origin is O.

The position vector 𝐩 of P is the displacement vector of P from O.


=== Notation ===
"
Definition:Radius,Radius,"Let $M = \struct {A, d}$ be a metric space or pseudometric space.

Let $a \in A$.

Let $\map {B_\epsilon} a$ be the open $\epsilon$-ball of $a$.


In $\map {B_\epsilon} a$, the value $\epsilon$ is referred to as the radius of the open $\epsilon$-ball.",Definition:Open Ball/Radius,['Definitions/Open Balls'],"Let M = A, d be a metric space or pseudometric space.

Let a ∈ A.

Let B_ϵ a be the open ϵ-ball of a.


In B_ϵ a, the value ϵ is referred to as the radius of the open ϵ-ball."
Definition:Range,Range,"Let $\RR \subseteq S \times T$ be a relation, or (usually) a mapping (which is, of course, itself a relation).


The range of $\RR$ can be defined as $T$.

As such, it is the same thing as the term codomain of $\RR$.
Let $\RR \subseteq S \times T$ be a relation, or (usually) a mapping (which is, of course, itself a relation).


The range of $\RR$ can be defined as:
:$\Rng \RR = \set {t \in T: \exists s \in S: \tuple {s, t} \in \RR}$

Defined like this, it is the same as what is defined as the image of $\RR$.
",Definition:Range of Relation,"['Definitions/Relation Theory', 'Definitions/Mapping Theory', 'Definitions/Ranges (Relation Theory)']","Let ⊆ S × T be a relation, or (usually) a mapping (which is, of course, itself a relation).


The range of  can be defined as T.

As such, it is the same thing as the term codomain of .
Let ⊆ S × T be a relation, or (usually) a mapping (which is, of course, itself a relation).


The range of  can be defined as:
:= t ∈ T: ∃ s ∈ S: s, t∈

Defined like this, it is the same as what is defined as the image of .
"
Definition:Range,Range,"Let $S$ be a set of observations of a quantitative variable.


The range of $S$ is defined as:
:$\map R S := \map \max S - \map \min S$

where $\map \max S$ and $\map \min S$ are the greatest value of $S$ and the least value of $S$ respectively.",Definition:Range (Statistics),['Definitions/Descriptive Statistics'],"Let S be a set of observations of a quantitative variable.


The range of S is defined as:
:R S := max S - min S

where max S and min S are the greatest value of S and the least value of S respectively."
Definition:Rank,Rank,"Let $A$ be a set.

Let $V$ denote the von Neumann hierarchy.


Then the rank of $A$ is the smallest ordinal $x$ such that $A \in \map V {x + 1}$, given that $x$ exists.",Definition:Rank (Set Theory),['Definitions/Axiom of Foundation'],"Let A be a set.

Let V denote the von Neumann hierarchy.


Then the rank of A is the smallest ordinal x such that A ∈ V x + 1, given that x exists."
Definition:Rank,Rank,"Let $H, K$ be Hilbert spaces.

Let $T: H \to K$ be a linear transformation.


Then $T$ is said to be a finite rank operator, or of finite rank,  its range, $\Rng T$, is finite dimensional.

Note that a finite rank operator is not necessarily bounded.
",Definition:Rank (Linear Algebra),['Definitions/Linear Algebra'],"Let H, K be Hilbert spaces.

Let T: H → K be a linear transformation.


Then T is said to be a finite rank operator, or of finite rank,  its range, T, is finite dimensional.

Note that a finite rank operator is not necessarily bounded.
"
Definition:Rank,Rank,"Let $\phi$ be a linear transformation from one vector space to another.

Let the image of $\phi$ be finite-dimensional.


Then its dimension is called the rank of $\phi$ and is denoted $\map \rho \phi$.",Definition:Rank/Linear Transformation,['Definitions/Linear Algebra'],"Let ϕ be a linear transformation from one vector space to another.

Let the image of ϕ be finite-dimensional.


Then its dimension is called the rank of ϕ and is denoted ρϕ."
Definition:Rank,Rank,"Let $f: \C \to \C$ be an entire function.

Let $\sequence {a_n}$ be the sequence of non-zero zeroes of $f$, repeated according to multiplicity.


The rank of $f$ is:
:the smallest positive integer $p \ge 0$ for which the series $\ds \sum_{n \mathop = 1}^\infty \size {a_n}^{-p - 1}$ converges
or:
:$\infty$ if there is no such integer.


If $f$ has finitely many zeroes, its rank is $0$.",Definition:Rank of Entire Function,['Definitions/Entire Functions'],"Let f: → be an entire function.

Let a_n be the sequence of non-zero zeroes of f, repeated according to multiplicity.


The rank of f is:
:the smallest positive integer p ≥ 0 for which the series ∑_n  = 1^∞a_n^-p - 1 converges
or:
:∞ if there is no such integer.


If f has finitely many zeroes, its rank is 0."
Definition:Rank,Rank,"Let $F$ be a free group.


The rank of $F$ is the dimension of its abelianization as a module over $\Z$.",Definition:Rank of Free Group,['Definitions/Group Theory'],"Let F be a free group.


The rank of F is the dimension of its abelianization as a module over ."
Definition:Rank,Rank,"Let $S$ be a set of discrete data which has been linearly ordered by a total ordering $\QQ$.

The ranking of $x \in S$ is the index of $x$ in the sequence induced on $S$ by $\QQ$.



Let $S$ be a set of sample data which has been assigned a ranking $\RR$.

The index of a given $x \in S$ according to $\RR$ is known as its rank.
",Definition:Rank (Statistics),['Definitions/Rankings'],"Let S be a set of discrete data which has been linearly ordered by a total ordering .

The ranking of x ∈ S is the index of x in the sequence induced on S by .



Let S be a set of sample data which has been assigned a ranking .

The index of a given x ∈ S according to  is known as its rank.
"
Definition:Rank Function,Rank Function,"Let $\struct {S, \RR}$ be a relational structure.

Let $\struct {T, \prec}$ be a strictly well-ordered set.

Let $\operatorname {rk}: S \to T$ be a mapping such that:
:$\forall x, y \in S: \paren {x \ne y \text { and } \tuple {x, y} \in \RR} \implies \map {\operatorname {rk} } x \prec \map {\operatorname {rk} } y$


$\operatorname {rk}$ is known as a rank function for $\RR$.",Definition:Rank Function for Relation,"['Definitions/Well-Founded Relations', 'Definitions/Relation Theory']","Let S, be a relational structure.

Let T, ≺ be a strictly well-ordered set.

Let rk: S → T be a mapping such that:
:∀ x, y ∈ S: x  y  and x, y∈rk x ≺rk y


rk is known as a rank function for ."
Definition:Rank Function,Rank Function,"Let $M = \struct {S, \mathscr I}$ be a matroid.


The rank function of $M$ is the mapping $\rho : \powerset S \to \Z$ from the power set of $S$ into the integers defined by:
:$\forall A \subseteq S : \map \rho A = \max \set {\size X : X \subseteq A \land X \in \mathscr I}$
where $\size A$ denotes the cardinality of $A$.
",Definition:Rank Function (Matroid),['Definitions/Matroid Theory'],"Let M = S, ℐ be a matroid.


The rank function of M is the mapping ρ :  S → from the power set of S into the integers defined by:
:∀ A ⊆ S : ρ A = max X : X ⊆ A  X ∈ℐ
where A denotes the cardinality of A.
"
Definition:Ray,Ray,"An infinite half-line is a line which terminates at an endpoint at one end, but has no such endpoint at the other.


=== Start Point ===
",Definition:Line/Infinite Half-Line,"['Definitions/Infinite Half-Lines', 'Definitions/Lines']","An infinite half-line is a line which terminates at an endpoint at one end, but has no such endpoint at the other.


=== Start Point ===
"
Definition:Ray,Ray,"Let $T = \struct {S, \tau}$ be a topological space.


A ray in $T$ is an embedding $R_{>0} \to S$.

Category:Definitions/Topology",Definition:Ray (Topology),['Definitions/Topology'],"Let T = S, τ be a topological space.


A ray in T is an embedding R_>0→ S.

Category:Definitions/Topology"
Definition:Ray,Ray,"Let $\struct {S, \preccurlyeq}$ be a totally ordered set.

Let $\prec$ be the reflexive reduction of $\preccurlyeq$.

Let $a \in S$ be any point in $S$.


The following sets are called open rays or open half-lines:

:$\set {x \in S: a \prec x}$ (the strict upper closure of $a$), denoted $a^\succ$
:$\set {x \in S: x \prec a}$ (the strict lower closure of $a$), denoted $a^\prec$.
Let $\struct {S, \preccurlyeq}$ be a totally ordered set.

Let $a \in S$ be any point in $S$.


The following sets are called closed rays or closed half-lines:

:$\set {x \in S: a \preccurlyeq x}$ (the upper closure of $a$), denoted $a^\succcurlyeq$
:$\set {x \in S: x \preccurlyeq a}$ (the lower closure of $a$), denoted $a^\preccurlyeq$.
Let $\struct {S, \preccurlyeq}$ be a totally ordered set.

Let $\prec$ be the reflexive reduction of $\preccurlyeq$.

Let $a \in S$ be any point in $S$.


An upward-pointing ray is a ray which is bounded below:

:an open ray $a^\succ:= \set {x \in S: a \prec x}$
:a closed ray $a^\succcurlyeq: \set {x \in S: a \preccurlyeq x}$
Let $\struct {S, \preccurlyeq}$ be a totally ordered set.

Let $\prec$ be the reflexive reduction of $\preccurlyeq$.

Let $a \in S$ be any point in $S$.


A downward-pointing ray is a ray which is bounded above:

:an open ray $a^\prec := \set {x \in S: x \prec a}$
:a closed ray $a^\preccurlyeq : \set {x \in S: x \preccurlyeq a}$
",Definition:Ray (Order Theory),"['Definitions/Order Theory', 'Definitions/Rays (Order Theory)']","Let S, ≼ be a totally ordered set.

Let ≺ be the reflexive reduction of ≼.

Let a ∈ S be any point in S.


The following sets are called open rays or open half-lines:

:x ∈ S: a ≺ x (the strict upper closure of a), denoted a^≻
:x ∈ S: x ≺ a (the strict lower closure of a), denoted a^≺.
Let S, ≼ be a totally ordered set.

Let a ∈ S be any point in S.


The following sets are called closed rays or closed half-lines:

:x ∈ S: a ≼ x (the upper closure of a), denoted a^≽
:x ∈ S: x ≼ a (the lower closure of a), denoted a^≼.
Let S, ≼ be a totally ordered set.

Let ≺ be the reflexive reduction of ≼.

Let a ∈ S be any point in S.


An upward-pointing ray is a ray which is bounded below:

:an open ray a^≻:= x ∈ S: a ≺ x
:a closed ray a^≽: x ∈ S: a ≼ x
Let S, ≼ be a totally ordered set.

Let ≺ be the reflexive reduction of ≼.

Let a ∈ S be any point in S.


A downward-pointing ray is a ray which is bounded above:

:an open ray a^≺ := x ∈ S: x ≺ a
:a closed ray a^≼ : x ∈ S: x ≼ a
"
Definition:Realization,Realization,"Let $S$ be a stochastic process.

Let $T$ be a time series of observations of $S$ which has been acquired as $S$ evolves, according to the underlying probability distribution of $S$.


Then $T$ is referred to as a realization of $S$.


Thus we can regard the observation $z_t$ at some timestamp $t$, for example $t = 25$, as the realization of a random variable with probability mass function $\map p {z_t}$.

Similarly the observations $z_{t_1}$ and $z_{t_2}$ at times $t_1$ and $t_2$ can be regarded as the realizations of two random variables with joint probability mass function $\map p {z_{t_1} }$ and $\map p {z_{t_2} }$.


Similarly, the observations making an equispaced time series can be described by an $N$-dimensional random variable $\tuple {z_1, z_2, \dotsc, z_N}$ with associated probability mass function $\map p {z_1, z_2, \dotsc, z_N}$.",Definition:Realization of Stochastic Process,['Definitions/Stochastic Processes'],"Let S be a stochastic process.

Let T be a time series of observations of S which has been acquired as S evolves, according to the underlying probability distribution of S.


Then T is referred to as a realization of S.


Thus we can regard the observation z_t at some timestamp t, for example t = 25, as the realization of a random variable with probability mass function p z_t.

Similarly the observations z_t_1 and z_t_2 at times t_1 and t_2 can be regarded as the realizations of two random variables with joint probability mass function p z_t_1 and p z_t_2.


Similarly, the observations making an equispaced time series can be described by an N-dimensional random variable z_1, z_2, …, z_N with associated probability mass function p z_1, z_2, …, z_N."
Definition:Realization,Realization,"Let $\MM$ be an $\LL$-structure.

Let $A$ be a subset of the universe of $\MM$.

Let $\LL_A$ be the language consisting of $\LL$ along with constant symbols for each element of $A$.

Viewing $\MM$ as an $\LL_A$-structure by interpreting each new constant as the element for which it is named, let $\map {\operatorname {Th}_A} \MM$ be the set of $\LL_A$-sentences satisfied by $\MM$.


An $n$-type over $A$ is a set $p$ of $\LL_A$-formulas in $n$ free variables such that $p \cup \map {\operatorname {Th}_A} \MM$ is satisfiable by some $\LL_A$-structure.




=== Complete Type ===

We say that an $n$-type $p$ is complete (over $A$) :
:for every $\LL_A$-formula $\phi$ in $n$ free variables, either $\phi \in p$ or $\phi \notin p$.


The set of complete $n$-types over $A$ is often denoted by $\map {S_n^\MM} A$.


Given an $n$-tuple $\bar b$ of elements from $\MM$, the type of $\bar b$ over $A$ is the complete $n$-type consisting of those $\LL_A$-formulas $\map \phi {x_1, \dotsc, x_n}$ such that $\MM \models \map \phi {\bar b}$.

It is often denoted by $\map {\operatorname {tp}^\MM} {\bar b / A}$.


=== Realization ===

Given an $\LL_A$-structure $\NN$, a type $p$ is realized by an element $\bar b$ of $\NN$ :
:$\forall \phi \in p: \NN \models \map \phi {\bar b}$.

Such an element $\bar b$ of $\NN$ is a realization of $p$.


=== Omission ===

We say that $\NN$ omits $p$  $p$ is not realized in $\NN$.

Then $p$ is an omission from $\NN$.",Definition:Type,['Definitions/Model Theory for Predicate Logic'],"Let  be an -structure.

Let A be a subset of the universe of .

Let _A be the language consisting of  along with constant symbols for each element of A.

Viewing  as an _A-structure by interpreting each new constant as the element for which it is named, let Th_A be the set of _A-sentences satisfied by .


An n-type over A is a set p of _A-formulas in n free variables such that p ∪Th_A is satisfiable by some _A-structure.




=== Complete Type ===

We say that an n-type p is complete (over A) :
:for every _A-formula ϕ in n free variables, either ϕ∈ p or ϕ∉ p.


The set of complete n-types over A is often denoted by S_n^ A.


Given an n-tuple b̅ of elements from , the type of b̅ over A is the complete n-type consisting of those _A-formulas ϕx_1, …, x_n such that ϕb̅.

It is often denoted by tp^b̅ / A.


=== Realization ===

Given an _A-structure , a type p is realized by an element b̅ of  :
:∀ϕ∈ p: ϕb̅.

Such an element b̅ of  is a realization of p.


=== Omission ===

We say that  omits p  p is not realized in .

Then p is an omission from ."
Definition:Reducible,Reducible,"Let $q = \dfrac a b$ be a vulgar fraction.

Then $q$ is defined as being reducible  $q$ is not in canonical form.

That is,  there exists $r \in \Z: r \ne 1$ such that $r$ is a divisor of both $a$ and $b$.

Such a fraction can therefore be reduced by dividing both $a$ and $b$ by $r$.
",Definition:Reducible Fraction,"['Definitions/Reducible Fractions', 'Definitions/Vulgar Fractions', 'Definitions/Fractions']","Let q =  a b be a vulgar fraction.

Then q is defined as being reducible  q is not in canonical form.

That is,  there exists r ∈: r  1 such that r is a divisor of both a and b.

Such a fraction can therefore be reduced by dividing both a and b by r.
"
Definition:Reducible,Reducible,"Let $\rho: G \to \GL V$ be a linear representation.


$\rho$ is reducible  there exists a non-trivial proper vector subspace $W$ of $V$ such that:
:$\forall g \in G: \map {\map \rho g} W \subseteq W$


That is, such that $W$ is invariant for every linear operator in the set $\set {\map \rho g: g \in G}$.",Definition:Reducible Linear Representation,['Definitions/Representation Theory'],"Let ρ: G → V be a linear representation.


ρ is reducible  there exists a non-trivial proper vector subspace W of V such that:
:∀ g ∈ G: ρ g W ⊆ W


That is, such that W is invariant for every linear operator in the set ρ g: g ∈ G."
Definition:Reducible,Reducible,"Let $M$ be a $G$-module.

Then $M$ is reducible  the corresponding linear representation is reducible.
Let $\rho: G \to \GL V$ be a linear representation.


$\rho$ is reducible  there exists a non-trivial proper vector subspace $W$ of $V$ such that:
:$\forall g \in G: \map {\map \rho g} W \subseteq W$


That is, such that $W$ is invariant for every linear operator in the set $\set {\map \rho g: g \in G}$.
",Definition:Reducible G-Module,['Definitions/Representation Theory'],"Let M be a G-module.

Then M is reducible  the corresponding linear representation is reducible.
Let ρ: G → V be a linear representation.


ρ is reducible  there exists a non-trivial proper vector subspace W of V such that:
:∀ g ∈ G: ρ g W ⊆ W


That is, such that W is invariant for every linear operator in the set ρ g: g ∈ G.
"
Definition:Reducible,Reducible,"Let $\Sigma, \Sigma'$ be finite sets.

Let:






be sets of finite strings over $\Sigma$ and $\Sigma'$ respectively, where:
:$\Sigma^*$ denotes the set of all finite strings over the alphabet $\Sigma$.

Let $f : \Sigma^* \to \Sigma'^*$ be a computable function such that, for all $w \in \Sigma^*$:
:$w \in L \iff \map f w \in L'$

Then, $f$ is a mapping reduction from $L$ to $L'$.


If any such $f$ exists, then $L$ is mapping reducible to $L'$, which is denoted as:
:$L \le_m L'$
",Definition:Mapping Reduction,"['Definitions/Mapping Reductions', 'Definitions/Turing Machines']","Let Σ, Σ' be finite sets.

Let:






be sets of finite strings over Σ and Σ' respectively, where:
:Σ^* denotes the set of all finite strings over the alphabet Σ.

Let f : Σ^* →Σ'^* be a computable function such that, for all w ∈Σ^*:
:w ∈ L  f w ∈ L'

Then, f is a mapping reduction from L to L'.


If any such f exists, then L is mapping reducible to L', which is denoted as:
:L ≤_m L'
"
Definition:Regular,Regular,"Let $\kappa$ be an infinite cardinal.


Then $\kappa$ is a regular cardinal :
:$\map {\mathrm {cf} } \kappa = \kappa$

That is,  the cofinality of $\kappa$ is equal to itself.

",Definition:Regular Cardinal,['Definitions/Cardinals'],"Let κ be an infinite cardinal.


Then κ is a regular cardinal :
:cfκ = κ

That is,  the cofinality of κ is equal to itself.

"
Definition:Regular,Regular,"Let $U \subset \C$ be an open set.

Let $f : U \to \C$ be a complex function.


Then $f$ is analytic in $U$  for every $z_0 \in U$ there exists a sequence $\sequence {a_n}: \N \to \C$ such that the series:
:$\ds \sum_{n \mathop = 0}^\infty a_n \paren {z - z_0}^n$
converges to $\map f z$ in a neighborhood of $z_0$ in $U$.",Definition:Analytic Function/Complex Plane,['Definitions/Analytic Complex Functions'],"Let U ⊂ be an open set.

Let f : U → be a complex function.


Then f is analytic in U  for every z_0 ∈ U there exists a sequence a_n: → such that the series:
:∑_n  = 0^∞ a_n z - z_0^n
converges to f z in a neighborhood of z_0 in U."
Definition:Regular,Regular,"Let $\struct {R, +, \circ}$ be a ring with unity.

Let $n \in \Z_{>0}$ be a (strictly) positive integer.

Let $\mathbf A$ be an element of the ring of square matrices $\struct {\map {\MM_R} n, +, \times}$.


Then $\mathbf A$ is invertible :
:$\exists \mathbf B \in \struct {\map {\MM_R} n, +, \times}: \mathbf A \mathbf B = \mathbf I_n = \mathbf B \mathbf A$
where $\mathbf I_n$ denotes the unit matrix of order $n$.


Such a $\mathbf B$ is the inverse of $\mathbf A$.

It is usually denoted $\mathbf A^{-1}$.",Definition:Invertible Matrix,"['Definitions/Inverse Matrices', 'Definitions/Matrices']","Let R, +, ∘ be a ring with unity.

Let n ∈_>0 be a (strictly) positive integer.

Let 𝐀 be an element of the ring of square matrices _R n, +, ×.


Then 𝐀 is invertible :
:∃𝐁∈_R n, +, ×: 𝐀𝐁 = 𝐈_n = 𝐁𝐀
where 𝐈_n denotes the unit matrix of order n.


Such a 𝐁 is the inverse of 𝐀.

It is usually denoted 𝐀^-1."
Definition:Regular,Regular,"Let $\struct {S, \circ}$ be a magma.

The mapping $\lambda_a: S \to S$ is defined as:

:$\forall x \in S: \map {\lambda_a} x = a \circ x$


This is known as the left regular representation of $\struct {S, \circ}$ with respect to $a$.
Let $\struct {S, \circ}$ be a magma.

The mapping $\rho_a: S \to S$ is defined as:

:$\forall x \in S: \map {\rho_a} x = x \circ a$


This is known as the right regular representation of $\struct {S, \circ}$ with respect to $a$.
",Definition:Regular Representations,"['Definitions/Abstract Algebra', 'Definitions/Group Theory', 'Definitions/Regular Representations']","Let S, ∘ be a magma.

The mapping λ_a: S → S is defined as:

:∀ x ∈ S: λ_a x = a ∘ x


This is known as the left regular representation of S, ∘ with respect to a.
Let S, ∘ be a magma.

The mapping ρ_a: S → S is defined as:

:∀ x ∈ S: ρ_a x = x ∘ a


This is known as the right regular representation of S, ∘ with respect to a.
"
Definition:Regular,Regular,"Let $\struct {S, \circ}$ be a magma.

The mapping $\lambda_a: S \to S$ is defined as:

:$\forall x \in S: \map {\lambda_a} x = a \circ x$


This is known as the left regular representation of $\struct {S, \circ}$ with respect to $a$.
",Definition:Regular Representations/Left Regular Representation,"['Definitions/Left Regular Representation', 'Definitions/Regular Representations']","Let S, ∘ be a magma.

The mapping λ_a: S → S is defined as:

:∀ x ∈ S: λ_a x = a ∘ x


This is known as the left regular representation of S, ∘ with respect to a.
"
Definition:Regular,Regular,"Let $\struct {S, \circ}$ be a magma.

The mapping $\rho_a: S \to S$ is defined as:

:$\forall x \in S: \map {\rho_a} x = x \circ a$


This is known as the right regular representation of $\struct {S, \circ}$ with respect to $a$.
",Definition:Regular Representations/Right Regular Representation,"['Definitions/Right Regular Representation', 'Definitions/Regular Representations']","Let S, ∘ be a magma.

The mapping ρ_a: S → S is defined as:

:∀ x ∈ S: ρ_a x = x ∘ a


This is known as the right regular representation of S, ∘ with respect to a.
"
Definition:Regular,Regular,,Definition:Regular Element,['Definitions/Abstract Algebra'],
Definition:Regular,Regular,"Let $A$ be a commutative ring with unity.

Let $a \in A$.


Then $a$ is regular  it is not a zero divisor.",Definition:Regular Element of Ring,['Definitions/Ring Theory'],"Let A be a commutative ring with unity.

Let a ∈ A.


Then a is regular  it is not a zero divisor."
Definition:Regular,Regular,"Let $T$ be a topological space.

Let $A \subseteq T$.


Then $A$ is regular open in $T$ :
:$A = A^{- \circ}$

That is,  $A$ equals the interior of its closure.",Definition:Regular Open Set,['Definitions/Topology'],"Let T be a topological space.

Let A ⊆ T.


Then A is regular open in T :
:A = A^- ∘

That is,  A equals the interior of its closure."
Definition:Regular,Regular,"Let $T$ be a topological space.

Let $A \subseteq T$.


Then $A$ is regular closed in $T$ :
:$A = A^{\circ -}$

That is,  $A$ equals the closure of its interior.",Definition:Regular Closed Set,['Definitions/Topology'],"Let T be a topological space.

Let A ⊆ T.


Then A is regular closed in T :
:A = A^∘ -

That is,  A equals the closure of its interior."
Definition:Regular,Regular,"Let $X$ and $Y$ be smooth manifolds.

Let $f: X \to Y$ be a smooth mapping.


Then a point $y \in Y$ is called a regular value of $f$  the pushforward of $f$ at $x$:
: $f_* \vert_x: T_x X \to T_y Y$

is surjective for every $x \in \map {f^{-1} } y \subseteq X$.",Definition:Regular Value,['Definitions/Topology'],"Let X and Y be smooth manifolds.

Let f: X → Y be a smooth mapping.


Then a point y ∈ Y is called a regular value of f  the pushforward of f at x:
: f_* |_x: T_x X → T_y Y

is surjective for every x ∈f^-1 y ⊆ X."
Definition:Regular,Regular,"Let $T = \struct {S, \tau}$ be a topological space.


$\struct {S, \tau}$ is a regular space :
:$\struct {S, \tau}$ is a $T_3$ space
:$\struct {S, \tau}$ is a $T_0$ (Kolmogorov) space.


That is:
:$\forall F \subseteq S: \relcomp S F \in \tau, y \in \relcomp S F: \exists U, V \in \tau: F \subseteq U, y \in V: U \cap V = \O$ 

:$\forall x, y \in S$, either:
::$\exists U \in \tau: x \in U, y \notin U$
::$\exists U \in \tau: y \in U, x \notin U$


",Definition:Regular Space,"['Definitions/Regular Spaces', 'Definitions/T3 Spaces', 'Definitions/T0 Spaces', 'Definitions/Separation Axioms']","Let T = S, τ be a topological space.


S, τ is a regular space :
:S, τ is a T_3 space
:S, τ is a T_0 (Kolmogorov) space.


That is:
:∀ F ⊆ S:  S F ∈τ, y ∈ S F: ∃ U, V ∈τ: F ⊆ U, y ∈ V: U ∩ V = Ø 

:∀ x, y ∈ S, either:
::∃ U ∈τ: x ∈ U, y ∉ U
::∃ U ∈τ: y ∈ U, x ∉ U


"
Definition:Regular,Regular,"Let $T = \struct {S, \tau}$ be a topological space.


$\struct {S, \tau}$ is a Tychonoff Space or completely regular space :
:$\struct {S, \tau}$ is a $T_{3 \frac 1 2}$ space
:$\struct {S, \tau}$ is a $T_0$ (Kolmogorov) space.


That is:

:For any closed set $F \subseteq S$ and any point $y \in S$ such that $y \notin F$, there exists an Urysohn function for $F$ and $\set y$.

:$\forall x, y \in S$, either:
::$\exists U \in \tau: x \in U, y \notin U$
::$\exists U \in \tau: y \in U, x \notin U$


",Definition:Tychonoff Space,"['Definitions/Tychonoff Spaces', 'Definitions/Separation Axioms']","Let T = S, τ be a topological space.


S, τ is a Tychonoff Space or completely regular space :
:S, τ is a T_3 1/2 space
:S, τ is a T_0 (Kolmogorov) space.


That is:

:For any closed set F ⊆ S and any point y ∈ S such that y ∉ F, there exists an Urysohn function for F and y.

:∀ x, y ∈ S, either:
::∃ U ∈τ: x ∈ U, y ∉ U
::∃ U ∈τ: y ∈ U, x ∉ U


"
Definition:Regular,Regular,"Let $G = \struct {V, E}$ be an simple graph whose vertices all have the same degree $r$.

Then $G$ is called regular of degree $r$, or $r$-regular.
Let $G = \struct {V, E}$ be an simple graph whose vertices all have the same degree $r$.

Then $G$ is called regular of degree $r$, or $r$-regular.
",Definition:Regular Graph,"['Definitions/Regular Graphs', 'Definitions/Graph Theory']","Let G = V, E be an simple graph whose vertices all have the same degree r.

Then G is called regular of degree r, or r-regular.
Let G = V, E be an simple graph whose vertices all have the same degree r.

Then G is called regular of degree r, or r-regular.
"
Definition:Regular,Regular,"A regular expression is an algebraic structure on an alphabet $\Sigma$ defined as follows:

* The empty-set regular expression, $\O$, is a regular expression.

* The empty-word regular expression, $\epsilon$, is a regular expression.

* Every $\sigma \in \Sigma$ is a regular expression. (These are called literals.)

* If $R_1$ and $R_2$ are regular expressions, $R_1 R_2$ is a regular expression (concatenation).

* If $R_1$ and $R_2$ are regular expressions, $R_1 \mid R_2$ is a regular expression (alternation).

* If $R$ is a regular expression, $R^*$ is a regular expression (Kleene star).


Every regular expression $R$ on an alphabet $\Sigma$ defines a language $\map L R \subseteq \Sigma^*$, where $\Sigma^*$ is the set of all (finite-length) words (sequences) of symbols in $\Sigma$:

* $\map L \O = \O$ (the empty set).

* $\map L \epsilon = \set {\sqbrk \,}$ (the set containing only the empty word).

* If $R$ is a literal $\sigma$, $\map L R = \set {\sqbrk \sigma}$ (i.e., the set containing only the single-symbol word “$\sigma$”).

* If $R$ is a concatenation $R_1 R_2$, $\map L R = \set {w_1 w_2: w_1 \in \map L {R_1}, w_2 \in \map L {R_2} }$, where $w_1 w_2$ is the concatenation of the words $w_1$ and $w_2$.

* If $R$ is an alternation $R_1 \mid R_2$, $\map L R = L {R_1} \cup \map L {R_2}$.

* If $R$ is a Kleene star $R_1^*$, $\map L R$ is the smallest set satisfying the following:
** $\sqbrk \, \in \map L R$ (the empty word is in the set);
** if $w_1 \in \map L R$ and $w_2 \in \map L {R_1}$, then $w_1 w_2 \in \map L R$.

Category:Definitions/Formal Systems",Definition:Regular Expression,['Definitions/Formal Systems'],"A regular expression is an algebraic structure on an alphabet Σ defined as follows:

* The empty-set regular expression, Ø, is a regular expression.

* The empty-word regular expression, ϵ, is a regular expression.

* Every σ∈Σ is a regular expression. (These are called literals.)

* If R_1 and R_2 are regular expressions, R_1 R_2 is a regular expression (concatenation).

* If R_1 and R_2 are regular expressions, R_1 | R_2 is a regular expression (alternation).

* If R is a regular expression, R^* is a regular expression (Kleene star).


Every regular expression R on an alphabet Σ defines a language L R ⊆Σ^*, where Σ^* is the set of all (finite-length) words (sequences) of symbols in Σ:

* L Ø = Ø (the empty set).

* L ϵ = (the set containing only the empty word).

* If R is a literal σ, L R = σ (i.e., the set containing only the single-symbol word “σ”).

* If R is a concatenation R_1 R_2, L R = w_1 w_2: w_1 ∈ L R_1, w_2 ∈ L R_2, where w_1 w_2 is the concatenation of the words w_1 and w_2.

* If R is an alternation R_1 | R_2, L R = L R_1∪ L R_2.

* If R is a Kleene star R_1^*, L R is the smallest set satisfying the following:
** ∈ L R (the empty word is in the set);
** if w_1 ∈ L R and w_2 ∈ L R_1, then w_1 w_2 ∈ L R.

Category:Definitions/Formal Systems"
Definition:Representation,Representation,"Let $G$ be a group.

Let $X$ be a set.

Let $\struct {\map \Gamma X, \circ}$ be the symmetric group on $X$.


A permutation representation of $G$ is a group homomorphism from $G$ to $\struct {\map \Gamma X, \circ}$.


=== Associated to Group Action ===
",Definition:Permutation Representation,['Definitions/Group Actions'],"Let G be a group.

Let X be a set.

Let Γ X, ∘ be the symmetric group on X.


A permutation representation of G is a group homomorphism from G to Γ X, ∘.


=== Associated to Group Action ===
"
Definition:Representation,Representation,"Let $R$ be a ring with unity.

Let $M$ be an abelian group.


A unital ring representation of $R$ on $M$ is a ring representation $R \to \map {\operatorname {End} } M$ which is unital.

That is, it is a unital ring homomorphism from $R$ to the endomorphism ring $\map {\operatorname {End} } M$.
",Definition:Ring Representation,['Definitions/Module Theory'],"Let R be a ring with unity.

Let M be an abelian group.


A unital ring representation of R on M is a ring representation R →End M which is unital.

That is, it is a unital ring homomorphism from R to the endomorphism ring End M.
"
Definition:Representation,Representation,"Let $\mathbf C$ be a locally small category.

Let $\mathbf{Set}$ be the category of sets.

Let $F : \mathbf C \to \mathbf{Set}$ be a covariant functor.


A representation of $F$ is a pair $\tuple {C, \eta}$ where $\eta : \map {\operatorname {Hom} } {C, \cdot} \to F$ is a natural isomorphism with the covariant hom functor of $C$.",Definition:Representation of Functor,['Definitions/Category Theory'],"Let 𝐂 be a locally small category.

Let 𝐒𝐞𝐭 be the category of sets.

Let F : 𝐂→𝐒𝐞𝐭 be a covariant functor.


A representation of F is a pair C, η where η : HomC, ·→ F is a natural isomorphism with the covariant hom functor of C."
Definition:Residue,Residue,"Let $f: \C \to \C$ be a complex function.

Let $z_0 \in U \subset \C$ such that $f$ is analytic in $U \setminus \set {z_0}$.


Then by Existence of Laurent Series, there is a Laurent series:
:$\ds \sum_{j \mathop = -\infty}^\infty a_j \paren {z - z_0}^j$
such that the sum converges to $f$ in $U - \set {z_0}$.  


The residue at a point $z = z_0$ of $f$ is defined as $a_{-1}$ in that Laurent series.

It is denoted $\Res f {z_0}$ or just $\map {\mathrm {Res} } {z_0}$ when $f$ is understood.
",Definition:Residue (Complex Analysis),['Definitions/Complex Analysis'],"Let f: → be a complex function.

Let z_0 ∈ U ⊂ such that f is analytic in U ∖z_0.


Then by Existence of Laurent Series, there is a Laurent series:
:∑_j  = -∞^∞ a_j z - z_0^j
such that the sum converges to f in U - z_0.  


The residue at a point z = z_0 of f is defined as a_-1 in that Laurent series.

It is denoted f z_0 or just Resz_0 when f is understood.
"
Definition:Residue,Residue,"Let $m, n \in \N$ be natural numbers.

Let $a \in \Z$ be an integer such that $a$ is not divisible by $m$.

Then $a$ is a residue of $m$ of order $n$ :
:$\exists x \in \Z: x^n \equiv a \pmod m$
where $\equiv$ denotes modulo congruence.


=== Nonresidue ===

Let $m, n \in \N$ be natural numbers.

Let $a \in \Z$ be an integer such that $a$ is not divisible by $m$.


$a$ is a nonresidue of $m$ of order $n$  there does not exist $x \in \Z$ such that:
:$x^n \equiv a \pmod m$

where $\equiv$ denotes modulo congruence.
",Definition:Residue (Number Theory),"['Definitions/Residues (Number Theory)', 'Definitions/Number Theory']","Let m, n ∈ be natural numbers.

Let a ∈ be an integer such that a is not divisible by m.

Then a is a residue of m of order n :
:∃ x ∈: x^n ≡ a  m
where ≡ denotes modulo congruence.


=== Nonresidue ===

Let m, n ∈ be natural numbers.

Let a ∈ be an integer such that a is not divisible by m.


a is a nonresidue of m of order n  there does not exist x ∈ such that:
:x^n ≡ a  m

where ≡ denotes modulo congruence.
"
Definition:Residue,Residue,"Let $m \in \Z_{\ne 0}$ be a non-zero integer.

Let $a, b \in \Z$.

Let $a \equiv b \pmod m$.


Then $b$ is a residue of $a$ modulo $m$.

Residue is another word for remainder, and is any integer congruent to $a$ modulo $m$.
Let $m \in \Z_{\ne 0}$ be a non-zero integer.

Let $a, b \in \Z$.

Let $a \equiv b \pmod m$.


Then $b$ is a residue of $a$ modulo $m$.

Residue is another word for remainder, and is any integer congruent to $a$ modulo $m$.
",Definition:Congruence (Number Theory)/Residue,"['Definitions/Congruence (Number Theory)', 'Definitions/Residue Classes']","Let m ∈_ 0 be a non-zero integer.

Let a, b ∈.

Let a ≡ b  m.


Then b is a residue of a modulo m.

Residue is another word for remainder, and is any integer congruent to a modulo m.
Let m ∈_ 0 be a non-zero integer.

Let a, b ∈.

Let a ≡ b  m.


Then b is a residue of a modulo m.

Residue is another word for remainder, and is any integer congruent to a modulo m.
"
Definition:Residue,Residue,"Let $R$ be a commutative local ring.

Let $m$ be its maximal ideal.


The residue field of $R$ is the quotient ring $R / m$.
",Definition:Residue Field of Local Ring,['Definitions/Local Rings'],"Let R be a commutative local ring.

Let m be its maximal ideal.


The residue field of R is the quotient ring R / m.
"
Definition:Right,Right,"The direction right is that way:
:$\to$",Definition:Right (Direction),['Definitions/Language Definitions'],"The direction right is that way:
:→"
Definition:Right,Right,"A right angle is an angle that is equal to half of a straight angle.


=== Measurement of Right Angle ===
",Definition:Right Angle,"['Definitions/Right Angles', 'Definitions/Angles']","A right angle is an angle that is equal to half of a straight angle.


=== Measurement of Right Angle ===
"
Definition:Right,Right,"In an equation:
:$\text {Expression $1$} = \text {Expression $2$}$
the term $\text {Expression $2$}$ is the right hand side.",Definition:Right Hand Side,['Definitions/Language Definitions'],"In an equation:
:Expression 1 = Expression 2
the term Expression 2 is the right hand side."
Definition:Right,Right,"Let $\RR \subseteq S \times T$ be a relation.


Then $\RR$ is right-total :
:$\forall t \in T: \exists s \in S: \tuple {s, t} \in \RR$


That is,  every element of $T$ is related to by some element of $S$.


That is, :
:$\Img \RR = T$
where $\Img \RR$ denotes the image of $\RR$.",Definition:Right-Total Relation,['Definitions/Relation Theory'],"Let ⊆ S × T be a relation.


Then  is right-total :
:∀ t ∈ T: ∃ s ∈ S: s, t∈


That is,  every element of T is related to by some element of S.


That is, :
:= T
where  denotes the image of ."
Definition:Right,Right,"Let $\RR \subseteq S \times S$ be a relation in $S$.


$\RR$ is right-Euclidean :

:$\tuple {x, y} \in \RR \land \tuple {x, z} \in \RR \implies \tuple {y, z} \in \RR$",Definition:Euclidean Relation/Right-Euclidean,['Definitions/Euclidean Relations'],"Let ⊆ S × S be a relation in S.


 is right-Euclidean :

:x, y∈x, z∈y, z∈"
Definition:Right,Right,"Let $A$ be a class.

Let $\RR$ be a relation on $A$.


An element $x$ of $A$ is right normal with respect to $\RR$ :
:$\forall y \in A: \map \RR {y, x}$ holds.",Definition:Right Normal Element of Relation,['Definitions/Relations'],"Let A be a class.

Let  be a relation on A.


An element x of A is right normal with respect to  :
:∀ y ∈ A: y, x holds."
Definition:Right,Right,"A mapping $f: X \to Y$ is right cancellable (or right-cancellable) :

:$\forall Z: \forall \paren {h_1, h_2: Y \to Z}: h_1 \circ f = h_2 \circ f \implies h_1 = h_2$

That is,  for any set $Z$:
:If $h_1$ and $h_2$ are mappings from $Y$ to $Z$
:then $h_1 \circ f = h_2 \circ f$ implies $h_1 = h_2$.",Definition:Right Cancellable Mapping,"['Definitions/Mapping Theory', 'Definitions/Cancellability']","A mapping f: X → Y is right cancellable (or right-cancellable) :

:∀ Z: ∀h_1, h_2: Y → Z: h_1 ∘ f = h_2 ∘ f  h_1 = h_2

That is,  for any set Z:
:If h_1 and h_2 are mappings from Y to Z
:then h_1 ∘ f = h_2 ∘ f implies h_1 = h_2."
Definition:Right,Right,"Let $S, T$ be sets where $S \ne \O$, that is, $S$ is not empty.

Let $f: S \to T$ be a mapping.


Let $g: T \to S$ be a mapping such that:
:$f \circ g = I_T$
where:
:$f \circ g$ denotes the composite mapping $g$ followed by $f$
:$I_T$ is the identity mapping on $T$.


Then $g: T \to S$ is called a right inverse (mapping) of $f$.",Definition:Right Inverse Mapping,['Definitions/Inverse Mappings'],"Let S, T be sets where S Ø, that is, S is not empty.

Let f: S → T be a mapping.


Let g: T → S be a mapping such that:
:f ∘ g = I_T
where:
:f ∘ g denotes the composite mapping g followed by f
:I_T is the identity mapping on T.


Then g: T → S is called a right inverse (mapping) of f."
Definition:Right,Right,"Let $\struct {S, \preccurlyeq}$ be an ordered set.

Let $a, b \in S$.


The right half-open interval between $a$ and $b$ is the set:

:$\hointr a b := a^\succcurlyeq \cap b^\prec = \set {s \in S: \paren {a \preccurlyeq s} \land \paren {s \prec b} }$

where:
:$a^\succcurlyeq$ denotes the upper closure of $a$
:$b^\prec$ denotes the strict lower closure of $b$.
",Definition:Interval/Ordered Set/Right Half-Open,['Definitions/Intervals'],"Let S, ≼ be an ordered set.

Let a, b ∈ S.


The right half-open interval between a and b is the set:

:a b := a^≽∩ b^≺ = s ∈ S: a ≼ ss ≺ b

where:
:a^≽ denotes the upper closure of a
:b^≺ denotes the strict lower closure of b.
"
Definition:Right,Right,"A right angle is an angle that is equal to half of a straight angle.


=== Measurement of Right Angle ===

",Definition:Orientation of Coordinate Axes/Cartesian Plane/Right-Handed,['Definitions/Orientation (Coordinate Axes)'],"A right angle is an angle that is equal to half of a straight angle.


=== Measurement of Right Angle ===

"
Definition:Right,Right,"A right angle is an angle that is equal to half of a straight angle.


=== Measurement of Right Angle ===

A right angle is an angle that is equal to half of a straight angle.


=== Measurement of Right Angle ===

",Definition:Orientation of Coordinate Axes/Cartesian 3-Space/Right-Handed,"['Definitions/Orientation (Coordinate Axes)', 'Definitions/Right-Hand Rule']","A right angle is an angle that is equal to half of a straight angle.


=== Measurement of Right Angle ===

A right angle is an angle that is equal to half of a straight angle.


=== Measurement of Right Angle ===

"
Definition:Right,Right,"Let $\Bbb I = \openint a b$ be an open real interval.

Let $f: \Bbb I \to \R$ be a real function.

Let $L \in \R$.


Suppose that:
:$\forall \epsilon \in \R_{>0}: \exists \delta \in \R_{>0}: \forall x \in \Bbb I: a < x < a + \delta \implies \size {\map f x - L} < \epsilon$
where $\R_{>0}$ denotes the set of strictly positive real numbers.

That is, for every real strictly positive $\epsilon$ there exists a real strictly positive $\delta$ such that every real number in the domain of $f$, greater than $a$ but within $\delta$ of $a$, has an image within $\epsilon$ of $L$.


:

Then $\map f x$ is said to tend to the limit $L$ as $x$ tends to $a$ from the right, and we write:
:$\map f x \to L$ as $x \to a^+$
or
:$\ds \lim_{x \mathop \to a^+} \map f x = L$


This is voiced
:the limit of $\map f x$ as $x$ tends to $a$ from the right
and such an $L$ is called:
:a limit from the right.
",Definition:Right-Hand Derivative,['Definitions/Derivatives'],"Let I =  a b be an open real interval.

Let f:  I → be a real function.

Let L ∈.


Suppose that:
:∀ϵ∈_>0: ∃δ∈_>0: ∀ x ∈ I: a < x < a + δ f x - L < ϵ
where _>0 denotes the set of strictly positive real numbers.

That is, for every real strictly positive ϵ there exists a real strictly positive δ such that every real number in the domain of f, greater than a but within δ of a, has an image within ϵ of L.


:

Then f x is said to tend to the limit L as x tends to a from the right, and we write:
:f x → L as x → a^+
or
:lim_x → a^+ f x = L


This is voiced
:the limit of f x as x tends to a from the right
and such an L is called:
:a limit from the right.
"
Definition:Right,Right,"Let $V$ be a vector space over the real numbers $\R$.

Let $f: \R \to V$ be a function.


A right difference quotient is an expression of the form:
:$\dfrac {\map f {x + h} - \map f x} h$
where $h > 0$ is a strictly positive real number.",Definition:Difference Quotient/Right,['Definitions/Difference Quotients'],"Let V be a vector space over the real numbers .

Let f: → V be a function.


A right difference quotient is an expression of the form:
:f x + h -  f x h
where h > 0 is a strictly positive real number."
Definition:Right,Right,"Let $\Bbb I = \openint a b$ be an open real interval.

Let $f: \Bbb I \to \R$ be a real function.

Let $L \in \R$.


Suppose that:
:$\forall \epsilon \in \R_{>0}: \exists \delta \in \R_{>0}: \forall x \in \Bbb I: a < x < a + \delta \implies \size {\map f x - L} < \epsilon$
where $\R_{>0}$ denotes the set of strictly positive real numbers.

That is, for every real strictly positive $\epsilon$ there exists a real strictly positive $\delta$ such that every real number in the domain of $f$, greater than $a$ but within $\delta$ of $a$, has an image within $\epsilon$ of $L$.


:

Then $\map f x$ is said to tend to the limit $L$ as $x$ tends to $a$ from the right, and we write:
:$\map f x \to L$ as $x \to a^+$
or
:$\ds \lim_{x \mathop \to a^+} \map f x = L$


This is voiced
:the limit of $\map f x$ as $x$ tends to $a$ from the right
and such an $L$ is called:
:a limit from the right.
Let $\Bbb I = \openint a b$ be an open real interval.

Let $f: \Bbb I \to \R$ be a real function.

Let $L \in \R$.


Suppose that:
:$\forall \epsilon \in \R_{>0}: \exists \delta \in \R_{>0}: \forall x \in \Bbb I: a < x < a + \delta \implies \size {\map f x - L} < \epsilon$
where $\R_{>0}$ denotes the set of strictly positive real numbers.

That is, for every real strictly positive $\epsilon$ there exists a real strictly positive $\delta$ such that every real number in the domain of $f$, greater than $a$ but within $\delta$ of $a$, has an image within $\epsilon$ of $L$.


:

Then $\map f x$ is said to tend to the limit $L$ as $x$ tends to $a$ from the right, and we write:
:$\map f x \to L$ as $x \to a^+$
or
:$\ds \lim_{x \mathop \to a^+} \map f x = L$


This is voiced
:the limit of $\map f x$ as $x$ tends to $a$ from the right
and such an $L$ is called:
:a limit from the right.
",Definition:Continuous Real Function/Right-Continuous,['Definitions/Continuous Real Functions'],"Let I =  a b be an open real interval.

Let f:  I → be a real function.

Let L ∈.


Suppose that:
:∀ϵ∈_>0: ∃δ∈_>0: ∀ x ∈ I: a < x < a + δ f x - L < ϵ
where _>0 denotes the set of strictly positive real numbers.

That is, for every real strictly positive ϵ there exists a real strictly positive δ such that every real number in the domain of f, greater than a but within δ of a, has an image within ϵ of L.


:

Then f x is said to tend to the limit L as x tends to a from the right, and we write:
:f x → L as x → a^+
or
:lim_x → a^+ f x = L


This is voiced
:the limit of f x as x tends to a from the right
and such an L is called:
:a limit from the right.
Let I =  a b be an open real interval.

Let f:  I → be a real function.

Let L ∈.


Suppose that:
:∀ϵ∈_>0: ∃δ∈_>0: ∀ x ∈ I: a < x < a + δ f x - L < ϵ
where _>0 denotes the set of strictly positive real numbers.

That is, for every real strictly positive ϵ there exists a real strictly positive δ such that every real number in the domain of f, greater than a but within δ of a, has an image within ϵ of L.


:

Then f x is said to tend to the limit L as x tends to a from the right, and we write:
:f x → L as x → a^+
or
:lim_x → a^+ f x = L


This is voiced
:the limit of f x as x tends to a from the right
and such an L is called:
:a limit from the right.
"
Definition:Right,Right,"Let $a, b \in \R$ be real numbers.


The right half-open (real) interval from $a$ to $b$ is the subset:
:$\hointr a b := \set {x \in \R: a \le x < b}$
",Definition:Real Interval/Half-Open/Right,['Definitions/Real Intervals'],"Let a, b ∈ be real numbers.


The right half-open (real) interval from a to b is the subset:
:a b := x ∈: a ≤ x < b
"
Definition:Right,Right,"There are two unbounded closed intervals involving a real number $a \in \R$, defined as:




",Definition:Real Interval/Unbounded Closed,['Definitions/Real Intervals'],"There are two unbounded closed intervals involving a real number a ∈, defined as:




"
Definition:Right,Right,"There are two unbounded open intervals involving a real number $a \in \R$, defined as:




",Definition:Real Interval/Unbounded Open,['Definitions/Real Intervals'],"There are two unbounded open intervals involving a real number a ∈, defined as:




"
Definition:Right,Right,"Let $\struct {S, \circ}$ be an algebraic structure.

An element $z_R \in S$ is called a right zero element (or just right zero) :
:$\forall x \in S: x \circ z_R = z_R$",Definition:Right Zero,['Definitions/Zero Elements'],"Let S, ∘ be an algebraic structure.

An element z_R ∈ S is called a right zero element (or just right zero) :
:∀ x ∈ S: x ∘ z_R = z_R"
Definition:Right,Right,"Let $\struct {S, \circ}$ be an algebraic structure.

An element $e_R \in S$ is called a right identity (element) :
:$\forall x \in S: x \circ e_R = x$
",Definition:Identity (Abstract Algebra)/Right Identity,['Definitions/Identity Elements'],"Let S, ∘ be an algebraic structure.

An element e_R ∈ S is called a right identity (element) :
:∀ x ∈ S: x ∘ e_R = x
"
Definition:Right,Right,"Let $S$ be a set.

For any $x, y \in S$, the right operation on $S$ is the binary operation defined as:
:$\forall x, y \in S: x \to y = y$",Definition:Right Operation,"['Definitions/Abstract Algebra', 'Definitions/Right Operation']","Let S be a set.

For any x, y ∈ S, the right operation on S is the binary operation defined as:
:∀ x, y ∈ S: x → y = y"
Definition:Right,Right,"Let $\struct {S, \circ}$ be an algebraic structure.


An element $x \in \struct {S, \circ}$ is right cancellable :

:$\forall a, b \in S: a \circ x = b \circ x \implies a = b$",Definition:Cancellable Element/Right Cancellable,['Definitions/Cancellability'],"Let S, ∘ be an algebraic structure.


An element x ∈S, ∘ is right cancellable :

:∀ a, b ∈ S: a ∘ x = b ∘ x  a = b"
Definition:Right,Right,"Let $\struct {S, \circ}$ be an algebraic structure.


An element $x \in \struct {S, \circ}$ is right cancellable :

:$\forall a, b \in S: a \circ x = b \circ x \implies a = b$
",Definition:Right Cancellable Operation,['Definitions/Cancellability'],"Let S, ∘ be an algebraic structure.


An element x ∈S, ∘ is right cancellable :

:∀ a, b ∈ S: a ∘ x = b ∘ x  a = b
"
Definition:Right,Right,"Let $S$ be a set on which is defined two binary operations, defined on all the elements of $S \times S$, denoted here as $\circ$ and $*$.

The operation $\circ$ is right distributive over the operation $*$ :

:$\forall a, b, c \in S: \paren {a * b} \circ c = \paren {a \circ c} * \paren {b \circ c}$",Definition:Distributive Operation/Right,['Definitions/Distributive Operations'],"Let S be a set on which is defined two binary operations, defined on all the elements of S × S, denoted here as ∘ and *.

The operation ∘ is right distributive over the operation * :

:∀ a, b, c ∈ S: a * b∘ c = a ∘ c * b ∘ c"
Definition:Right,Right,"Let $\struct {S, \circ}$ be a monoid whose identity is $e_S$.

An element $x_R \in S$ is called a right inverse of $x$ :
:$x \circ x_R = e_S$
",Definition:Inverse (Abstract Algebra)/Right Inverse,['Definitions/Inverse Elements'],"Let S, ∘ be a monoid whose identity is e_S.

An element x_R ∈ S is called a right inverse of x :
:x ∘ x_R = e_S
"
Definition:Right,Right,"Let $\struct {S, \circ}$ be a magma.

The mapping $\rho_a: S \to S$ is defined as:

:$\forall x \in S: \map {\rho_a} x = x \circ a$


This is known as the right regular representation of $\struct {S, \circ}$ with respect to $a$.
",Definition:Quasigroup/Right Quasigroup,['Definitions/Quasigroups'],"Let S, ∘ be a magma.

The mapping ρ_a: S → S is defined as:

:∀ x ∈ S: ρ_a x = x ∘ a


This is known as the right regular representation of S, ∘ with respect to a.
"
Definition:Right,Right,"Let $x$ and $y$ be elements which are operated on by a given operation $\circ$.

The right-hand product of $x$ by $y$ is the product $x \circ y$.",Definition:Operation/Binary Operation/Product/Right,['Definitions/Operations'],"Let x and y be elements which are operated on by a given operation ∘.

The right-hand product of x by y is the product x ∘ y."
Definition:Right,Right,"Let $\struct {S, \circ, \preceq}$ be a positively totally ordered semigroup.


Then $\struct {S, \circ, \preceq}$ is a right naturally totally ordered semigroup :

:$\forall a, b \in S: a \prec b \implies \exists x \in S: b = a \circ x$",Definition:Right Naturally Totally Ordered Semigroup,['Definitions/Naturally Ordered Semigroup'],"Let S, ∘, ≼ be a positively totally ordered semigroup.


Then S, ∘, ≼ is a right naturally totally ordered semigroup :

:∀ a, b ∈ S: a ≺ b ∃ x ∈ S: b = a ∘ x"
Definition:Right,Right,"Let $\struct {S, \circ}$ be an algebraic structure.

Let $\struct {H, \circ}$ be a subgroup of $\struct {S, \circ}$.


The right coset of $y$ modulo $H$, or right coset of $H$ by $y$, is:

:$H \circ y = \set {x \in S: \exists h \in H: x = h \circ y}$


That is, it is the subset product with singleton:

:$H \circ y = H \circ \set y$",Definition:Coset/Right Coset,['Definitions/Cosets'],"Let S, ∘ be an algebraic structure.

Let H, ∘ be a subgroup of S, ∘.


The right coset of y modulo H, or right coset of H by y, is:

:H ∘ y = x ∈ S: ∃ h ∈ H: x = h ∘ y


That is, it is the subset product with singleton:

:H ∘ y = H ∘ y"
Definition:Right,Right,"Let $G$ be a group.

Let $H$ be a subgroup of $G$.


We can use $H$ to define a relation on $G$ as follows:

:$\RR^r_H = \set {\tuple {x, y} \in G \times G: x y^{-1} \in H}$

This is called right congruence modulo $H$.
Let $\struct {S, \circ}$ be an algebraic structure.

Let $\struct {H, \circ}$ be a subgroup of $\struct {S, \circ}$.


The right coset of $y$ modulo $H$, or right coset of $H$ by $y$, is:

:$H \circ y = \set {x \in S: \exists h \in H: x = h \circ y}$


That is, it is the subset product with singleton:

:$H \circ y = H \circ \set y$
",Definition:Coset Space/Right Coset Space,['Definitions/Cosets'],"Let G be a group.

Let H be a subgroup of G.


We can use H to define a relation on G as follows:

:^r_H = x, y∈ G × G: x y^-1∈ H

This is called right congruence modulo H.
Let S, ∘ be an algebraic structure.

Let H, ∘ be a subgroup of S, ∘.


The right coset of y modulo H, or right coset of H by y, is:

:H ∘ y = x ∈ S: ∃ h ∈ H: x = h ∘ y


That is, it is the subset product with singleton:

:H ∘ y = H ∘ y
"
Definition:Right,Right,"Let $G$ be a group.

Let $H$ be a subgroup of $G$.

Let $S \subseteq G$ be a subset of $G$.


$S$ is a right transversal for $H$ in $G$  every right coset of $H$ contains exactly one element of $S$.
Let $\struct {S, \circ}$ be an algebraic structure.

Let $\struct {H, \circ}$ be a subgroup of $\struct {S, \circ}$.


The right coset of $y$ modulo $H$, or right coset of $H$ by $y$, is:

:$H \circ y = \set {x \in S: \exists h \in H: x = h \circ y}$


That is, it is the subset product with singleton:

:$H \circ y = H \circ \set y$
",Definition:Transversal (Group Theory)/Right Transversal,['Definitions/Transversals (Group Theory)'],"Let G be a group.

Let H be a subgroup of G.

Let S ⊆ G be a subset of G.


S is a right transversal for H in G  every right coset of H contains exactly one element of S.
Let S, ∘ be an algebraic structure.

Let H, ∘ be a subgroup of S, ∘.


The right coset of y modulo H, or right coset of H by y, is:

:H ∘ y = x ∈ S: ∃ h ∈ H: x = h ∘ y


That is, it is the subset product with singleton:

:H ∘ y = H ∘ y
"
Definition:Right,Right,"Let $X$ be a set.

Let $\struct {G, \circ}$ be a group whose identity is $e$.


A right group action is a mapping $\phi: X \times G \to X$ such that:

:$\forall \tuple {x, g} \in X \times G : x * g := \map \phi {x, g} \in X$

in such a way that the right group action axioms are satisfied:
",Definition:Group Action/Right Group Action,['Definitions/Group Actions'],"Let X be a set.

Let G, ∘ be a group whose identity is e.


A right group action is a mapping ϕ: X × G → X such that:

:∀x, g∈ X × G : x * g := ϕx, g∈ X

in such a way that the right group action axioms are satisfied:
"
Definition:Right,Right,"Let $\struct {G, \circ}$ be a group.

Let $\powerset G$ be the power set of $G$.


The (right) subset product action of $G$ is the group action $*: G \times \powerset G \to \powerset G$:
:$\forall g \in G, S \in \powerset G: g * S = S \circ g$",Definition:Subset Product Action/Right,['Definitions/Subset Product Action'],"Let G, ∘ be a group.

Let G be the power set of G.


The (right) subset product action of G is the group action *: G × G → G:
:∀ g ∈ G, S ∈ G: g * S = S ∘ g"
Definition:Right,Right,"Let $G$ be a group.

Let $H$ be a subgroup of $G$.


We can use $H$ to define a relation on $G$ as follows:

:$\RR^r_H = \set {\tuple {x, y} \in G \times G: x y^{-1} \in H}$

This is called right congruence modulo $H$.",Definition:Congruence Modulo Subgroup/Right Congruence,['Definitions/Congruence Modulo Subgroup'],"Let G be a group.

Let H be a subgroup of G.


We can use H to define a relation on G as follows:

:^r_H = x, y∈ G × G: x y^-1∈ H

This is called right congruence modulo H."
Definition:Right,Right,"Let $\struct {R, +, \circ}$ be a ring.


A right zero divisor (in $R$) is an element $x \in R$ such that:
: $\exists y \in R^*: y \circ x = 0_R$

where $R^*$ is defined as $R \setminus \set {0_R}$.",Definition:Right Zero Divisor,['Definitions/Zero Divisors'],"Let R, +, ∘ be a ring.


A right zero divisor (in R) is an element x ∈ R such that:
: ∃ y ∈ R^*: y ∘ x = 0_R

where R^* is defined as R ∖0_R."
Definition:Right,Right,"Let $R$ be a ring.

Let $M$ be an abelian group.

Let $\circ : M \times R \to M$ be a mapping from the cartesian product $M \times R$.


$\circ$ is a right linear ring action of $R$ on $M$  $\circ$ satisfies the right ring action axioms:
",Definition:Linear Ring Action/Right,['Definitions/Linear Ring Actions'],"Let R be a ring.

Let M be an abelian group.

Let ∘ : M × R → M be a mapping from the cartesian product M × R.


∘ is a right linear ring action of R on M  ∘ satisfies the right ring action axioms:
"
Definition:Right,Right,"Let $\struct {R, +, \circ}$ be a ring.

Let $\struct {J, +}$ be a subgroup of $\struct {R, +}$.


$J$ is a right ideal of $R$ :
:$\forall j \in J: \forall r \in R: j \circ r \in J$

that is, :
:$\forall r \in R: J \circ r \subseteq J$
",Definition:Ideal of Ring/Right Ideal,['Definitions/Ideal Theory'],"Let R, +, ∘ be a ring.

Let J, + be a subgroup of R, +.


J is a right ideal of R :
:∀ j ∈ J: ∀ r ∈ R: j ∘ r ∈ J

that is, :
:∀ r ∈ R: J ∘ r ⊆ J
"
Definition:Right,Right,"Let $\struct {R, +, \circ}$ be a ring.

Let $\struct {J, +}$ be a subgroup of $\struct {R, +}$.


$J$ is a right ideal of $R$ :
:$\forall j \in J: \forall r \in R: j \circ r \in J$

that is, :
:$\forall r \in R: J \circ r \subseteq J$
Let $\struct {R, +, \circ}$ be a ring.

Let $\struct {J, +}$ be a subgroup of $\struct {R, +}$.


$J$ is a right ideal of $R$ :
:$\forall j \in J: \forall r \in R: j \circ r \in J$

that is, :
:$\forall r \in R: J \circ r \subseteq J$
",Definition:Maximal Ideal of Ring/Right,['Definitions/Maximal Ideals of Rings'],"Let R, +, ∘ be a ring.

Let J, + be a subgroup of R, +.


J is a right ideal of R :
:∀ j ∈ J: ∀ r ∈ R: j ∘ r ∈ J

that is, :
:∀ r ∈ R: J ∘ r ⊆ J
Let R, +, ∘ be a ring.

Let J, + be a subgroup of R, +.


J is a right ideal of R :
:∀ j ∈ J: ∀ r ∈ R: j ∘ r ∈ J

that is, :
:∀ r ∈ R: J ∘ r ⊆ J
"
Definition:Right,Right,"Let $\struct {R, +_R, \times_R}$ be a ring.

Let $\struct {G, +_G}$ be an abelian group.


A right module over $R$ is an $R$-algebraic structure $\struct {G, +_G, \circ}_R$ with one operation $\circ$, the (right) ring action, which satisfies the right module axioms:

",Definition:Right Module,"['Definitions/Right Modules', 'Definitions/Module Theory']","Let R, +_R, ×_R be a ring.

Let G, +_G be an abelian group.


A right module over R is an R-algebraic structure G, +_G, ∘_R with one operation ∘, the (right) ring action, which satisfies the right module axioms:

"
Definition:Right,Right,"A right-truncatable prime is a prime number which remains prime when any number of digits are removed from the right hand end.


=== Sequence ===
",Definition:Right-Truncatable Prime,"['Definitions/Right-Truncatable Primes', 'Definitions/Prime Numbers', 'Definitions/Recreational Mathematics']","A right-truncatable prime is a prime number which remains prime when any number of digits are removed from the right hand end.


=== Sequence ===
"
Definition:Right,Right,,Definition:Right Cancellable,['Definitions/Cancellability'],
Definition:Right Cancellable,Right Cancellable,"Let $\struct {S, \circ}$ be an algebraic structure.


An element $x \in \struct {S, \circ}$ is right cancellable :

:$\forall a, b \in S: a \circ x = b \circ x \implies a = b$",Definition:Cancellable Element/Right Cancellable,['Definitions/Cancellability'],"Let S, ∘ be an algebraic structure.


An element x ∈S, ∘ is right cancellable :

:∀ a, b ∈ S: a ∘ x = b ∘ x  a = b"
Definition:Right Cancellable,Right Cancellable,"Let $\struct {S, \circ}$ be an algebraic structure.


An element $x \in \struct {S, \circ}$ is right cancellable :

:$\forall a, b \in S: a \circ x = b \circ x \implies a = b$
",Definition:Right Cancellable Operation,['Definitions/Cancellability'],"Let S, ∘ be an algebraic structure.


An element x ∈S, ∘ is right cancellable :

:∀ a, b ∈ S: a ∘ x = b ∘ x  a = b
"
Definition:Right Cancellable,Right Cancellable,"A mapping $f: X \to Y$ is right cancellable (or right-cancellable) :

:$\forall Z: \forall \paren {h_1, h_2: Y \to Z}: h_1 \circ f = h_2 \circ f \implies h_1 = h_2$

That is,  for any set $Z$:
:If $h_1$ and $h_2$ are mappings from $Y$ to $Z$
:then $h_1 \circ f = h_2 \circ f$ implies $h_1 = h_2$.",Definition:Right Cancellable Mapping,"['Definitions/Mapping Theory', 'Definitions/Cancellability']","A mapping f: X → Y is right cancellable (or right-cancellable) :

:∀ Z: ∀h_1, h_2: Y → Z: h_1 ∘ f = h_2 ∘ f  h_1 = h_2

That is,  for any set Z:
:If h_1 and h_2 are mappings from Y to Z
:then h_1 ∘ f = h_2 ∘ f implies h_1 = h_2."
Definition:Right Inverse,Right Inverse,"Let $S, T$ be sets where $S \ne \O$, that is, $S$ is not empty.

Let $f: S \to T$ be a mapping.


Let $g: T \to S$ be a mapping such that:
:$f \circ g = I_T$
where:
:$f \circ g$ denotes the composite mapping $g$ followed by $f$
:$I_T$ is the identity mapping on $T$.


Then $g: T \to S$ is called a right inverse (mapping) of $f$.",Definition:Right Inverse Mapping,['Definitions/Inverse Mappings'],"Let S, T be sets where S Ø, that is, S is not empty.

Let f: S → T be a mapping.


Let g: T → S be a mapping such that:
:f ∘ g = I_T
where:
:f ∘ g denotes the composite mapping g followed by f
:I_T is the identity mapping on T.


Then g: T → S is called a right inverse (mapping) of f."
Definition:Right Inverse,Right Inverse,"Let $\struct {S, \circ}$ be a monoid whose identity is $e_S$.

An element $x_R \in S$ is called a right inverse of $x$ :
:$x \circ x_R = e_S$
",Definition:Inverse (Abstract Algebra)/Right Inverse,['Definitions/Inverse Elements'],"Let S, ∘ be a monoid whose identity is e_S.

An element x_R ∈ S is called a right inverse of x :
:x ∘ x_R = e_S
"
Definition:Right Inverse,Right Inverse,"
",Definition:Inverse Matrix/Right,['Definitions/Inverse Matrices'],"
"
Definition:Root,Root,"Let $\map E x$ be a mathematical expression representing an equation which is dependent upon a variable $x$.

A root of $\map E x$ is a constant which, when substituted for $x$ in $\map E x$, makes $\map E x$ a true statement.


=== Extraction of Root ===

The process of finding roots of a given equation is referred to as extraction.
",Definition:Root of Equation,"['Definitions/Roots of Equations', 'Definitions/Roots', 'Definitions/Algebra']","Let E x be a mathematical expression representing an equation which is dependent upon a variable x.

A root of E x is a constant which, when substituted for x in E x, makes E x a true statement.


=== Extraction of Root ===

The process of finding roots of a given equation is referred to as extraction.
"
Definition:Root,Root,"Let $f: R \to R$ be a mapping on a ring $R$.

Let $x \in R$.


Then the values of $x$ for which $\map f x = 0_R$ are known as the roots of the mapping $f$.
",Definition:Root of Mapping,"['Definitions/Roots of Mappings', 'Definitions/Ring Theory', 'Definitions/Field Theory', 'Definitions/Real Analysis', 'Definitions/Complex Analysis']","Let f: R → R be a mapping on a ring R.

Let x ∈ R.


Then the values of x for which f x = 0_R are known as the roots of the mapping f.
"
Definition:Root,Root,"Let $R$ be a commutative ring with unity.

Let $f \in R \sqbrk x$ be a polynomial over $R$.


A root in $R$ of $f$ is an element $x \in R$ for which $\map f x = 0$, where $\map f x$ denotes the image of $f$ under the evaluation homomorphism at $x$.",Definition:Root of Polynomial,"['Definitions/Roots of Polynomials', 'Definitions/Polynomial Theory', 'Definitions/Ring Theory']","Let R be a commutative ring with unity.

Let f ∈ R  x be a polynomial over R.


A root in R of f is an element x ∈ R for which f x = 0, where f x denotes the image of f under the evaluation homomorphism at x."
Definition:Root,Root,"Let $T$ be a rooted tree.

The root node of $T$ is the node of $T$ which is distinguished from the others by being the ancestor node of every node of $T$.
Let $T$ be a rooted tree.

The root node of $T$ is the node of $T$ which is distinguished from the others by being the ancestor node of every node of $T$.
",Definition:Rooted Tree,"['Definitions/Rooted Trees', 'Definitions/Graph Theory', 'Definitions/Tree Theory']","Let T be a rooted tree.

The root node of T is the node of T which is distinguished from the others by being the ancestor node of every node of T.
Let T be a rooted tree.

The root node of T is the node of T which is distinguished from the others by being the ancestor node of every node of T.
"
Definition:Root,Root,"Let $T$ be a rooted tree.

The root node of $T$ is the node of $T$ which is distinguished from the others by being the ancestor node of every node of $T$.
",Definition:Labeled Tree for Propositional Logic/Hypothesis Set,['Definitions/Propositional Tableaus'],"Let T be a rooted tree.

The root node of T is the node of T which is distinguished from the others by being the ancestor node of every node of T.
"
Definition:Row,Row,"A row of a truth table is one of the horizontal lines that consists of instances of the symbols $T$ and $F$.

Each row contains the truth values of each of the boolean interpretations of the statement forms according to the propositional variables that comprise them.

There are as many rows in a truth table as there are combinations of $T$ and $F$ for all the propositional variables that constitute the statement forms.",Definition:Truth Table/Row,['Definitions/Truth Tables'],"A row of a truth table is one of the horizontal lines that consists of instances of the symbols T and F.

Each row contains the truth values of each of the boolean interpretations of the statement forms according to the propositional variables that comprise them.

There are as many rows in a truth table as there are combinations of T and F for all the propositional variables that constitute the statement forms."
Definition:Row,Row,"A row matrix is a $1 \times n$ matrix:

:$\mathbf R = \begin{bmatrix}
r_{1 1} & r_{1 2} & \cdots & r_{1 n}
\end{bmatrix}$


That is, it is a matrix with only one row.
",Definition:Matrix/Row,['Definitions/Matrices'],"A row matrix is a 1 × n matrix:

:𝐑 = [ r_1 1 r_1 2     ⋯ r_1 n ]


That is, it is a matrix with only one row.
"
Definition:Row,Row,"Let $\mathbf L$ be a Latin square.

The rows of $\mathbf L$ are the lines of elements reading across the page.",Definition:Latin Square/Row,['Definitions/Latin Squares'],"Let 𝐋 be a Latin square.

The rows of 𝐋 are the lines of elements reading across the page."
Definition:Scalar Field,Scalar Field,"Let $\struct {G, +_G, \circ}_K$ be a vector space, where:

:$\struct {K, +_K, \times_K}$ is a field

:$\struct {G, +_G}$ is an abelian group $\struct {G, +_G}$

:$\circ: K \times G \to G$ is a binary operation.


Then the field $\struct {K, +_K, \times_K}$ is called the scalar field of $\struct {G, +_G, \circ}_K$.


If the scalar field is understood, then $\struct {G, +_G, \circ}_K$ can be rendered $\struct {G, +_G, \circ}$.
",Definition:Scalar Field (Linear Algebra),"['Definitions/Vector Algebra', 'Definitions/Linear Algebra']","Let G, +_G, ∘_K be a vector space, where:

:K, +_K, ×_K is a field

:G, +_G is an abelian group G, +_G

:∘: K × G → G is a binary operation.


Then the field K, +_K, ×_K is called the scalar field of G, +_G, ∘_K.


If the scalar field is understood, then G, +_G, ∘_K can be rendered G, +_G, ∘.
"
Definition:Scalar Field,Scalar Field,"Let $F$ be a field which acts on a region of space $S$.

Let the point-function giving rise to $F$ be a scalar quantity.


Then $F$ is a scalar field.
",Definition:Scalar Field (Physics),"['Definitions/Scalar Fields (Physics)', 'Definitions/Fields (Physics)', 'Definitions/Physics']","Let F be a field which acts on a region of space S.

Let the point-function giving rise to F be a scalar quantity.


Then F is a scalar field.
"
Definition:Section,Section,"Let a geometrical line $A$ cross over (or intersect) another line $B$.

The point where they cross is called the section of the $B$ by $A$ (or equivalently, of $A$ by $B$).


Category:Definitions/Geometry
Let $F$ be a $3$-dimensional figure.

A plane section of $F$ is the intersection of $F$ with a plane.
",Definition:Section (Geometry),['Definitions/Geometry'],"Let a geometrical line A cross over (or intersect) another line B.

The point where they cross is called the section of the B by A (or equivalently, of A by B).


Category:Definitions/Geometry
Let F be a 3-dimensional figure.

A plane section of F is the intersection of F with a plane.
"
Definition:Section,Section,"Let a geometrical line $A$ cross over (or intersect) another line $B$.

The point where they cross is called the section of the $B$ by $A$ (or equivalently, of $A$ by $B$).


Category:Definitions/Geometry",Definition:Section of Line by Line,['Definitions/Geometry'],"Let a geometrical line A cross over (or intersect) another line B.

The point where they cross is called the section of the B by A (or equivalently, of A by B).


Category:Definitions/Geometry"
Definition:Section,Section,"Let $F$ be a $3$-dimensional figure.

A plane section of $F$ is the intersection of $F$ with a plane.
The intersection of two lines $AB$ and $CD$ is denoted by $AB \cap CD$.

The intersection of two geometric figures is the set of points shared by both figures.


Note that this use of $\cap$ is consistent with that of its more usual context of set intersection.


When two lines intersect, they are said to cut each other.
",Definition:Plane Section,"['Definitions/Plane Sections', 'Definitions/Solid Geometry']","Let F be a 3-dimensional figure.

A plane section of F is the intersection of F with a plane.
The intersection of two lines AB and CD is denoted by AB ∩ CD.

The intersection of two geometric figures is the set of points shared by both figures.


Note that this use of ∩ is consistent with that of its more usual context of set intersection.


When two lines intersect, they are said to cut each other.
"
Definition:Section,Section,"Let $F$ be a $3$-dimensional figure.

A plane section of $F$ is the intersection of $F$ with a plane.
",Definition:Cross-Section,"['Definitions/Plane Sections', 'Definitions/Solid Geometry']","Let F be a 3-dimensional figure.

A plane section of F is the intersection of F with a plane.
"
Definition:Section,Section,"Let $M, E$ be topological spaces. 

Let $\pi: E \to M$ be a continuous surjection. 

Let $I_M: M \to M$ be the identity mapping on $M$. 


Then a section of $E$ is a continuous mapping $s: M \to E$ such that $\pi \circ s = I_M$.",Definition:Section (Topology),['Definitions/Topology'],"Let M, E be topological spaces. 

Let π: E → M be a continuous surjection. 

Let I_M: M → M be the identity mapping on M. 


Then a section of E is a continuous mapping s: M → E such that π∘ s = I_M."
Definition:Section,Section,"To bisect a finite geometrical object is to cut it in half, that is, into two equal parts.

The act of cutting in half is known as bisection.


=== Bisector ===
",Definition:Bisection,"['Definitions/Bisection', 'Definitions/Geometry']","To bisect a finite geometrical object is to cut it in half, that is, into two equal parts.

The act of cutting in half is known as bisection.


=== Bisector ===
"
Definition:Segment,Segment,"
:

That is, it is the angle the base makes with the circumference where they meet.


It can also be defined as the angle between the base and the tangent to the circle at the end of the base:

:


Category:Definitions/Segments of Circles

:

:


:

Such a segment is said to admit the angle specified.


Category:Definitions/Segments of Circles

:



Category:Definitions/Segments of Circles
Let $AB$ be a chord of a circle $\CC$ defined by the points $A$ and $B$ on the circumference of $\CC$.

The major segment of $\CC$  $AB$ is the segment between $AB$ and the major arc of $\CC$ between $A$ and $B$.
Let $AB$ be a chord of a circle $\CC$ defined by the points $A$ and $B$ on the circumference of $\CC$.

The minor segment of $\CC$  $AB$ is the segment between $AB$ and the minor arc of $\CC$ between $A$ and $B$.
",Definition:Segment of Circle,"['Definitions/Segments of Circles', 'Definitions/Circles']","
:

That is, it is the angle the base makes with the circumference where they meet.


It can also be defined as the angle between the base and the tangent to the circle at the end of the base:

:


Category:Definitions/Segments of Circles

:

:


:

Such a segment is said to admit the angle specified.


Category:Definitions/Segments of Circles

:



Category:Definitions/Segments of Circles
Let AB be a chord of a circle  defined by the points A and B on the circumference of .

The major segment of   AB is the segment between AB and the major arc of  between A and B.
Let AB be a chord of a circle  defined by the points A and B on the circumference of .

The minor segment of   AB is the segment between AB and the minor arc of  between A and B.
"
Definition:Segment,Segment,"Let $\struct {S, \preceq}$ be a well-ordered set.

Let $a \in S$.


The initial segment (of $S$) determined by $a$ is defined as:

:$S_a := \set {b \in S: b \preceq a \land b \ne a}$

which can also be rendered as:

:$S_a := \set {b \in S: b \prec a}$


That is, $S_a$ is the set of all elements of $S$ that strictly precede $a$.

That is, $S_a$ is the strict lower closure of $a$ (in $S$).


By extension, $S_a$ is described as an initial segment (of $S$).


=== Class Theoretical Definition ===

",Definition:Initial Segment,"['Definitions/Initial Segments', 'Definitions/Order Theory']","Let S, ≼ be a well-ordered set.

Let a ∈ S.


The initial segment (of S) determined by a is defined as:

:S_a := b ∈ S: b ≼ a  b  a

which can also be rendered as:

:S_a := b ∈ S: b ≺ a


That is, S_a is the set of all elements of S that strictly precede a.

That is, S_a is the strict lower closure of a (in S).


By extension, S_a is described as an initial segment (of S).


=== Class Theoretical Definition ===

"
Definition:Segment,Segment,"Let $\struct {S, \preccurlyeq}$ be an ordered set.

Let $a \in S$.


The lower closure of $a$ (in $S$) is defined as:

:$a^\preccurlyeq := \set {b \in S: b \preccurlyeq a}$


That is, $a^\preccurlyeq$ is the set of all elements of $S$ that precede $a$.


=== Class Theory ===

",Definition:Lower Closure/Element,['Definitions/Lower Closures'],"Let S, ≼ be an ordered set.

Let a ∈ S.


The lower closure of a (in S) is defined as:

:a^≼ := b ∈ S: b ≼ a


That is, a^≼ is the set of all elements of S that precede a.


=== Class Theory ===

"
Definition:Segment,Segment,"Let $\mathbf A$ be a matrix with $m$ rows and $n$ columns.


A submatrix of $\mathbf A$ is a matrix formed by selecting from $\mathbf A$:
:a subset of the rows
and:
:a subset of the columns
and forming a new matrix by using those entries, in the same relative positions, that appear in both the rows and columns of those selected.",Definition:Submatrix,['Definitions/Matrix Theory'],"Let 𝐀 be a matrix with m rows and n columns.


A submatrix of 𝐀 is a matrix formed by selecting from 𝐀:
:a subset of the rows
and:
:a subset of the columns
and forming a new matrix by using those entries, in the same relative positions, that appear in both the rows and columns of those selected."
Definition:Separable,Separable,"A topological space $T = \struct {S, \tau}$ is separable  there exists a countable subset of $S$ which is everywhere dense in $T$.


=== Normed Vector Space ===


",Definition:Separable Space,"['Definitions/Countability Axioms', 'Definitions/Separable Spaces']","A topological space T = S, τ is separable  there exists a countable subset of S which is everywhere dense in T.


=== Normed Vector Space ===


"
Definition:Separable,Separable,"A topological space $T = \struct {S, \tau}$ is second-countable or satisfies the Second Axiom of Countability  its topology has a countable basis.",Definition:Second-Countable Space,['Definitions/Countability Axioms'],"A topological space T = S, τ is second-countable or satisfies the Second Axiom of Countability  its topology has a countable basis."
Definition:Separable,Separable,"A first order ordinary differential equation which can be expressed in the form:
:$\dfrac {\d y} {\d x} = \map g x \map h y$
is known as a separable differential equation.


Its general solution is found by solving the integration:
:$\ds \int \frac {\d y} {\map h y} = \int \map g x \rd x + C$


=== General Form ===

",Definition:Separable Differential Equation,"['Definitions/Separable Differential Equations', 'Definitions/Ordinary Differential Equations']","A first order ordinary differential equation which can be expressed in the form:
:ỵx̣ =  g x  h y
is known as a separable differential equation.


Its general solution is found by solving the integration:
:∫ỵ/ h y = ∫ g x  x + C


=== General Form ===

"
Definition:Separated,Separated,"Let $\struct {S, \tau}$ be a topological space.

Let $x, y \in S$ such that both of the following hold:

:$\exists U \in \tau: x \in U, y \notin U$
:$\exists V \in \tau: y \in V, x \notin V$


Then $x$ and $y$ are separated points.
",Definition:Separated Points,['Definitions/Topology'],"Let S, τ be a topological space.

Let x, y ∈ S such that both of the following hold:

:∃ U ∈τ: x ∈ U, y ∉ U
:∃ V ∈τ: y ∈ V, x ∉ V


Then x and y are separated points.
"
Definition:Separated,Separated,"Let $T = \struct {S, \tau}$ be a topological space.


Let $A, B \subseteq S$ such that:
:$\exists N_A, N_B \subseteq S: \exists U, V \in \tau: A \subseteq U \subseteq N_A, B \subseteq V \subseteq N_B: N_A \cap N_B = \O$


That is, that $A$ and $B$ both have neighborhoods in $T$ which are disjoint.


Then $A$ and $B$ are described as separated by neighborhoods.


Category:Definitions/Separated by Neighborhoods
",Definition:Points Separated by Neighborhoods/Neighborhoods,['Definitions/Separated by Neighborhoods'],"Let T = S, τ be a topological space.


Let A, B ⊆ S such that:
:∃ N_A, N_B ⊆ S: ∃ U, V ∈τ: A ⊆ U ⊆ N_A, B ⊆ V ⊆ N_B: N_A ∩ N_B = Ø


That is, that A and B both have neighborhoods in T which are disjoint.


Then A and B are described as separated by neighborhoods.


Category:Definitions/Separated by Neighborhoods
"
Definition:Separated,Separated,"Let $T = \struct {S, \tau}$ be a topological space.


Let $A, B \subseteq S$ such that:
:$\exists N_A, N_B \subseteq S: \exists U, V \in \tau: A \subseteq U \subseteq N_A, B \subseteq V \subseteq N_B: N_A \cap N_B = \O$


That is, that $A$ and $B$ both have neighborhoods in $T$ which are disjoint.


Then $A$ and $B$ are described as separated by neighborhoods.


Category:Definitions/Separated by Neighborhoods",Definition:Sets Separated by Neighborhoods/Neighborhoods,['Definitions/Separated by Neighborhoods'],"Let T = S, τ be a topological space.


Let A, B ⊆ S such that:
:∃ N_A, N_B ⊆ S: ∃ U, V ∈τ: A ⊆ U ⊆ N_A, B ⊆ V ⊆ N_B: N_A ∩ N_B = Ø


That is, that A and B both have neighborhoods in T which are disjoint.


Then A and B are described as separated by neighborhoods.


Category:Definitions/Separated by Neighborhoods"
Definition:Separated,Separated,"Let $T = \struct {S, \tau}$ be a topological space.


Let $x, y \in S$ such that:

:$\exists N_x, N_y \subseteq S: \exists U, V \in \tau: x \subseteq U \subseteq N_x, y \subseteq V \subseteq N_y: N_x^- \cap N_y^- = \O$

where $N_x^-$ and $N_y^-$ are the closures in $T$ of $N_x$ and $N_y$ respectively.


That is, that $x$ and $y$ both have neighborhoods in $T$ whose closures are disjoint.


Then $x$ and $y$ are described as separated by closed neighborhoods.


Thus two points are separated by closed neighborhoods $x$ and $y$  the two singleton sets $\set x$ and $\set y$ are separated (as sets) by closed neighborhoods.
Let $T = \struct {S, \tau}$ be a topological space.


Let $x, y \in S$ such that:

:$\exists N_x, N_y \subseteq S: \exists U, V \in \tau: x \subseteq U \subseteq N_x, y \subseteq V \subseteq N_y: N_x^- \cap N_y^- = \O$

where $N_x^-$ and $N_y^-$ are the closures in $T$ of $N_x$ and $N_y$ respectively.


That is, that $x$ and $y$ both have neighborhoods in $T$ whose closures are disjoint.


Then $x$ and $y$ are described as separated by closed neighborhoods.


Thus two points are separated by closed neighborhoods $x$ and $y$  the two singleton sets $\set x$ and $\set y$ are separated (as sets) by closed neighborhoods.
Let $T = \struct {S, \tau}$ be a topological space.


Let $A, B \subseteq S$ such that:
:$\exists N_A, N_B \subseteq S: \exists U, V \in \tau: A \subseteq U \subseteq N_A, B \subseteq V \subseteq N_B: N_A^- \cap N_B^- = \O$
where $N_A^-$ and $N_B^-$ are the closures in $T$ of $N_A$ and $N_B$ respectively.

That is, that $A$ and $B$ both have neighborhoods in $T$ whose closures are disjoint.


Then $A$ and $B$ are described as separated by closed neighborhoods.


Category:Definitions/Separated by Closed Neighborhoods
",Definition:Separated by Closed Neighborhoods/Points,['Definitions/Separated by Closed Neighborhoods'],"Let T = S, τ be a topological space.


Let x, y ∈ S such that:

:∃ N_x, N_y ⊆ S: ∃ U, V ∈τ: x ⊆ U ⊆ N_x, y ⊆ V ⊆ N_y: N_x^- ∩ N_y^- = Ø

where N_x^- and N_y^- are the closures in T of N_x and N_y respectively.


That is, that x and y both have neighborhoods in T whose closures are disjoint.


Then x and y are described as separated by closed neighborhoods.


Thus two points are separated by closed neighborhoods x and y  the two singleton sets x and y are separated (as sets) by closed neighborhoods.
Let T = S, τ be a topological space.


Let x, y ∈ S such that:

:∃ N_x, N_y ⊆ S: ∃ U, V ∈τ: x ⊆ U ⊆ N_x, y ⊆ V ⊆ N_y: N_x^- ∩ N_y^- = Ø

where N_x^- and N_y^- are the closures in T of N_x and N_y respectively.


That is, that x and y both have neighborhoods in T whose closures are disjoint.


Then x and y are described as separated by closed neighborhoods.


Thus two points are separated by closed neighborhoods x and y  the two singleton sets x and y are separated (as sets) by closed neighborhoods.
Let T = S, τ be a topological space.


Let A, B ⊆ S such that:
:∃ N_A, N_B ⊆ S: ∃ U, V ∈τ: A ⊆ U ⊆ N_A, B ⊆ V ⊆ N_B: N_A^- ∩ N_B^- = Ø
where N_A^- and N_B^- are the closures in T of N_A and N_B respectively.

That is, that A and B both have neighborhoods in T whose closures are disjoint.


Then A and B are described as separated by closed neighborhoods.


Category:Definitions/Separated by Closed Neighborhoods
"
Definition:Separated,Separated,"Let $T = \struct {S, \tau}$ be a topological space.


Let $A, B \subseteq S$ such that:
:$\exists N_A, N_B \subseteq S: \exists U, V \in \tau: A \subseteq U \subseteq N_A, B \subseteq V \subseteq N_B: N_A^- \cap N_B^- = \O$
where $N_A^-$ and $N_B^-$ are the closures in $T$ of $N_A$ and $N_B$ respectively.

That is, that $A$ and $B$ both have neighborhoods in $T$ whose closures are disjoint.


Then $A$ and $B$ are described as separated by closed neighborhoods.


Category:Definitions/Separated by Closed Neighborhoods
",Definition:Separated by Closed Neighborhoods/Sets,['Definitions/Separated by Closed Neighborhoods'],"Let T = S, τ be a topological space.


Let A, B ⊆ S such that:
:∃ N_A, N_B ⊆ S: ∃ U, V ∈τ: A ⊆ U ⊆ N_A, B ⊆ V ⊆ N_B: N_A^- ∩ N_B^- = Ø
where N_A^- and N_B^- are the closures in T of N_A and N_B respectively.

That is, that A and B both have neighborhoods in T whose closures are disjoint.


Then A and B are described as separated by closed neighborhoods.


Category:Definitions/Separated by Closed Neighborhoods
"
Definition:Side,Side,":

The line segments which make up a polygon are known as its sides.

Thus, in the polygon above, the sides are identified as $a, b, c, d$ and $e$.
:

The line segments which make up a polygon are known as its sides.

Thus, in the polygon above, the sides are identified as $a, b, c, d$ and $e$.
",Definition:Polygon/Side,"['Definitions/Sides of Polygons', 'Definitions/Polygons']",":

The line segments which make up a polygon are known as its sides.

Thus, in the polygon above, the sides are identified as a, b, c, d and e.
:

The line segments which make up a polygon are known as its sides.

Thus, in the polygon above, the sides are identified as a, b, c, d and e.
"
Definition:Side,Side,"From the definition of surface, it follows that a plane locally separates space into two sides.

Thus the sides of a plane are the parts of that space into which the plane separates it.


Category:Definitions/Surfaces",Definition:Plane Surface/Side,['Definitions/Surfaces'],"From the definition of surface, it follows that a plane locally separates space into two sides.

Thus the sides of a plane are the parts of that space into which the plane separates it.


Category:Definitions/Surfaces"
Definition:Side,Side,"Let $S$ be a surface.

By definition, $S$ locally separates space into two sides.

Thus the sides of $S$ are the parts of that space into which $S$ separates it.


Category:Definitions/Surfaces",Definition:Side of Surface,['Definitions/Surfaces'],"Let S be a surface.

By definition, S locally separates space into two sides.

Thus the sides of S are the parts of that space into which S separates it.


Category:Definitions/Surfaces"
Definition:Side,Side,"The side of a plane number is one of the (natural) numbers which are its divisors.


=== Example ===


Category:Definitions/Euclidean Number Theory",Definition:Plane Number/Side,['Definitions/Euclidean Number Theory'],"The side of a plane number is one of the (natural) numbers which are its divisors.


=== Example ===


Category:Definitions/Euclidean Number Theory"
Definition:Side,Side,"The side of a solid number is one of the (natural) numbers which are its divisors.


=== Example ===


Category:Definitions/Euclidean Number Theory",Definition:Solid Number/Side,['Definitions/Euclidean Number Theory'],"The side of a solid number is one of the (natural) numbers which are its divisors.


=== Example ===


Category:Definitions/Euclidean Number Theory"
Definition:Sign,Sign,,Definition:Formal Language/Alphabet/Sign,['Definitions/Alphabets (Formal Language)'],
Definition:Sign,Sign,"Let $n \in \N$ be a natural number.

Let $\N_n$ denote the set of natural numbers $\set {1, 2, \ldots, n}$.

Let $\tuple {x_1, x_2, \ldots, x_n}$ be an ordered $n$-tuple of real numbers.

Let $\pi$ be a permutation of $\N_n$.

Let $\map {\Delta_n} {x_1, x_2, \ldots, x_n}$ be the product of differences of $\tuple {x_1, x_2, \ldots, x_n}$.

Let $\pi \cdot \map {\Delta_n} {x_1, x_2, \ldots, x_n}$ be defined as:

:$\pi \cdot \map {\Delta_n} {x_1, x_2, \ldots, x_n} := \map {\Delta_n} {x_{\map \pi 1}, x_{\map \pi 2}, \ldots, x_{\map \pi n} }$


The sign of $\pi \in S_n$ is defined as:

:$\map \sgn \pi = \begin {cases}
\dfrac {\Delta_n} {\pi \cdot \Delta_n} & : \Delta_n \ne 0 \\
0 & : \Delta_n = 0 \end {cases}$",Definition:Sign of Permutation,"['Definitions/Sign of Permutation', 'Definitions/Permutation Theory', 'Definitions/Algebra']","Let n ∈ be a natural number.

Let _n denote the set of natural numbers 1, 2, …, n.

Let x_1, x_2, …, x_n be an ordered n-tuple of real numbers.

Let π be a permutation of _n.

Let Δ_nx_1, x_2, …, x_n be the product of differences of x_1, x_2, …, x_n.

Let π·Δ_nx_1, x_2, …, x_n be defined as:

:π·Δ_nx_1, x_2, …, x_n := Δ_nx_π 1, x_π 2, …, x_π n


The sign of π∈ S_n is defined as:

:π = Δ_nπ·Δ_n    : Δ_n  0 

0     : Δ_n = 0"
Definition:Sign,Sign,"In the context of arithmetic and algebra, the term sign is used to mean one of the operators:

* Addition: $+$
* Subtraction: $-$
* Multiplication: $\times$
* Division: $\div$

It can also be used to describe a general operator in the context of abstract algebra: $\circ$ and so on.
",Definition:Sign (Arithmetic),"['Definitions/Arithmetic', 'Definitions/Algebra', 'Definitions/Abstract Algebra']","In the context of arithmetic and algebra, the term sign is used to mean one of the operators:

* Addition: +
* Subtraction: -
* Multiplication: ×
* Division: ÷

It can also be used to describe a general operator in the context of abstract algebra: ∘ and so on.
"
Definition:Sign,Sign,"The sign of a number is the symbol indicating whether it is:
: positive, denoted by the symbol $+$
or:
: negative, denoted by the symbol $-$.


Hence a number's sign has evolved to define the fact of the number being positive or negative independently of the symbol itself.


Thus:
:the sign of $3.14159$ is positive
and
:the sign of $-75$ is negative.",Definition:Sign of Number,['Definitions/Numbers'],"The sign of a number is the symbol indicating whether it is:
: positive, denoted by the symbol +
or:
: negative, denoted by the symbol -.


Hence a number's sign has evolved to define the fact of the number being positive or negative independently of the symbol itself.


Thus:
:the sign of 3.14159 is positive
and
:the sign of -75 is negative."
Definition:Signature,Signature,"Let $\LL_1$ be the language of predicate logic.


Then a signature for $\LL_1$ is an explicit choice of the alphabet of $\LL_1$.

That is to say, it amounts to choosing, for each $n \in \N$:

:A collection $\FF_n$ of $n$-ary function symbols
:A collection $\PP_n$ of $n$-ary relation symbols

It is often conceptually enlightening to explicitly address the $0$-ary function symbols separately, as constant symbols.
",Definition:Signature (Logic),['Definitions/Formal Languages'],"Let _1 be the language of predicate logic.


Then a signature for _1 is an explicit choice of the alphabet of _1.

That is to say, it amounts to choosing, for each n ∈:

:A collection _n of n-ary function symbols
:A collection _n of n-ary relation symbols

It is often conceptually enlightening to explicitly address the 0-ary function symbols separately, as constant symbols.
"
Definition:Signature,Signature,"Let $n \in \N$ be a natural number.

Let $\N_n$ denote the set of natural numbers $\set {1, 2, \ldots, n}$.

Let $\tuple {x_1, x_2, \ldots, x_n}$ be an ordered $n$-tuple of real numbers.

Let $\pi$ be a permutation of $\N_n$.

Let $\map {\Delta_n} {x_1, x_2, \ldots, x_n}$ be the product of differences of $\tuple {x_1, x_2, \ldots, x_n}$.

Let $\pi \cdot \map {\Delta_n} {x_1, x_2, \ldots, x_n}$ be defined as:

:$\pi \cdot \map {\Delta_n} {x_1, x_2, \ldots, x_n} := \map {\Delta_n} {x_{\map \pi 1}, x_{\map \pi 2}, \ldots, x_{\map \pi n} }$


The sign of $\pi \in S_n$ is defined as:

:$\map \sgn \pi = \begin {cases}
\dfrac {\Delta_n} {\pi \cdot \Delta_n} & : \Delta_n \ne 0 \\
0 & : \Delta_n = 0 \end {cases}$",Definition:Sign of Permutation,"['Definitions/Sign of Permutation', 'Definitions/Permutation Theory', 'Definitions/Algebra']","Let n ∈ be a natural number.

Let _n denote the set of natural numbers 1, 2, …, n.

Let x_1, x_2, …, x_n be an ordered n-tuple of real numbers.

Let π be a permutation of _n.

Let Δ_nx_1, x_2, …, x_n be the product of differences of x_1, x_2, …, x_n.

Let π·Δ_nx_1, x_2, …, x_n be defined as:

:π·Δ_nx_1, x_2, …, x_n := Δ_nx_π 1, x_π 2, …, x_π n


The sign of π∈ S_n is defined as:

:π = Δ_nπ·Δ_n    : Δ_n  0 

0     : Δ_n = 0"
Definition:Similar,Similar,"Let $G$ be a vector space over a field $K$.

Let $\beta \in K$.

Let $s_\beta: G \to G$ be the mapping on $G$ defined as:
:$\forall \mathbf x \in G: \map {s_\beta} {\mathbf x} = \beta \mathbf x$


$s_\beta$ is called a similarity (mapping).


=== Scale Factor ===

",Definition:Similar Figures,['Definitions/Euclidean Geometry'],"Let G be a vector space over a field K.

Let β∈ K.

Let s_β: G → G be the mapping on G defined as:
:∀𝐱∈ G: s_β𝐱 = β𝐱


s_β is called a similarity (mapping).


=== Scale Factor ===

"
Definition:Similar,Similar,"Similar triangles are triangles whose corresponding angles are the same, but whose corresponding sides may be of different lengths.

:

Thus $\triangle ABC$ is similar to $\triangle DEF$:
:$\angle ABC = \angle EFD$
:$\angle BCA = \angle EDF$
:$\angle CAB = \angle DEF$
Similar triangles are triangles whose corresponding angles are the same, but whose corresponding sides may be of different lengths.

:

Thus $\triangle ABC$ is similar to $\triangle DEF$:
:$\angle ABC = \angle EFD$
:$\angle BCA = \angle EDF$
:$\angle CAB = \angle DEF$
",Definition:Similar Triangles,['Definitions/Triangles'],"Similar triangles are triangles whose corresponding angles are the same, but whose corresponding sides may be of different lengths.

:

Thus ABC is similar to DEF:
:∠ ABC = ∠ EFD
:∠ BCA = ∠ EDF
:∠ CAB = ∠ DEF
Similar triangles are triangles whose corresponding angles are the same, but whose corresponding sides may be of different lengths.

:

Thus ABC is similar to DEF:
:∠ ABC = ∠ EFD
:∠ BCA = ∠ EDF
:∠ CAB = ∠ DEF
"
Definition:Similar,Similar,"
:



Category:Definitions/Segments of Circles",Definition:Segment of Circle/Similar,['Definitions/Segments of Circles'],"
:



Category:Definitions/Segments of Circles"
Definition:Similar,Similar,"



Category:Definitions/Solid Geometry",Definition:Similar Solid Figures,['Definitions/Solid Geometry'],"



Category:Definitions/Solid Geometry"
Definition:Similar,Similar,"Let $h_1$ and $h_2$ be the lengths of the axes of two right circular cones.

Let $d_1$ and $d_2$ be the lengths of the diameters of the bases of the two right circular cones.

Then the two right circular cones are similar :

:$\dfrac {h_1} {h_2} = \dfrac {d_1} {d_2}$



:



Category:Definitions/Right Circular Cones",Definition:Right Circular Cone/Similar Cones,['Definitions/Right Circular Cones'],"Let h_1 and h_2 be the lengths of the axes of two right circular cones.

Let d_1 and d_2 be the lengths of the diameters of the bases of the two right circular cones.

Then the two right circular cones are similar :

:h_1h_2 = d_1d_2



:



Category:Definitions/Right Circular Cones"
Definition:Similar,Similar,"Let $h_1$ and $h_2$ be the heights of two cylinders.

Let $d_1$ and $d_2$ be the diameters of the bases of the two cylinders.

Then the two cylinders are similar :

:$\dfrac {h_1} {h_2} = \dfrac {d_1} {d_2}$



:



Category:Definitions/Cylinders",Definition:Cylinder/Similar Cylinders,['Definitions/Cylinders'],"Let h_1 and h_2 be the heights of two cylinders.

Let d_1 and d_2 be the diameters of the bases of the two cylinders.

Then the two cylinders are similar :

:h_1h_2 = d_1d_2



:



Category:Definitions/Cylinders"
Definition:Similar,Similar,"Two rectilineal figures are similar :
:They have corresponding angles, all of which are equal
:They have corresponding sides, all of which are proportional.


=== Informal Definition ===

Two geometric figures are similar if they have the same shape but not necessarily the same size.

It is intuitively understood what it means for two figures to have the same shape.


=== Algebraic Definition ===

Two geometric figures are similar if one can be transformed into the other by means of a similarity mapping.


=== Euclid's Definition ===


:

",Definition:Similar Planes,['Definitions/Euclidean Geometry'],"Two rectilineal figures are similar :
:They have corresponding angles, all of which are equal
:They have corresponding sides, all of which are proportional.


=== Informal Definition ===

Two geometric figures are similar if they have the same shape but not necessarily the same size.

It is intuitively understood what it means for two figures to have the same shape.


=== Algebraic Definition ===

Two geometric figures are similar if one can be transformed into the other by means of a similarity mapping.


=== Euclid's Definition ===


:

"
Definition:Similar,Similar,"

Category:Definitions/Angles
Category:Definitions/Solid Geometry",Definition:Similar Inclination,"['Definitions/Angles', 'Definitions/Solid Geometry']","

Category:Definitions/Angles
Category:Definitions/Solid Geometry"
Definition:Similar,Similar,"



Category:Definitions/Solid Geometry


Category:Definitions/Angles
Category:Definitions/Solid Geometry
",Definition:Similar Situation,['Definitions/Solid Geometry'],"



Category:Definitions/Solid Geometry


Category:Definitions/Angles
Category:Definitions/Solid Geometry
"
Definition:Similar,Similar,"Let $m$ and $n$ be plane numbers.

Let:
:$m = p_1 \times p_2$ where $p_1 \le p_2$
:$n = q_1 \times q_2$ where $q_1 \le q_2$

Then $m$ and $n$ are similar :
:$p_1 : q_1 = p_2 : q_2$


That is:
:$\dfrac {p_1} {q_1} = \dfrac {p_2} {q_2}$



:


Category:Definitions/Euclidean Number Theory",Definition:Plane Number/Similar Numbers,['Definitions/Euclidean Number Theory'],"Let m and n be plane numbers.

Let:
:m = p_1 × p_2 where p_1 ≤ p_2
:n = q_1 × q_2 where q_1 ≤ q_2

Then m and n are similar :
:p_1 : q_1 = p_2 : q_2


That is:
:p_1q_1 = p_2q_2



:


Category:Definitions/Euclidean Number Theory"
Definition:Similar,Similar,"Let $m$ and $n$ be solid numbers.

Let:
: $m = p_1 \times p_2 \times p_3$ where $p_1 \le p_2 \le p_3$
: $n = q_1 \times q_2 \times q_3$ where $q_1 \le q_2 \le q_3$

Then $m$ and $n$ are similar :
:$p_1 : q_1 = p_2 : q_2 = p_3 : q_3$



:


Category:Definitions/Euclidean Number Theory",Definition:Solid Number/Similar Numbers,['Definitions/Euclidean Number Theory'],"Let m and n be solid numbers.

Let:
: m = p_1 × p_2 × p_3 where p_1 ≤ p_2 ≤ p_3
: n = q_1 × q_2 × q_3 where q_1 ≤ q_2 ≤ q_3

Then m and n are similar :
:p_1 : q_1 = p_2 : q_2 = p_3 : q_3



:


Category:Definitions/Euclidean Number Theory"
Definition:Similar,Similar,,Definition:Similarity,[],
Definition:Similar,Similar,"Let $G$ be a vector space over a field $K$.

Let $\beta \in K$.

Let $s_\beta: G \to G$ be the mapping on $G$ defined as:
:$\forall \mathbf x \in G: \map {s_\beta} {\mathbf x} = \beta \mathbf x$


$s_\beta$ is called a similarity (mapping).


=== Scale Factor ===
",Definition:Similarity Mapping,"['Definitions/Linear Algebra', 'Definitions/Similarity Mappings']","Let G be a vector space over a field K.

Let β∈ K.

Let s_β: G → G be the mapping on G defined as:
:∀𝐱∈ G: s_β𝐱 = β𝐱


s_β is called a similarity (mapping).


=== Scale Factor ===
"
Definition:Similar,Similar,"Let $S$ and $T$ be sets.

Then $S$ and $T$ are equivalent :
:there exists a bijection $f: S \to T$ between the elements of $S$ and those of $T$.

That is,  they have the same cardinality.


This can be written $S \sim T$.


If $S$ and $T$ are not equivalent we write $S \nsim T$.",Definition:Set Equivalence,"['Definitions/Set Equivalence', 'Definitions/Set Theory']","Let S and T be sets.

Then S and T are equivalent :
:there exists a bijection f: S → T between the elements of S and those of T.

That is,  they have the same cardinality.


This can be written S ∼ T.


If S and T are not equivalent we write S  T."
Definition:Similarity,Similarity,"Let $G$ be a vector space over a field $K$.

Let $\beta \in K$.

Let $s_\beta: G \to G$ be the mapping on $G$ defined as:
:$\forall \mathbf x \in G: \map {s_\beta} {\mathbf x} = \beta \mathbf x$


$s_\beta$ is called a similarity (mapping).


=== Scale Factor ===
",Definition:Similarity Mapping,"['Definitions/Linear Algebra', 'Definitions/Similarity Mappings']","Let G be a vector space over a field K.

Let β∈ K.

Let s_β: G → G be the mapping on G defined as:
:∀𝐱∈ G: s_β𝐱 = β𝐱


s_β is called a similarity (mapping).


=== Scale Factor ===
"
Definition:Similarity,Similarity,"Let $S$ be a geometric object.

$S$ has the property of self-similarity :
:every point of $S$ is contained in a copy of $S$ at a smaller scale.



",Definition:Self-Similarity,['Definitions/Fractals'],"Let S be a geometric object.

S has the property of self-similarity :
:every point of S is contained in a copy of S at a smaller scale.



"
Definition:Similarity,Similarity,"Let $S$ and $T$ be sets.

Then $S$ and $T$ are equivalent :
:there exists a bijection $f: S \to T$ between the elements of $S$ and those of $T$.

That is,  they have the same cardinality.


This can be written $S \sim T$.


If $S$ and $T$ are not equivalent we write $S \nsim T$.",Definition:Set Equivalence,"['Definitions/Set Equivalence', 'Definitions/Set Theory']","Let S and T be sets.

Then S and T are equivalent :
:there exists a bijection f: S → T between the elements of S and those of T.

That is,  they have the same cardinality.


This can be written S ∼ T.


If S and T are not equivalent we write S  T."
Definition:Simple,Simple,"A group $G$ is simple  it has only $G$ and the trivial group as normal subgroups.

That is,  the composition length of $G$ is $1$.",Definition:Simple Group,['Definitions/Normality in Groups'],"A group G is simple  it has only G and the trivial group as normal subgroups.

That is,  the composition length of G is 1."
Definition:Simple,Simple,"Let $E / F$ be a field extension.


Then $E$ is a simple extension over $F$ :
:$\exists \alpha \in E: E = F \sqbrk \alpha$
where $F \sqbrk \alpha$ is the field extension generated by $\alpha$.",Definition:Simple Field Extension,['Definitions/Field Extensions'],"Let E / F be a field extension.


Then E is a simple extension over F :
:∃α∈ E: E = F α
where F α is the field extension generated by α."
Definition:Simple,Simple,"Let $\R$ be the set of real numbers.

Let $n \ge 0$ be a natural number.


A simple finite continued fraction of length $n$ is a finite continued fraction in $\R$ of length $n$ whose partial denominators are integers that are strictly positive, except perhaps the first.

That is, it is a finite sequence $a: \closedint 0 n \to \Z$ with $a_n > 0$ for $n > 0$.
Let $\R$ be the field of real numbers.


A simple infinite continued fraction is a infinite continued fraction in $\R$ whose partial denominators are integers that are strictly positive, except perhaps the first.

That is, it is a sequence $a: \N_{\ge 0} \to \Z$ with $a_n > 0$ for $n > 0$.
",Definition:Continued Fraction/Simple,"['Definitions/Simple Continued Fractions', 'Definitions/Continued Fractions']","Let  be the set of real numbers.

Let n ≥ 0 be a natural number.


A simple finite continued fraction of length n is a finite continued fraction in  of length n whose partial denominators are integers that are strictly positive, except perhaps the first.

That is, it is a finite sequence a:  0 n → with a_n > 0 for n > 0.
Let  be the field of real numbers.


A simple infinite continued fraction is a infinite continued fraction in  whose partial denominators are integers that are strictly positive, except perhaps the first.

That is, it is a sequence a: _≥ 0→ with a_n > 0 for n > 0.
"
Definition:Simple,Simple,"Let $\R$ be the set of real numbers.

Let $n \ge 0$ be a natural number.


A simple finite continued fraction of length $n$ is a finite continued fraction in $\R$ of length $n$ whose partial denominators are integers that are strictly positive, except perhaps the first.

That is, it is a finite sequence $a: \closedint 0 n \to \Z$ with $a_n > 0$ for $n > 0$.
",Definition:Continued Fraction/Simple/Finite,['Definitions/Simple Continued Fractions'],"Let  be the set of real numbers.

Let n ≥ 0 be a natural number.


A simple finite continued fraction of length n is a finite continued fraction in  of length n whose partial denominators are integers that are strictly positive, except perhaps the first.

That is, it is a finite sequence a:  0 n → with a_n > 0 for n > 0.
"
Definition:Simple,Simple,"Let $\R^n$ be a real cartesian space of $n$ dimensions.

Let $C_1, \ldots, C_n$ be directed smooth curves in $\R^n$.

Let $C_i$ be parameterized by the smooth path $\rho_i: \closedint {a_i} {b_i} \to \R^n$ for all $i \in \set {1, 2, \ldots, n}$.

Let $C$ be the contour in $\R^n$ defined by the finite sequence $C_1, \ldots, C_n$.


$C$ is a simple contour :

:$(1): \quad$ For all $i, j \in \set {1, \ldots, n}, t_1 \in \hointr {a_i} {b_i}, t_2 \in \hointr {a_j} {b_j}$ with $t_1 \ne t_2$, we have $\map {\rho_i} {t_1} \ne \map {\rho_j} {t_2}$

:$(2): \quad$ For all $k \in \set {1, \ldots, n}, t \in \hointr {a_k} {b_k}$ where either $k \ne 1$ or $t \ne a_1$, we have $\map {\rho_k} t \ne \map {\rho_n} {b_n}$.


Thus a simple contour is a contour that does not intersect itself.


=== Complex Plane ===

The definition carries over to the complex plane, in which context it is usually applied:

",Definition:Contour/Simple,['Definitions/Vector Analysis'],"Let ^n be a real cartesian space of n dimensions.

Let C_1, …, C_n be directed smooth curves in ^n.

Let C_i be parameterized by the smooth path ρ_i: a_ib_i→^n for all i ∈1, 2, …, n.

Let C be the contour in ^n defined by the finite sequence C_1, …, C_n.


C is a simple contour :

:(1): For all i, j ∈1, …, n, t_1 ∈a_ib_i, t_2 ∈a_jb_j with t_1  t_2, we have ρ_it_1ρ_jt_2

:(2): For all k ∈1, …, n, t ∈a_kb_k where either k  1 or t  a_1, we have ρ_k t ρ_nb_n.


Thus a simple contour is a contour that does not intersect itself.


=== Complex Plane ===

The definition carries over to the complex plane, in which context it is usually applied:

"
Definition:Simple,Simple,"A simple graph is a graph which is:

:An undirected graph, that is, the edges are defined as doubleton sets of vertices and not ordered pairs

:Not a multigraph, that is, there is no more than one edge between each pair of vertices

:Not a loop-graph, that is, there are no loops, that is, edges which start and end at the same vertex

:Not a weighted graph, that is, the edges are not mapped to a number.


=== Formal Definition ===

",Definition:Simple Graph,"['Definitions/Simple Graphs', 'Definitions/Graph Theory']","A simple graph is a graph which is:

:An undirected graph, that is, the edges are defined as doubleton sets of vertices and not ordered pairs

:Not a multigraph, that is, there is no more than one edge between each pair of vertices

:Not a loop-graph, that is, there are no loops, that is, edges which start and end at the same vertex

:Not a weighted graph, that is, the edges are not mapped to a number.


=== Formal Definition ===

"
Definition:Simple,Simple,"Let $D = \struct {V, E}$ be a digraph.

If the relation $E$ in $D$ is also specifically asymmetric, then $D$ is called a simple digraph.

That is, in a simple digraph there are no pairs of arcs (like there are between $v_1$ and $v_4$ in the diagram above) which go in both directions between two vertices.
",Definition:Digraph/Simple Digraph,['Definitions/Digraphs'],"Let D = V, E be a digraph.

If the relation E in D is also specifically asymmetric, then D is called a simple digraph.

That is, in a simple digraph there are no pairs of arcs (like there are between v_1 and v_4 in the diagram above) which go in both directions between two vertices.
"
Definition:Simple,Simple,"Let $G = \struct {V, E}$ be a multigraph.


A simple edge is an edge $u v$ of $G$ which is the only edge of $G$ which is incident to both $u$ and $v$.


Category:Definitions/Edges of Graphs
Category:Definitions/Multigraphs",Definition:Multigraph/Simple Edge,"['Definitions/Edges of Graphs', 'Definitions/Multigraphs']","Let G = V, E be a multigraph.


A simple edge is an edge u v of G which is the only edge of G which is incident to both u and v.


Category:Definitions/Edges of Graphs
Category:Definitions/Multigraphs"
Definition:Simple,Simple,"Let $\struct {X, \Sigma}$ be a measurable space.

Then the space of simple functions on $\struct {X, \Sigma}$, denoted $\map \EE \Sigma$, is the collection of all simple functions $f: X \to \R$:

:$\map \EE \Sigma := \set {f: X \to \R: \text{$f$ is a simple function} }$


=== Space of Positive Simple Functions ===

The space of positive simple functions on $\struct {X, \Sigma}$, denoted $\map {\EE^+} \Sigma$, is the subset of positive simple functions in $\map \EE \Sigma$:

:$\map {\EE^+} \Sigma := \set {f: X \to \R: \text {$f$ is a positive simple function} }$
",Definition:Simple Function,['Definitions/Measure Theory'],"Let X, Σ be a measurable space.

Then the space of simple functions on X, Σ, denoted Σ, is the collection of all simple functions f: X →:

:Σ := f: X →: f is a simple function


=== Space of Positive Simple Functions ===

The space of positive simple functions on X, Σ, denoted ^+Σ, is the subset of positive simple functions in Σ:

:^+Σ := f: X →: f is a positive simple function
"
Definition:Simple,Simple,,Definition:Semisimple,[],
Definition:Singular,Singular,"Let $\kappa$ be an infinite cardinal.


Then $\kappa$ is a singular cardinal  $\map {\mathrm {cf} } \kappa < \kappa$.

That is, the cofinality of $\kappa$ is less than itself.

",Definition:Singular Cardinal,['Definitions/Cardinals'],"Let κ be an infinite cardinal.


Then κ is a singular cardinal  cfκ < κ.

That is, the cofinality of κ is less than itself.

"
Definition:Singular,Singular,"=== Real Analysis ===

Let $C$ be a locus.


=== Complex Analysis ===
",Definition:Singular Point,"['Definitions/Singular Points', 'Definitions/Singularity Theory', 'Definitions/Analysis']","=== Real Analysis ===

Let C be a locus.


=== Complex Analysis ===
"
Definition:Singular,Singular,,Definition:Singularity,[],
Definition:Skew,Skew,"Let $L_1$ and $L_2$ be two straight lines in $3$-dimensional Euclidean space.


$L_1$ and $L_2$ are said to be skew , when produced, they are neither intersecting nor parallel.",Definition:Skew Lines,"['Definitions/Skew Lines', 'Definitions/Solid Geometry']","Let L_1 and L_2 be two straight lines in 3-dimensional Euclidean space.


L_1 and L_2 are said to be skew , when produced, they are neither intersecting nor parallel."
Definition:Skew,Skew,A skew field is a division ring whose ring product is specifically not commutative.,Definition:Skew Field,"['Definitions/Ring Theory', 'Definitions/Field Theory']",A skew field is a division ring whose ring product is specifically not commutative.
Definition:Skew,Skew,"Let $X$ be a random variable with mean $\mu$ and standard deviation $\sigma$.

The coefficient of skewness of $X$ is the coefficient:
:$\gamma_1 = \expect {\paren {\dfrac {X - \mu} \sigma}^3}$
where $\mu_i$ denotes the $i$th central moment of $X$.
",Definition:Skewness,"['Definitions/Skewness', 'Definitions/Statistics', 'Definitions/Probability Theory']","Let X be a random variable with mean μ and standard deviation σ.

The coefficient of skewness of X is the coefficient:
:γ_1 = X - μσ^3
where μ_i denotes the ith central moment of X.
"
Definition:Small,Small,"Let $A$ denote an arbitrary class.


Then $A$ is said to be small :

:$\exists x: x = A$

where $=$ denotes class equality and $x$ is a set variable.


That is, a class is small  it is equal to some set variable.


To denote that a class $A$ is small, the notation $\map \MM A$ may be used.

Thus:
:$\map \MM A \iff \exists x: x = A$",Definition:Small Class,"['Definitions/Class Theory', 'Definitions/Set Theory']","Let A denote an arbitrary class.


Then A is said to be small :

:∃ x: x = A

where = denotes class equality and x is a set variable.


That is, a class is small  it is equal to some set variable.


To denote that a class A is small, the notation A may be used.

Thus:
:A ∃ x: x = A"
Definition:Small,Small,"Let $\mathbf C$ be a metacategory.


Then $\mathbf C$ is said to be small  both of the following hold:

:The collection of objects $\mathbf C_0$ is a set;
:The collection of morphisms $\mathbf C_1$ is a set.",Definition:Small Category,['Definitions/Category Theory'],"Let 𝐂 be a metacategory.


Then 𝐂 is said to be small  both of the following hold:

:The collection of objects 𝐂_0 is a set;
:The collection of morphisms 𝐂_1 is a set."
Definition:Smooth,Smooth,"A real function is smooth  it is of differentiability class $C^\infty$.

That is,  it admits of continuous derivatives of all orders.


",Definition:Smooth Real Function,"['Definitions/Differentiable Real Functions', 'Definitions/Topology', 'Definitions/Differentiability Classes']","A real function is smooth  it is of differentiability class C^∞.

That is,  it admits of continuous derivatives of all orders.


"
Definition:Smooth,Smooth,"Let $M$ be a second-countable locally Euclidean space of dimension $d$. 

Let $\mathscr F$ be a smooth differentiable structure on $M$.


Then $\struct {M, \mathscr F}$ is called a smooth manifold of dimension $d$.
A real function is smooth  it is of differentiability class $C^\infty$.

That is,  it admits of continuous derivatives of all orders.



",Definition:Smooth Mapping,['Definitions/Manifolds'],"Let M be a second-countable locally Euclidean space of dimension d. 

Let ℱ be a smooth differentiable structure on M.


Then M, ℱ is called a smooth manifold of dimension d.
A real function is smooth  it is of differentiability class C^∞.

That is,  it admits of continuous derivatives of all orders.



"
Definition:Smooth,Smooth,"Let $M$ be a topological space.

Let $d$ be a natural number.


A $d$-dimensional smooth differentiable structure $\mathscr F$ on $M$ is a $d$-dimensional differentiable structure on $M$ which is of class $\CC^k$ for every $k \in \N$.

Category:Definitions/Manifolds
Let $M$ be a second-countable locally Euclidean space of dimension $d$. 

Let $\mathscr F$ be a smooth differentiable structure on $M$.


Then $\struct {M, \mathscr F}$ is called a smooth manifold of dimension $d$.
",Definition:Topological Manifold/Smooth Manifold,"['Definitions/Smooth Manifolds', 'Definitions/Topological Manifolds', 'Definitions/Differentiable Manifolds']","Let M be a topological space.

Let d be a natural number.


A d-dimensional smooth differentiable structure ℱ on M is a d-dimensional differentiable structure on M which is of class ^k for every k ∈.

Category:Definitions/Manifolds
Let M be a second-countable locally Euclidean space of dimension d. 

Let ℱ be a smooth differentiable structure on M.


Then M, ℱ is called a smooth manifold of dimension d.
"
Definition:Smooth,Smooth,"Let $M, N$ be smooth manifolds. 

Denote $m := \dim M$ and $n := \dim N$. 

Let $\phi: M \to N$ be a mapping. 


Then $\phi$ is a smooth mapping :
:for every chart $\struct {U, \kappa}$ on $M$ and every chart $\struct {V, \xi}$ on $N$ such that $V \cap \map \phi U \ne \O$, the mapping:
::$\ds \xi \circ \phi \circ \kappa^{-1}: \map \kappa U \subseteq \R^m \to \map \xi {V \cap \map \phi U} \subseteq \R^n$
:is smooth.
Let $M, N$ be smooth manifolds. 

Denote $m := \dim M$ and $n := \dim N$. 

Let $\phi: M \to N$ be a mapping. 


Then $\phi$ is a smooth mapping :
:for every chart $\struct {U, \kappa}$ on $M$ and every chart $\struct {V, \xi}$ on $N$ such that $V \cap \map \phi U \ne \O$, the mapping:
::$\ds \xi \circ \phi \circ \kappa^{-1}: \map \kappa U \subseteq \R^m \to \map \xi {V \cap \map \phi U} \subseteq \R^n$
:is smooth.
",Definition:Smooth Homotopy,['Definitions/Homotopy Theory'],"Let M, N be smooth manifolds. 

Denote m :=  M and n :=  N. 

Let ϕ: M → N be a mapping. 


Then ϕ is a smooth mapping :
:for every chart U, κ on M and every chart V, ξ on N such that V ∩ϕ U Ø, the mapping:
::ξ∘ϕ∘κ^-1: κ U ⊆^m →ξV ∩ϕ U⊆^n
:is smooth.
Let M, N be smooth manifolds. 

Denote m :=  M and n :=  N. 

Let ϕ: M → N be a mapping. 


Then ϕ is a smooth mapping :
:for every chart U, κ on M and every chart V, ξ on N such that V ∩ϕ U Ø, the mapping:
::ξ∘ϕ∘κ^-1: κ U ⊆^m →ξV ∩ϕ U⊆^n
:is smooth.
"
Definition:Solution,Solution,"Let $P: X \to \set {\T, \F}$ be a propositional function defined on a domain $X$.

Let $S = \map {P^{-1} } \T$ be the fiber of truth (under $P$).


Then an element of $S$ is known as a solution of $P$.


This terminology is usual when $P$ is an equation in the context of algebra.",Definition:Fiber of Truth/Solution,"['Definitions/Mapping Theory', 'Definitions/Algebra']","Let P: X →, be a propositional function defined on a domain X.

Let S = P^-1 be the fiber of truth (under P).


Then an element of S is known as a solution of P.


This terminology is usual when P is an equation in the context of algebra."
Definition:Solution,Solution,"Let $\Phi$ be a differential equation.

The general solution to $\Phi$ is the set of all functions $\phi$ that satisfy $\Phi$.



",Definition:Differential Equation/Solution/General Solution,"['Definitions/General Solutions to Differential Equations', 'Definitions/Solutions to Differential Equations', 'Definitions/Differential Equations']","Let Φ be a differential equation.

The general solution to Φ is the set of all functions ϕ that satisfy Φ.



"
Definition:Solution,Solution,"Let $P: X \to \set {\T, \F}$ be a propositional function defined on a domain $X$.


The fiber of truth (under $P$) is the preimage, or fiber, of $\T$ under $P$:
:$\map {P^{-1} } \T := \set {x \in X: \map P x = \T}$


That is, the elements of $X$ whose image under $P$ is $\T$.


=== Solution ===

Let $\Phi$ be a differential equation.

Let $S$ denote the solution set of $\Phi$.

A particular solution of $\Phi$ is the element of $S$, or subset of $S$, which satisfies a particular boundary condition of $\Phi$.
",Definition:Differential Equation/Solution/Particular Solution,"['Definitions/Particular Solutions to Differential Equations', 'Definitions/Solutions to Differential Equations']","Let P: X →, be a propositional function defined on a domain X.


The fiber of truth (under P) is the preimage, or fiber, of  under P:
:P^-1 := x ∈ X:  P x =


That is, the elements of X whose image under P is .


=== Solution ===

Let Φ be a differential equation.

Let S denote the solution set of Φ.

A particular solution of Φ is the element of S, or subset of S, which satisfies a particular boundary condition of Φ.
"
Definition:Solution,Solution,"Let $P: X \to \set {\T, \F}$ be a propositional function defined on a domain $X$.

Let $S = \map {P^{-1} } \T$ be the fiber of truth (under $P$).


Then an element of $S$ is known as a solution of $P$.


This terminology is usual when $P$ is an equation in the context of algebra.
",Definition:Fiber of Truth,"['Definitions/Mapping Theory', 'Definitions/Symbolic Logic']","Let P: X →, be a propositional function defined on a domain X.

Let S = P^-1 be the fiber of truth (under P).


Then an element of S is known as a solution of P.


This terminology is usual when P is an equation in the context of algebra.
"
Definition:Solution,Solution,"An ordered $n$-tuple $\tuple {x_1, x_2, \ldots, x_n}$ which satisfies each of the equations in a system of $m$ simultaneous equations in $n$ variables is called a solution of the system.",Definition:Simultaneous Equations/Solution,['Definitions/Simultaneous Equations'],"An ordered n-tuple x_1, x_2, …, x_n which satisfies each of the equations in a system of m simultaneous equations in n variables is called a solution of the system."
Definition:Solution,Solution,"Let $G$ be a game.

A solution of $G$ is a systematic description of the outcomes that may emerge in a family of games.
",Definition:Solution of Game,['Definitions/Game Theory'],"Let G be a game.

A solution of G is a systematic description of the outcomes that may emerge in a family of games.
"
Definition:Solution,Solution,"Let:
:$\map P x \equiv 0 \pmod n$
be a polynomial congruence.


A solution of $\map P x \equiv 0 \pmod n$ is a residue class modulo $n$ such that any element of that class satisfies the congruence.",Definition:Polynomial Congruence/Solution,['Definitions/Polynomial Congruences'],"Let:
:P x ≡ 0  n
be a polynomial congruence.


A solution of P x ≡ 0  n is a residue class modulo n such that any element of that class satisfies the congruence."
Definition:Sound,Sound,"Sound is the vibration of an elastic medium whose frequency ranges between approximately $20 \, \text {Hz}$ and $20 \, \text {kHz}$.

It is characterised by the ability of humans to detect it with their aural sensory systems.


Category:Definitions/Sound
Category:Definitions/Acoustics
Category:Definitions/Physics",Definition:Sound (Physics),"['Definitions/Sound', 'Definitions/Acoustics', 'Definitions/Physics']","Sound is the vibration of an elastic medium whose frequency ranges between approximately 20  Hz and 20  kHz.

It is characterised by the ability of humans to detect it with their aural sensory systems.


Category:Definitions/Sound
Category:Definitions/Acoustics
Category:Definitions/Physics"
Definition:Sound,Sound,,Definition:Sound Argument,['Definitions/Logical Arguments'],
Definition:Space,Space,"Let $S$ be a set.

Let $\tau$ be a topology on $S$.

That is, let $\tau \subseteq \powerset S$ satisfy the open set axioms:


Then the ordered pair $\struct {S, \tau}$ is called a topological space.

The elements of $\tau$ are called open sets of $\struct {S, \tau}$.


In a topological space $\struct {S, \tau}$, we consider $S$ to be the universal set.",Definition:Topological Space,"['Definitions/Topological Spaces', 'Definitions/Topology', 'Definitions/Abstract Spaces']","Let S be a set.

Let τ be a topology on S.

That is, let τ⊆ S satisfy the open set axioms:


Then the ordered pair S, τ is called a topological space.

The elements of τ are called open sets of S, τ.


In a topological space S, τ, we consider S to be the universal set."
Definition:Space,Space,"Let $S$ be a set.

Let $\tau$ be a topology on $S$.

That is, let $\tau \subseteq \powerset S$ satisfy the open set axioms:


Then the ordered pair $\struct {S, \tau}$ is called a topological space.

The elements of $\tau$ are called open sets of $\struct {S, \tau}$.


In a topological space $\struct {S, \tau}$, we consider $S$ to be the universal set.
",Definition:Hausdorff Space,"['Definitions/Separation Axioms', 'Definitions/Hausdorff Spaces']","Let S be a set.

Let τ be a topology on S.

That is, let τ⊆ S satisfy the open set axioms:


Then the ordered pair S, τ is called a topological space.

The elements of τ are called open sets of S, τ.


In a topological space S, τ, we consider S to be the universal set.
"
Definition:Space,Space,"The vector space axioms are the defining properties of a vector space.

Let $\struct {G, +_G, \circ}_K$ be a vector space over $K$ where:

:$G$ is a set of objects, called vectors.

:$+_G: G \times G \to G$ is a binary operation on $G$

:$\struct {K, +, \cdot}$ is a division ring whose unity is $1_K$

:$\circ: K \times G \to G$ is a binary operation

The usual situation is for $K$ to be one of the standard number fields $\R$ or $\C$.


The vector space axioms consist of the abelian group axioms:










together with the properties of a unitary module:







",Definition:Vector Space,"['Definitions/Vector Spaces', 'Definitions/Vector Algebra', 'Definitions/Linear Algebra', 'Definitions/Abstract Spaces']","The vector space axioms are the defining properties of a vector space.

Let G, +_G, ∘_K be a vector space over K where:

:G is a set of objects, called vectors.

:+_G: G × G → G is a binary operation on G

:K, +, · is a division ring whose unity is 1_K

:∘: K × G → G is a binary operation

The usual situation is for K to be one of the standard number fields  or .


The vector space axioms consist of the abelian group axioms:










together with the properties of a unitary module:







"
Definition:Space,Space,"Let $V = \struct {G, +_G, \circ}_K$ be a vector space over $K$, where:

:$\struct {G, +_G}$ is an abelian group

:$\struct {K, +_K, \times_K}$ is the scalar field of $V$.


The elements of the abelian group $\struct {G, +_G}$ are called vectors.
",Definition:Column Space,"['Definitions/Column Space', 'Definitions/Matrix Theory', 'Definitions/Linear Algebra']","Let V = G, +_G, ∘_K be a vector space over K, where:

:G, +_G is an abelian group

:K, +_K, ×_K is the scalar field of V.


The elements of the abelian group G, +_G are called vectors.
"
Definition:Space,Space,"Let $\R$ be the set of real numbers.


Then the $\R$-module $\R^n$ is called the real ($n$-dimensional) vector space.
Let:
$\quad \mathbf A_{m \times n} = \begin {bmatrix}
a_{11} & a_{12} & \cdots & a_{1n} \\
a_{21} & a_{22} & \cdots & a_{2n} \\
\vdots & \vdots & \ddots & \vdots \\
a_{m1} & a_{m2} & \cdots & a_{mn} \\
\end {bmatrix}$,  $\mathbf x_{n \times 1} = \begin {bmatrix} x_1 \\ x_2 \\ \vdots \\ x_n \end {bmatrix}$, $\mathbf 0_{m \times 1} = \begin {bmatrix} 0 \\ 0 \\ \vdots \\ 0 \end {bmatrix}$

be matrices where each column is a member of a real vector space.

The set of all solutions to $\mathbf A \mathbf x = \mathbf 0$:

:$\map {\mathrm N} {\mathbf A} = \set {\mathbf x \in \R^n : \mathbf {A x} = \mathbf 0}$

is called the null space of $\mathbf A$.



",Definition:Null Space,"['Definitions/Null Spaces', 'Definitions/Linear Algebra']","Let  be the set of real numbers.


Then the -module ^n is called the real (n-dimensional) vector space.
Let:
𝐀_m × n = [ a_11 a_12    ⋯ a_1n; a_21 a_22    ⋯ a_2n;    ⋮    ⋮    ⋱    ⋮; a_m1 a_m2    ⋯ a_mn;      ],  𝐱_n × 1 = [ x_1; x_2;   ⋮; x_n ], 0_m × 1 = [ 0; 0; ⋮; 0 ]

be matrices where each column is a member of a real vector space.

The set of all solutions to 𝐀𝐱 = 0:

:N𝐀 = 𝐱∈^n : 𝐀 𝐱 = 0

is called the null space of 𝐀.



"
Definition:Space,Space,"Ordinary space (or just space) is a word used to mean the universe we live in.

The intuitive belief is that space is $3$-dimensional and therefore isomorphic to the real vector space $\R^3$.


Hence ordinary space is usually taken as an alternative term for Euclidean $3$-dimensional space.
Let $\R$ be the set of real numbers.


Then the $\R$-module $\R^n$ is called the real ($n$-dimensional) vector space.
Let $S$ be one of the standard number fields $\Q$, $\R$, $\C$.

Let $S^n$ be a cartesian space for $n \in \N_{\ge 1}$.

Let $d: S^n \times S^n \to \R$ be the usual (Euclidean) metric on $S^n$.

Then $\tuple {S^n, d}$ is a Euclidean space.
",Definition:Ordinary Space,"['Definitions/Ordinary Space', 'Definitions/Geometry', 'Definitions/Projective Geometry', 'Definitions/Physics']","Ordinary space (or just space) is a word used to mean the universe we live in.

The intuitive belief is that space is 3-dimensional and therefore isomorphic to the real vector space ^3.


Hence ordinary space is usually taken as an alternative term for Euclidean 3-dimensional space.
Let  be the set of real numbers.


Then the -module ^n is called the real (n-dimensional) vector space.
Let S be one of the standard number fields , , .

Let S^n be a cartesian space for n ∈_≥ 1.

Let d: S^n × S^n → be the usual (Euclidean) metric on S^n.

Then S^n, d is a Euclidean space.
"
Definition:Space,Space,"Let $\R$ be the set of real numbers.


Then the $\R$-module $\R^n$ is called the real ($n$-dimensional) vector space.
",Definition:Real Vector Space,"['Definitions/Real Vector Spaces', 'Definitions/Examples of Vector Spaces', 'Definitions/Real Analysis', 'Definitions/Analytic Geometry']","Let  be the set of real numbers.


Then the -module ^n is called the real (n-dimensional) vector space.
"
Definition:Spectrum,Spectrum,"Let $A$ be a commutative ring with unity.


The prime spectrum of $A$ is the set of prime ideals $\mathfrak p$ of $A$:

:$\Spec A = \set {\mathfrak p \lhd A: \mathfrak p \text{ is prime} }$

where $\mathfrak p \lhd A$ indicates that $\mathfrak p$ is an ideal of $A$.",Definition:Prime Spectrum of Ring,['Definitions/Commutative Algebra'],"Let A be a commutative ring with unity.


The prime spectrum of A is the set of prime ideals 𝔭 of A:

:A = 𝔭 A: 𝔭 is prime

where 𝔭 A indicates that 𝔭 is an ideal of A."
Definition:Spectrum,Spectrum,"Let $A$ be a commutative ring with unity.


The maximal spectrum of $A$ is the set of maximal ideals of $A$:

:$\operatorname{Max} \: \Spec A = \set {\mathfrak m \lhd A : \mathfrak m \text { is maximal} }$

where $I \lhd A$ indicates that $I$ is an ideal of $A$.


The notation $\operatorname {Max} \: \Spec A$ is also a shorthand for the locally ringed space
:$\struct {\operatorname {Max} \: \Spec A, \tau, \OO_{\map {\operatorname {Max Spec} } A} }$
where:
:$\tau$ is the Zariski topology on $\map {\operatorname {Max Spec} } A$
:$\OO_{\map {\operatorname {Max Spec} } A}$ is the structure sheaf of $\map {\operatorname {Max Spec} } A$",Definition:Maximal Spectrum of Ring,['Definitions/Commutative Algebra'],"Let A be a commutative ring with unity.


The maximal spectrum of A is the set of maximal ideals of A:

:Max  A = 𝔪 A : 𝔪 is maximal

where I  A indicates that I is an ideal of A.


The notation Max  A is also a shorthand for the locally ringed space
:Max  A, τ, _Max Spec A
where:
:τ is the Zariski topology on Max Spec A
:_Max Spec A is the structure sheaf of Max Spec A"
Definition:Spectrum,Spectrum,"Let $\struct {X, \norm \cdot_X}$ be a Banach space over $\C$. 

Let $A : X \to X$ be a bounded linear operator.

Let $\map \rho A$ be the resolvent set of $A$. 

Let: 

:$\map \sigma A = \C \setminus \map \rho A$


We say that $\map \sigma A$ is the spectrum of $A$.",Definition:Spectrum (Spectral Theory)/Bounded Linear Operator,"['Spectra (Spectral Theory)', 'Definitions/Spectra (Spectral Theory)', 'Definitions/Bounded Linear Operators', 'Definitions/Banach Spaces', 'Definitions/Spectra (Spectral Theory)']","Let X, ·_X be a Banach space over . 

Let A : X → X be a bounded linear operator.

Let ρ A be the resolvent set of A. 

Let: 

:σ A = ∖ρ A


We say that σ A is the spectrum of A."
Definition:Structure,Structure,"Let $\LL_1$ be the language of predicate logic.


A structure $\AA$ for $\LL_1$ comprises:

:$(1): \quad$ A non-empty set $A$;
:$(2): \quad$ For each function symbol $f$ of arity $n$, a mapping $f_\AA: A^n \to A$;
:$(3): \quad$ For each predicate symbol $p$ of arity $n$, a mapping $p_\AA: A^n \to \Bbb B$

where $\Bbb B$ denotes the set of truth values.

$A$ is called the underlying set of $\AA$.

$f_\AA$ and $p_\AA$ are called the interpretations of $f$ and $p$ in $\AA$, respectively.


We remark that function symbols of arity $0$ are interpreted as constants in $A$.

To avoid pathological situations with the interpretation of arity-$0$ function symbols, it is essential that $A$ be non-empty.

Also, the predicate symbols may be interpreted as relations via their characteristic functions.



",Definition:Formal Semantics/Structure,['Definitions/Formal Semantics'],"Let _1 be the language of predicate logic.


A structure Å for _1 comprises:

:(1): A non-empty set A;
:(2): For each function symbol f of arity n, a mapping f_Å: A^n → A;
:(3): For each predicate symbol p of arity n, a mapping p_Å: A^n → B

where B denotes the set of truth values.

A is called the underlying set of Å.

f_Å and p_Å are called the interpretations of f and p in Å, respectively.


We remark that function symbols of arity 0 are interpreted as constants in A.

To avoid pathological situations with the interpretation of arity-0 function symbols, it is essential that A be non-empty.

Also, the predicate symbols may be interpreted as relations via their characteristic functions.



"
Definition:Structure,Structure,"Let $\LL_1$ be the language of predicate logic.


A structure $\AA$ for $\LL_1$ comprises:

:$(1): \quad$ A non-empty set $A$;
:$(2): \quad$ For each function symbol $f$ of arity $n$, a mapping $f_\AA: A^n \to A$;
:$(3): \quad$ For each predicate symbol $p$ of arity $n$, a mapping $p_\AA: A^n \to \Bbb B$

where $\Bbb B$ denotes the set of truth values.

$A$ is called the underlying set of $\AA$.

$f_\AA$ and $p_\AA$ are called the interpretations of $f$ and $p$ in $\AA$, respectively.


We remark that function symbols of arity $0$ are interpreted as constants in $A$.

To avoid pathological situations with the interpretation of arity-$0$ function symbols, it is essential that $A$ be non-empty.

Also, the predicate symbols may be interpreted as relations via their characteristic functions.


",Definition:Structure for Predicate Logic,"['Definitions/Predicate Logic', 'Definitions/Formal Semantics', 'Definitions/Model Theory for Predicate Logic']","Let _1 be the language of predicate logic.


A structure Å for _1 comprises:

:(1): A non-empty set A;
:(2): For each function symbol f of arity n, a mapping f_Å: A^n → A;
:(3): For each predicate symbol p of arity n, a mapping p_Å: A^n → B

where B denotes the set of truth values.

A is called the underlying set of Å.

f_Å and p_Å are called the interpretations of f and p in Å, respectively.


We remark that function symbols of arity 0 are interpreted as constants in A.

To avoid pathological situations with the interpretation of arity-0 function symbols, it is essential that A be non-empty.

Also, the predicate symbols may be interpreted as relations via their characteristic functions.


"
Definition:Structure,Structure,"A relational structure is an ordered pair $\struct {S, \RR}$, where:
:$S$ is a set
:$\RR$ is an endorelation on $S$.",Definition:Relational Structure,"['Definitions/Relation Theory', 'Definitions/Relational Structures']","A relational structure is an ordered pair S,, where:
:S is a set
: is an endorelation on S."
Definition:Structure Sheaf,Structure Sheaf,"Let $\struct {X, \OO_X}$ be a ringed space.


The structure sheaf of $\struct {X, \OO_X}$ is the term $\OO_X$.
",Definition:Ringed Space,"['Definitions/Algebraic Geometry', 'Definitions/Ringed Spaces']","Let X, _X be a ringed space.


The structure sheaf of X, _X is the term _X.
"
Definition:Structure Sheaf,Structure Sheaf,"A ringed space is a pair $\struct {X, \OO_X}$ where:
:$X$ is a topological space
:$\OO_X$ is a sheaf of commutative rings with unity on $X$.



=== Structure Sheaf ===

",Definition:Structure Sheaf of Spectrum of Ring,['Definitions/Algebraic Geometry'],"A ringed space is a pair X, _X where:
:X is a topological space
:_X is a sheaf of commutative rings with unity on X.



=== Structure Sheaf ===

"
Definition:Subadditive Function,Subadditive Function,"Let $\struct {S, +_S}$ and $\struct {T, +_T, \preceq}$ be semigroups such that $\struct {T, +_T, \preceq}$ is ordered.


Let $f: S \to T$ be a mapping from $S$ to $T$ which satisfies the relation:
:$\forall a, b \in S: \map f {a +_S b} \preceq \map f a +_T \map f b$


Then $f$ is defined as being subadditive.


The usual context in which this is encountered is where $S$ and $T$ are both the set of real numbers $\R$ (or a subset of them).",Definition:Subadditive Function (Conventional),"['Definitions/Subadditive Functions', 'Definitions/Abstract Algebra', 'Definitions/Analysis']","Let S, +_S and T, +_T, ≼ be semigroups such that T, +_T, ≼ is ordered.


Let f: S → T be a mapping from S to T which satisfies the relation:
:∀ a, b ∈ S:  f a +_S b≼ f a +_T  f b


Then f is defined as being subadditive.


The usual context in which this is encountered is where S and T are both the set of real numbers  (or a subset of them)."
Definition:Subadditive Function,Subadditive Function,"Let $\SS$ be an algebra of sets.

Let $f: \SS \to \overline \R$ be a function, where $\overline \R$ denotes the extended set of real numbers.


Then $f$ is defined to be subadditive (or sub-additive) :
:$\forall S, T \in \SS: \map f {S \cup T} \le \map f S + \map f T$


That is, for any two elements of $\SS$, $f$ applied to their union is not greater than the sum of $f$ of the individual elements.",Definition:Subadditive Function (Measure Theory),"['Definitions/Set Systems', 'Definitions/Measure Theory']","Let  be an algebra of sets.

Let f: → be a function, where  denotes the extended set of real numbers.


Then f is defined to be subadditive (or sub-additive) :
:∀ S, T ∈:  f S ∪ T≤ f S +  f T


That is, for any two elements of , f applied to their union is not greater than the sum of f of the individual elements."
Definition:Subadditive Function,Subadditive Function,"Let $\Sigma$ be a $\sigma$-algebra over a set $X$.

Let $f: \Sigma \to \overline \R$ be a function, where $\overline \R$ denotes the set of extended real numbers.


Then $f$ is defined as countably subadditive  for any sequence $\sequence {E_n}_{n \mathop \in \N}$ of elements of $\Sigma$:

:$\ds \map f {\bigcup_{n \mathop = 0}^\infty E_n} \le \sum_{n \mathop = 0}^\infty \map f {E_n}$",Definition:Countably Subadditive Function,"['Definitions/Set Systems', 'Definitions/Measure Theory']","Let Σ be a σ-algebra over a set X.

Let f: Σ→ be a function, where  denotes the set of extended real numbers.


Then f is defined as countably subadditive  for any sequence E_n_n ∈ of elements of Σ:

:f ⋃_n  = 0^∞ E_n≤∑_n  = 0^∞ f E_n"
Definition:Subdivision,Subdivision,"Let $G = \struct {V, E}$ be a graph.


The edge subdivision operation for an edge $\set {u, v} \in E$ is the deletion of $\set {u, v}$ from $G$ and the addition of two edges $\set {u, w}$ and $\set {w, v}$ along with the new vertex $w$. 


This operation generates a new graph $H$:
:$H = \struct {V \cup \set w, \paren {E \setminus \set {u, v} } \cup \set {\set {u, w}, \set {w, v} } }$
Let $G = \struct {V, E}$ be a graph.


A graph which has been derived from $G$ by a sequence of edge subdivision operations is called a subdivision of $G$.
",Definition:Subdivision (Graph Theory),"['Definitions/Subdivisions (Graph Theory)', 'Definitions/Graph Theory']","Let G = V, E be a graph.


The edge subdivision operation for an edge u, v∈ E is the deletion of u, v from G and the addition of two edges u, w and w, v along with the new vertex w. 


This operation generates a new graph H:
:H = V ∪ w, E ∖u, v∪u, w, w, v
Let G = V, E be a graph.


A graph which has been derived from G by a sequence of edge subdivision operations is called a subdivision of G.
"
Definition:Subdivision,Subdivision,"Let $G = \struct {V, E}$ be a graph.


The edge subdivision operation for an edge $\set {u, v} \in E$ is the deletion of $\set {u, v}$ from $G$ and the addition of two edges $\set {u, w}$ and $\set {w, v}$ along with the new vertex $w$. 


This operation generates a new graph $H$:
:$H = \struct {V \cup \set w, \paren {E \setminus \set {u, v} } \cup \set {\set {u, w}, \set {w, v} } }$",Definition:Subdivision (Graph Theory)/Edge,"['Definitions/Subdivisions (Graph Theory)', 'Definitions/Edges of Graphs']","Let G = V, E be a graph.


The edge subdivision operation for an edge u, v∈ E is the deletion of u, v from G and the addition of two edges u, w and w, v along with the new vertex w. 


This operation generates a new graph H:
:H = V ∪ w, E ∖u, v∪u, w, w, v"
Definition:Subdivision,Subdivision,"Let $G = \struct {V, E}$ be a graph.


The edge subdivision operation for an edge $\set {u, v} \in E$ is the deletion of $\set {u, v}$ from $G$ and the addition of two edges $\set {u, w}$ and $\set {w, v}$ along with the new vertex $w$. 


This operation generates a new graph $H$:
:$H = \struct {V \cup \set w, \paren {E \setminus \set {u, v} } \cup \set {\set {u, w}, \set {w, v} } }$
",Definition:Subdivision (Graph Theory)/Graph,['Definitions/Subdivisions (Graph Theory)'],"Let G = V, E be a graph.


The edge subdivision operation for an edge u, v∈ E is the deletion of u, v from G and the addition of two edges u, w and w, v along with the new vertex w. 


This operation generates a new graph H:
:H = V ∪ w, E ∖u, v∪u, w, w, v
"
Definition:Subspace,Subspace,"Let $T = \struct {S, \tau}$ be a topological space.

Let $H \subseteq S$ be a non-empty subset of $S$.


Define:
:$\tau_H := \set {U \cap H: U \in \tau} \subseteq \powerset H$

where $\powerset H$ denotes the power set of $H$.


Then the topological space $T_H = \struct {H, \tau_H}$ is called a (topological) subspace of $T$.


The set $\tau_H$ is referred to as the subspace topology on $H$ (induced by $\tau$).",Definition:Topological Subspace,['Definitions/Topology'],"Let T = S, τ be a topological space.

Let H ⊆ S be a non-empty subset of S.


Define:
:τ_H := U ∩ H: U ∈τ⊆ H

where H denotes the power set of H.


Then the topological space T_H = H, τ_H is called a (topological) subspace of T.


The set τ_H is referred to as the subspace topology on H (induced by τ)."
Definition:Subspace,Subspace,"Let $\struct {A, d}$ be a metric space.

Let $H \subseteq A$.

Let $d_H: H \times H \to \R$ be the restriction $d \restriction_{H \times H}$ of $d$ to $H$.

That is, let $\forall x, y \in H: \map {d_H} {x, y} = \map d {x, y}$.


Then $d_H$ is the metric induced on $H$ by $d$ or the subspace metric of $d$ (with respect to $H$).


The metric space $\struct {H, d_H}$ is called a metric subspace of $\struct {A, d}$.",Definition:Metric Subspace,"['Definitions/Metric Subspaces', 'Definitions/Metric Spaces']","Let A, d be a metric space.

Let H ⊆ A.

Let d_H: H × H → be the restriction d _H × H of d to H.

That is, let ∀ x, y ∈ H: d_Hx, y =  d x, y.


Then d_H is the metric induced on H by d or the subspace metric of d (with respect to H).


The metric space H, d_H is called a metric subspace of A, d."
Definition:Subspace,Subspace,"Let $K$ be a division ring.

Let $\struct {S, +, \circ}_K$ be a $K$-algebraic structure with one operation.

Let $\struct {T, +_T, \circ_T}_K$ be a vector subspace of $\struct {S, +, \circ}_K$.


If $T$ is a proper subset of $S$, then $\struct {T, +_T, \circ_T}_K$ is a proper (vector) subspace of $\struct {S, +, \circ}_K$.
",Definition:Vector Subspace,"['Definitions/Linear Algebra', 'Definitions/Vector Algebra']","Let K be a division ring.

Let S, +, ∘_K be a K-algebraic structure with one operation.

Let T, +_T, ∘_T_K be a vector subspace of S, +, ∘_K.


If T is a proper subset of S, then T, +_T, ∘_T_K is a proper (vector) subspace of S, +, ∘_K.
"
Definition:Substatement,Substatement,A substatement of a compound statement is one of the statements that comprise it.,Definition:Compound Statement/Substatement,['Definitions/Compound Statements'],A substatement of a compound statement is one of the statements that comprise it.
Definition:Substatement,Substatement,A substatement of a statement form $\mathbf A$ is another statement form which occurs as a part of $\mathbf A$.,Definition:Statement Form/Substatement,['Definitions/Symbolic Logic'],A substatement of a statement form 𝐀 is another statement form which occurs as a part of 𝐀.
Definition:Substitution,Substitution,"Let $S$ and $T$ be non-empty sets such that $T$ is not a subset of $S$.

Let:
:$s \in S$
:$t \in T \setminus S$
where $\setminus$ denotes set difference.

Let $S'$ be the set defined as:
:$S' = \paren {S \setminus \set s} \cup \set t$

That is, $S'$ is the set obtained by removing $s$ and replacing it with $t$ which is not in $S$.


The operation of replacing $s$ with $t$ is known as substitution.


Category:Definitions/Set Theory",Definition:Substitution (Set Theory),['Definitions/Set Theory'],"Let S and T be non-empty sets such that T is not a subset of S.

Let:
:s ∈ S
:t ∈ T ∖ S
where ∖ denotes set difference.

Let S' be the set defined as:
:S' = S ∖ s∪ t

That is, S' is the set obtained by removing s and replacing it with t which is not in S.


The operation of replacing s with t is known as substitution.


Category:Definitions/Set Theory"
Definition:Substitution,Substitution,"Let $\FF$ be a formal language with alphabet $\AA$.

Let $\mathbf B$ be a well-formed formula of $\FF$.

Let $\mathbf A$ be a well-formed part of $\mathbf B$.

Let $\mathbf A'$ be another well-formed formula.


Then the substitution of $\mathbf A'$ for $\mathbf A$ in $\mathbf B$ is the collation resulting from $\mathbf B$ by replacing all occurrences of $\mathbf A$ in $\mathbf B$ by $\mathbf A'$.

It is denoted as $\map {\mathbf B} {\mathbf A' \mathbin {//} \mathbf A}$.


Note that it is not immediate that $\map {\mathbf B} {\mathbf A' \mathbin {//} \mathbf A}$ is a well-formed formula of $\FF$.

This is either accepted as an axiom or proven as a theorem about the formal language $\FF$.


=== Example ===
",Definition:Substitution (Formal Systems)/Well-Formed Part,['Definitions/Formal Languages'],"Let  be a formal language with alphabet Å.

Let 𝐁 be a well-formed formula of .

Let 𝐀 be a well-formed part of 𝐁.

Let 𝐀' be another well-formed formula.


Then the substitution of 𝐀' for 𝐀 in 𝐁 is the collation resulting from 𝐁 by replacing all occurrences of 𝐀 in 𝐁 by 𝐀'.

It is denoted as 𝐁𝐀' //𝐀.


Note that it is not immediate that 𝐁𝐀' //𝐀 is a well-formed formula of .

This is either accepted as an axiom or proven as a theorem about the formal language .


=== Example ===
"
Definition:Substitution,Substitution,"Let $\FF$ be a formal language with alphabet $\AA$.

Let $\mathbf B$ be a well-formed formula of $\FF$.

Let $p$ be a letter of $\FF$.

Let $\mathbf A$ be another well-formed formula.

Then the substitution of $\mathbf A$ for $p$ in $\mathbf B$ is the collation resulting from $\mathbf B$ by replacing all occurrences of $p$ in $\mathbf B$ by $\mathbf A$.

It is denoted as $\map {\mathbf B} {\mathbf A \mathbin {//} p}$.


Note that it is not immediate that $\map {\mathbf B} {\mathbf A \mathbin {//} p}$ is a well-formed formula of $\FF$.

This is either accepted as an axiom or proven as a theorem about the formal language $\FF$.",Definition:Substitution (Formal Systems)/Letter,['Definitions/Formal Languages'],"Let  be a formal language with alphabet Å.

Let 𝐁 be a well-formed formula of .

Let p be a letter of .

Let 𝐀 be another well-formed formula.

Then the substitution of 𝐀 for p in 𝐁 is the collation resulting from 𝐁 by replacing all occurrences of p in 𝐁 by 𝐀.

It is denoted as 𝐁𝐀// p.


Note that it is not immediate that 𝐁𝐀// p is a well-formed formula of .

This is either accepted as an axiom or proven as a theorem about the formal language ."
Definition:Substitution,Substitution,"Let $\mathbf C$ be a WFF of the language of predicate logic $\LL_1$.

Consider the (abbreviated) WFF $Q x: \mathbf C$ where $Q$ is a quantifier.

Let $y$ be another variable such that:

:$y$ is freely substitutable for $x$ in $\mathbf C$
:$y$ does not occur freely in $\mathbf C$.


Let $\mathbf C'$ be the WFF resulting from substituting $y$ for all free occurrences of $x$ in $\mathbf C$.

The change from $Q x: \mathbf C$ to $Q y: \mathbf C'$ is called alphabetic substitution.",Definition:Alphabetic Substitution,['Definitions/Predicate Logic'],"Let 𝐂 be a WFF of the language of predicate logic _1.

Consider the (abbreviated) WFF Q x: 𝐂 where Q is a quantifier.

Let y be another variable such that:

:y is freely substitutable for x in 𝐂
:y does not occur freely in 𝐂.


Let 𝐂' be the WFF resulting from substituting y for all free occurrences of x in 𝐂.

The change from Q x: 𝐂 to Q y: 𝐂' is called alphabetic substitution."
Definition:Substitution,Substitution,"Let $S_1$ be a statement form.

Let $p$ be a metasymbol which occurs one or more times in $S_1$.

Let $T$ be a statement.

Let $S_2$ be the string formed by replacing every occurrence of $p$ in $S_1$ with $T$.


Then $S_2$ results from the substitution of $p$ by $T$ in $S_1$.

$S_2$ is called a substitution instance of $S_1$.",Definition:Substitution (Formal Systems)/Metasymbol,['Definitions/Formal Systems'],"Let S_1 be a statement form.

Let p be a metasymbol which occurs one or more times in S_1.

Let T be a statement.

Let S_2 be the string formed by replacing every occurrence of p in S_1 with T.


Then S_2 results from the substitution of p by T in S_1.

S_2 is called a substitution instance of S_1."
Definition:Substitution,Substitution,"=== Mapping ===

Let $S$ be a set.

Let $f: S^t \to S$ be a mapping.

Let $\left\{{g_1: S^k \to S, g_2: S^k \to S, \ldots, g_t: S^k \to S}\right\}$ be a set of mappings.

Let the mapping $h: S^k \to S$ be defined as:
:$h \left({s_1, s_2, \ldots, s_k}\right) = f \left({g_1 \left({s_1, s_2, \ldots, s_k}\right), g_2 \left({s_1, s_2, \ldots, s_k}\right), \ldots, g_t \left({s_1, s_2, \ldots, s_k}\right)}\right)$


Then $h$ is said to be obtained from $f, g_1, g_2, \ldots, g_k$ by substitution.


The definition can be generalized in the following ways:
* It can apply to mappings which operate on variously different sets.
* Each of $g_1, g_2, \ldots, g_t$ may have different arities. If $g$ is a mapping of $m$ variables where $m > k$, we can always consider it a mapping of $k$ variables in which the additional variables play no part. So if $g_i$ is a mapping of $k_i$ variables, we can take $k = \max \left\{{k_i: i = 1, 2, \ldots, t}\right\}$ and then each $g_i$ is then a mapping of $k$ variables.


=== Partial Function ===

Let $f: \N^t \to \N$ be a partial function.

Let $\left\{{g_1: \N^k \to \N, g_2: \N^k \to \N, \ldots, g_t: \N^k \to \N}\right\}$ be a set of partial functions.

Let the partial function $h: \N^k \to \N$ be defined as:
:$h \left({n_1, n_2, \ldots, n_k}\right) \approx f \left({g_1 \left({n_1, n_2, \ldots, n_k}\right), g_2 \left({n_1, n_2, \ldots, n_k}\right), \ldots, g_t \left({n_1, n_2, \ldots, n_k}\right)}\right)$
where $\approx$ is as defined in Partial Function Equality.


Then $h$ is said to be obtained from $f, g_1, g_2, \ldots, g_k$ by substitution.


Note that $h \left({n_1, n_2, \ldots, n_k}\right)$ is defined only when:
* All of $g_1 \left({n_1, n_2, \ldots, n_k}\right), g_2 \left({n_1, n_2, \ldots, n_k}\right), \ldots, g_t \left({n_1, n_2, \ldots, n_k}\right)$ are defined
* $f \left({g_1 \left({n_1, n_2, \ldots, n_k}\right), g_2 \left({n_1, n_2, \ldots, n_k}\right), \ldots, g_t \left({n_1, n_2, \ldots, n_k}\right)}\right)$ is defined.",Definition:Substitution (Mathematical Logic),['Definitions/Mathematical Logic'],"=== Mapping ===

Let S be a set.

Let f: S^t → S be a mapping.

Let {g_1: S^k → S, g_2: S^k → S, …, g_t: S^k → S} be a set of mappings.

Let the mapping h: S^k → S be defined as:
:h (s_1, s_2, …, s_k) = f (g_1 (s_1, s_2, …, s_k), g_2 (s_1, s_2, …, s_k), …, g_t (s_1, s_2, …, s_k))


Then h is said to be obtained from f, g_1, g_2, …, g_k by substitution.


The definition can be generalized in the following ways:
* It can apply to mappings which operate on variously different sets.
* Each of g_1, g_2, …, g_t may have different arities. If g is a mapping of m variables where m > k, we can always consider it a mapping of k variables in which the additional variables play no part. So if g_i is a mapping of k_i variables, we can take k = max{k_i: i = 1, 2, …, t} and then each g_i is then a mapping of k variables.


=== Partial Function ===

Let f: ^t → be a partial function.

Let {g_1: ^k →, g_2: ^k →, …, g_t: ^k →} be a set of partial functions.

Let the partial function h: ^k → be defined as:
:h (n_1, n_2, …, n_k) ≈ f (g_1 (n_1, n_2, …, n_k), g_2 (n_1, n_2, …, n_k), …, g_t (n_1, n_2, …, n_k))
where ≈ is as defined in Partial Function Equality.


Then h is said to be obtained from f, g_1, g_2, …, g_k by substitution.


Note that h (n_1, n_2, …, n_k) is defined only when:
* All of g_1 (n_1, n_2, …, n_k), g_2 (n_1, n_2, …, n_k), …, g_t (n_1, n_2, …, n_k) are defined
* f (g_1 (n_1, n_2, …, n_k), g_2 (n_1, n_2, …, n_k), …, g_t (n_1, n_2, …, n_k)) is defined."
Definition:Subtree,Subtree,"Let $T = \struct {V, E}$ be a tree.


A subtree of $T$ is a subgraph of $T$ that is also a tree.",Definition:Subtree (Graph Theory),"['Definitions/Subgraphs', 'Definitions/Tree Theory']","Let T = V, E be a tree.


A subtree of T is a subgraph of T that is also a tree."
Definition:Subtree,Subtree,"Let $T = \struct {V, E}$ be a tree.


A subtree of $T$ is a subgraph of $T$ that is also a tree.
",Definition:Rooted Subtree,"['Definitions/Subgraphs', 'Definitions/Rooted Trees']","Let T = V, E be a tree.


A subtree of T is a subgraph of T that is also a tree.
"
Definition:Subtree,Subtree,"Let $\struct {T, \preceq}$ be a tree.

A subtree of $\struct {T, \preceq}$ is an ordered subset $\struct {S, \preceq}$ with the property that:
:for every $\forall s \in S: \forall t \in T: t \preceq s \implies t \in S$

That is, such that $\struct {S, \preceq}$ is a lower closure of $\struct {T, \preceq}$.


Category:Definitions/Set Theory",Definition:Tree (Set Theory)/Subtree,['Definitions/Set Theory'],"Let T, ≼ be a tree.

A subtree of T, ≼ is an ordered subset S, ≼ with the property that:
:for every ∀ s ∈ S: ∀ t ∈ T: t ≼ s  t ∈ S

That is, such that S, ≼ is a lower closure of T, ≼.


Category:Definitions/Set Theory"
Definition:Successor,Successor,"Let $\preceq$ be an ordering.

Let $a, b$ be such that $a \preceq b$.


Then $b$ succeeds $a$.

$a$ is then described as being a successor of $b$.
",Definition:Succeed,"['Definitions/Order Theory', 'Definitions/Successor Elements']","Let ≼ be an ordering.

Let a, b be such that a ≼ b.


Then b succeeds a.

a is then described as being a successor of b.
"
Definition:Successor,Successor,"Let $\struct {S, \preceq}$ be an ordered set.

Let $a, b \in S$.


Then $a$ is an immediate successor (element) to $b$  $b$ is an immediate predecessor (element) to $a$.

That is, :
:$(1): \quad b \prec a$
:$(2): \quad \nexists c \in S: b \prec c \prec a$

That is, there exists no element strictly between $b$ and $a$ in the ordering $\preceq$.

That is:
:$a \prec b$ and $\openint a b = \O$
where $\openint a b$ denotes the open interval from $a$ to $b$.


We say that $a$ immediately succeeds $b$.


=== Class Theory ===

",Definition:Immediate Successor Element,['Definitions/Successor Elements'],"Let S, ≼ be an ordered set.

Let a, b ∈ S.


Then a is an immediate successor (element) to b  b is an immediate predecessor (element) to a.

That is, :
:(1):    b ≺ a
:(2):   ∄ c ∈ S: b ≺ c ≺ a

That is, there exists no element strictly between b and a in the ordering ≼.

That is:
:a ≺ b and a b = Ø
where a b denotes the open interval from a to b.


We say that a immediately succeeds b.


=== Class Theory ===

"
Definition:Successor,Successor,"Let $V$ be a basic universe.

Let $s: V \to V$ denote the successor mapping on $V$.


For $x \in V$, the result of applying the successor mapping on $x$ is denoted $x^+$:

:$x^+ := \map s x = x \cup \set x$

$x^+$ is referred to as the successor (set) of $x$.
",Definition:Successor Ordinal,"['Definitions/Ordinals', 'Definitions/Successor Mapping']","Let V be a basic universe.

Let s: V → V denote the successor mapping on V.


For x ∈ V, the result of applying the successor mapping on x is denoted x^+:

:x^+ :=  s x = x ∪ x

x^+ is referred to as the successor (set) of x.
"
Definition:Sum,Sum,"Let $a + b$ denote the operation of addition on two objects $a$ and $b$.

Then the result $a + b$ is referred to as the sum of $a$ and $b$.


Note that the nature of $a$ and $b$ has deliberately been left unspecified.

They could be, for example, numbers, matrices or more complex expressions constructed from such elements.",Definition:Addition/Sum,['Definitions/Addition'],"Let a + b denote the operation of addition on two objects a and b.

Then the result a + b is referred to as the sum of a and b.


Note that the nature of a and b has deliberately been left unspecified.

They could be, for example, numbers, matrices or more complex expressions constructed from such elements."
Definition:Sum,Sum,"Let $\struct {G, +}$ be a commutative monoid.


Let $F \subseteq G$ be a finite subset of $G$.


Let $\set {e_1, e_2, \ldots, e_n}$ be a finite enumeration of $F$.

Let $\tuple {e_1, e_2, \ldots, e_n}$ be the ordered tuple formed from the bijection $e: \closedint 1 n \to F$.


The summation over $F$, denoted $\ds \sum_{g \mathop \in F} g$, is defined as the summation over $\tuple{e_1, e_2, \ldots, e_n}$:
:$\ds \sum_{g \mathop \in F} g = \sum_{i \mathop = 1}^n e_i$
Let $\struct {G, +}$ be a commutative monoid.


Let $\family {g_i}_{i \mathop \in I}$ be an indexed subset of $G$ where the indexing set $I$ is finite.


Let $\set {e_1, e_2, \ldots, e_n}$ be a finite enumeration of $I$.

Let $\tuple {g_{e_1}, g_{e_2}, \ldots, g_{e_n} }$ be the ordered tuple formed from the composite mapping $g \circ e: \closedint 1 n \to G$.


The summation over $I$, denoted $\ds \sum_{i \mathop \in I} g_i$, is defined as the summation over $\tuple {g_{e_1}, g_{e_2}, \ldots, g_{e_n} }$:
:$\ds \sum_{i \mathop \in I} g_i = \sum_{k \mathop = 1}^n g_{e_k}$
",Definition:Summation,"['Definitions/Summations', 'Definitions/Algebra', 'Definitions/Abstract Algebra']","Let G, + be a commutative monoid.


Let F ⊆ G be a finite subset of G.


Let e_1, e_2, …, e_n be a finite enumeration of F.

Let e_1, e_2, …, e_n be the ordered tuple formed from the bijection e:  1 n → F.


The summation over F, denoted ∑_g ∈ F g, is defined as the summation over e_1, e_2, …, e_n:
:∑_g ∈ F g = ∑_i  = 1^n e_i
Let G, + be a commutative monoid.


Let g_i_i ∈ I be an indexed subset of G where the indexing set I is finite.


Let e_1, e_2, …, e_n be a finite enumeration of I.

Let g_e_1, g_e_2, …, g_e_n be the ordered tuple formed from the composite mapping g ∘ e:  1 n → G.


The summation over I, denoted ∑_i ∈ I g_i, is defined as the summation over g_e_1, g_e_2, …, g_e_n:
:∑_i ∈ I g_i = ∑_k  = 1^n g_e_k
"
Definition:Support,Support,"Let $S$ be a set.

Let $f: S \to \R$ be a real-valued function.


The support of $f$ is the set of elements $x$ of $S$ whose values under $f$ are non-zero.

That is:
:$\map \supp f := \set {x \in S: \map f x \ne 0}$


That is, the support of a function whose codomain is the set of real numbers is generally defined to be the subset of its domain which maps to anywhere that is not $0$.

Category:Definitions/Real Analysis",Definition:Support of Mapping to Algebraic Structure/Real-Valued Function,['Definitions/Real Analysis'],"Let S be a set.

Let f: S → be a real-valued function.


The support of f is the set of elements x of S whose values under f are non-zero.

That is:
:f := x ∈ S:  f x  0


That is, the support of a function whose codomain is the set of real numbers is generally defined to be the subset of its domain which maps to anywhere that is not 0.

Category:Definitions/Real Analysis"
Definition:Support,Support,"Let $A$ be a commutative ring with unity.

Let $M$ be a unitary $A$-module.


The support $\map \supp M$ of $M$ is the set of prime ideals $P$ of $A$ such that the localization of $M$ at $P$ is nonzero:
:$\map \supp M = \set {P \in \Spec A : M_P \ne 0}$

where $\Spec A$ is the spectrum of $A$.

Category:Definitions/Module Theory",Definition:Support of Module,['Definitions/Module Theory'],"Let A be a commutative ring with unity.

Let M be a unitary A-module.


The support M of M is the set of prime ideals P of A such that the localization of M at P is nonzero:
:M = P ∈ A : M_P  0

where A is the spectrum of A.

Category:Definitions/Module Theory"
Definition:Support,Support,"Let $\family {\struct {S_i, \circ_i} }_{i \mathop \in I}$ be a family of algebraic structures with identity.

Let $\ds S = \prod_{i \mathop \in I} S_i$ be their direct product.

Let $e_i$ be an identity of $S_i$ for all $i \in I$.

Let $m = \family {m_i}_{i \mathop \in I} \in S$.


The support of $m$ is defined as:
:$\supp \set {i \in I: m_i \ne e_i}$




=== Finite Support ===

The element is said to have finite support  its support is a finite set.
",Definition:Support of Element of Direct Product,"['Definitions/Abstract Algebra', 'Definitions/Direct Products']","Let S_i, ∘_i_i ∈ I be a family of algebraic structures with identity.

Let S = ∏_i ∈ I S_i be their direct product.

Let e_i be an identity of S_i for all i ∈ I.

Let m = m_i_i ∈ I∈ S.


The support of m is defined as:
:i ∈ I: m_i  e_i




=== Finite Support ===

The element is said to have finite support  its support is a finite set.
"
Definition:Support,Support,"Let $f: \R^n \to \R$ be a continuous real-valued function.

The support of $f$ is the closure of the set of elements $x$ of $\R^n$ whose values under $f$ are non-zero.

That is:
:$\map \supp f = \cl \set {x \in \R^n: \map f x \ne 0}$


Category:Definitions/Real Analysis",Definition:Support of Continuous Mapping/Real-Valued,['Definitions/Real Analysis'],"Let f: ^n → be a continuous real-valued function.

The support of f is the closure of the set of elements x of ^n whose values under f are non-zero.

That is:
:f = x ∈^n:  f x  0


Category:Definitions/Real Analysis"
Definition:Support,Support,,Definition:Support of Distribution,['Definitions/Real Analysis'],
Definition:Support,Support,"Let $S$ be a set.

Let $f$ be a permutation on $S$.


The support of $f$ is the subset of moved elements:
:$\map \supp f = \set {x \in S: \map f x \ne x}$



",Definition:Support of Permutation,"['Definitions/Mapping Theory', 'Definitions/Permutation Theory']","Let S be a set.

Let f be a permutation on S.


The support of f is the subset of moved elements:
:f = x ∈ S:  f x  x



"
Definition:Support,Support,"Let $S$ be a set

Let $E \subseteq S$ be a subset.

Let $\chi_E: S \to \set {0, 1}$ be the characteristic function of $E$.


The support of $\chi_E$, denoted $\map \supp {\chi_E}$, is the set $E$.

That is:

:$\map \supp {\chi_E} = \set {x \in S: \map {\chi_E} x = 1}$
",Definition:Characteristic Function (Set Theory)/Set/Support,['Definitions/Characteristic Functions of Sets'],"Let S be a set

Let E ⊆ S be a subset.

Let χ_E: S →0, 1 be the characteristic function of E.


The support of χ_E, denoted χ_E, is the set E.

That is:

:χ_E = x ∈ S: χ_E x = 1
"
Definition:Supremum,Supremum,"Let $\struct {S, \preccurlyeq}$ be an ordered set.

Let $T \subseteq S$ admit a supremum $\sup T$.


If $T$ is finite, $\sup T$ is called a finite supremum.
",Definition:Supremum of Set,['Definitions/Suprema'],"Let S, ≼ be an ordered set.

Let T ⊆ S admit a supremum sup T.


If T is finite, sup T is called a finite supremum.
"
Definition:Supremum,Supremum,"Let $T \subseteq \R$ be a subset of the real numbers.


A real number $c \in \R$ is the supremum of $T$ in $\R$ :

:$(1): \quad c$ is an upper bound of $T$ in $\R$
:$(2): \quad c \le d$ for all upper bounds $d$ of $T$ in $\R$.


If there exists a supremum of $T$ (in $\R$), we say that:
:$T$ admits a supremum (in $\R$) or
:$T$ has a supremum (in $\R$).


The supremum of $T$ is denoted $\sup T$ or $\map \sup T$.


=== Definition by Propositional Function ===
",Definition:Supremum of Set/Real Numbers,['Definitions/Suprema'],"Let T ⊆ be a subset of the real numbers.


A real number c ∈ is the supremum of T in  :

:(1):    c is an upper bound of T in 
:(2):    c ≤ d for all upper bounds d of T in .


If there exists a supremum of T (in ), we say that:
:T admits a supremum (in ) or
:T has a supremum (in ).


The supremum of T is denoted sup T or sup T.


=== Definition by Propositional Function ===
"
Definition:Symmetry,Symmetry,"A symmetry mapping of a geometric figure is a bijection from the figure to itself which preserves the distance between points.

In other words, it is a self-congruence.


Intuitively and informally, a symmetry mapping is a movement of the figure so that it looks exactly the same after it has been moved.",Definition:Symmetry Mapping,"['Definitions/Geometry', 'Definitions/Mapping Theory', 'Definitions/Symmetry Mappings']","A symmetry mapping of a geometric figure is a bijection from the figure to itself which preserves the distance between points.

In other words, it is a self-congruence.


Intuitively and informally, a symmetry mapping is a movement of the figure so that it looks exactly the same after it has been moved."
Definition:Symmetry,Symmetry,"Let $P$ be a geometric figure.

Let $S_P$ be the set of all symmetries of $P$.

Let $\struct {S_P, \circ}$ be the algebraic structure such that $\circ$ denotes the composition of mappings.


Then $\struct {S_P, \circ}$ is called the symmetry group of $P$.
",Definition:Symmetry Group,"['Definitions/Examples of Groups', 'Definitions/Symmetry Groups']","Let P be a geometric figure.

Let S_P be the set of all symmetries of P.

Let S_P, ∘ be the algebraic structure such that ∘ denotes the composition of mappings.


Then S_P, ∘ is called the symmetry group of P.
"
Definition:System,System,A system is a configuration of electrical devices designed for a specific utilitarian purpose.,Definition:System (Electronics),"['Definitions/Systems (Electronics)', 'Definitions/Electronics']",A system is a configuration of electrical devices designed for a specific utilitarian purpose.
Definition:System,System,"Let $T = \struct {S, \tau}$ be a topological space.

A neighborhood system is a family $\family {\NN_x}_{x \mathop \in S}$ indexed by points of $S$, such that $\NN_x$ is a local basis at $x$ for $x \in S$.",Definition:Neighborhood System,['Definitions/Topology'],"Let T = S, τ be a topological space.

A neighborhood system is a family _x_x ∈ S indexed by points of S, such that _x is a local basis at x for x ∈ S."
Definition:System,System,"Let $M = \struct {A, d}$ be a metric space.

Let $a \in A$.

Let $\NN_a$ be the set of all neighborhoods of $a$ in $M$.


Then $\NN_a$ is the system of neighborhoods of the point $a$.
",Definition:System of Neighborhoods,"['Definitions/Systems of Neighborhoods', 'Definitions/Neighborhoods']","Let M = A, d be a metric space.

Let a ∈ A.

Let _a be the set of all neighborhoods of a in M.


Then _a is the system of neighborhoods of the point a.
"
Definition:System,System,"Let $\struct {S, \tau}$ be a topological space.

Let $f: S \to S$ be a continuous mapping.


Then $\struct {S, f}$ is called a topological dynamical system.",Definition:Topological Dynamical System,['Definitions/Dynamical Systems Theory'],"Let S, τ be a topological space.

Let f: S → S be a continuous mapping.


Then S, f is called a topological dynamical system."
Definition:System,System,,Definition:Tableau Proof (Propositional Tableaus)/Proof System,"['Definitions/Propositional Tableaus', 'Definitions/Proof Systems']",
Definition:System,System,"A key feature of collations is the presence of methods to collate a number of collations into a new one.

A collection of collations, together with a collection of such collation methods may be called a collation system.


For example, words and the method of concatenation.


Category:Definitions/Collations",Definition:Collation/Collation System,['Definitions/Collations'],"A key feature of collations is the presence of methods to collate a number of collations into a new one.

A collection of collations, together with a collection of such collation methods may be called a collation system.


For example, words and the method of concatenation.


Category:Definitions/Collations"
Definition:System,System,"A formal system is a formal language $\LL$ together with a deductive apparatus for $\LL$.


Let $\FF$ be a formal system consisting of a formal language with deductive apparatus $\DD$.

By applying the formal grammar of $\LL$, one constructs well-formed formulae in $\LL$.

Of such a well-formed formula, one can then use the deductive apparatus $\DD$ to determine whether or not it is a theorem in $\FF$.
A formal system is a formal language $\LL$ together with a deductive apparatus for $\LL$.


Let $\FF$ be a formal system consisting of a formal language with deductive apparatus $\DD$.

By applying the formal grammar of $\LL$, one constructs well-formed formulae in $\LL$.

Of such a well-formed formula, one can then use the deductive apparatus $\DD$ to determine whether or not it is a theorem in $\FF$.
",Definition:Formal System,['Definitions/Formal Systems'],"A formal system is a formal language  together with a deductive apparatus for .


Let  be a formal system consisting of a formal language with deductive apparatus .

By applying the formal grammar of , one constructs well-formed formulae in .

Of such a well-formed formula, one can then use the deductive apparatus  to determine whether or not it is a theorem in .
A formal system is a formal language  together with a deductive apparatus for .


Let  be a formal system consisting of a formal language with deductive apparatus .

By applying the formal grammar of , one constructs well-formed formulae in .

Of such a well-formed formula, one can then use the deductive apparatus  to determine whether or not it is a theorem in .
"
Definition:System,System,"A system of differential equations is a set of simultaneous differential equations.

The solutions for each of the differential equations are in general expected to be consistent.


=== Autonomous System ===


Category:Definitions/Differential Equations
",Definition:Differential Equation/System/Autonomous,['Definitions/Differential Equations'],"A system of differential equations is a set of simultaneous differential equations.

The solutions for each of the differential equations are in general expected to be consistent.


=== Autonomous System ===


Category:Definitions/Differential Equations
"
Definition:System,System,"A Cartesian coordinate system is a coordinate system in which the position of a point is determined by its relation to a set of perpendicular straight lines.

These straight lines are referred to as coordinate axes.
",Definition:Rectangular Coordinate System,['Definitions/Cartesian Coordinate Systems'],"A Cartesian coordinate system is a coordinate system in which the position of a point is determined by its relation to a set of perpendicular straight lines.

These straight lines are referred to as coordinate axes.
"
Definition:System,System,"A mathematical system is a set $\SS = \struct {E, O, A}$ where:

:$E$ is a non-empty set of elements

:$O$ is a set of relations and operations on the elements of $E$

:$A$ is a set of axioms concerning the elements of $E$ and $O$.


=== Abstract System ===

A mathematical system $\SS = \struct {E, O, A}$ is classed as abstract  the elements of $E$ and $O$ are defined only by their properties as specified in $A$.


=== Concrete System ===

A mathematical system $\SS = \struct {E, O, A}$ is classed as concrete  the elements of $E$ and $O$ are understood as objects independently of their existence in $\SS$ itself.


The distinction between abstract and concrete is of questionable value from a modern standpoint, as it is a moot point, for example, as to whether the natural numbers exist independently of Peano's axioms or are specifically defined by them.


=== Algebraic System ===

A mathematical system $\SS = \struct {E, O, A}$ is classed as algebraic  it has many of the properties of the set of integers.

This is usually because such a system is itself an abstraction of certain properties of the integers.

The axioms are usually not considered as separate entities from the operations, as their nature is implicit in the operations themselves.


Specifically, an algebraic system can be defined as follows:

",Definition:Algebraic System,['Definitions/Abstract Algebra'],"A mathematical system is a set = E, O, A where:

:E is a non-empty set of elements

:O is a set of relations and operations on the elements of E

:A is a set of axioms concerning the elements of E and O.


=== Abstract System ===

A mathematical system = E, O, A is classed as abstract  the elements of E and O are defined only by their properties as specified in A.


=== Concrete System ===

A mathematical system = E, O, A is classed as concrete  the elements of E and O are understood as objects independently of their existence in  itself.


The distinction between abstract and concrete is of questionable value from a modern standpoint, as it is a moot point, for example, as to whether the natural numbers exist independently of Peano's axioms or are specifically defined by them.


=== Algebraic System ===

A mathematical system = E, O, A is classed as algebraic  it has many of the properties of the set of integers.

This is usually because such a system is itself an abstraction of certain properties of the integers.

The axioms are usually not considered as separate entities from the operations, as their nature is implicit in the operations themselves.


Specifically, an algebraic system can be defined as follows:

"
Definition:System,System,"An algebraic system is a mathematical system $\SS = \struct {E, O}$ where:

:$E$ is a non-empty set of elements

:$O$ is a set of finitary operations on $E$.
An algebraic system is a mathematical system $\SS = \struct {E, O}$ where:

:$E$ is a non-empty set of elements

:$O$ is a set of finitary operations on $E$.
",Definition:Mathematical System,['Definitions/Abstract Algebra'],"An algebraic system is a mathematical system = E, O where:

:E is a non-empty set of elements

:O is a set of finitary operations on E.
An algebraic system is a mathematical system = E, O where:

:E is a non-empty set of elements

:O is a set of finitary operations on E.
"
Definition:System,System,The solar system is the system of celestial bodies which are under the direct influence of the gravitational field of the sun.,Definition:Solar System,"['Definitions/Astronomy', 'Definitions/Celestial Mechanics', 'Definitions/Solar System']",The solar system is the system of celestial bodies which are under the direct influence of the gravitational field of the sun.
Definition:System,System,A number system is a technique for representing numbers.,Definition:Number System,['Definitions/Number Systems'],A number system is a technique for representing numbers.
Definition:System,System,"The vigesimal system is base $20$ notation.

That is, every number $x \in \R$ is expressed in the form:
:$\ds x = \sum_{j \mathop \in \Z} r_j 20^j$
where:
:$\forall j \in \Z: r_j \in \set {0, 1, \ldots, 19}$",Definition:Vigesimal System,['Definitions/Number Bases'],"The vigesimal system is base 20 notation.

That is, every number x ∈ is expressed in the form:
:x = ∑_j ∈ r_j 20^j
where:
:∀ j ∈: r_j ∈0, 1, …, 19"
Definition:System,System,"A positional number system is a number system with the following properties:

:It has a set of numerals which represent a subset of the numbers.

:The number being represented is written as a string of these numerals, which represent a different value according to their position in the numerals.

The design of the positional number system is such that all numbers can be represented by such a string, which may or may not be infinite in length.
",Definition:Factorial Number System,['Definitions/Number Bases'],"A positional number system is a number system with the following properties:

:It has a set of numerals which represent a subset of the numbers.

:The number being represented is written as a string of these numerals, which represent a different value according to their position in the numerals.

The design of the positional number system is such that all numbers can be represented by such a string, which may or may not be infinite in length.
"
Definition:System,System,"A decimal system is a system of measurement in which the standard multiples and fractions of the units of measurement are powers of $10$.
A system of measurement is a set of fundamental units of measurement with which one can measure any measurable physical property.
",Definition:Decimal System,"['Definitions/Decimal System', 'Definitions/Decimal', 'Definitions/Number Systems', 'Definitions/Units of Measurement']","A decimal system is a system of measurement in which the standard multiples and fractions of the units of measurement are powers of 10.
A system of measurement is a set of fundamental units of measurement with which one can measure any measurable physical property.
"
Definition:System,System,"The combinatorial number system is a system for representing a positive integer $m$ by a sequence of digits which are the upper coefficient of a sequence of $n$ binomial coefficients for some $n \in \Z_{>0}$:

:$m := k_1 k_2 k_3 \ldots k_n$

where:
:$m = \dbinom {k_1} 1 + \dbinom {k_2} 2 + \dbinom {k_3} 3 + \cdots + \dbinom {k_n} n$
:$0 \le k_1 < k_2 < \cdots < k_n$",Definition:Combinatorial Number System,['Definitions/Binomial Coefficients'],"The combinatorial number system is a system for representing a positive integer m by a sequence of digits which are the upper coefficient of a sequence of n binomial coefficients for some n ∈_>0:

:m := k_1 k_2 k_3 … k_n

where:
:m = k_1 1 + k_2 2 + k_3 3 + ⋯ + k_n n
:0 ≤ k_1 < k_2 < ⋯ < k_n"
Definition:System,System,"Zeckendorf representation is a system for representing a positive integer $m$ by a sequence of digits which are the indices of a sequence of $r$ Fibonacci numbers:

:$n := k_1 k_2 k_3 \ldots k_r$

where:
:$n = F_{k_1} + F_{k_2} + F_{k_3} + \cdots + F_{k_r}$
:$k_1 \gg k_2 \gg k_3 \gg \cdots \gg k_r \gg 0$

where $n \gg k$ denotes that $n \ge k + 2$.",Definition:Zeckendorf Representation,"['Definitions/Fibonacci Numbers', 'Definitions/Number Bases']","Zeckendorf representation is a system for representing a positive integer m by a sequence of digits which are the indices of a sequence of r Fibonacci numbers:

:n := k_1 k_2 k_3 … k_r

where:
:n = F_k_1 + F_k_2 + F_k_3 + ⋯ + F_k_r
:k_1 ≫ k_2 ≫ k_3 ≫⋯≫ k_r ≫ 0

where n ≫ k denotes that n ≥ k + 2."
Definition:System,System,"A number system is a technique for representing numbers.
",Definition:Positional Numeral System,['Definitions/Number Systems'],"A number system is a technique for representing numbers.
"
Definition:System,System,"A number system is a technique for representing numbers.
A positional number system is a number system with the following properties:

:It has a set of numerals which represent a subset of the numbers.

:The number being represented is written as a string of these numerals, which represent a different value according to their position in the numerals.

The design of the positional number system is such that all numbers can be represented by such a string, which may or may not be infinite in length.
A positional number system is a number system with the following properties:

:It has a set of numerals which represent a subset of the numbers.

:The number being represented is written as a string of these numerals, which represent a different value according to their position in the numerals.

The design of the positional number system is such that all numbers can be represented by such a string, which may or may not be infinite in length.
Decimal notation is the quotidian technique of expressing numbers in base $10$.

Every number $x \in \R$ is expressed in the form:
:$\ds x = \sum_{j \mathop \in \Z} r_j 10^j$
where:
:$\forall j \in \Z: r_j \in \set {0, 1, 2, 3, 4, 5, 6, 7, 8, 9}$
Sexagesimal notation is the technique of expressing numbers in base $60$.
",Definition:Babylonian Number System,"['Definitions/Numeral Systems', 'Definitions/Babylonian Number System', 'Definitions/Babylonian Mathematics']","A number system is a technique for representing numbers.
A positional number system is a number system with the following properties:

:It has a set of numerals which represent a subset of the numbers.

:The number being represented is written as a string of these numerals, which represent a different value according to their position in the numerals.

The design of the positional number system is such that all numbers can be represented by such a string, which may or may not be infinite in length.
A positional number system is a number system with the following properties:

:It has a set of numerals which represent a subset of the numbers.

:The number being represented is written as a string of these numerals, which represent a different value according to their position in the numerals.

The design of the positional number system is such that all numbers can be represented by such a string, which may or may not be infinite in length.
Decimal notation is the quotidian technique of expressing numbers in base 10.

Every number x ∈ is expressed in the form:
:x = ∑_j ∈ r_j 10^j
where:
:∀ j ∈: r_j ∈0, 1, 2, 3, 4, 5, 6, 7, 8, 9
Sexagesimal notation is the technique of expressing numbers in base 60.
"
Definition:System,System,"The metric system is the colloquial term for the system of measurement based on the metre.

Its main characteristic is that its units are constructed on a decimal system.
A system of measurement is a set of fundamental units of measurement with which one can measure any measurable physical property.
A decimal system is a system of measurement in which the standard multiples and fractions of the units of measurement are powers of $10$.
",Definition:Metric System,"['Definitions/Metric System', 'Definitions/Units of Measurement']","The metric system is the colloquial term for the system of measurement based on the metre.

Its main characteristic is that its units are constructed on a decimal system.
A system of measurement is a set of fundamental units of measurement with which one can measure any measurable physical property.
A decimal system is a system of measurement in which the standard multiples and fractions of the units of measurement are powers of 10.
"
Definition:System,System,"A physical system is a portion of the physical universe which has been chosen for investigation for a particular purpose.


Category:Definitions/Physics",Definition:Physical System,['Definitions/Physics'],"A physical system is a portion of the physical universe which has been chosen for investigation for a particular purpose.


Category:Definitions/Physics"
Definition:System,System,"Let $X$ be a set, and let $\DD \subseteq \powerset X$ be a collection of subsets of $X$.


Then $\DD$ is called a Dynkin system (on $X$)  it satisfies the following conditions:

:$(1): \quad X \in \DD$
:$(2): \quad \forall D \in \DD: X \setminus D \in \DD$
:$(3): \quad$ For all pairwise disjoint sequences $\sequence {D_n}_{n \mathop \in \N}$ in $\DD$, $\ds \bigcup_{n \mathop \in \N} D_n \in \DD$",Definition:Dynkin System,"['Definitions/Set Systems', 'Definitions/Measure Theory', 'Definitions/Dynkin Systems']","Let X be a set, and let ⊆ X be a collection of subsets of X.


Then  is called a Dynkin system (on X)  it satisfies the following conditions:

:(1):    X ∈
:(2):   ∀ D ∈: X ∖ D ∈
:(3): For all pairwise disjoint sequences D_n_n ∈ in , ⋃_n ∈ D_n ∈"
Definition:System,System,"Let $X$ be a set, and let $\DD \subseteq \powerset X$ be a collection of subsets of $X$.


Then $\DD$ is called a Dynkin system (on $X$)  it satisfies the following conditions:

:$(1): \quad X \in \DD$
:$(2): \quad \forall D \in \DD: X \setminus D \in \DD$
:$(3): \quad$ For all pairwise disjoint sequences $\sequence {D_n}_{n \mathop \in \N}$ in $\DD$, $\ds \bigcup_{n \mathop \in \N} D_n \in \DD$
Let $X$ be a set, and let $\DD \subseteq \powerset X$ be a collection of subsets of $X$.


Then $\DD$ is called a Dynkin system (on $X$)  it satisfies the following conditions:

:$(1): \quad X \in \DD$
:$(2): \quad \forall D \in \DD: X \setminus D \in \DD$
:$(3): \quad$ For all pairwise disjoint sequences $\sequence {D_n}_{n \mathop \in \N}$ in $\DD$, $\ds \bigcup_{n \mathop \in \N} D_n \in \DD$
",Definition:Dynkin System Generated by Collection of Subsets,['Definitions/Dynkin Systems'],"Let X be a set, and let ⊆ X be a collection of subsets of X.


Then  is called a Dynkin system (on X)  it satisfies the following conditions:

:(1):    X ∈
:(2):   ∀ D ∈: X ∖ D ∈
:(3): For all pairwise disjoint sequences D_n_n ∈ in , ⋃_n ∈ D_n ∈
Let X be a set, and let ⊆ X be a collection of subsets of X.


Then  is called a Dynkin system (on X)  it satisfies the following conditions:

:(1):    X ∈
:(2):   ∀ D ∈: X ∖ D ∈
:(3): For all pairwise disjoint sequences D_n_n ∈ in , ⋃_n ∈ D_n ∈
"
Definition:System,System,"A dynamical system is a non-linear system in which a function describes the time dependence of a point in a geometrical space.


=== Flow ===

A non-linear system is a system of differential equations which are non-linear.
",Definition:Dynamical System,"['Definitions/Dynamical Systems', 'Definitions/Dynamical Systems Theory']","A dynamical system is a non-linear system in which a function describes the time dependence of a point in a geometrical space.


=== Flow ===

A non-linear system is a system of differential equations which are non-linear.
"
Definition:System,System,"A numeral system is:
:a set of symbols that is used to represent a specific subset of the set of numbers (usually natural numbers), referred to as numerals
:a set of rules which define how to combine the numerals so as to be able to express other numbers.",Definition:Numeral System,"['Definitions/Numeral Systems', 'Definitions/Numbers']","A numeral system is:
:a set of symbols that is used to represent a specific subset of the set of numbers (usually natural numbers), referred to as numerals
:a set of rules which define how to combine the numerals so as to be able to express other numbers."
Definition:System,System,"Let $S$ be a finite set.

Let $\mathscr F$ be a set of subsets of $S$ satisfying the independence system axioms:


The ordered pair $I = \struct {S, \mathscr F}$ is called an independence system on $S$.
",Definition:Independence System,['Definitions/Matroid Theory'],"Let S be a finite set.

Let ℱ be a set of subsets of S satisfying the independence system axioms:


The ordered pair I = S, ℱ is called an independence system on S.
"
Definition:System,System,"Let $L = \left({S, \preceq}\right)$ be an ordered set.


The system of $L$ is an ordered subset of $L$.
",Definition:Closure System,['Definitions/Order Theory'],"Let L = (S, ≼) be an ordered set.


The system of L is an ordered subset of L.
"
Definition:System,System,"There are five main classes of number:

:$(1): \quad$ The natural numbers: $\N = \set {0, 1, 2, 3, \ldots}$
::$(1 \text a): \quad$ The non-zero natural numbers: $\N_{>0} = \set {1, 2, 3, \ldots}$
:$(2): \quad$ The integers: $\Z = \set {\ldots, -3, -2, -1, 0, 1, 2, 3, \ldots}$
:$(3): \quad$ The rational numbers: $\Q = \set {p / q: p, q \in \Z, q \ne 0}$
:$(4): \quad$ The real numbers: $\R = \set {x: x = \sequence {s_n} }$ where $\sequence {s_n}$ is a Cauchy sequence in $\Q$
:$(5): \quad$ The complex numbers: $\C = \set {a + i b: a, b \in \R, i^2 = -1}$


It is possible to categorize numbers further, for example:

:The set of algebraic numbers $\mathbb A$ is the subset of the complex numbers which are roots of polynomials with rational coefficients.  The algebraic numbers include the rational numbers, $\sqrt 2$, and the golden section $\varphi$.

:The set of transcendental numbers is the set of all the real numbers which are not algebraic.  The transcendental numbers include $\pi, e$ and $\sqrt 2^{\sqrt 2}$.

:The set of prime numbers (sometimes referred to as $\mathbb P$) is the subset of the integers which have exactly two positive divisors, $1$ and the number itself.  The first several positive primes are $2, 3, 5, 7, 11, 13, \ldots$",Definition:Number,['Definitions/Numbers'],"There are five main classes of number:

:(1): The natural numbers: = 0, 1, 2, 3, …
::(1 a): The non-zero natural numbers: _>0 = 1, 2, 3, …
:(2): The integers: = …, -3, -2, -1, 0, 1, 2, 3, …
:(3): The rational numbers: = p / q: p, q ∈, q  0
:(4): The real numbers: = x: x = s_n where s_n is a Cauchy sequence in 
:(5): The complex numbers: = a + i b: a, b ∈, i^2 = -1


It is possible to categorize numbers further, for example:

:The set of algebraic numbers 𝔸 is the subset of the complex numbers which are roots of polynomials with rational coefficients.  The algebraic numbers include the rational numbers, √(2), and the golden section φ.

:The set of transcendental numbers is the set of all the real numbers which are not algebraic.  The transcendental numbers include π, e and √(2)^√(2).

:The set of prime numbers (sometimes referred to as ℙ) is the subset of the integers which have exactly two positive divisors, 1 and the number itself.  The first several positive primes are 2, 3, 5, 7, 11, 13, …"
Definition:System,System,"A system of measurement is a set of fundamental units of measurement with which one can measure any measurable physical property.
",Definition:Imperial/Mass,"['Definitions/Imperial', 'Definitions/Mass']","A system of measurement is a set of fundamental units of measurement with which one can measure any measurable physical property.
"
Definition:System,System,"A Steiner triple system of order $v$ is a BIBD with block size $3$, and each pair of points occurring together in exactly $1$ block (called a triple).",Definition:Steiner Triple System,['Definitions/Design Theory'],"A Steiner triple system of order v is a BIBD with block size 3, and each pair of points occurring together in exactly 1 block (called a triple)."
Definition:System,System,"Let $m \in \Z_{\ne 0}$ be a non-zero integer.


Let $S := \set {r_1, r_2, \dotsb, r_s}$ be a set of integers with the properties that:

:$(1): \quad i \ne j \implies r_i \not \equiv r_j \pmod m$

:$(2): \quad \forall n \in \Z: \exists r_i \in S: n \equiv r_i \pmod m$


Then $S$ is a complete residue system modulo $m$.",Definition:Complete Residue System,['Definitions/Residue Classes'],"Let m ∈_ 0 be a non-zero integer.


Let S := r_1, r_2, …, r_s be a set of integers with the properties that:

:(1):    i  j  r_i ≢r_j  m

:(2):   ∀ n ∈: ∃ r_i ∈ S: n ≡ r_i  m


Then S is a complete residue system modulo m."
Definition:System,System,"Let $m \in \Z_{> 0}$ be a (strictly) positive integer.


The reduced residue system modulo $m$, denoted $\Z'_m$, is the set of all residue classes of $k$ (modulo $m$) which are prime to $m$:

:$\Z'_m = \set {\eqclass k m \in \Z_m: k \perp m}$


Thus $\Z'_m$ is the set of all coprime residue classes modulo $m$:
:$\Z'_m = \set {\eqclass {a_1} m, \eqclass {a_2} m, \ldots, \eqclass {a_{\map \phi m} } m}$
where:
:$\forall k: a_k \perp m$
:$\map \phi m$ denotes the Euler phi function of $m$.
",Definition:Reduced Residue System,"['Definitions/Residue Classes', 'Definitions/Reduced Residue Systems', 'Definitions/Modulo Arithmetic']","Let m ∈_> 0 be a (strictly) positive integer.


The reduced residue system modulo m, denoted '_m, is the set of all residue classes of k (modulo m) which are prime to m:

:'_m =  k m ∈_m: k ⊥ m


Thus '_m is the set of all coprime residue classes modulo m:
:'_m = a_1 m, a_2 m, …, a_ϕ m m
where:
:∀ k: a_k ⊥ m
:ϕ m denotes the Euler phi function of m.
"
Definition:System,System,"A set of sets is a set, whose elements are themselves all sets.


Those elements can themselves be assumed to be subsets of some particular fixed set which is frequently referred to as the universe.",Definition:Set of Sets,"['Definitions/Set Systems', 'Definitions/Set Theory']","A set of sets is a set, whose elements are themselves all sets.


Those elements can themselves be assumed to be subsets of some particular fixed set which is frequently referred to as the universe."
Definition:System,System,"A system of simultaneous equations is a set of equations:

:$\forall i \in \set {1, 2, \ldots, m} : \map {f_i} {x_1, x_2, \ldots x_n} = \beta_i$


That is:









=== Linear Equations ===
",Definition:Simultaneous Equations,"['Definitions/Linear Algebra', 'Definitions/Simultaneous Equations']","A system of simultaneous equations is a set of equations:

:∀ i ∈1, 2, …, m : f_ix_1, x_2, … x_n = β_i


That is:









=== Linear Equations ===
"
Definition:System,System,A system of measurement is a set of fundamental units of measurement with which one can measure any measurable physical property.,Definition:System of Measurement,"['Definitions/Physics', 'Definitions/Units of Measurement', 'Definitions/Systems of Measurement']",A system of measurement is a set of fundamental units of measurement with which one can measure any measurable physical property.
Definition:System,System,"A system of measurement is a set of fundamental units of measurement with which one can measure any measurable physical property.
The imperial system is a system of measurement based on traditional established folk measures.

Its base units can be understood as being the FPS Base Units
",Definition:US Volume System,"['Definitions/Systems of Measurement', 'Definitions/US Volume System']","A system of measurement is a set of fundamental units of measurement with which one can measure any measurable physical property.
The imperial system is a system of measurement based on traditional established folk measures.

Its base units can be understood as being the FPS Base Units
"
Definition:System,System,"Let $L = \left({S, \preceq}\right)$ be an ordered set.


The system of $L$ is an ordered subset of $L$.",Definition:System (Order Theory),['Definitions/Order Theory'],"Let L = (S, ≼) be an ordered set.


The system of L is an ordered subset of L."
Definition:System,System,"Let $m \in \Z_{> 0}$.

The least positive reduced residue system modulo $m$ is the set of integers:
:$\set {a_1, a_2, \ldots, a_{\map \phi m} }$
with the following properties:
:$\map \phi m$ is the Euler $\phi$ function
:$\forall i: 0 < a_i < m$
:each of which is prime to $m$
:no two of which are congruent modulo $m$.",Definition:Reduced Residue System/Least Positive,['Definitions/Residue Classes'],"Let m ∈_> 0.

The least positive reduced residue system modulo m is the set of integers:
:a_1, a_2, …, a_ϕ m
with the following properties:
:ϕ m is the Euler ϕ function
:∀ i: 0 < a_i < m
:each of which is prime to m
:no two of which are congruent modulo m."
Definition:System,System,"Let $B = \struct {E, M, \pi, F}$ be a fiber bundle. 

Let $\UU = \set {U_\alpha \subseteq M: \alpha \in I}$ be an open cover of $M$ with index set $I$. 

Let $\struct {U_\alpha, \chi_\alpha}$ be local trivializations for all $\alpha \in I$. 


The set $\set {\struct {U_\alpha, \chi_\alpha}: \alpha \in I}$ is called a system of local trivializations of $E$ on $M$.",Definition:Fiber Bundle/System of Local Trivializations,['Definitions/Fiber Bundles'],"Let B = E, M, π, F be a fiber bundle. 

Let = U_α⊆ M: α∈ I be an open cover of M with index set I. 

Let U_α, χ_α be local trivializations for all α∈ I. 


The set U_α, χ_α: α∈ I is called a system of local trivializations of E on M."
Definition:System,System,"A system of differential equations is autonomous if all of the differential equations which it comprises are themselves autonomous.


Category:Definitions/Differential Equations
",Definition:Differential Equation/System,['Definitions/Differential Equations'],"A system of differential equations is autonomous if all of the differential equations which it comprises are themselves autonomous.


Category:Definitions/Differential Equations
"
Definition:Table,Table,"A reference table is a set of pages, arranged usually in book form, containing arrays of (usually) numbers arranged in rows and columns for ease of look-up.

They have been generally superseded by computers now, but facility in their use is generally considered advantageous.
",Definition:Reference Table,"['Definitions/Reference Tables', 'Definitions/Tools', 'Definitions/Proof Techniques']","A reference table is a set of pages, arranged usually in book form, containing arrays of (usually) numbers arranged in rows and columns for ease of look-up.

They have been generally superseded by computers now, but facility in their use is generally considered advantageous.
"
Definition:Table,Table,"A Cayley table is a technique for describing an algebraic structure (usually a finite group) by putting all the products in a square array:

$\qquad \begin {array} {c|cccc}
\circ & a & b & c & d \\
\hline
a & a & a & b & a \\
b & b & c & a & d \\
c & d & e & f & a \\
d & c & d & a & b \\
\end {array}$


The column down the  denotes the first (leading) operand of the operation.

The row across the top denotes the second (following) operand of the operation.

Thus, in the above Cayley table:
:$c \circ a = d$


If desired, the symbol denoting the operation itself can be put in the upper left corner, but this is not essential if there is no ambiguity.


The order in which the rows and columns are placed is immaterial.

However, it is conventional, when representing an algebraic structure with an identity element, to place that element at the head of the first row and column.


=== Entry ===

A Cayley table is a technique for describing an algebraic structure (usually a finite group) by putting all the products in a square array:

$\qquad \begin {array} {c|cccc}
\circ & a & b & c & d \\
\hline
a & a & a & b & a \\
b & b & c & a & d \\
c & d & e & f & a \\
d & c & d & a & b \\
\end {array}$


The column down the  denotes the first (leading) operand of the operation.

The row across the top denotes the second (following) operand of the operation.

Thus, in the above Cayley table:
:$c \circ a = d$


If desired, the symbol denoting the operation itself can be put in the upper left corner, but this is not essential if there is no ambiguity.


The order in which the rows and columns are placed is immaterial.

However, it is conventional, when representing an algebraic structure with an identity element, to place that element at the head of the first row and column.


=== Entry ===

",Definition:Cayley Table,"['Definitions/Cayley Tables', 'Definitions/Abstract Algebra']","A Cayley table is a technique for describing an algebraic structure (usually a finite group) by putting all the products in a square array:

[ ∘ a b c d; a a a b a; b b c a d; c d e f a; d c d a b;   ]


The column down the  denotes the first (leading) operand of the operation.

The row across the top denotes the second (following) operand of the operation.

Thus, in the above Cayley table:
:c ∘ a = d


If desired, the symbol denoting the operation itself can be put in the upper left corner, but this is not essential if there is no ambiguity.


The order in which the rows and columns are placed is immaterial.

However, it is conventional, when representing an algebraic structure with an identity element, to place that element at the head of the first row and column.


=== Entry ===

A Cayley table is a technique for describing an algebraic structure (usually a finite group) by putting all the products in a square array:

[ ∘ a b c d; a a a b a; b b c a d; c d e f a; d c d a b;   ]


The column down the  denotes the first (leading) operand of the operation.

The row across the top denotes the second (following) operand of the operation.

Thus, in the above Cayley table:
:c ∘ a = d


If desired, the symbol denoting the operation itself can be put in the upper left corner, but this is not essential if there is no ambiguity.


The order in which the rows and columns are placed is immaterial.

However, it is conventional, when representing an algebraic structure with an identity element, to place that element at the head of the first row and column.


=== Entry ===

"
Definition:Tableau Proof,Tableau Proof,"A tableau proof for a proof system is a technique for presenting a logical argument in the form of a formal proof in a straightforward, standard form.

On , the proof system is usually natural deduction.


A tableau proof is a sequence of lines specifying the order of premises, assumptions, inferences and conclusion in support of an argument.


Each line of a tableau proof has a particular format. It consists of the following parts:

* Line: The line number of the proof. This is a simple numbering from 1 upwards.
* Pool: The list of all the lines containing the pool of assumptions for the formula introduced on this line.
* Formula: The propositional formula introduced on this line.
* Rule: The justification for introducing this line. This should be the rule of inference being used to derive this line.
* Depends on: The lines (if any) upon which this line directly depends. For premises and assumptions, this field will be empty.


Optionally, a comment may be added to explicitly point out possible intricacies.

If any assumptions are discharged on a certain line, for the sake of clarity it is preferred that such be mentioned explicitly in a comment.


At the end of a tableau proof, the only lines upon which the proof depends may be those which contain the premises.


=== Length ===
",Definition:Tableau Proof (Formal Systems),['Definitions/Proof Systems'],"A tableau proof for a proof system is a technique for presenting a logical argument in the form of a formal proof in a straightforward, standard form.

On , the proof system is usually natural deduction.


A tableau proof is a sequence of lines specifying the order of premises, assumptions, inferences and conclusion in support of an argument.


Each line of a tableau proof has a particular format. It consists of the following parts:

* Line: The line number of the proof. This is a simple numbering from 1 upwards.
* Pool: The list of all the lines containing the pool of assumptions for the formula introduced on this line.
* Formula: The propositional formula introduced on this line.
* Rule: The justification for introducing this line. This should be the rule of inference being used to derive this line.
* Depends on: The lines (if any) upon which this line directly depends. For premises and assumptions, this field will be empty.


Optionally, a comment may be added to explicitly point out possible intricacies.

If any assumptions are discharged on a certain line, for the sake of clarity it is preferred that such be mentioned explicitly in a comment.


At the end of a tableau proof, the only lines upon which the proof depends may be those which contain the premises.


=== Length ===
"
Definition:Tableau Proof,Tableau Proof,"Let $\mathbf H$ be a set of WFFs of propositional logic.

Let $\mathbf A$ be a WFF.


A tableau proof of $\mathbf A$ from $\mathbf H$ is a tableau confutation of $\mathbf H \cup \set {\neg \mathbf A}$.


This definition also applies when $\mathbf H = \O$.

Then a tableau proof of $\mathbf A$ is a tableau confutation of $\set {\neg \mathbf A}$.


If there exists a tableau proof of $\mathbf A$ from $\mathbf H$, one can write:
:$\mathbf H \vdash_{\mathrm{PT} } \mathbf A$

Specifically, the notation:
:$\vdash_{\mathrm{PT} } \mathbf A$
means that there exists a tableau proof of $\mathbf A$.


=== Proof System ===
",Definition:Tableau Proof (Propositional Tableaus),"['Definitions/Propositional Tableaus', 'Definitions/Proof Systems']","Let 𝐇 be a set of WFFs of propositional logic.

Let 𝐀 be a WFF.


A tableau proof of 𝐀 from 𝐇 is a tableau confutation of 𝐇∪𝐀.


This definition also applies when 𝐇 = Ø.

Then a tableau proof of 𝐀 is a tableau confutation of 𝐀.


If there exists a tableau proof of 𝐀 from 𝐇, one can write:
:𝐇⊢_PT𝐀

Specifically, the notation:
:⊢_PT𝐀
means that there exists a tableau proof of 𝐀.


=== Proof System ===
"
Definition:Tangent,Tangent,":

In the above right triangle, we are concerned about the angle $\theta$.

The tangent of $\angle \theta$ is defined as being $\dfrac{\text{Opposite}} {\text{Adjacent}}$.",Definition:Tangent Function/Definition from Triangle,['Definitions/Tangent Function'],":

In the above right triangle, we are concerned about the angle θ.

The tangent of ∠θ is defined as being OppositeAdjacent."
Definition:Tangent,Tangent,"Let $x \in \R$ be a real number.

The real function $\tan x$ is defined as:

:$\tan x = \dfrac {\sin x} {\cos x}$

where:
: $\sin x$ is the sine of $x$
: $\cos x$ is the cosine of $x$.

The definition is valid for all $x \in \R$ such that $\cos x \ne 0$.",Definition:Tangent Function/Real,['Definitions/Tangent Function'],"Let x ∈ be a real number.

The real function tan x is defined as:

:tan x = sin xcos x

where:
: sin x is the sine of x
: cos x is the cosine of x.

The definition is valid for all x ∈ such that cos x  0."
Definition:Tangent,Tangent,"Let $z \in \C$ be a complex number.

The complex function $\tan z$ is defined as:

:$\tan z = \dfrac {\sin z} {\cos z}$

where:
: $\sin z$ is the sine of $z$
: $\cos z$ is the cosine of $z$.

The definition is valid for all $z \in \C$ such that $\cos z \ne 0$.",Definition:Tangent Function/Complex,['Definitions/Tangent Function'],"Let z ∈ be a complex number.

The complex function tan z is defined as:

:tan z = sin zcos z

where:
: sin z is the sine of z
: cos z is the cosine of z.

The definition is valid for all z ∈ such that cos z  0."
Definition:Tensor Product,Tensor Product,"Let $A$ and $B$ be abelian groups.


=== Definition 1: by universal property ===

Their tensor product is a pair $\struct {A \otimes B, \theta}$ where:
:$A \otimes B$ is an abelian group
:$\theta : A \times B \to A \otimes B$ is a biadditive mapping such that, for every ordered pair $\struct {C, \omega}$ where:
:$C$ is an abelian group
:$\omega : A \times B \to C$ is a biadditive mapping
there exists a unique group homomorphism $g : A \otimes B \to C$ such that $\omega = g \circ \theta$.


=== Definition 2: construction ===

Their tensor product is the pair $\struct {A \otimes B, \theta}$ where:
:$A \otimes B$ is the quotient of the free abelian group $\Z^{\paren {A \times B} }$ on the cartesian product $A \times B$ by the subgroup generated by the elements of the form:
:::$\tuple {a_1 + a_2, b} - \tuple {a_1, b} - \tuple {a_2, b}$
:::$\tuple {a, b_1 + b_2} - \tuple {a, b_1} - \tuple {a, b_2}$
::for $a, a_1, a_2 \in A$, $b, b_1, b_2 \in B$, where we denote $\tuple {a, b}$ for its image under the canonical mapping $A \times B \to \Z^{\paren {A \times B} }$.
:$\theta : A \times B \to A \otimes B$ is the composition of the canonical mapping $A \times B \to \Z^{\paren {A \times B} }$ with the quotient group epimorphism $\Z^{\paren {A \times B} } \to A \otimes B$.",Definition:Tensor Product of Abelian Groups,['Definitions/Abelian Groups'],"Let A and B be abelian groups.


=== Definition 1: by universal property ===

Their tensor product is a pair A ⊗ B, θ where:
:A ⊗ B is an abelian group
:θ : A × B → A ⊗ B is a biadditive mapping such that, for every ordered pair C, ω where:
:C is an abelian group
:ω : A × B → C is a biadditive mapping
there exists a unique group homomorphism g : A ⊗ B → C such that ω = g ∘θ.


=== Definition 2: construction ===

Their tensor product is the pair A ⊗ B, θ where:
:A ⊗ B is the quotient of the free abelian group ^A × B on the cartesian product A × B by the subgroup generated by the elements of the form:
:::a_1 + a_2, b - a_1, b - a_2, b
:::a, b_1 + b_2 - a, b_1 - a, b_2
::for a, a_1, a_2 ∈ A, b, b_1, b_2 ∈ B, where we denote a, b for its image under the canonical mapping A × B →^A × B.
:θ : A × B → A ⊗ B is the composition of the canonical mapping A × B →^A × B with the quotient group epimorphism ^A × B→ A ⊗ B."
Definition:Tensor Product,Tensor Product,"=== Commutative ring ===

Let $R$ be a commutative ring with unity.

Let $M$ and $N$ be $R$-modules.


=== Definition 1 ===

Their tensor product is a pair $\struct {M \otimes_R N, \theta}$ where:
:$M \otimes_R N$ is an $R$-module
:$\theta : M \times N \to M \otimes_R N$ is an $R$-bilinear mapping
satisfying the following universal property:
:For every pair $\struct {P, \omega}$ of an $R$-module and an $R$-bilinear mapping $\omega : M \times N \to P$, there exists a unique $R$-module homomorphism $f: M \otimes_R N \to P$ with $\omega = f \circ \theta$.


=== Definition 2 ===

Their tensor product is the pair $\struct {M \otimes_R N, \theta}$, where:
:$M \otimes_R N$ is the quotient of the free $R$-module $R^{\paren {M \times N} }$ on the direct product $M \times N$, by the submodule generated by the set of elements of the form:
::$\tuple {\lambda m_1 + m_2, n} - \lambda \tuple {m_1, n} - \tuple {m_2, n}$
::$\tuple {m, \lambda n_1 + n_2} - \lambda \tuple {m, n_1} - \tuple {m, n_2}$
::for $m, m_1, m_2 \in M$, $n, n_1, n_2 \in N$ and $\lambda \in R$, where we denote $\tuple {m, n}$ for its image under the canonical mapping $M \times N \to R^{\paren {M \times N} }$.
:$\theta : M \times N \to M \otimes_R N$ is the composition of the canonical mapping $M \times N \to R^{\paren {M \times N} }$ with the quotient module homomorphism $R^{\paren {M \times N} } \to M \otimes_R N$.


=== Noncommutative ring ===

Let $R$ be a ring.

Let $M$ be a $R$-right module.

Let $N$ be a $R$-left module.


First construct a left module as a direct sum of all free left modules with a basis that is a single ordered pair in $M \times N$ which is denoted $\map R {m, n}$.

:$T = \ds \bigoplus_{s \mathop \in M \mathop \times N} R s$


That this is indeed a module is demonstrated in Tensor Product is Module.


Next for all $m, m' \in M$, $n, n' \in N$ and $r \in R$ we construct the following free left modules.

:$L_{m, m', n}$ with a basis of $\tuple {m + m', n}$, $\tuple {m, n}$ and $\tuple {m', n}$
:$R_{m, n, n'}$ with a basis of $\tuple {m, n + n'}$, $\tuple {m, n}$ and $\tuple {m, n'}$
:$A_{r, m, n}$ with a basis of $r \tuple {m, n}$ and $\tuple {m r, n}$
:$B_{r, m, n}$ with a basis of $r \tuple {m, n}$ and $\tuple {m, r n}$

Let:

:$D = \ds \map {\bigoplus_{r \in R, n, n' \in N, m, m' \in M} } {L_{m, m', n} \oplus R_{m, n, n'} \oplus A_{r, m, n} \oplus B_{r, m, n} }$

The tensor product $M \otimes_R N$ is then our quotient module $T / D$.",Definition:Tensor Product of Modules,"['Definitions/Module Theory', 'Definitions/Tensor Algebra', 'Definitions/Homological Algebra']","=== Commutative ring ===

Let R be a commutative ring with unity.

Let M and N be R-modules.


=== Definition 1 ===

Their tensor product is a pair M ⊗_R N, θ where:
:M ⊗_R N is an R-module
:θ : M × N → M ⊗_R N is an R-bilinear mapping
satisfying the following universal property:
:For every pair P, ω of an R-module and an R-bilinear mapping ω : M × N → P, there exists a unique R-module homomorphism f: M ⊗_R N → P with ω = f ∘θ.


=== Definition 2 ===

Their tensor product is the pair M ⊗_R N, θ, where:
:M ⊗_R N is the quotient of the free R-module R^M × N on the direct product M × N, by the submodule generated by the set of elements of the form:
::λ m_1 + m_2, n - λm_1, n - m_2, n
::m, λ n_1 + n_2 - λm, n_1 - m, n_2
::for m, m_1, m_2 ∈ M, n, n_1, n_2 ∈ N and λ∈ R, where we denote m, n for its image under the canonical mapping M × N → R^M × N.
:θ : M × N → M ⊗_R N is the composition of the canonical mapping M × N → R^M × N with the quotient module homomorphism R^M × N→ M ⊗_R N.


=== Noncommutative ring ===

Let R be a ring.

Let M be a R-right module.

Let N be a R-left module.


First construct a left module as a direct sum of all free left modules with a basis that is a single ordered pair in M × N which is denoted R m, n.

:T = ⊕_s ∈ M × N R s


That this is indeed a module is demonstrated in Tensor Product is Module.


Next for all m, m' ∈ M, n, n' ∈ N and r ∈ R we construct the following free left modules.

:L_m, m', n with a basis of m + m', n, m, n and m', n
:R_m, n, n' with a basis of m, n + n', m, n and m, n'
:A_r, m, n with a basis of r m, n and m r, n
:B_r, m, n with a basis of r m, n and m, r n

Let:

:D = ⊕_r ∈ R, n, n' ∈ N, m, m' ∈ ML_m, m', n⊕ R_m, n, n'⊕ A_r, m, n⊕ B_r, m, n

The tensor product M ⊗_R N is then our quotient module T / D."
Definition:Term,Term,"Part of specifying the language of predicate logic $\LL_1$ is the introduction of terms.


The terms of $\LL_1$ are identified by the following bottom-up grammar:






Colloquially, we can think of a term as an expression signifying an object.",Definition:Language of Predicate Logic/Formal Grammar/Term,"['Definitions/Predicate Logic', 'Definitions/Language of Predicate Logic']","Part of specifying the language of predicate logic _1 is the introduction of terms.


The terms of _1 are identified by the following bottom-up grammar:






Colloquially, we can think of a term as an expression signifying an object."
Definition:Term,Term,"A term is either a variable or a constant.


Let $a \circ b$ be an expression.

Then each of $a$ and $b$ are known as the terms of the expression.


The word term is usually used when the operation $\circ$ is addition, that is $+$.
",Definition:Logical Term,['Definitions/Predicate Logic'],"A term is either a variable or a constant.


Let a ∘ b be an expression.

Then each of a and b are known as the terms of the expression.


The word term is usually used when the operation ∘ is addition, that is +.
"
Definition:Term,Term,"A term is either a variable or a constant.


Let $a \circ b$ be an expression.

Then each of $a$ and $b$ are known as the terms of the expression.


The word term is usually used when the operation $\circ$ is addition, that is $+$.
A term is either a variable or a constant.


Let $a \circ b$ be an expression.

Then each of $a$ and $b$ are known as the terms of the expression.


The word term is usually used when the operation $\circ$ is addition, that is $+$.
A term is either a variable or a constant.


Let $a \circ b$ be an expression.

Then each of $a$ and $b$ are known as the terms of the expression.


The word term is usually used when the operation $\circ$ is addition, that is $+$.
",Definition:Term of Expression,"['Definitions/Algebra', 'Definitions/Symbolic Logic']","A term is either a variable or a constant.


Let a ∘ b be an expression.

Then each of a and b are known as the terms of the expression.


The word term is usually used when the operation ∘ is addition, that is +.
A term is either a variable or a constant.


Let a ∘ b be an expression.

Then each of a and b are known as the terms of the expression.


The word term is usually used when the operation ∘ is addition, that is +.
A term is either a variable or a constant.


Let a ∘ b be an expression.

Then each of a and b are known as the terms of the expression.


The word term is usually used when the operation ∘ is addition, that is +.
"
Definition:Term,Term,"Let $P = a_n x^n + a_{n - 1} x^{n - 1} + \cdots + a_1 x + a_0$ be a polynomial.

Each of the expressions $a_i x^i$, for $0 \le i \le n$, is referred to as a term of $P$.",Definition:Polynomial/Term,['Definitions/Polynomial Theory'],"Let P = a_n x^n + a_n - 1 x^n - 1 + ⋯ + a_1 x + a_0 be a polynomial.

Each of the expressions a_i x^i, for 0 ≤ i ≤ n, is referred to as a term of P."
Definition:Term,Term,"The elements of a sequence are known as its terms.


Let $\sequence {x_n}$ be a sequence.

Then the $k$th term of $\sequence {x_n}$ is the ordered pair $\tuple {k, x_k}$.


=== Index ===

The elements of a sequence are known as its terms.


Let $\sequence {x_n}$ be a sequence.

Then the $k$th term of $\sequence {x_n}$ is the ordered pair $\tuple {k, x_k}$.


=== Index ===

",Definition:Term of Sequence,['Definitions/Sequences'],"The elements of a sequence are known as its terms.


Let x_n be a sequence.

Then the kth term of x_n is the ordered pair k, x_k.


=== Index ===

The elements of a sequence are known as its terms.


Let x_n be a sequence.

Then the kth term of x_n is the ordered pair k, x_k.


=== Index ===

"
Definition:Term,Term,"Let $I$ and $S$ be sets.

Let $x: I \to S$ be a mapping.

Let $x_i$ denote the image of an element $i \in I$ of the domain $I$ of $x$.

Let $\family {x_i}_{i \mathop \in I}$ denote the set of the images of all the element $i \in I$ under $x$.


The image of $x$ at an index $i$ is referred to as a term of the (indexed) family, and is denoted $x_i$.


=== Notation ===

",Definition:Indexing Set/Term,['Definitions/Indexed Families'],"Let I and S be sets.

Let x: I → S be a mapping.

Let x_i denote the image of an element i ∈ I of the domain I of x.

Let x_i_i ∈ I denote the set of the images of all the element i ∈ I under x.


The image of x at an index i is referred to as a term of the (indexed) family, and is denoted x_i.


=== Notation ===

"
Definition:Term,Term,"Let $n \in \N_{>0}$.

Let $\sequence {a_k}_{k \mathop \in \N^*_n}$ be an ordered tuple.

The ordered pair $\tuple {k, a_k}$ is called the $k$th term of the ordered tuple for each $k \in \N^*_n$.",Definition:Ordered Tuple/Term,['Definitions/Ordered Tuples'],"Let n ∈_>0.

Let a_k_k ∈^*_n be an ordered tuple.

The ordered pair k, a_k is called the kth term of the ordered tuple for each k ∈^*_n."
Definition:Torsion,Torsion,"Let $G$ be a group.


An element of finite order of $G$ is also known as a torsion element of $G$.


Category:Definitions/Order of Group Elements",Definition:Order of Group Element/Finite/Also known as,['Definitions/Order of Group Elements'],"Let G be a group.


An element of finite order of G is also known as a torsion element of G.


Category:Definitions/Order of Group Elements"
Definition:Torsion,Torsion,"Let $G$ be a group.


An element of finite order of $G$ is also known as a torsion element of $G$.


Category:Definitions/Order of Group Elements
",Definition:Torsion Subgroup,['Definitions/Examples of Subgroups'],"Let G be a group.


An element of finite order of G is also known as a torsion element of G.


Category:Definitions/Order of Group Elements
"
Definition:Torsion,Torsion,"Let $R$ be a commutative ring with unity.

Let $M$ be a unitary module over $R$.

Let $m \in M$.


Then $m$ is a torsion element  there exists a regular element $a \in R$ with $a m = 0$.",Definition:Torsion Element of Module,['Definitions/Module Theory'],"Let R be a commutative ring with unity.

Let M be a unitary module over R.

Let m ∈ M.


Then m is a torsion element  there exists a regular element a ∈ R with a m = 0."
Definition:Torsion,Torsion,"Let $R$ be a commutative ring with unity.

Let $M$ be a unitary module over $R$.

Let $m \in M$.


Then $m$ is a torsion element  there exists a regular element $a \in R$ with $a m = 0$.
",Definition:Torsion Submodule,['Definitions/Module Theory'],"Let R be a commutative ring with unity.

Let M be a unitary module over R.

Let m ∈ M.


Then m is a torsion element  there exists a regular element a ∈ R with a m = 0.
"
Definition:Torsion,Torsion,"Let $R$ be a commutative ring with unity.

Let $M$ be a unitary module over $R$.

Let $m \in M$.


Then $m$ is a torsion element  there exists a regular element $a \in R$ with $a m = 0$.
Let $R$ be a commutative ring with unity.

Let $M$ be a unitary module over $R$.


The torsion submodule $T(M)$ of $M$ is the submodule of all torsion elements of $M$.
",Definition:Torsion Module,"['Definitions/Module Theory', 'Definitions/Commutative Algebra']","Let R be a commutative ring with unity.

Let M be a unitary module over R.

Let m ∈ M.


Then m is a torsion element  there exists a regular element a ∈ R with a m = 0.
Let R be a commutative ring with unity.

Let M be a unitary module over R.


The torsion submodule T(M) of M is the submodule of all torsion elements of M.
"
Definition:Trace,Trace,"Let $V$ be a vector space.

Let $A: V \to V$ be a linear operator of $V$.


The trace of $A$ is the trace of the matrix of $A$ with respect to some basis.


Category:Definitions/Linear Algebra
",Definition:Trace (Field Theory),['Definitions/Field Theory'],"Let V be a vector space.

Let A: V → V be a linear operator of V.


The trace of A is the trace of the matrix of A with respect to some basis.


Category:Definitions/Linear Algebra
"
Definition:Trace,Trace,"Let $A = \sqbrk a_n$ be a square matrix of order $n$.


The trace of $A$ is:

:$\ds \map \tr A = \sum_{i \mathop = 1}^n a_{ii}$


=== Using Einstein Summation Convention ===

",Definition:Trace (Linear Algebra)/Matrix,"['Definitions/Traces of Matrices', 'Definitions/Matrix Theory']","Let A =  a_n be a square matrix of order n.


The trace of A is:

:A = ∑_i  = 1^n a_ii


=== Using Einstein Summation Convention ===

"
Definition:Trace,Trace,"Let $X$ be a set, and let $\Sigma$ be a $\sigma$-algebra on $X$.

Let $E \subseteq X$ be a subset of $X$.


Then the trace $\sigma$-algebra (of $E$ in $\Sigma$), $\Sigma_E$, is defined as:

:$\Sigma_E := \set {E \cap S: S \in \Sigma}$


It is a $\sigma$-algebra on $E$, as proved on Trace $\sigma$-Algebra is $\sigma$-Algebra.",Definition:Trace Sigma-Algebra,"['Definitions/Trace Sigma-Algebras', 'Definitions/Sigma-Algebras', 'Definitions/Trace Sigma-Algebras']","Let X be a set, and let Σ be a σ-algebra on X.

Let E ⊆ X be a subset of X.


Then the trace σ-algebra (of E in Σ), Σ_E, is defined as:

:Σ_E := E ∩ S: S ∈Σ


It is a σ-algebra on E, as proved on Trace σ-Algebra is σ-Algebra."
Definition:Trace,Trace,"Let $P$ be a URM program.

The trace table of $P$ consists of:
* The stage of computation;
* The number of the instruction of $P$ that is about to be performed;
* A list of the contents of all the registers used by $P$ at this point.

Thus the trace table is a list of the states of the URM program at each stage.",Definition:Trace Table,['Definitions/Mathematical Logic'],"Let P be a URM program.

The trace table of P consists of:
* The stage of computation;
* The number of the instruction of P that is about to be performed;
* A list of the contents of all the registers used by P at this point.

Thus the trace table is a list of the states of the URM program at each stage."
Definition:Trace,Trace,,Definition:Trace of Tensor,['Definitions/Riemannian Geometry'],
Definition:Transcendental,Transcendental,"Let $\struct {R, +, \circ}$ be a commutative ring with unity whose zero is $0_R$ and whose unity is $1_R$.

Let $\struct {D, +, \circ}$ be an integral subdomain of $R$.

Let $x \in R$.


Then $x$ is transcendental over $D$ :
:$\ds \forall n \in \Z_{\ge 0}: \sum_{k \mathop = 0}^n a_k \circ x^k = 0_R \implies \forall k: 0 \le k \le n: a_k = 0_R$


That is, $x$ is transcendental over $D$  the only way to express $0_R$ as a polynomial in $x$ over $D$ is by the null polynomial.
",Definition:Transcendental (Abstract Algebra),"['Definitions/Ring Theory', 'Definitions/Field Extensions', 'Definitions/Polynomial Theory']","Let R, +, ∘ be a commutative ring with unity whose zero is 0_R and whose unity is 1_R.

Let D, +, ∘ be an integral subdomain of R.

Let x ∈ R.


Then x is transcendental over D :
:∀ n ∈_≥ 0: ∑_k  = 0^n a_k ∘ x^k = 0_R ∀ k: 0 ≤ k ≤ n: a_k = 0_R


That is, x is transcendental over D  the only way to express 0_R as a polynomial in x over D is by the null polynomial.
"
Definition:Transcendental,Transcendental,"Let $E / F$ be a field extension.

Let $\alpha \in E$.


Then $\alpha$ is transcendental over $F$ :
: $\nexists \map f x \in F \sqbrk x \setminus \set 0: \map f \alpha = 0$
where $\map f x$ denotes a polynomial in $x$ over $F$.
Let $E / F$ be a field extension.

Let $\alpha \in E$.


Then $\alpha$ is transcendental over $F$ :
: $\nexists \map f x \in F \sqbrk x \setminus \set 0: \map f \alpha = 0$
where $\map f x$ denotes a polynomial in $x$ over $F$.
",Definition:Transcendental (Abstract Algebra)/Field Extension,"['Definitions/Field Extensions', 'Definitions/Polynomial Theory']","Let E / F be a field extension.

Let α∈ E.


Then α is transcendental over F :
: ∄ f x ∈ F  x ∖ 0:  f α = 0
where f x denotes a polynomial in x over F.
Let E / F be a field extension.

Let α∈ E.


Then α is transcendental over F :
: ∄ f x ∈ F  x ∖ 0:  f α = 0
where f x denotes a polynomial in x over F.
"
Definition:Transcendental,Transcendental,"Let $\struct {R, +, \circ}$ be a commutative ring with unity whose zero is $0_R$ and whose unity is $1_R$.

Let $\struct {D, +, \circ}$ be an integral subdomain of $R$.

Let $x \in R$.


Then $x$ is transcendental over $D$ :
:$\ds \forall n \in \Z_{\ge 0}: \sum_{k \mathop = 0}^n a_k \circ x^k = 0_R \implies \forall k: 0 \le k \le n: a_k = 0_R$


That is, $x$ is transcendental over $D$  the only way to express $0_R$ as a polynomial in $x$ over $D$ is by the null polynomial.",Definition:Transcendental (Abstract Algebra)/Ring,"['Definitions/Ring Theory', 'Definitions/Polynomial Theory']","Let R, +, ∘ be a commutative ring with unity whose zero is 0_R and whose unity is 1_R.

Let D, +, ∘ be an integral subdomain of R.

Let x ∈ R.


Then x is transcendental over D :
:∀ n ∈_≥ 0: ∑_k  = 0^n a_k ∘ x^k = 0_R ∀ k: 0 ≤ k ≤ n: a_k = 0_R


That is, x is transcendental over D  the only way to express 0_R as a polynomial in x over D is by the null polynomial."
Definition:Transcendental,Transcendental,"Let $F$ be a field.

Let $z$ be a complex number.

$z$ is a transcendental number over $F$  $z$ cannot be expressed as a root of a polynomial with coefficients in $F$.
",Definition:Transcendental Number,"['Definitions/Transcendental Numbers', 'Definitions/Numbers', 'Definitions/Analysis']","Let F be a field.

Let z be a complex number.

z is a transcendental number over F  z cannot be expressed as a root of a polynomial with coefficients in F.
"
Definition:Transcendental,Transcendental,"Let $f$ be an entire function that has an essential singularity at $\infty$.

Then $f$ is a transcendental entire function.
",Definition:Entire Function/Transcendental,"['Definitions/Entire Functions', 'Definitions/Complex Analysis']","Let f be an entire function that has an essential singularity at ∞.

Then f is a transcendental entire function.
"
Definition:Transfer Function,Transfer Function,"Let $C, D \subseteq \C$ with $z \in C \implies z + 1 \in C$.

Let $F: C \to D$ and $H: D \to D$ be holomorphic functions.

Let $\map H {\map F z} = \map F {z + 1}$ for all $z \in C$.

Then $F$ is said to be a superfunction of $H$, and $H$ is called a transfer function of $F$.

That is, superfunctions are iterations of transfer functions.
Let $C, D \subseteq \C$ with $z \in C \implies z + 1 \in C$.

Let $F: C \to D$ and $H: D \to D$ be holomorphic functions.

Let $\map H {\map F z} = \map F {z + 1}$ for all $z \in C$.

Then $F$ is said to be a superfunction of $H$, and $H$ is called a transfer function of $F$.

That is, superfunctions are iterations of transfer functions.
",Definition:Superfunction,['Definitions/Number Theory'],"Let C, D ⊆ with z ∈ C  z + 1 ∈ C.

Let F: C → D and H: D → D be holomorphic functions.

Let H  F z =  F z + 1 for all z ∈ C.

Then F is said to be a superfunction of H, and H is called a transfer function of F.

That is, superfunctions are iterations of transfer functions.
Let C, D ⊆ with z ∈ C  z + 1 ∈ C.

Let F: C → D and H: D → D be holomorphic functions.

Let H  F z =  F z + 1 for all z ∈ C.

Then F is said to be a superfunction of H, and H is called a transfer function of F.

That is, superfunctions are iterations of transfer functions.
"
Definition:Transfer Function,Transfer Function,"A transfer function, in the context of time series analysis, is a function of time which theoretically models the future output for each possible input.
",Definition:Transfer Function (Time Series Analysis),['Definitions/Time Series Analysis'],"A transfer function, in the context of time series analysis, is a function of time which theoretically models the future output for each possible input.
"
Definition:Transformation,Transformation,"Let $R$ be a commutative ring with unity.

Let $G$ be an $R$-module.

Let $G^*$ be the algebraic dual of $G$.

Let $G^{**}$ be the double dual of $G^*$.


For each $x \in G$, we define the mapping $x^\wedge: G^* \to R$ as:
:$\forall t \in G^*: \map {x^\wedge} t = \map t x$


The mapping $J: G \to G^{**}$ defined as:
:$\forall x \in G: \map J x = x^\wedge$
is called the evaluation linear transformation from $G$ into $G^{**}$.


It is usual to denote the mapping $t: G^* \to R$ as follows:

:$\forall x \in G, t \in G^*: \innerprod x t := \map t x$",Definition:Evaluation Linear Transformation/Module Theory,"['Definitions/Evaluation Linear Transformations', 'Definitions/Module Theory']","Let R be a commutative ring with unity.

Let G be an R-module.

Let G^* be the algebraic dual of G.

Let G^** be the double dual of G^*.


For each x ∈ G, we define the mapping x^∧: G^* → R as:
:∀ t ∈ G^*: x^∧ t =  t x


The mapping J: G → G^** defined as:
:∀ x ∈ G:  J x = x^∧
is called the evaluation linear transformation from G into G^**.


It is usual to denote the mapping t: G^* → R as follows:

:∀ x ∈ G, t ∈ G^*:  x t :=  t x"
Definition:Transformation,Transformation,"A complex transformation is a mapping on the complex plane $f: \C \to \C$ which is specifically not a multifunction.


Let $z = x + i y$ be a complex variable.

Let $w = u + i v = \map f z$.


Then $w$ can be expressed as:
:$u + i v = \map f {x + i y}$

such that:
:$u = \map u {x, y}$
and:
:$v = \map v {x, y}$
are real functions of two variables.


Thus a point $P = \tuple {x, y}$ in the complex plane is transformed to a point $P' = \tuple {\map u {x, y}, \map v {x, y} }$ by $f$.

Thus $P'$ is the image of $P$ under $f$.",Definition:Complex Transformation,['Definitions/Complex Functions'],"A complex transformation is a mapping on the complex plane f: → which is specifically not a multifunction.


Let z = x + i y be a complex variable.

Let w = u + i v =  f z.


Then w can be expressed as:
:u + i v =  f x + i y

such that:
:u =  u x, y
and:
:v =  v x, y
are real functions of two variables.


Thus a point P = x, y in the complex plane is transformed to a point P' =  u x, y,  v x, y by f.

Thus P' is the image of P under f."
Definition:Transformation,Transformation,"Let $G$ be a group whose identity is $e$.

Let $X$ be a set.

Let $\phi: G \times X \to X$ be a group action.


Then $G$ is an effective transformation group for $\phi$  $\phi$ is faithful.",Definition:Effective Transformation Group,['Definitions/Group Actions'],"Let G be a group whose identity is e.

Let X be a set.

Let ϕ: G × X → X be a group action.


Then G is an effective transformation group for ϕ  ϕ is faithful."
Definition:Transformation,Transformation,"Let $\mathbf C$ and $\mathbf D$ be categories.

Let $F, G : \mathbf C \to \mathbf D$ be covariant functors.


A natural transformation $\eta$ from $F$ to $G$ is a mapping on $\mathbf C$ such that:


:$(1): \quad$ For all $x \in \mathbf C$, $\eta_x$ is a morphism from $\map F x$ to $\map G x$.

:$(2): \quad$ For all $x, y \in C$ and morphism $f: x \to y$, the following diagram commutes:


$\quad \quad \xymatrix{
\map F x \ar[d]^{\eta_x} \ar[r]^{\map F f} & \map F y \ar[d]^{\eta_y} \\
\map G x \ar[r]^{\map G f}                 & \map G y}$",Definition:Natural Transformation/Covariant Functors,['Definitions/Natural Transformations'],"Let 𝐂 and 𝐃 be categories.

Let F, G : 𝐂→𝐃 be covariant functors.


A natural transformation η from F to G is a mapping on 𝐂 such that:


:(1): For all x ∈𝐂, η_x is a morphism from F x to G x.

:(2): For all x, y ∈ C and morphism f: x → y, the following diagram commutes:


F x [d]^η_x[r]^ F f    F y [d]^η_y
 G x [r]^ G f    G y"
Definition:Transformation,Transformation,"Let $\mathbf C$ and $\mathbf D$ be categories.

Let $F, G: \mathbf C \to \mathbf D$ be contravariant functors.


A natural transformation $\eta$ from $F$ to $G$ is a mapping on $\mathbf C$ such that:

:$(1): \quad$ For all $x \in \mathbf C$, $\eta_x$ is a morphism from $\map F x$ to $\map G x$.

:$(2): \quad$ For all $x, y \in C$ and morphism $f: x \to y$, the following diagram commutes:


$\quad \quad \xymatrix{
\map F x \ar[d]^{\eta_x} & \map F y \ar[d]^{\eta_y} \ar[l]^{\map F f}  \\
\map G x                 & \map G y \ar[l]^{\map G f} }$",Definition:Natural Transformation/Contravariant Functors,['Definitions/Natural Transformations'],"Let 𝐂 and 𝐃 be categories.

Let F, G: 𝐂→𝐃 be contravariant functors.


A natural transformation η from F to G is a mapping on 𝐂 such that:

:(1): For all x ∈𝐂, η_x is a morphism from F x to G x.

:(2): For all x, y ∈ C and morphism f: x → y, the following diagram commutes:


F x [d]^η_x    F y [d]^η_y[l]^ F f
 G x                     G y [l]^ G f"
Definition:Transformation,Transformation,"The Lorentz transformation is a transformation which changes the position and motion in one inertial frame of reference to a different inertial frame of reference.

The equations governing such a transformation must satisfy the postulates of the special theory of relativity.




The Lorentz transformation is a transformation which changes the position and motion in one inertial frame of reference to a different inertial frame of reference.

The equations governing such a transformation must satisfy the postulates of the special theory of relativity.



",Definition:Lorentz Transformation,"['Definitions/Lorentz Transformations', 'Definitions/Special Theory of Relativity']","The Lorentz transformation is a transformation which changes the position and motion in one inertial frame of reference to a different inertial frame of reference.

The equations governing such a transformation must satisfy the postulates of the special theory of relativity.




The Lorentz transformation is a transformation which changes the position and motion in one inertial frame of reference to a different inertial frame of reference.

The equations governing such a transformation must satisfy the postulates of the special theory of relativity.



"
Definition:Transformation,Transformation,"Let $\map f x$ be a polynomial over a field $k$:

:$\map f x = a_n x^n + a_{n - 1} x^{n - 1} + a_{n - 2} x^{n - 2} + \cdots + a_1 x + a_0$


Then the Tschirnhaus transformation is the linear substitution $x = y - \dfrac {a_{n - 1} } {n a_n}$.

The Tschirnhaus transformation produces a resulting polynomial $\map {f'} y$ which is depressed, as shown on Tschirnhaus Transformation yields Depressed Polynomial.

This technique is used in the derivation of Cardano's Formula for the roots of the general cubic.

",Definition:Tschirnhaus Transformation,"['Definitions/Tschirnhaus Transformations', 'Definitions/Polynomial Theory']","Let f x be a polynomial over a field k:

:f x = a_n x^n + a_n - 1 x^n - 1 + a_n - 2 x^n - 2 + ⋯ + a_1 x + a_0


Then the Tschirnhaus transformation is the linear substitution x = y - a_n - 1n a_n.

The Tschirnhaus transformation produces a resulting polynomial f' y which is depressed, as shown on Tschirnhaus Transformation yields Depressed Polynomial.

This technique is used in the derivation of Cardano's Formula for the roots of the general cubic.

"
Definition:Transitive,Transitive,"Let $\RR$ be a relation on a set $S$.


The transitive closure of $\RR$ is defined as the smallest transitive relation on $S$ which contains $\RR$ as a subset.


The transitive closure of $\RR$ is denoted $\RR^+$.
Let $\RR$ be a relation on a set $S$.


The transitive closure of $\RR$ is defined as the intersection of all transitive relations on $S$ which contain $\RR$.


The transitive closure of $\RR$ is denoted $\RR^+$.
",Definition:Transitive Closure (Relation Theory),"['Definitions/Transitive Closures', 'Definitions/Transitive Relations', 'Definitions/Examples of Closure Operators']","Let  be a relation on a set S.


The transitive closure of  is defined as the smallest transitive relation on S which contains  as a subset.


The transitive closure of  is denoted ^+.
Let  be a relation on a set S.


The transitive closure of  is defined as the intersection of all transitive relations on S which contain .


The transitive closure of  is denoted ^+.
"
Definition:Transitive,Transitive,"Let $\RR$ be a relation on a set $S$.

The reflexive transitive closure of $\RR$ is denoted $\RR^*$, and is defined as the smallest reflexive and transitive relation on $S$ which contains $\RR$.
Let $\RR$ be a relation on a set $S$.

The reflexive transitive closure of $\RR$ is denoted $\RR^*$, and is defined as the reflexive closure of the transitive closure of $\RR$:

:$\RR^* = \paren {\RR^+}^=$
Let $\RR$ be a relation on a set $S$.

The reflexive transitive closure of $\RR$ is denoted $\RR^*$, and is defined as the transitive closure of the reflexive closure of $\RR$:
:$\RR^* = \paren {\RR^=}^+$
",Definition:Reflexive Transitive Closure,"['Definitions/Transitive Relations', 'Definitions/Reflexive Relations', 'Definitions/Reflexive Transitive Closures']","Let  be a relation on a set S.

The reflexive transitive closure of  is denoted ^*, and is defined as the smallest reflexive and transitive relation on S which contains .
Let  be a relation on a set S.

The reflexive transitive closure of  is denoted ^*, and is defined as the reflexive closure of the transitive closure of :

:^* = ^+^=
Let  be a relation on a set S.

The reflexive transitive closure of  is denoted ^*, and is defined as the transitive closure of the reflexive closure of :
:^* = ^=^+
"
Definition:Transitive,Transitive,"Let $G$ be a group.

Let $S$ be a set.

Let $*: G \times S \to S$ be a group action.

Let $n\geq1$ be a natural number.


The group action is $n$-transitive  for any two ordered $n$-tuples $(x_1, \ldots, x_n)$ and $(y_1, \ldots, y_n)$ of pairwise distinct elements of $S$, there exists $g\in G$ such that:
:$\forall i\in \{1, \ldots, n\} : g * x_i = y_i$


Category:Definitions/Group Actions
",Definition:Transitive Group Action,['Definitions/Group Actions'],"Let G be a group.

Let S be a set.

Let *: G × S → S be a group action.

Let n≥1 be a natural number.


The group action is n-transitive  for any two ordered n-tuples (x_1, …, x_n) and (y_1, …, y_n) of pairwise distinct elements of S, there exists g∈ G such that:
:∀ i∈{1, …, n} : g * x_i = y_i


Category:Definitions/Group Actions
"
Definition:Transitive,Transitive,"Let $S_n$ denote the symmetric group on $n$ letters for $n \in \N$.

Let $H$ be a subgroup of $S_n$.

Let $H$ be such that:
:for every pair of elements $i, j \in \N_n$ there exists $\pi \in H$ such that $\map \pi i = j$.


Then $H$ is called a transitive subgroup of $S_n$.
",Definition:Transitive Subgroup,['Definitions/Symmetric Groups'],"Let S_n denote the symmetric group on n letters for n ∈.

Let H be a subgroup of S_n.

Let H be such that:
:for every pair of elements i, j ∈_n there exists π∈ H such that π i = j.


Then H is called a transitive subgroup of S_n.
"
Definition:Transitive,Transitive,"Let $A$ denote a class, which can be either a set or a proper class.

Then $A$ is transitive  every element of $A$ is also a subclass of $A$.


That is, $A$ is transitive :
:$x \in A \implies x \subseteq A$

or:
:$\forall x: \forall y: \paren {x \in y \land y \in A \implies x \in A}$",Definition:Transitive Class,"['Definitions/Class Theory', 'Definitions/Transitive Classes']","Let A denote a class, which can be either a set or a proper class.

Then A is transitive  every element of A is also a subclass of A.


That is, A is transitive :
:x ∈ A  x ⊆ A

or:
:∀ x: ∀ y: x ∈ y  y ∈ A  x ∈ A"
Definition:Transitive,Transitive,"=== Definition 1 ===


The following is not equivalent to the above, but they are almost the same.

=== Definition 2 ===
",Definition:Transitive Closure (Set Theory),['Definitions/Relational Closures'],"=== Definition 1 ===


The following is not equivalent to the above, but they are almost the same.

=== Definition 2 ===
"
Definition:Transpose,Transpose,"Let $R$ be a commutative ring.

Let $G$ and $H$ be $R$-modules.

Let $G^*$ and $H^*$ be the algebraic duals of $G$ and $H$ respectively.


Let $\map {\LL_R} {G, H}$ be the set of all linear transformations from $G$ to $H$.

Let $u \in \map {\LL_R} {G, H}$.


The transpose of $u$ is the mapping $u^\intercal: H^* \to G^*$ defined as:
:$\forall y \in H^*: \map {u^\intercal} y = y \circ u$
where $y \circ u$ is the composition of $y$ and $u$.",Definition:Transpose of Linear Transformation,['Definitions/Linear Transformations'],"Let R be a commutative ring.

Let G and H be R-modules.

Let G^* and H^* be the algebraic duals of G and H respectively.


Let _RG, H be the set of all linear transformations from G to H.

Let u ∈_RG, H.


The transpose of u is the mapping u^⊺: H^* → G^* defined as:
:∀ y ∈ H^*: u^⊺ y = y ∘ u
where y ∘ u is the composition of y and u."
Definition:Transpose,Transpose,"Let $\mathbf A = \sqbrk \alpha_{m n}$ be an $m \times n$ matrix over a set.


Then the transpose of $\mathbf A$ is denoted $\mathbf A^\intercal$ and is defined as:

:$\mathbf A^\intercal = \sqbrk \beta_{n m}: \forall i \in \closedint 1 n, j \in \closedint 1 m: \beta_{i j} = \alpha_{j i}$
",Definition:Transpose of Matrix,['Definitions/Matrix Theory'],"Let 𝐀 = α_m n be an m × n matrix over a set.


Then the transpose of 𝐀 is denoted 𝐀^⊺ and is defined as:

:𝐀^⊺ = β_n m: ∀ i ∈ 1 n, j ∈ 1 m: β_i j = α_j i
"
Definition:Triangle,Triangle,":

A triangle is a polygon with exactly three sides.


Thus a triangle is a $2$-simplex.


Because it is a polygon, it follows that it also has three vertices and three angles.",Definition:Triangle (Geometry),"['Definitions/Triangles', 'Definitions/Polygons', 'Definitions/Simplices']",":

A triangle is a polygon with exactly three sides.


Thus a triangle is a 2-simplex.


Because it is a polygon, it follows that it also has three vertices and three angles."
Definition:Triangle,Triangle,"The complete graph $K_3$ of order $3$ is called a triangle.

:

Category:Definitions/Complete Graphs
Category:Definitions/Examples of Graphs",Definition:Triangle (Graph Theory),"['Definitions/Complete Graphs', 'Definitions/Examples of Graphs']","The complete graph K_3 of order 3 is called a triangle.

:

Category:Definitions/Complete Graphs
Category:Definitions/Examples of Graphs"
Definition:Trivial,Trivial,"Let $\Delta_S$ be the diagonal relation on a set $S$.

As $\Delta_S$ is an equivalence, we can form the quotient mapping:
:$q_{\Delta_S}: S \to S / \Delta_S$.


This quotient mapping is called the trivial quotient of $S$.",Definition:Trivial Quotient,['Definitions/Quotient Mappings'],"Let Δ_S be the diagonal relation on a set S.

As Δ_S is an equivalence, we can form the quotient mapping:
:q_Δ_S: S → S / Δ_S.


This quotient mapping is called the trivial quotient of S."
Definition:Trivial,Trivial,"The trivial relation is the relation $\RR \subseteq S \times T$ in $S$ to $T$ such that every element of $S$ relates to every element in $T$:

:$\RR: S \times T: \forall \tuple {s, t} \in S \times T: \tuple {s, t} \in \RR$


That is:
:$\RR = S \times T$
the relation which equals the product of the sets on which it is defined.
",Definition:Trivial Relation,"['Definitions/Examples of Relations', 'Definitions/Examples of Equivalence Relations']","The trivial relation is the relation ⊆ S × T in S to T such that every element of S relates to every element in T:

:: S × T: ∀s, t∈ S × T: s, t∈


That is:
:= S × T
the relation which equals the product of the sets on which it is defined.
"
Definition:Trivial,Trivial,A trivial group is a group with only one element $e$.,Definition:Trivial Group,['Definitions/Examples of Groups'],A trivial group is a group with only one element e.
Definition:Trivial,Trivial,"A ring $\struct {R, +, \circ}$ is a trivial ring :

:$\forall x, y \in R: x \circ y = 0_R$",Definition:Trivial Ring,['Definitions/Ring Theory'],"A ring R, +, ∘ is a trivial ring :

:∀ x, y ∈ R: x ∘ y = 0_R"
Definition:Trivial,Trivial,"Let $\struct {G, +}$ be a finite abelian group.

Let $\struct {\C_{\ne 0}, \times}$ be the multiplicative group of complex numbers.


A character of $G$ is a group homomorphism:

:$\chi: G \to \C_{\ne 0}$",Definition:Character (Number Theory),['Definitions/Analytic Number Theory'],"Let G, + be a finite abelian group.

Let _ 0, × be the multiplicative group of complex numbers.


A character of G is a group homomorphism:

:χ: G →_ 0"
Definition:Trivial,Trivial,"Let $\struct {D, +, \circ}$ be an integral domain.

Let $\struct {U_D, \circ}$ be the group of units of $\struct {D, +, \circ}$.


A factorization in $\struct {D, +, \circ}$ of the form $x = z \circ y$, where neither $y$ nor $z$ is a unit of $D$, is called a non-trivial factorization.
",Definition:Trivial Factorization,['Definitions/Factorization'],"Let D, +, ∘ be an integral domain.

Let U_D, ∘ be the group of units of D, +, ∘.


A factorization in D, +, ∘ of the form x = z ∘ y, where neither y nor z is a unit of D, is called a non-trivial factorization.
"
Definition:Trivial,Trivial,"Let $S$ be a set.


A filter $\FF$ on $S$ by definition specifically does not include the empty set $\O$.

If a filter $\FF$ were to include $\O$, then from Empty Set is Subset of All Sets it would follow that every subset of $S$ would have to be in $\FF$, and so $\FF = \powerset S$.


Such a ""filter"" is called the trivial filter on $S$.


Category:Definitions/Filters on Sets
",Definition:Filter on Set/Trivial Filter,['Definitions/Filters on Sets'],"Let S be a set.


A filter  on S by definition specifically does not include the empty set Ø.

If a filter  were to include Ø, then from Empty Set is Subset of All Sets it would follow that every subset of S would have to be in , and so =  S.


Such a ""filter"" is called the trivial filter on S.


Category:Definitions/Filters on Sets
"
Definition:Trivial,Trivial,"Let $S \ne \O$ be a set.

Let $\tau = \set {S, \O}$.


Then $\tau$ is called the indiscrete topology on $S$.


A topological space $\struct {S, \set {S, \O} }$ is known as an indiscrete space.",Definition:Indiscrete Topology,"['Definitions/Indiscrete Topology', 'Definitions/Examples of Topologies']","Let S Ø be a set.

Let τ = S, Ø.


Then τ is called the indiscrete topology on S.


A topological space S, S, Ø is known as an indiscrete space."
Definition:Trivial,Trivial,"A trivial topological space is a topological space with only one element.


The open sets of a trivial topological space $T = \struct {\set s, \tau}$ are $\O$ and $\set s$.
A trivial topological space is a topological space with only one element.


The open sets of a trivial topological space $T = \struct {\set s, \tau}$ are $\O$ and $\set s$.
",Definition:Trivial Topological Space,['Definitions/Examples of Topologies'],"A trivial topological space is a topological space with only one element.


The open sets of a trivial topological space T =  s, τ are Ø and s.
A trivial topological space is a topological space with only one element.


The open sets of a trivial topological space T =  s, τ are Ø and s.
"
Definition:Trivial,Trivial,"Let $V$ be a vector space with zero vector $\mathbf 0$.


Then the set $(\mathbf 0) := \left\{{\mathbf 0}\right\}$ is called the zero subspace of $V$.


This name is appropriate as $(\mathbf 0)$ is in fact a subspace of $V$, as proved in Zero Subspace is Subspace.




Category:Definitions/Vector Spaces",Definition:Zero Subspace,['Definitions/Vector Spaces'],"Let V be a vector space with zero vector 0.


Then the set (0) := {0} is called the zero subspace of V.


This name is appropriate as (0) is in fact a subspace of V, as proved in Zero Subspace is Subspace.




Category:Definitions/Vector Spaces"
Definition:Trivial,Trivial," Division Ring 


 Vector Space 
",Definition:Trivial Norm,"['Definitions/Examples of Norms', 'Definitions/Trivial Norms']"," Division Ring 


 Vector Space 
"
Definition:Trivial,Trivial,"The trivial zeroes of the Riemann $\zeta$ function are the strictly negative even integers :

:$\set {n \in \Z: n = -2 \times k: k \in \N_{\ne 0} } = \set {-2, -4, -6, \ldots}$",Definition:Riemann Zeta Function/Zero/Trivial,['Definitions/Riemann Zeta Function'],"The trivial zeroes of the Riemann ζ function are the strictly negative even integers :

:n ∈: n = -2 × k: k ∈_ 0 = -2, -4, -6, …"
Definition:Union,Union,"Let $\mathbb S$ be a set of sets.

Let $\sequence {S_n}_{n \mathop \in \N}$ be a sequence in $\mathbb S$.

Let $S$ be the union of $\sequence {S_n}_{n \mathop \in \N}$:
:$\ds S = \bigcup_{n \mathop \in \N} S_n$


Then $S$ is a countable union of sets in $\mathbb S$.
Let $S = S_1 \cup S_2 \cup \ldots \cup S_n$.

Then:
:$\ds S = \bigcup_{i \mathop \in \N^*_n} S_i = \set {x: \exists i \in \N^*_n: x \in S_i}$
where $\N^*_n = \set {1, 2, 3, \ldots, n}$.


If it is clear from the context that $i \in \N^*_n$, we can also write $\ds \bigcup_{\N^*_n} S_i$.



",Definition:Set Union,"['Definitions/Set Theory', 'Definitions/Set Union']","Let 𝕊 be a set of sets.

Let S_n_n ∈ be a sequence in 𝕊.

Let S be the union of S_n_n ∈:
:S = ⋃_n ∈ S_n


Then S is a countable union of sets in 𝕊.
Let S = S_1 ∪ S_2 ∪…∪ S_n.

Then:
:S = ⋃_i ∈^*_n S_i = x: ∃ i ∈^*_n: x ∈ S_i
where ^*_n = 1, 2, 3, …, n.


If it is clear from the context that i ∈^*_n, we can also write ⋃_^*_n S_i.



"
Definition:Union,Union,"Let:

:$(1): \quad \RR_1 \subseteq S_1 \times T_1$ be a relation on $S_1 \times T_1$

:$(2): \quad \RR_2 \subseteq S_2 \times T_2$ be a relation on $S_2 \times T_2$

Let $\RR_1$ and $\RR_2$ be combinable, that is, that they agree on $S_1 \cap S_2$.


Then the union relation (or combined relation) $\RR$ of $\RR_1$ and $\RR_2$ is:

:$\RR \subseteq \paren {S_1 \cup S_2} \times \paren {T_1 \cup T_2}: \map \RR s =
\begin{cases}
 \map {\RR_1} s : & s \in S_1 \\
 \map {\RR_2} s : & s \in S_2
\end{cases}$",Definition:Union Relation,['Definitions/Relation Theory'],"Let:

:(1):   _1 ⊆ S_1 × T_1 be a relation on S_1 × T_1

:(2):   _2 ⊆ S_2 × T_2 be a relation on S_2 × T_2

Let _1 and _2 be combinable, that is, that they agree on S_1 ∩ S_2.


Then the union relation (or combined relation)  of _1 and _2 is:

:⊆S_1 ∪ S_2×T_1 ∪ T_2:  s =
_1 s :     s ∈ S_1 
_2 s :     s ∈ S_2"
Definition:Union,Union,"Let $I$ be an indexing set.

Let $\family {S_i}_{i \mathop \in I}$ be a family of sets indexed by $I$.


Then the union of $\family {S_i}$ is defined as:

:$\ds \bigcup_{i \mathop \in I} S_i := \set {x: \exists i \in I: x \in S_i}$


=== In the context of the Universal Set ===

In treatments of set theory in which the concept of the universal set is recognised, this can be expressed as follows.



=== Subsets of General Set ===
This definition is the same when the universal set $\mathbb U$ is replaced by any set $X$, which may or may not be a universal set:

",Definition:Set Union/Family of Sets,"['Definitions/Set Union', 'Definitions/Indexed Families']","Let I be an indexing set.

Let S_i_i ∈ I be a family of sets indexed by I.


Then the union of S_i is defined as:

:⋃_i ∈ I S_i := x: ∃ i ∈ I: x ∈ S_i


=== In the context of the Universal Set ===

In treatments of set theory in which the concept of the universal set is recognised, this can be expressed as follows.



=== Subsets of General Set ===
This definition is the same when the universal set 𝕌 is replaced by any set X, which may or may not be a universal set:

"
Definition:Union,Union,,Definition:Disjoint Union,[],
Definition:Unit,Unit,"Let $\struct {R, +, \circ}$ be a non-null ring.

Then $\struct {R, +, \circ}$ is a ring with unity  the multiplicative semigroup $\struct {R, \circ}$ has an identity element.

Such an identity element is known as a unity.


It follows that such a $\struct {R, \circ}$ is a monoid.
Let $\struct {R, +, \circ}$ be a ring.

If the semigroup $\struct {R, \circ}$ has an identity, this identity is referred to as the unity of the ring $\struct {R, +, \circ}$.

It is (usually) denoted $1_R$, where the subscript denotes the particular ring to which $1_R$ belongs (or often $1$ if there is no danger of ambiguity).


The ring $R$ itself is then referred to as a ring with unity.
Let $\struct {R, +, \circ}$ be a ring with unity.


Then the set $U_R$ of units of $\struct {R, +, \circ}$ is called the group of units of $\struct {R, +, \circ}$.


This can be denoted explicitly as $\struct {U_R, \circ}$.
",Definition:Unit of Ring,"['Definitions/Units of Rings', 'Definitions/Ring Theory', 'Definitions/Factorization']","Let R, +, ∘ be a non-null ring.

Then R, +, ∘ is a ring with unity  the multiplicative semigroup R, ∘ has an identity element.

Such an identity element is known as a unity.


It follows that such a R, ∘ is a monoid.
Let R, +, ∘ be a ring.

If the semigroup R, ∘ has an identity, this identity is referred to as the unity of the ring R, +, ∘.

It is (usually) denoted 1_R, where the subscript denotes the particular ring to which 1_R belongs (or often 1 if there is no danger of ambiguity).


The ring R itself is then referred to as a ring with unity.
Let R, +, ∘ be a ring with unity.


Then the set U_R of units of R, +, ∘ is called the group of units of R, +, ∘.


This can be denoted explicitly as U_R, ∘.
"
Definition:Unit,Unit,"Let $\SS$ be a system of sets.

Let $U \in \SS$ such that:
:$\forall A \in \SS: A \cap U = A$


Then $U$ is the unit of $\SS$.


Note that, for a given system of sets, if $U$ exists then it is unique.

",Definition:Unit of System of Sets,['Definitions/Set Systems'],"Let  be a system of sets.

Let U ∈ such that:
:∀ A ∈: A ∩ U = A


Then U is the unit of .


Note that, for a given system of sets, if U exists then it is unique.

"
Definition:Unit,Unit,"Let $R$ be a commutative ring.

Let $\struct {A, *}$ be an algebra over $R$. 


Then $\struct {A, *}$ is a unital algebra  the algebraic structure $\struct {A, \oplus}$ has an identity element.

That is:
:$\exists 1_A \in A: \forall a \in A: a * 1_A = 1_A * a = a$
",Definition:Unit of Algebra,['Definitions/Unital Algebras'],"Let R be a commutative ring.

Let A, * be an algebra over R. 


Then A, * is a unital algebra  the algebraic structure A, ⊕ has an identity element.

That is:
:∃ 1_A ∈ A: ∀ a ∈ A: a * 1_A = 1_A * a = a
"
Definition:Unit,Unit,"Let $R$ be a commutative ring.

Let $\struct {A, *}$ be an algebra over $R$. 


Then $\struct {A, *}$ is a unital algebra  the algebraic structure $\struct {A, \oplus}$ has an identity element.

That is:
:$\exists 1_A \in A: \forall a \in A: a * 1_A = 1_A * a = a$",Definition:Unital Algebra,['Definitions/Algebras'],"Let R be a commutative ring.

Let A, * be an algebra over R. 


Then A, * is a unital algebra  the algebraic structure A, ⊕ has an identity element.

That is:
:∃ 1_A ∈ A: ∀ a ∈ A: a * 1_A = 1_A * a = a"
Definition:Unit,Unit,"Let $\struct {R, +, \circ}$ be a non-null ring.

Then $\struct {R, +, \circ}$ is a ring with unity  the multiplicative semigroup $\struct {R, \circ}$ has an identity element.

Such an identity element is known as a unity.


It follows that such a $\struct {R, \circ}$ is a monoid.
",Definition:Unity (Abstract Algebra)/Ring,"['Definitions/Ring Theory', 'Definitions/Unity']","Let R, +, ∘ be a non-null ring.

Then R, +, ∘ is a ring with unity  the multiplicative semigroup R, ∘ has an identity element.

Such an identity element is known as a unity.


It follows that such a R, ∘ is a monoid.
"
Definition:Unit,Unit,"A unit of measurement is a specified magnitude of a given physical quantity, defined by convention.

It is used as a standard for measurement of that physical quantity.

Any other value of the physical quantity can be expressed as a multiple of that unit of measurement.",Definition:Unit of Measurement,"['Definitions/Physics', 'Definitions/Units of Measurement']","A unit of measurement is a specified magnitude of a given physical quantity, defined by convention.

It is used as a standard for measurement of that physical quantity.

Any other value of the physical quantity can be expressed as a multiple of that unit of measurement."
Definition:Unit,Unit,,Definition:Unity,['Definitions/Unity'],
Definition:Unity,Unity,"Let $\struct {R, +, \circ}$ be a ring.

If the semigroup $\struct {R, \circ}$ has an identity, this identity is referred to as the unity of the ring $\struct {R, +, \circ}$.

It is (usually) denoted $1_R$, where the subscript denotes the particular ring to which $1_R$ belongs (or often $1$ if there is no danger of ambiguity).


The ring $R$ itself is then referred to as a ring with unity.
Let $\struct {F, +, \times}$ be a field.

The identity element of the multiplicative group $\struct {F^*, \times}$ of $F$ is called the multiplicative identity of $F$.

It is often denoted $e_F$ or $1_F$, or, if there is no danger of ambiguity, $e$ or $1$.
",Definition:Unity (Abstract Algebra),"['Definitions/Unity', 'Definitions/Identity Elements']","Let R, +, ∘ be a ring.

If the semigroup R, ∘ has an identity, this identity is referred to as the unity of the ring R, +, ∘.

It is (usually) denoted 1_R, where the subscript denotes the particular ring to which 1_R belongs (or often 1 if there is no danger of ambiguity).


The ring R itself is then referred to as a ring with unity.
Let F, +, × be a field.

The identity element of the multiplicative group F^*, × of F is called the multiplicative identity of F.

It is often denoted e_F or 1_F, or, if there is no danger of ambiguity, e or 1.
"
Definition:Unity,Unity,"Let $\struct {R, +, \circ}$ be a non-null ring.

Then $\struct {R, +, \circ}$ is a ring with unity  the multiplicative semigroup $\struct {R, \circ}$ has an identity element.

Such an identity element is known as a unity.


It follows that such a $\struct {R, \circ}$ is a monoid.
",Definition:Unity (Abstract Algebra)/Ring,"['Definitions/Ring Theory', 'Definitions/Unity']","Let R, +, ∘ be a non-null ring.

Then R, +, ∘ is a ring with unity  the multiplicative semigroup R, ∘ has an identity element.

Such an identity element is known as a unity.


It follows that such a R, ∘ is a monoid.
"
Definition:Unity,Unity,"Let $\struct {F, +, \times}$ be a field.

The identity element of the multiplicative group $\struct {F^*, \times}$ of $F$ is called the multiplicative identity of $F$.

It is often denoted $e_F$ or $1_F$, or, if there is no danger of ambiguity, $e$ or $1$.",Definition:Multiplicative Identity,"['Definitions/Field Theory', 'Definitions/Unity']","Let F, +, × be a field.

The identity element of the multiplicative group F^*, × of F is called the multiplicative identity of F.

It is often denoted e_F or 1_F, or, if there is no danger of ambiguity, e or 1."
Definition:Universal,Universal,"Sets are considered to be subsets of some large universal set, also called the universe.

Exactly what this universe is will vary depending on the subject and context.

When discussing particular sets, it should be made clear just what that universe is.

However, note that from There Exists No Universal Set, this universe cannot be everything that there is.


The traditional symbol used to signify the universe is $\mathfrak A$.

However, this is old-fashioned and inconvenient, so some newer texts have taken to using $\mathbb U$ or just $U$ instead.


With this notation, this definition can be put into symbols as:
:$\forall S: S \subseteq \mathbb U$


The use of $\mathbb U$ or a variant is not universal: some sources use $X$.",Definition:Universe (Set Theory),['Definitions/Set Theory'],"Sets are considered to be subsets of some large universal set, also called the universe.

Exactly what this universe is will vary depending on the subject and context.

When discussing particular sets, it should be made clear just what that universe is.

However, note that from There Exists No Universal Set, this universe cannot be everything that there is.


The traditional symbol used to signify the universe is 𝔄.

However, this is old-fashioned and inconvenient, so some newer texts have taken to using 𝕌 or just U instead.


With this notation, this definition can be put into symbols as:
:∀ S: S ⊆𝕌


The use of 𝕌 or a variant is not universal: some sources use X."
Definition:Universal,Universal,"The universal class is the class of which all sets are elements.


The universal class is defined most commonly in literature as:

:$V = \set {x: x = x}$

where $x$ ranges over all sets.


It can be briefly defined as the class of all sets.",Definition:Universal Class,"['Definitions/Class Theory', 'Definitions/Universal Class']","The universal class is the class of which all sets are elements.


The universal class is defined most commonly in literature as:

:V = x: x = x

where x ranges over all sets.


It can be briefly defined as the class of all sets."
Definition:Universal,Universal,"Let $C$ be a category.


A universal object of $C$ is an object that is initial or terminal.",Definition:Universal Object,['Definitions/Category Theory'],"Let C be a category.


A universal object of C is an object that is initial or terminal."
Definition:Universal,Universal,"A universal cover is a covering space which is simply connected.



Category:Definitions/Topology",Definition:Universal Cover,['Definitions/Topology'],"A universal cover is a covering space which is simply connected.



Category:Definitions/Topology"
Definition:Universe,Universe,"The universe of discourse is the term used to mean everything we are talking about.


When introducing the symbols:
:$\forall$ (the universal quantifier)
or:
:$\exists$ (the existential quantifier)
it is understood that the objects referred to are those in the specified universe of discourse.

It is usual to define that universe.
The universe of discourse is the term used to mean everything we are talking about.


When introducing the symbols:
:$\forall$ (the universal quantifier)
or:
:$\exists$ (the existential quantifier)
it is understood that the objects referred to are those in the specified universe of discourse.

It is usual to define that universe.
The universe of discourse is the term used to mean everything we are talking about.


When introducing the symbols:
:$\forall$ (the universal quantifier)
or:
:$\exists$ (the existential quantifier)
it is understood that the objects referred to are those in the specified universe of discourse.

It is usual to define that universe.
",Definition:Universe of Discourse,"['Definitions/Universe of Discourse', 'Definitions/Logic']","The universe of discourse is the term used to mean everything we are talking about.


When introducing the symbols:
:∀ (the universal quantifier)
or:
:∃ (the existential quantifier)
it is understood that the objects referred to are those in the specified universe of discourse.

It is usual to define that universe.
The universe of discourse is the term used to mean everything we are talking about.


When introducing the symbols:
:∀ (the universal quantifier)
or:
:∃ (the existential quantifier)
it is understood that the objects referred to are those in the specified universe of discourse.

It is usual to define that universe.
The universe of discourse is the term used to mean everything we are talking about.


When introducing the symbols:
:∀ (the universal quantifier)
or:
:∃ (the existential quantifier)
it is understood that the objects referred to are those in the specified universe of discourse.

It is usual to define that universe.
"
Definition:Universe,Universe,"Sets are considered to be subsets of some large universal set, also called the universe.

Exactly what this universe is will vary depending on the subject and context.

When discussing particular sets, it should be made clear just what that universe is.

However, note that from There Exists No Universal Set, this universe cannot be everything that there is.


The traditional symbol used to signify the universe is $\mathfrak A$.

However, this is old-fashioned and inconvenient, so some newer texts have taken to using $\mathbb U$ or just $U$ instead.


With this notation, this definition can be put into symbols as:
:$\forall S: S \subseteq \mathbb U$


The use of $\mathbb U$ or a variant is not universal: some sources use $X$.
",Definition:Grothendieck Universe,"['Definitions/Tarski-Grothendieck Set Theory', 'Definitions/Category Theory']","Sets are considered to be subsets of some large universal set, also called the universe.

Exactly what this universe is will vary depending on the subject and context.

When discussing particular sets, it should be made clear just what that universe is.

However, note that from There Exists No Universal Set, this universe cannot be everything that there is.


The traditional symbol used to signify the universe is 𝔄.

However, this is old-fashioned and inconvenient, so some newer texts have taken to using 𝕌 or just U instead.


With this notation, this definition can be put into symbols as:
:∀ S: S ⊆𝕌


The use of 𝕌 or a variant is not universal: some sources use X.
"
Definition:Universe,Universe,"The universe of discourse is the term used to mean everything we are talking about.


When introducing the symbols:
:$\forall$ (the universal quantifier)
or:
:$\exists$ (the existential quantifier)
it is understood that the objects referred to are those in the specified universe of discourse.

It is usual to define that universe.
",Definition:Population,"['Definitions/Statistics', 'Definitions/Applied Mathematics']","The universe of discourse is the term used to mean everything we are talking about.


When introducing the symbols:
:∀ (the universal quantifier)
or:
:∃ (the existential quantifier)
it is understood that the objects referred to are those in the specified universe of discourse.

It is usual to define that universe.
"
Definition:Universe,Universe,"The physical universe, or usually just universe, is commonly defined as, and understood to be, .


=== Real-World ===
",Definition:Physical Universe,"['Definitions/Applied Mathematics', 'Definitions/Physics']","The physical universe, or usually just universe, is commonly defined as, and understood to be, .


=== Real-World ===
"
Definition:Upper Bound,Upper Bound,"Let $\struct {S, \preceq}$ be an ordered set.

Let $T$ be a subset of $S$.


An upper bound for $T$ (in $S$) is an element $M \in S$ such that:
:$\forall t \in T: t \preceq M$

That is, $M$ succeeds every element of $T$.


=== Subset of Real Numbers ===

The concept is usually encountered where $\struct {S, \preceq}$ is the set of real numbers under the usual ordering $\struct {\R, \le}$:

",Definition:Upper Bound of Set,['Definitions/Boundedness'],"Let S, ≼ be an ordered set.

Let T be a subset of S.


An upper bound for T (in S) is an element M ∈ S such that:
:∀ t ∈ T: t ≼ M

That is, M succeeds every element of T.


=== Subset of Real Numbers ===

The concept is usually encountered where S, ≼ is the set of real numbers under the usual ordering , ≤:

"
Definition:Upper Bound,Upper Bound,"When considering the upper bound of a set of numbers, it is commonplace to ignore the set and instead refer just to the number itself.

Thus the construction:

:The set of numbers which fulfil the propositional function $\map P n$ is bounded above with the upper bound $N$

would be reported as:

:The number $n$ such that $\map P n$ has the upper bound $N$.


This construct obscures the details of what is actually being stated. Its use on  is considered an abuse of notation and so discouraged.


This also applies in the case where it is the upper bound of a mapping which is under discussion.


Category:Definitions/Numbers
Category:Definitions/Boundedness
",Definition:Upper Bound of Set/Real Numbers,['Definitions/Boundedness'],"When considering the upper bound of a set of numbers, it is commonplace to ignore the set and instead refer just to the number itself.

Thus the construction:

:The set of numbers which fulfil the propositional function P n is bounded above with the upper bound N

would be reported as:

:The number n such that P n has the upper bound N.


This construct obscures the details of what is actually being stated. Its use on  is considered an abuse of notation and so discouraged.


This also applies in the case where it is the upper bound of a mapping which is under discussion.


Category:Definitions/Numbers
Category:Definitions/Boundedness
"
Definition:Upper Bound,Upper Bound,"Let $f: S \to T$ be a mapping whose codomain is an ordered set $\struct {T, \preceq}$.


Let $f$ be bounded above in $T$ by $H \in T$.


Then $H$ is an upper bound of $f$.


=== Real-Valued Function ===

The concept is usually encountered where $\struct {T, \preceq}$ is the set of real numbers under the usual ordering $\struct {\R, \le}$:

",Definition:Upper Bound of Mapping,['Definitions/Boundedness'],"Let f: S → T be a mapping whose codomain is an ordered set T, ≼.


Let f be bounded above in T by H ∈ T.


Then H is an upper bound of f.


=== Real-Valued Function ===

The concept is usually encountered where T, ≼ is the set of real numbers under the usual ordering , ≤:

"
Definition:Upper Bound,Upper Bound,"Let $f: S \to T$ be a mapping whose codomain is an ordered set $\struct {T, \preceq}$.


Let $f$ be bounded above in $T$ by $H \in T$.


Then $H$ is an upper bound of $f$.


=== Real-Valued Function ===

The concept is usually encountered where $\struct {T, \preceq}$ is the set of real numbers under the usual ordering $\struct {\R, \le}$:


",Definition:Upper Bound of Sequence,['Definitions/Boundedness'],"Let f: S → T be a mapping whose codomain is an ordered set T, ≼.


Let f be bounded above in T by H ∈ T.


Then H is an upper bound of f.


=== Real-Valued Function ===

The concept is usually encountered where T, ≼ is the set of real numbers under the usual ordering , ≤:


"
Definition:Vacuous,Vacuous,"Let $P \implies Q$ be a conditional statement.

Suppose that $P$ is false.

Then the statement $P \implies Q$ is a vacuous truth, or is vacuously true.


It is frequently encountered in the form:
:$\forall x: \map P x \implies \map Q x$
when the propositional function $\map P x$ is false for all $x$.

Such a statement is also a vacuous truth.


For example, the statement:
:All cats who are expert chess-players are also fluent in ancient Sanskrit
is (vacuously) true, because (as far as the author knows) there are no cats who are expert chess-players.",Definition:Vacuous Truth,['Definitions/Logic'],"Let P  Q be a conditional statement.

Suppose that P is false.

Then the statement P  Q is a vacuous truth, or is vacuously true.


It is frequently encountered in the form:
:∀ x:  P x  Q x
when the propositional function P x is false for all x.

Such a statement is also a vacuous truth.


For example, the statement:
:All cats who are expert chess-players are also fluent in ancient Sanskrit
is (vacuously) true, because (as far as the author knows) there are no cats who are expert chess-players."
Definition:Vacuous,Vacuous,"The empty set is a set which has no elements.

That is, $x \in \O$ is false, whatever $x$ is.


It is usually denoted by some variant of a zero with a line through it, for example $\O$ or $\emptyset$, and can always be represented as $\set {}$.",Definition:Empty Set,"['Definitions/Empty Set', 'Definitions/Set Theory']","The empty set is a set which has no elements.

That is, x ∈Ø is false, whatever x is.


It is usually denoted by some variant of a zero with a line through it, for example Ø or ∅, and can always be represented as ."
Definition:Vacuous,Vacuous,"Take the summation:
:$\ds \sum_{\map \Phi j} a_j$
where $\map \Phi j$ is a propositional function of $j$.

Suppose that there are no values of $j$ for which $\map \Phi j$ is true.

Then $\ds \sum_{\map \Phi j} a_j$ is defined as being $0$.

This summation is called a vacuous summation.


This is because:
:$\forall a: a + 0 = a$
where $a$ is a number.

Hence for all $j$ for which $\map \Phi j$ is false, the sum is unaffected.


This is most frequently seen in the form:
:$\ds \sum_{j \mathop = m}^n a_j = 0$
where $m > n$.

In this case, $j$ can not at the same time be both greater than or equal to $m$ and less than or equal to $n$.


Some sources consider such a treatment as abuse of notation.",Definition:Summation/Vacuous Summation,['Definitions/Summations'],"Take the summation:
:∑_Φ j a_j
where Φ j is a propositional function of j.

Suppose that there are no values of j for which Φ j is true.

Then ∑_Φ j a_j is defined as being 0.

This summation is called a vacuous summation.


This is because:
:∀ a: a + 0 = a
where a is a number.

Hence for all j for which Φ j is false, the sum is unaffected.


This is most frequently seen in the form:
:∑_j  = m^n a_j = 0
where m > n.

In this case, j can not at the same time be both greater than or equal to m and less than or equal to n.


Some sources consider such a treatment as abuse of notation."
Definition:Vacuous,Vacuous,"Take the composite expressed as a continued product:
:$\ds \prod_{\map R j} a_j$
where $\map R j$ is a propositional function of $j$.

Suppose that there are no values of $j$ for which $\map R j$ is true.

Then $\ds \prod_{\map R j} a_j$ is defined to be $1$.

Beware: not zero.

This composite is called a vacuous product.


This is because:
:$\forall a: a \times 1 = a$
where $a$ is a number.

Hence for all $j$ for which $\map R j$ is false, the value of the product is unaffected.


This is most frequently seen in the form:
:$\ds \prod_{j \mathop = m}^n a_j = 1$
where $m > n$.

In this case, $j$ can not at the same time be both greater than or equal to $m$ and less than or equal to $n$.",Definition:Continued Product/Vacuous Product,['Definitions/Continued Products'],"Take the composite expressed as a continued product:
:∏_ R j a_j
where R j is a propositional function of j.

Suppose that there are no values of j for which R j is true.

Then ∏_ R j a_j is defined to be 1.

Beware: not zero.

This composite is called a vacuous product.


This is because:
:∀ a: a × 1 = a
where a is a number.

Hence for all j for which R j is false, the value of the product is unaffected.


This is most frequently seen in the form:
:∏_j  = m^n a_j = 1
where m > n.

In this case, j can not at the same time be both greater than or equal to m and less than or equal to n."
Definition:Valid,Valid,"A valid argument is a logical argument in which the premises provide conclusive reasons for the conclusion.


When a proof is valid, we may say one of the following:
* The conclusion follows from the premises;
* The premises entail the conclusion;
* The conclusion is true on the strength of the premises;
* The conclusion is drawn from the premises;
* The conclusion is deduced from the premises;
* The conclusion is derived from the premises.


=== Proof ===

If all the premises of a valid argument are true, then the conclusion must also therefore be true.

It is not possible for the premises of a valid argument to be true, but for the conclusion to be false.


A valid argument is a logical argument in which the premises provide conclusive reasons for the conclusion.


When a proof is valid, we may say one of the following:
* The conclusion follows from the premises;
* The premises entail the conclusion;
* The conclusion is true on the strength of the premises;
* The conclusion is drawn from the premises;
* The conclusion is deduced from the premises;
* The conclusion is derived from the premises.


=== Proof ===

If all the premises of a valid argument are true, then the conclusion must also therefore be true.

It is not possible for the premises of a valid argument to be true, but for the conclusion to be false.


A valid argument is a logical argument in which the premises provide conclusive reasons for the conclusion.


When a proof is valid, we may say one of the following:
* The conclusion follows from the premises;
* The premises entail the conclusion;
* The conclusion is true on the strength of the premises;
* The conclusion is drawn from the premises;
* The conclusion is deduced from the premises;
* The conclusion is derived from the premises.


=== Proof ===

If all the premises of a valid argument are true, then the conclusion must also therefore be true.

It is not possible for the premises of a valid argument to be true, but for the conclusion to be false.


",Definition:Valid Argument,"['Definitions/Valid Arguments', 'Definitions/Logical Arguments']","A valid argument is a logical argument in which the premises provide conclusive reasons for the conclusion.


When a proof is valid, we may say one of the following:
* The conclusion follows from the premises;
* The premises entail the conclusion;
* The conclusion is true on the strength of the premises;
* The conclusion is drawn from the premises;
* The conclusion is deduced from the premises;
* The conclusion is derived from the premises.


=== Proof ===

If all the premises of a valid argument are true, then the conclusion must also therefore be true.

It is not possible for the premises of a valid argument to be true, but for the conclusion to be false.


A valid argument is a logical argument in which the premises provide conclusive reasons for the conclusion.


When a proof is valid, we may say one of the following:
* The conclusion follows from the premises;
* The premises entail the conclusion;
* The conclusion is true on the strength of the premises;
* The conclusion is drawn from the premises;
* The conclusion is deduced from the premises;
* The conclusion is derived from the premises.


=== Proof ===

If all the premises of a valid argument are true, then the conclusion must also therefore be true.

It is not possible for the premises of a valid argument to be true, but for the conclusion to be false.


A valid argument is a logical argument in which the premises provide conclusive reasons for the conclusion.


When a proof is valid, we may say one of the following:
* The conclusion follows from the premises;
* The premises entail the conclusion;
* The conclusion is true on the strength of the premises;
* The conclusion is drawn from the premises;
* The conclusion is deduced from the premises;
* The conclusion is derived from the premises.


=== Proof ===

If all the premises of a valid argument are true, then the conclusion must also therefore be true.

It is not possible for the premises of a valid argument to be true, but for the conclusion to be false.


"
Definition:Valid,Valid,"Let $\LL$ be a formal language.

Part of specifying a formal semantics $\mathscr M$ for $\LL$ is to define a notion of validity.


Concretely, a precise meaning needs to be assigned to the phrase:

:""The $\LL$-WFF $\phi$ is valid in the $\mathscr M$-structure $\MM$.""

It can be expressed symbolically as:

:$\MM \models_{\mathscr M} \phi$",Definition:Formal Semantics/Valid,['Definitions/Formal Semantics'],"Let  be a formal language.

Part of specifying a formal semantics ℳ for  is to define a notion of validity.


Concretely, a precise meaning needs to be assigned to the phrase:

:""The -WFF ϕ is valid in the ℳ-structure .""

It can be expressed symbolically as:

:_ℳϕ"
Definition:Value,Value,"A variable $x$ may be (temporarily, conceptually) identified with a particular object.

If so, then that object is called the value of $x$.",Definition:Variable/Value,"['Definitions/Predicate Logic', 'Definitions/Algebra', 'Definitions/Variables']","A variable x may be (temporarily, conceptually) identified with a particular object.

If so, then that object is called the value of x."
Definition:Value,Value,"Let $f: S \to T$ be a mapping.

Let $s \in S$.

The image of $s$ (under $f$) is defined as:

:$\Img s = \map f s = \ds \bigcup \set {t \in T: \tuple {s, t} \in f}$

That is, $\map f s$ is the element of the codomain of $f$ related to $s$ by $f$.


By the nature of a mapping, $\map f s$ is guaranteed to exist and to be unique for any given $s$ in the domain of $f$.",Definition:Image (Relation Theory)/Mapping/Element,['Definitions/Images'],"Let f: S → T be a mapping.

Let s ∈ S.

The image of s (under f) is defined as:

:s =  f s = ⋃t ∈ T: s, t∈ f

That is, f s is the element of the codomain of f related to s by f.


By the nature of a mapping, f s is guaranteed to exist and to be unique for any given s in the domain of f."
Definition:Value,Value,"Let $G$ be a game.


The value of $G$ is the payoff resulting from a solution of $G$.
",Definition:Value of Game,"['Definitions/Values of Games', 'Definitions/Game Theory']","Let G be a game.


The value of G is the payoff resulting from a solution of G.
"
Definition:Value,Value,"Let $K$ be a physical constant.

The value of $K$ is defined as the number of units of the specific physical quantity that go to make up $K$.


Category:Definitions/Physics
",Definition:Value (Physics),"['Definitions/Physics', 'Definitions/Physics']","Let K be a physical constant.

The value of K is defined as the number of units of the specific physical quantity that go to make up K.


Category:Definitions/Physics
"
Definition:Vanishing Ideal,Vanishing Ideal,"Let $A$ be a commutative ring with unity.

Let $V \subseteq \Spec A$ be a set of prime ideals of $A$.


Its vanishing ideal is its intersection, the set of elements of $A$ that are in each $\mathfrak p \in V$:
:$\map I V = \bigcap V$",Definition:Vanishing Ideal of Set of Prime Ideals,['Definitions/Commutative Algebra'],"Let A be a commutative ring with unity.

Let V ⊆ A be a set of prime ideals of A.


Its vanishing ideal is its intersection, the set of elements of A that are in each 𝔭∈ V:
:I V = ⋂ V"
Definition:Vanishing Ideal,Vanishing Ideal,"Let $k$ be a field.

Let $n \ge 0$ be a natural number.

Let $k \sqbrk {X_1, \ldots, X_n}$ be the polynomial ring in $n$ variables over $k$.

Let $S \subseteq \mathbb A^n_k$ be a subset of the standard affine space over $k$.


Its vanishing ideal is the ideal:
:$\map I S = \set {f \in k \sqbrk {X_1, \ldots, X_n} : \forall x \in S : \map f x = 0}$",Definition:Vanishing Ideal of Subset of Affine Space,['Definitions/Algebraic Geometry'],"Let k be a field.

Let n ≥ 0 be a natural number.

Let k X_1, …, X_n be the polynomial ring in n variables over k.

Let S ⊆𝔸^n_k be a subset of the standard affine space over k.


Its vanishing ideal is the ideal:
:I S = f ∈ k X_1, …, X_n : ∀ x ∈ S :  f x = 0"
Definition:Vertex,Vertex,"

Let $G = \struct {V, E}$ be a graph.

The vertices (singular: vertex) are the elements of $V$.

Informally, the vertices are the points that are connected by the edges.


In the above, the vertices are the points $A, B, C, D, E, F, G$ which are marked as dots.
",Definition:Graph (Graph Theory)/Vertex,['Definitions/Vertices of Graphs'],"

Let G = V, E be a graph.

The vertices (singular: vertex) are the elements of V.

Informally, the vertices are the points that are connected by the edges.


In the above, the vertices are the points A, B, C, D, E, F, G which are marked as dots.
"
Definition:Vertex,Vertex,":

A corner of a polygon is known as a vertex.

Thus, in the polygon above, the vertices are $A, B, C, D$ and $E$.
The vertices of a polyhedron are the vertices of the polygons which constitute its faces.
",Definition:Vertex (Geometry),['Definitions/Geometry'],":

A corner of a polygon is known as a vertex.

Thus, in the polygon above, the vertices are A, B, C, D and E.
The vertices of a polyhedron are the vertices of the polygons which constitute its faces.
"
Definition:Vertex,Vertex,":

A corner of a polygon is known as a vertex.

Thus, in the polygon above, the vertices are $A, B, C, D$ and $E$.",Definition:Polygon/Vertex,"['Definitions/Polygons', 'Definitions/Vertices (Geometry)']",":

A corner of a polygon is known as a vertex.

Thus, in the polygon above, the vertices are A, B, C, D and E."
Definition:Vertex,Vertex,The vertices of a polyhedron are the vertices of the polygons which constitute its faces.,Definition:Polyhedron/Vertex,"['Definitions/Polyhedra', 'Definitions/Vertices (Geometry)']",The vertices of a polyhedron are the vertices of the polygons which constitute its faces.
Definition:Vertex,Vertex,The point at which the arms of an angle meet is known as the vertex of that angle.,Definition:Angle/Vertex,['Definitions/Angles'],The point at which the arms of an angle meet is known as the vertex of that angle.
Definition:Vertex,Vertex,"The point at which the arms of an angle meet is known as the vertex of that angle.
",Definition:Solid Angle/Vertex,['Definitions/Solid Angles'],"The point at which the arms of an angle meet is known as the vertex of that angle.
"
Definition:Vertex,Vertex,"Consider a cone consisting of the set of all straight lines joining the boundary of a plane figure $PQR$ to a point $A$ not in the same plane of $PQR$:


:


In the above diagram, the point $A$ is known as the apex of the cone.",Definition:Cone (Geometry)/Apex,['Definitions/Cones'],"Consider a cone consisting of the set of all straight lines joining the boundary of a plane figure PQR to a point A not in the same plane of PQR:


:


In the above diagram, the point A is known as the apex of the cone."
Definition:Vertex,Vertex,":


Let $K$ be an ellipse.

A vertex of $K$ is either of the two endpoints of the major axis of $K$.


In the above diagram, $V_1$ and $V_2$ are the vertices of $K$.
:


Let $P$ be a parabola.

The vertex of $P$ is the point where the axis intersects $P$.


In the above diagram, $V$ is the vertex of $P$.

",Definition:Vertex of Conic Section,"['Definitions/Vertices of Conic Sections', 'Definitions/Conic Sections']",":


Let K be an ellipse.

A vertex of K is either of the two endpoints of the major axis of K.


In the above diagram, V_1 and V_2 are the vertices of K.
:


Let P be a parabola.

The vertex of P is the point where the axis intersects P.


In the above diagram, V is the vertex of P.

"
Definition:Vertex,Vertex,":


Let $K$ be an ellipse.

A vertex of $K$ is either of the two endpoints of the major axis of $K$.


In the above diagram, $V_1$ and $V_2$ are the vertices of $K$.
",Definition:Ellipse/Vertex,"['Definitions/Vertices of Ellipses', 'Definitions/Vertices of Conic Sections', 'Definitions/Ellipses']",":


Let K be an ellipse.

A vertex of K is either of the two endpoints of the major axis of K.


In the above diagram, V_1 and V_2 are the vertices of K.
"
Definition:Vertex,Vertex,":


Let $P$ be a parabola.

The vertex of $P$ is the point where the axis intersects $P$.


In the above diagram, $V$ is the vertex of $P$.
:


Let $P$ be a parabola.

The vertex of $P$ is the point where the axis intersects $P$.


In the above diagram, $V$ is the vertex of $P$.
",Definition:Parabola/Vertex,"['Definitions/Vertices of Conic Sections', 'Definitions/Parabolas']",":


Let P be a parabola.

The vertex of P is the point where the axis intersects P.


In the above diagram, V is the vertex of P.
:


Let P be a parabola.

The vertex of P is the point where the axis intersects P.


In the above diagram, V is the vertex of P.
"
Definition:Vertex,Vertex,,Definition:Hyperbola/Vertex,"['Definitions/Vertices of Conic Sections', 'Definitions/Hyperbolas']",
Definition:Walk,Walk,"Let $G = \struct {V, E}$ be a graph.

A walk $W$ on $G$ is:
:an alternating sequence of vertices $v_1, v_2, \ldots$ and edges $e_1, e_2, \ldots$ of $G$
:beginning and ending with a vertex
:in which edge $e_j$ of $W$ is incident with the vertex $v_j$ and the vertex $v_{j + 1}$.


A walk between two vertices $u$ and $v$ is called a $u$-$v$ walk.


To describe a walk on a simple graph it is sufficient to list just the vertices in order, as the edges (being unique between vertices) are unambiguous.


=== Closed ===

=== Open ===
",Definition:Walk (Graph Theory),"['Definitions/Graph Theory', 'Definitions/Walks']","Let G = V, E be a graph.

A walk W on G is:
:an alternating sequence of vertices v_1, v_2, … and edges e_1, e_2, … of G
:beginning and ending with a vertex
:in which edge e_j of W is incident with the vertex v_j and the vertex v_j + 1.


A walk between two vertices u and v is called a u-v walk.


To describe a walk on a simple graph it is sufficient to list just the vertices in order, as the edges (being unique between vertices) are unambiguous.


=== Closed ===

=== Open ===
"
Definition:Walk,Walk,"Let $G = \struct {V, E}$ be a graph.

A walk $W$ on $G$ is:
:an alternating sequence of vertices $v_1, v_2, \ldots$ and edges $e_1, e_2, \ldots$ of $G$
:beginning and ending with a vertex
:in which edge $e_j$ of $W$ is incident with the vertex $v_j$ and the vertex $v_{j + 1}$.


A walk between two vertices $u$ and $v$ is called a $u$-$v$ walk.


To describe a walk on a simple graph it is sufficient to list just the vertices in order, as the edges (being unique between vertices) are unambiguous.


=== Closed ===

=== Open ===

Let $G = \struct {V, E}$ be a graph.

A walk $W$ on $G$ is:
:an alternating sequence of vertices $v_1, v_2, \ldots$ and edges $e_1, e_2, \ldots$ of $G$
:beginning and ending with a vertex
:in which edge $e_j$ of $W$ is incident with the vertex $v_j$ and the vertex $v_{j + 1}$.


A walk between two vertices $u$ and $v$ is called a $u$-$v$ walk.


To describe a walk on a simple graph it is sufficient to list just the vertices in order, as the edges (being unique between vertices) are unambiguous.


=== Closed ===

=== Open ===

",Definition:Walk (Graph Theory)/Closed,['Definitions/Walks'],"Let G = V, E be a graph.

A walk W on G is:
:an alternating sequence of vertices v_1, v_2, … and edges e_1, e_2, … of G
:beginning and ending with a vertex
:in which edge e_j of W is incident with the vertex v_j and the vertex v_j + 1.


A walk between two vertices u and v is called a u-v walk.


To describe a walk on a simple graph it is sufficient to list just the vertices in order, as the edges (being unique between vertices) are unambiguous.


=== Closed ===

=== Open ===

Let G = V, E be a graph.

A walk W on G is:
:an alternating sequence of vertices v_1, v_2, … and edges e_1, e_2, … of G
:beginning and ending with a vertex
:in which edge e_j of W is incident with the vertex v_j and the vertex v_j + 1.


A walk between two vertices u and v is called a u-v walk.


To describe a walk on a simple graph it is sufficient to list just the vertices in order, as the edges (being unique between vertices) are unambiguous.


=== Closed ===

=== Open ===

"
Definition:Walk,Walk,"Let $G = \struct {V, A}$ be a digraph.


A directed walk in $G$ is a finite or infinite sequence $\sequence {x_k}$ such that:

:$\forall k \in \N: k + 1 \in \Dom {\sequence {x_k} }: \tuple {x_k, x_{k + 1} } \in A$
",Definition:Directed Walk,"['Definitions/Digraphs', 'Definitions/Walks']","Let G = V, A be a digraph.


A directed walk in G is a finite or infinite sequence x_k such that:

:∀ k ∈: k + 1 ∈x_k: x_k, x_k + 1∈ A
"
Definition:Walk,Walk,"Let $\sequence {X_n}_{n \mathop \ge 0}$ be a Markov chain whose state space is the set of integers $\Z$.

Let $\sequence {X_n}$ be such that $X_{n + 1}$ is an element of the set $\set {X_n + 1, X_n, X_n - 1}$.

Then $\sequence {X_n}$ is a one-dimensional random walk.
",Definition:Random Walk,"['Definitions/Stochastic Processes', 'Definitions/Markov Chains', 'Definitions/Random Walks']","Let X_n_n ≥ 0 be a Markov chain whose state space is the set of integers .

Let X_n be such that X_n + 1 is an element of the set X_n + 1, X_n, X_n - 1.

Then X_n is a one-dimensional random walk.
"
Definition:Wavelength,Wavelength,"Let $\phi$ be a periodic wave expressed as:
:$\forall x, t \in \R: \map \phi {x, t} = \map f {x - c t}$


The wavelength $\lambda$ of $\phi$ is the period of the wave profile of $\phi$.",Definition:Periodic Wave/Wavelength,"['Definitions/Periodic Waves', 'Definitions/Wavelength']","Let ϕ be a periodic wave expressed as:
:∀ x, t ∈: ϕx, t =  f x - c t


The wavelength λ of ϕ is the period of the wave profile of ϕ."
Definition:Wavelength,Wavelength,"The wavelength of a wave is is the distance over which the wave's shape repeats.


",Definition:Wavelength (Physics),['Definitions/Physics'],"The wavelength of a wave is is the distance over which the wave's shape repeats.


"
Definition:Weak,Weak,"In a conditional $p \implies q$, the statement $q$ is weaker than $p$.",Definition:Conditional/Language of Conditional/Weak,['Definitions/Conditional'],"In a conditional p  q, the statement q is weaker than p."
Definition:Weak,Weak,,Definition:Weak Convergence,[],
Definition:Weak,Weak,"Let $T = \struct {S, \tau}$ be a topological space.


Then $T$ is weakly locally compact  every point of $S$ has a compact neighborhood.",Definition:Weakly Locally Compact Space,['Definitions/Compact Spaces'],"Let T = S, τ be a topological space.


Then T is weakly locally compact  every point of S has a compact neighborhood."
Definition:Weight,Weight,"Let $N = \left({V, E, w}\right)$ be a network.

The mapping $w: E \to \R$ is known as the weight function of $N$.
Let $N = \struct {V, E, w}$ be a network with weight function $w: E \to \R$.


The values of the elements of $E$ under $w$ are known as the weights of the edges of $N$.


The weights of a network $N$ can be depicted by writing the appropriate numbers next to the edges of the underlying graph of $N$.
Let $N = \struct {V, E, w}$ be a network with weight function $w: E \to \R$.


The values of the elements of $E$ under $w$ are known as the weights of the edges of $N$.


The weights of a network $N$ can be depicted by writing the appropriate numbers next to the edges of the underlying graph of $N$.
",Definition:Network/Weight,['Definitions/Network Theory'],"Let N = (V, E, w) be a network.

The mapping w: E → is known as the weight function of N.
Let N = V, E, w be a network with weight function w: E →.


The values of the elements of E under w are known as the weights of the edges of N.


The weights of a network N can be depicted by writing the appropriate numbers next to the edges of the underlying graph of N.
Let N = V, E, w be a network with weight function w: E →.


The values of the elements of E under w are known as the weights of the edges of N.


The weights of a network N can be depicted by writing the appropriate numbers next to the edges of the underlying graph of N.
"
Definition:Weight,Weight,"Let $N = \left({V, E, w}\right)$ be a network.

The mapping $w: E \to \R$ is known as the weight function of $N$.",Definition:Network/Weight Function,['Definitions/Network Theory'],"Let N = (V, E, w) be a network.

The mapping w: E → is known as the weight function of N."
Definition:Weight,Weight,"A weight function on a set $S$ is a mapping from $S$ to the real numbers:
:$w: S \to \R$


It is common for the requirements of a specific application under discussion for the codomain of $w$ to be restricted to the positive reals:
:$w: S \to \R_{\ge 0}$


The thing that determines whether a given mapping is a weight function depends more on how it is used.",Definition:Weight Function,"['Definitions/Statistics', 'Definitions/Discrete Mathematics', 'Definitions/Analysis']","A weight function on a set S is a mapping from S to the real numbers:
:w: S →


It is common for the requirements of a specific application under discussion for the codomain of w to be restricted to the positive reals:
:w: S →_≥ 0


The thing that determines whether a given mapping is a weight function depends more on how it is used."
Definition:Weight,Weight,"Let $C$ be a codeword of a linear code.

The weight of $C$ is the number of non-zero terms of $C$.",Definition:Weight of Linear Codeword,['Definitions/Linear Codes'],"Let C be a codeword of a linear code.

The weight of C is the number of non-zero terms of C."
Definition:Weight,Weight,,Definition:Weight,[],
Definition:Word,Word,"Let $\struct {M, \circ}$ be a magma.

Let $S \subseteq M$ be a subset.

A word in $S$ is the product of a finite number of elements of $S$.


The set of words in $S$ is denoted $\map W S$:
:$\map W S := \set {s_1 \circ s_2 \circ \cdots \circ s_n: n \in \N_{>0}: s_i \in S, 1 \le i \le n}$


Note that there is nothing in this definition preventing any of the elements of $S$ being repeated, neither is anything said about the order of these elements.


=== Monoid ===

Let $\struct {M, \circ}$ be a magma.

Let $S \subseteq M$ be a subset.

A word in $S$ is the product of a finite number of elements of $S$.


The set of words in $S$ is denoted $\map W S$:
:$\map W S := \set {s_1 \circ s_2 \circ \cdots \circ s_n: n \in \N_{>0}: s_i \in S, 1 \le i \le n}$


Note that there is nothing in this definition preventing any of the elements of $S$ being repeated, neither is anything said about the order of these elements.


=== Monoid ===

",Definition:Word (Abstract Algebra),"['Definitions/Words (Abstract Algebra)', 'Definitions/Group Theory', 'Definitions/Abstract Algebra']","Let M, ∘ be a magma.

Let S ⊆ M be a subset.

A word in S is the product of a finite number of elements of S.


The set of words in S is denoted W S:
:W S := s_1 ∘ s_2 ∘⋯∘ s_n: n ∈_>0: s_i ∈ S, 1 ≤ i ≤ n


Note that there is nothing in this definition preventing any of the elements of S being repeated, neither is anything said about the order of these elements.


=== Monoid ===

Let M, ∘ be a magma.

Let S ⊆ M be a subset.

A word in S is the product of a finite number of elements of S.


The set of words in S is denoted W S:
:W S := s_1 ∘ s_2 ∘⋯∘ s_n: n ∈_>0: s_i ∈ S, 1 ≤ i ≤ n


Note that there is nothing in this definition preventing any of the elements of S being repeated, neither is anything said about the order of these elements.


=== Monoid ===

"
Definition:Word,Word,"Let $S$ be a set.


A group word on $S$ is an ordered tuple on the set of literals $S^\pm$ of $S$.",Definition:Group Word on Set,['Definitions/Group Words'],"Let S be a set.


A group word on S is an ordered tuple on the set of literals S^± of S."
Definition:Word,Word,"Let $\AA$ be an alphabet.


Then a word in $\AA$ is a juxtaposition of finitely many (primitive) symbols of $\AA$.

Words are the most ubiquitous of collations used for formal languages.",Definition:Word (Formal Systems),['Definitions/Collations'],"Let Å be an alphabet.


Then a word in Å is a juxtaposition of finitely many (primitive) symbols of Å.

Words are the most ubiquitous of collations used for formal languages."
Definition:Word,Word,"A word in natural language is intuitively understood as a sequence of sounds which expresses a concept.

When written down, it appears as a sequence of letters, each one of which either is, or contributes to, a phoneme.",Definition:Word (Natural Language),['Definitions/Language Definitions'],"A word in natural language is intuitively understood as a sequence of sounds which expresses a concept.

When written down, it appears as a sequence of letters, each one of which either is, or contributes to, a phoneme."
Definition:Word,Word,"Let $\struct {G, \circ}$ be a group.

Let $S$ be a generating set for $G$ which is closed under inverses (that is, $x^{-1} \in S \iff x \in S$).


The word metric on $G$ with respect to $S$ is the metric $d_S$ defined as follows:

:For any $g, h \in G$, let $\map {d_S} {g, h}$ be the minimum length among the finite sequences $\tuple {x_1, \dots, x_n}$ with each $x_i \in S$ such that $g \circ x_1 \circ \cdots \circ x_n = h$.


Informally, $\map {d_S} {g, h}$ is the smallest number of elements from $S$ that one needs to multiply by to get from $g$ to $h$.",Definition:Word Metric,"['Definitions/Group Theory', 'Definitions/Examples of Metric Spaces']","Let G, ∘ be a group.

Let S be a generating set for G which is closed under inverses (that is, x^-1∈ S  x ∈ S).


The word metric on G with respect to S is the metric d_S defined as follows:

:For any g, h ∈ G, let d_Sg, h be the minimum length among the finite sequences x_1, …, x_n with each x_i ∈ S such that g ∘ x_1 ∘⋯∘ x_n = h.


Informally, d_Sg, h is the smallest number of elements from S that one needs to multiply by to get from g to h."
Definition:Zero,Zero,"The zero ordinal, denoted $0$, is the empty set $\O$.",Definition:Zero (Ordinal),['Definitions/Ordinals'],"The zero ordinal, denoted 0, is the empty set Ø."
Definition:Zero,Zero,"The cardinal associated with the empty set $\O$ is called zero, and is denoted $0$.


More informally, this means that zero is defined as being the number of elements in the empty set.",Definition:Zero (Cardinal),['Definitions/Cardinals'],"The cardinal associated with the empty set Ø is called zero, and is denoted 0.


More informally, this means that zero is defined as being the number of elements in the empty set."
Definition:Zero,Zero,"Let $\C$ denote the set of complex numbers.

The zero of $\C$ is the complex number:
:$0 + 0 i$",Definition:Zero (Number)/Complex,"['Definitions/Complex Numbers', 'Definitions/Zero']","Let  denote the set of complex numbers.

The zero of  is the complex number:
:0 + 0 i"
Definition:Zero,Zero,"Let $x \in \R$ be a number.

Let $b \in \Z$ such that $b > 1$ be a number base in which $x$ is represented.

By the Basis Representation Theorem, $x$ can be expressed uniquely in the form:

:$\ds x = \sum_{j \mathop \in \Z}^m r_j b^j$


Any instance of $r_j$ being equal to $0$ is known as a zero (digit) of $n$.",Definition:Zero Digit,"['Definitions/Zero Digit', 'Definitions/Zero', 'Definitions/Digits', 'Definitions/Numbers']","Let x ∈ be a number.

Let b ∈ such that b > 1 be a number base in which x is represented.

By the Basis Representation Theorem, x can be expressed uniquely in the form:

:x = ∑_j ∈^m r_j b^j


Any instance of r_j being equal to 0 is known as a zero (digit) of n."
Definition:Zero,Zero,"Let $\struct {S, \circ, \preceq}$ be a naturally ordered semigroup.

Then from , $\struct {S, \circ, \preceq}$ has a smallest element.


This smallest element of $\struct {S, \circ, \preceq}$ is called zero and has the symbol $0$.

That is:
:$\forall n \in S: 0 \preceq n$",Definition:Zero (Number)/Naturally Ordered Semigroup,"['Definitions/Naturally Ordered Semigroup', 'Definitions/Zero']","Let S, ∘, ≼ be a naturally ordered semigroup.

Then from , S, ∘, ≼ has a smallest element.


This smallest element of S, ∘, ≼ is called zero and has the symbol 0.

That is:
:∀ n ∈ S: 0 ≼ n"
Definition:Zero,Zero,"Let $\struct {S, \circ}$ be an algebraic structure.

An element $z_L \in S$ is called a left zero element (or just left zero) :
:$\forall x \in S: z_L \circ x = z_L$
Let $\struct {S, \circ}$ be an algebraic structure.

An element $z_R \in S$ is called a right zero element (or just right zero) :
:$\forall x \in S: x \circ z_R = z_R$
",Definition:Zero Element,"['Definitions/Abstract Algebra', 'Definitions/Zero Elements']","Let S, ∘ be an algebraic structure.

An element z_L ∈ S is called a left zero element (or just left zero) :
:∀ x ∈ S: z_L ∘ x = z_L
Let S, ∘ be an algebraic structure.

An element z_R ∈ S is called a right zero element (or just right zero) :
:∀ x ∈ S: x ∘ z_R = z_R
"
Definition:Zero,Zero,"Let $\struct {R, +, \circ}$ be a ring.

The identity for ring addition is called the ring zero (of $\struct {R, +, \circ}$).


It is denoted $0_R$ (or just $0$ if there is no danger of ambiguity).",Definition:Ring Zero,['Definitions/Ring Theory'],"Let R, +, ∘ be a ring.

The identity for ring addition is called the ring zero (of R, +, ∘).


It is denoted 0_R (or just 0 if there is no danger of ambiguity)."
Definition:Zero,Zero,"Let $\struct {F, +, \times}$ be a field.

The identity for field addition is called the field zero (of $\struct {F, +, \times}$).


It is denoted $0_F$ (or just $0$ if there is no danger of ambiguity).",Definition:Field Zero,['Definitions/Field Theory'],"Let F, +, × be a field.

The identity for field addition is called the field zero (of F, +, ×).


It is denoted 0_F (or just 0 if there is no danger of ambiguity)."
Definition:Zero,Zero,"Let $f: R \to R$ be a mapping on a ring $R$.

Let $x \in R$.


Then the values of $x$ for which $\map f x = 0_R$ are known as the roots of the mapping $f$.",Definition:Root of Mapping,"['Definitions/Roots of Mappings', 'Definitions/Ring Theory', 'Definitions/Field Theory', 'Definitions/Real Analysis', 'Definitions/Complex Analysis']","Let f: R → R be a mapping on a ring R.

Let x ∈ R.


Then the values of x for which f x = 0_R are known as the roots of the mapping f."
Definition:Zero Locus,Zero Locus,"Let $k$ be a field.

Let $n\geq1$ be a natural number.

Let $A = k \sqbrk {X_1, \ldots, X_n}$ be the polynomial ring in $n$ variables over $k$.

Let $I \subseteq A$ be a set.


Then the zero locus of $I$ is the set:

:$\map V I = \set {x \in k^n : \forall f \in I: \map f x = 0}$",Definition:Zero Locus of Set of Polynomials,['Definitions/Algebraic Geometry'],"Let k be a field.

Let n≥1 be a natural number.

Let A = k X_1, …, X_n be the polynomial ring in n variables over k.

Let I ⊆ A be a set.


Then the zero locus of I is the set:

:V I = x ∈ k^n : ∀ f ∈ I:  f x = 0"
Definition:Zero Locus,Zero Locus,"Let $A$ be a commutative ring with unity.

Let $S \subseteq A$ be a subset.


The vanishing set of $S$ is the set of prime ideals of $A$ containing $S$:
:$\map V S = \set {\mathfrak p \in \Spec A: \mathfrak p \supseteq S}$",Definition:Vanishing Set of Subset of Ring,['Definitions/Zariski Topology'],"Let A be a commutative ring with unity.

Let S ⊆ A be a subset.


The vanishing set of S is the set of prime ideals of A containing S:
:V S = 𝔭∈ A: 𝔭⊇ S"
